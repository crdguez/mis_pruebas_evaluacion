\documentclass[addpoints,spanish, 12pt,a4paper]{exam}
%\documentclass[answers, spanish, 12pt,a4paper]{exam}
\printanswers
\renewcommand*\half{.5}
\pointpoints{punto}{puntos}
\hpword{Puntos:}
\vpword{Puntos:}
\htword{Total}
\vtword{Total}
\hsword{Resultado:}
\hqword{Ejercicio:}
\vqword{Ejercicio:}

\usepackage[utf8]{inputenc}
\usepackage[spanish]{babel}
\usepackage{eurosym}
%\usepackage[spanish,es-lcroman, es-tabla, es-noshorthands]{babel}


\usepackage[margin=1in]{geometry}
\usepackage{amsmath,amssymb}
\usepackage{multicol}
\usepackage{yhmath}

\pointsinrightmargin % Para poner las puntuaciones a la derecha. Se puede cambiar. Si se comenta, sale a la izquierda.
\extrawidth{-2.4cm} %Un poquito más de margen por si ponemos extremostextos largos.
\marginpointname{ \emph{\points}}

\usepackage{graphicx}

\graphicspath{{../img/}} 

\newcommand{\class}{Matemáticas I}
\newcommand{\examdate}{\today}
\newcommand{\examnum}{Extraordinaria}
\newcommand{\tipo}{A}


\newcommand{\timelimit}{90 minutos}

\renewcommand{\solutiontitle}{\noindent\textbf{Solución:}\enspace}


\pagestyle{head}
\firstpageheader{\includegraphics[width=0.2\columnwidth]{header_left}}{\textbf{Departamento de Matemáticas\linebreak \class}\linebreak \examnum}{\includegraphics[width=0.1\columnwidth]{header_right}}
\runningheader{\class}{\examnum}{Página \thepage\ de \numpages}
\runningheadrule


\usepackage{pgf,tikz,pgfplots}
\pgfplotsset{compat=1.15}
\usepackage{mathrsfs}
\usetikzlibrary{arrows}


\begin{document}

\noindent
\begin{tabular*}{\textwidth}{l @{\extracolsep{\fill}} r @{\extracolsep{6pt}} }
\textbf{Nombre:} \makebox[3.5in]{\hrulefill} & \textbf{Fecha:}\makebox[1in]{\hrulefill} \\
 & \\
\textbf{Tiempo: \timelimit} & Tipo: \tipo 
\end{tabular*}
\rule[2ex]{\textwidth}{2pt}
Esta prueba tiene \numquestions\ ejercicios. La puntuación máxima es de \numpoints. La nota final de la prueba será la parte proporcional de la puntuación obtenida sobre la puntuación máxima. 



\begin{center}


\addpoints
 %\gradetable[h][questions]
	\pointtable[h][questions]
\end{center}

\noindent
\rule[2ex]{\textwidth}{2pt}
\begin{questions}
\question[2] Resuelve mediante expresiones algebraicas y, en caso que se pueda, por
Gauss:
\begin{parts}
%     \part Tres amigas, Elena, Carmen y Cristina entran en una tienda de deportes en la que sólo hay
% tres tipos de artículos. Carmen
% se compra 1 par de zapatillas, 2 sudaderas y 2 pantalones. Elena se compra 2 pares de zapatillas, 1 sudadera y 1 pantalón, y Cristina se compra 2 pares de zapatillas
% y 3 pantalones. Carmen se ha gastado en total 80 euros, Elena y Cristina 70 euros cada una. ¿Cuánto vale
% cada artículo? 
% \begin{solution}
%     \[
% \left[
% \begin{array}{ccc|c}
% 1 & 2 & 2 & 80 \\
% 2 & 1 & 1 & 70 \\
% 2 & 0 & 3 & 70
% \end{array}
% \right]
% \]

% \[
% \left[
% \begin{array}{ccc|c}
% 1 & 2 & 2 & 80 \\
% 0 & 1 & 1 & 30 \\
% 0 & 0 & 1 & 10
% \end{array}
% \right]
% \]

% \[
% \begin{aligned}
% z &= 20 \quad \text{(zapatillas)} \\
% s &= 20 \quad \text{(sudadera)} \\
% p &= 10 \quad \text{(pantalón)}
% \end{aligned}
% \]

% \end{solution}

\part Un padre decide repartir su fortuna de 480 monedas de oro entre sus tres hijas: Ana,
Carla y Pilar. La cantidad que recibe Ana
es el doble de la suma de las cantidades que reciben Carla y Pilar. Además, la suma de
las cantidades que reciben Ana y Pilar es igual al triple de la cantidad que recibe Carla.

\begin{solution}
$\left\{\begin{matrix}x + y + z = 480\\x = 2 y + 2 z\\x + z = 3 y\\\end{matrix}\right.$ \\  Por Gauss: \\ $\left(\begin{matrix}1 & 1 & 1 & 480\\1 & -2 & -2 & 0\\1 & -3 & 1 & 0\end{matrix}\right)\rightarrow\left(\begin{matrix}1 & 1 & 1 & 480\\0 & -3 & -3 & -480\\0 & 0 & 4 & 160\end{matrix}\right)\to$ Sol:$\left\{\left( 320, \  120, \  40\right)\right\}$ \\ Por Matriz inversa: \\ $X=A^{-1}\cdot b=\left(\begin{matrix}\frac{2}{3} & \frac{1}{3} & 0\\\frac{1}{4} & 0 & - \frac{1}{4}\\\frac{1}{12} & - \frac{1}{3} & \frac{1}{4}\end{matrix}\right)\cdot\left(\begin{matrix}480\\0\\0\end{matrix}\right)=\left(\begin{matrix}320\\120\\40\end{matrix}\right)$. \ Ya que $\left(\begin{matrix}1 & 1 & 1\\1 & -2 & -2\\1 & -3 & 1\end{matrix}\right)\xrightarrow{traspuesta}\left(\begin{matrix}1 & 1 & 1\\1 & -2 & -3\\1 & -2 & 1\end{matrix}\right)\xrightarrow{adjunta}\left(\begin{matrix}-8 & -4 & 0\\-3 & 0 & 3\\-1 & 4 & -3\end{matrix}\right)\xrightarrow{inversa}\left(\begin{matrix}\frac{2}{3} & \frac{1}{3} & 0\\\frac{1}{4} & 0 & - \frac{1}{4}\\\frac{1}{12} & - \frac{1}{3} & \frac{1}{4}\end{matrix}\right)$. \\ Por Cramer: \\ $det(A)=-12$ \ $\Delta_0$, $s_0$: $\left( -3840, \  320\right)$ \ $\Delta_1$, $s_1$: $\left( -1440, \  120\right)$ \ $\Delta_2$, $s_2$: $\left( -480, \  40\right)$ \ \end{solution}

    \part $\left\{\begin{matrix}{2^x} + {2^y} = 24\\ \\{2^{x + y}} = 128\end{matrix}\right. $

\begin{solution}
    $\rightarrow \left [ \left \{ x : 3, \quad y : 4\right \}, \quad \left \{ x : 4, \quad y : 3\right \}\right ]$
\end{solution}

% \part Resuelve la inecuación: $ \dfrac{{{x^3} - 5{x^2} + 2x + 8}}{1-x^2} \leq 0 $  \begin{solution}  $  \left(1, 2\right] \cup \left[4, \infty\right) $  \end{solution}
        % \part Resuelve la inecuación: $1-\dfrac{{x + 3}}{{x + 6}} \geq 0 $ \begin{solution} $\rightarrow \left(-6, \infty\right)$\end{solution}

\end{parts}

% \question[1] Responde a las siguientes cuestiones:
% \begin{parts}
%             % \part Resuelve la inecuación: $ \dfrac{{{x^3} - 5{x^2} + 2x + 8}}{1-x^2} \leq 0 $  \begin{solution}  $  \left(1, 2\right] \cup \left[4, \infty\right) $  \end{solution}
%         \part Resuelve la inecuación: $1-\dfrac{{x + 3}}{{x + 6}} \geq 0 $ \begin{solution} $\rightarrow \left(-6, \infty\right)$\end{solution}
% \end{parts}

\question Responde a las siguientes cuestiones:
\begin{parts}
    % \part[2] Resuelve la ecuación $1 + \cos{x} = \sen^2{x}$
    % \begin{solution}
    %     $[-90, 90, 180]$
    % \end{solution}
    % \part Resuelve la ecuación $\sen{x} = \cos^2{x}-1$
    % \begin{solution}
    %     $[0, 180, 270]$
    % \end{solution}
    \part[1] Resuelve la ecuación $$ \sen{x}- \sqrt{3}\cos{x}=0 $$  \begin{solution}  $ \left [ 60, \quad 240\right ] $  \end{solution}
    % \part[2]  Resuelve la ecuación $ 2\cos^2{x}-\sqrt{3}\cos{x}=0 $  \begin{solution}  $ \left [ 30, \quad 90, \quad 270, \quad 330\right ] $  \end{solution}
    % \part Resuelve la ecuación  $ \cos{2x}-3\cos{x}+1=0 $  \begin{solution}  $ \left [ -90, \quad 90\right ] $  \end{solution}
    % \part[1] Desde un punto del suelo se ve la copa de un pino bajo un ángulo de 42°. Si nos alejamos 2,5 m hacia otro punto del suelo, alineado con el anterior y con el pie del pino, vemos la copa bajo un ángulo de 24°. Calcula la altura del pino. \begin{solution}
    %     Aplicamos el teorema del seno para saber la hipotenusa del triángulo pequeño. Después calculamos la altura con el seno
    %     $$h=2.5\frac{\sin{24\sin{42}}}{\sin{18}}\approx2,2 m$$
    %     \end{solution}
            \part[2] Desde un punto del suelo se ve la copa de un pino bajo un ángulo de 42°. Si nos alejamos 2,5 m hacia otro punto del suelo, alineado con el anterior y con el pie del pino, vemos la copa bajo un ángulo de 24°. Calcula la distancia del pino a la posición final del observador. \begin{solution}
        Aplicamos el teorema del seno para saber la hipotenusa del triángulo pequeño. Después calculamos la altura con el seno
        $$h=2.5 + 2.5\frac{\sin{24\cos{42}}}{\sin{18}}\approx4,95 m$$
        \end{solution}
\end{parts}

% \question Resolver las siguientes ecuaciones:
% \begin{parts} \part[1]  $ \cos{2x}-3\cos{x}+1=0 $  \begin{solution}  $ \left [ -90, \quad 90\right ] $  \end{solution} \part[1]  $ 2\cos^2{x}-\sqrt{3}\cos{x}=0 $  \begin{solution}  $ \left [ 30, \quad 90, \quad 270, \quad 330\right ] $  \end{solution}
% \end{parts}

% \question Responde a las siguientes cuestiones relacionadas con complejos:
% \begin{parts} \part[1]  Calcula $ \dfrac{(1+2i)i^7}{(3-2i)-(2+i)} $  \begin{solution}  $ \frac{1}{2} + \frac{i}{2} $  \end{solution}
% \part[1] Escribe los conjugados de los siguientes números complejos en forma polar:  $ -4 $ , $2i$, $ 2 - 2\sqrt {3}i $\begin{solution}
%     $4_{180}$, $2_{270}, 4_{60}$
% \end{solution}
% \end{parts}

% \question[1] Escribe los siguientes números complejos en forma polar con el argumento en radianes:
% \begin{parts} \part  $ -4 $  \begin{solution}  $ 4_{\pi} $  \end{solution} \part  $ 2i $  \begin{solution}  $ 2_{\frac{\pi}{2}} $  \end{solution} \part  $ 2 - 2\sqrt {3}i $  \begin{solution}  $ 4_{- \frac{\pi}{3}} $  \end{solution}
% \end{parts}

\question[2] Resuelve las siguientes cuestiones:
        \begin{parts}
        \part Comprueba analíticamente que los puntos A(-3,2), B(1,9) y C(4,-2) forman un triángulo rectángulo en A.
\begin{solution}
    $\vec{AB}=(4,7)$ , $\vec{AC}=(7,-4)$ y $\vec{AB}\cdot\vec{AC}=0$
\end{solution}
        % \part  Determina el ángulo formado por las rectas: $ r\equiv2x-y-2=0\  y \ s\equiv3x+2y-4=0 $  \begin{solution}  $ 119.74488129694222 $  \end{solution}
        % \part Determina la ecuación general o implícita de la recta perpendicular a la recta \( 3x - 4y = 12 \) y  que pasa por el punto \( P(2, 3) \)
        % \begin{solution}
        %     $4x+3y-17=0$
        % \end{solution}
                \part Determina la ecuación de la recta que pase por el punto \( P(2, 3) \) y que sea perpendicular a la recta que pasa por los puntos $Q(0,3)$ y $R(4,0)$  
        \begin{solution}
            $4x+3y=17$
        \end{solution}
        \end{parts}

\question Dada la función:$f(x)=\dfrac{- x^{2} - x + 3}{x^{2} + x - 2}$. Calcular:

        \begin{parts} \part[0\half] Dominio de $f(x)$  \begin{solution}   $Dom(f)=\left(-\infty, -2\right) \cup \left(-2, 1\right) \cup \left(1, \infty\right)$\\  \end{solution} \part[1\half] Asíntotas verticales, horizontales y oblicuas, en caso que existan  \begin{solution}   Asíntotas:\\A.V. $x=-2$\\, A.V. $x=1$\\A.H. $y=-1$\\A.H. $y=-1$\\A.O. $y=-1$ \\A.O. $y=-1$ \\   \end{solution}
        \end{parts}

% \question[2] Determina \( a \) y \( b \) para que \( f(x) \) sea continua en todo \( \mathbb{R} \):
% \[
% f(x) =
% \begin{cases}
%     x + a & \text{si } x \leq 2 \\
%     3x - 2 & \text{si } 2 < x < 5 \\
%     ax + b & \text{si } x \geq 5
% \end{cases}
% \]

% \question[2] La función $B(x) = \dfrac{-x^2 + 9x - 16}{x}$ representa, en miles de euros, el beneficio neto de un proceso de venta, siendo \( x \) el número de artículos vendidos. Calcule el número de artículos que deben venderse para obtener el beneficio máximo y determine dicho beneficio máximo.
% \begin{solution}
%     $B'(x)=-1 + \frac{16}{x^{2}} \to x=4 \to B(4)=1 \to 1000 \euro$
% \end{solution}


\question[2] La función de costes de una empresa es 
$C(q)=q^3+3q+10$, donde $q$ representa las unidades producidas.  Sabiendo que el precio de venta, en euros, de cada unidad
producida es $p = 30$, se desea conocer:
\begin{parts}
    \part La función de beneficio de esta empresa.
    \begin{solution}
        $f(x)=- x^{3} + 27 x - 10$
    \end{solution}
    \part El número de unidades producidas que maximiza el beneficio de la empresa. 
    \begin{solution}
        $f'(x)=27 - 3 x^{2} \land f''(x)=-6x \to f'(3)=0 \land f''(3) = -18 \to x=3$
    \end{solution}
    \part El beneficio máximo que puede lograr la empresa.
    \begin{solution}
    $f(3)=44$    
    \end{solution}
    
\end{parts}

% \question Se considera la función real:
% \[
% f(x) = x^3 + ax^2 + bx \quad a, b \in \mathbb{R}
% \]

% \begin{parts}
% \part[1] ¿Qué valores deben tomar \( a \) y \( b \) para que \( f(x) \) tenga un máximo relativo en el punto \( P(1,4) \)?
% \begin{solution}
%     $f'(x)=2 a x + b + 3 x^{2} \to \left\{\begin{matrix}
%         f'(1)=0\\
%         f(1)=4
%     \end{matrix}\right. \to \left\{\begin{matrix}
%         2 a + b + 3\\
%         a + b - 3 = 0
%     \end{matrix}\right. \to \left\{ a : -6, \  b : 9\right\}
% $
% \end{solution}
% \end{parts}

%         \question (2 puntos) \textbf{SOLO PARA AQUELLOS QUE VAN A SUBIR NOTA} La temperatura media en los meses de invierno en varias ciudades y el gasto medio por habitante en
% calefacción ha sido: \\ \\\begin{tabular}{lrrrr}
% \hline
%                       &   Dic &   Ene &   Feb &   Mar \\
% \hline
%  Temperatura (grados) &  10 &  12 &  14 &  16 \\
%  Gasto (euros)        & 150 & 120 & 102 &  90 \\
% \hline
% \end{tabular} \\ \\
% Se proporciona la siguiente tabla de frecuencias:

% \begin{tabular}{rrrrrr}
% \hline
%     &   x &   y &   xy &   x2 &    y2 \\
% \hline
%   &  10 & 150 & 1500 &  100 & 22500 \\
%   &  12 & 120 & 1440 &  144 & 14400 \\
%   &  14 & 102 & 1428 &  196 & 10404 \\
%   &  16 &  90 & 1440 &  256 &  8100 \\
%   Sumas &  52 & 462 & 5808 &  696 & 55404 \\
% \hline

% \end{tabular}
%         \begin{parts} 
% %         \part Haz una tabla de frecuencias con los datos que necesites para hace el resto de apartados  \begin{solution}   \begin{tabular}{rrrrrr}
% % \hline
% %     &   x &   y &   xy &   x2 &    y2 \\
% % \hline
% %   0 &  10 & 150 & 1500 &  100 & 22500 \\
% %   1 &  12 & 120 & 1440 &  144 & 14400 \\
% %   2 &  14 & 102 & 1428 &  196 & 10404 \\
% %   3 &  16 &  90 & 1440 &  256 &  8100 \\
% %   4 &  52 & 462 & 5808 &  696 & 55404 \\
% % \hline
% % \end{tabular}   \end{solution} 

% % \part Calcula el gasto medio  \begin{solution}   {'media': 115.5}   \end{solution} 

% \part Halla el coeficiente de correlación lineal e interprétalo  \begin{solution}   {'media de x': 13.0, 'desviación de x': 2.23606797749979, 'media de y': 115.5, 'desviación de y': 22.599778759979046, 'covarianza': -49.5, 'coeficiente de correlación': -0.9795260923726159}   \end{solution} \part Estima el gasto medio por habitante de una ciudad si la temperatura media hubiera sido 11ºC  \begin{solution}   $y = - 9.9 x + 244.2$ \\La estimación para x=11 es: 135.3   \end{solution}
%         \end{parts}

%EJERCICIO 1
%     \question[1\half] Resuelve las siguientes ecuaciones:
%         \begin{parts}
%         \part   \[
%                 \log(x^3) = 3
%                 \]
        
%         \part   \[
%                 \log_2(x+1) - \log_2(x) = 1
%                 \]
    
%         \part   \[
%                 \sqrt{x+4} + \sqrt{x-2} = 4
%                 \]
%         \end{parts}
% %EJERCICIO 2
%     \question[1] Resuelve el siguiente sistema de ecuaciones utilizando el método de Gauss:

%         \[
%         \begin{cases}
%         x + 2y - z = 3 \\
%         2x - y + 3z = 9 \\
%         -x + 3y + 2z = 4
%         \end{cases}
%         \]
%     %EJERCICIO 3
%     \question[1\half] Resuelve el siguiente sistema de inecuaciones y representa gráficamente:
%         \[
%         \begin{cases}
%         x - y \geq 4 \\
%         3x + 2y < 14
%         \end{cases}
%         \]
% %EJERCICIO 4
%     \question[1] Calcula los siguientes límites:
%         \begin{parts}
%             \part \( \displaystyle \lim_{x \to 1} \frac{x^2 - 1}{x - 1} \)
%             \part \( \displaystyle \lim_{x \to 3} \frac{x^3 - 4x^2 + 5x}{x^3 - 6x^2 + 11x - 6} \)
%         \end{parts}
% %EJERCICIO 5
%     \question[1\half] Di cuáles son las tres condiciones de continuidad, después, estudia la continuidad de la siguiente función en todo su dominio:
%         \[
%         f(x) = 
%         \begin{cases}
%             x^2 + 2x & \text{si } x < 1 \\
%             4x - 3 & \text{si } x \geq 1
%         \end{cases}
%         \]
%     \newpage
% %EJERCICIO 6
%     \question[1\half] Calcula la derivada de las siguientes funciones:
%         \begin{parts}
%             \part \[f(x) = \frac{3x^2 + 4x - 5 }{x+2} \]
%             \part \[f(x) = \frac{1}{x} + 2x^3 \]
%             \part  \[f(x) = \sin(2x+3) + \cos(5x) \]
%         \end{parts}
    
% %EJERCICIO 7    
%     \question Dada la siguiente función y conociendo sus asíntotas como se muestran en la gráfica, calcula:
% \[
% f(x) = x^3 + 2x^2 - x + 4
% \]

% \begin{parts}
%     \part [0\half] Limites en el + \(\infty\) y \(- \infty\)
%     \part [0\half] Derivada 
%     \part [0\half] Máximos y mínimos
%     \part [0\half] Con todo lo anterior, representa la función
% \end{parts}

%    \question[1\half] Resuelve las siguientes ecuaciones:
%         \begin{parts}
%         \part   \[
%                 4\log({x}) = 2
%                 \]
   
%         \part   \[
%                 \log_3(x) - \log_3(x-1) = 3
%                 \]

%         \part   \[
%                 \sqrt{x+5} + \sqrt{x-1} = 6
%                 \]
%         \end{parts}
    
%     \question[1\half] Resuelve el siguiente sistema de inecuaciones y representa gráficamente:
%         \[
%         \begin{cases}
%             x + y > 9 \\
%             -2x +3y  \leq 12
%         \end{cases}
%         \]

%     \question[1] 
%         \begin{parts}
%             \part Escribe el teorema del seno y el teorema del coseno.
%             \part En un triángulo se conocen dos lados y el ángulo comprendido entre ellos: 
%                 \[
%                 a = 7,\quad b = 9,\quad C = 60^\circ
%                 \]
%                 Calcula el lado \( c \) y los ángulos \( A \) y \( B \).
%         \end{parts}

%     \question[1] Calcula los siguientes límites:
%         \begin{parts}
%             \part \( \displaystyle \lim_{x \to 2} \frac{x^2-4}{x-2} \)
%             \part \( \displaystyle \lim_{x \to 2} \frac{x^3 - 5x^2+6x}{x^3-7x^2+16x-12} \)
%         \end{parts}
    
%     \question[1\half] Di cuáles son las tres condiciones de continuidad, después, estudia la continuidad de la siguiente función en todo su dominio:
%         \[
%         f(x) = 
%         \begin{cases}
%             x^2 - 1 & \text{si } x < 2 \\
%             3x - 4 & \text{si } x \geq 2
%         \end{cases}
%         \]
%     \newpage
    
%     \question[1\half] Calcula la derivada de las siguientes funciones:
%         \begin{parts}
%             \part   \[
%                     f(x) = \frac{x^2 + 3x - 1}{x - 2}
%                     \]
%             \part   \[
%                     f(x) = \sin{(3x^2 + 2)}
%                     \]
%             \part   \[
%                     f(x) = e^{\log{2x}}
%                     \]
%         \end{parts}
    
   
    
    
%     \question Dada la siguiente función y conociendo sus asíntotas como se muestran en la gráfica, calcula:
%         \[
%         f(x) = \frac{x^2 +3x + 11}{x + 1} 
%         \]
     
%         \begin{parts}
%             \part [0\half] Dominio
%             \part [1] Derivada y extremos relativos
%             \part [0\half] Representa la función
%         \end{parts}        
%         \begin{tikzpicture}
%             \begin{axis}[ 
%                 axis lines = middle, 
%                 grid = both,              % Cuadrícula activa
%                 xmin = -10, xmax = 10,    % Rango de -10 a 10 en X
%                 ymin = -10, ymax = 10,    % Rango de -10 a 10 en Y
%                 xlabel = \( x \), ylabel = \( y \),
%                 enlargelimits, 
%                 width=15cm, height=15cm,    % Tamaño del gráfico
%                 tick label style={font=\small},
%                 axis line style={-Stealth},
%                 every axis label/.append style={font=\small},
%                 major grid style={dashed, thin, gray},   % Estilo de la cuadrícula mayor
%                 minor grid style={dotted, thin, gray},  % Estilo de la cuadrícula menor
%                 grid style={gray!30}        % Color y transparencia de la cuadrícula
%             ]
%                 % Recta vertical x = -1 (discontinua y roja)
%                 \addplot[red, dashed, thick] coordinates {(-1,-15)(-1,15)};
                
%                 % Recta diagonal y = x + 2 (discontinua y roja)
%                 \addplot[red, dashed, thick, domain=-15:15] {x + 2};
        
%             \end{axis}
%         \end{tikzpicture}
       

    \addpoints




\addpoints
\end{questions}
\end{document}
\grid
