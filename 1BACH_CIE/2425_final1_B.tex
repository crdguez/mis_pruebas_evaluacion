\documentclass[addpoints,spanish, 12pt,a4paper]{exam}
%\documentclass[answers, spanish, 12pt,a4paper]{exam}
% \printanswers
\renewcommand*\half{.5}
\pointpoints{punto}{puntos}
\hpword{Puntos:}
\vpword{Puntos:}
\htword{Total}
\vtword{Total}
\hsword{Resultado:}
\hqword{Ejercicio:}
\vqword{Ejercicio:}

\usepackage[utf8]{inputenc}
\usepackage[spanish]{babel}
\usepackage{eurosym}
%\usepackage[spanish,es-lcroman, es-tabla, es-noshorthands]{babel}


\usepackage[margin=1in]{geometry}
\usepackage{amsmath,amssymb}
\usepackage{multicol}
\usepackage{yhmath}

\pointsinrightmargin % Para poner las puntuaciones a la derecha. Se puede cambiar. Si se comenta, sale a la izquierda.
\extrawidth{-2.4cm} %Un poquito más de margen por si ponemos textos largos.
\marginpointname{ \emph{\points}}

\usepackage{graphicx}

\graphicspath{{../img/}} 

\newcommand{\class}{1º Bachillerato CIE}
\newcommand{\examdate}{\today}
\newcommand{\examnum}{Final 1ªEv.}
\newcommand{\tipo}{A}


\newcommand{\timelimit}{45 minutos}

\renewcommand{\solutiontitle}{\noindent\textbf{Solución:}\enspace}


\pagestyle{head}
\firstpageheader{\includegraphics[width=0.2\columnwidth]{header_left}}{\textbf{Departamento de Matemáticas\linebreak \class}\linebreak \examnum}{\includegraphics[width=0.1\columnwidth]{header_right}}
\runningheader{\class}{\examnum}{Página \thepage\ de \numpages}
\runningheadrule


\usepackage{pgf,tikz,pgfplots}
\pgfplotsset{compat=1.15}
\usepackage{mathrsfs}
\usetikzlibrary{arrows}


\begin{document}

\noindent
\begin{tabular*}{\textwidth}{l @{\extracolsep{\fill}} r @{\extracolsep{6pt}} }
\textbf{Nombre:} \makebox[3.5in]{\hrulefill} & \textbf{Fecha:}\makebox[1in]{\hrulefill} \\
 & \\
\textbf{Tiempo: \timelimit} & Tipo: \tipo 
\end{tabular*}
\rule[2ex]{\textwidth}{2pt}
Esta prueba tiene \numquestions\ ejercicios. La puntuación máxima es de \numpoints. 
La nota final de la prueba será la parte proporcional de la puntuación obtenida sobre la puntuación máxima. 

\begin{center}


\addpoints
 %\gradetable[h][questions]
	\pointtable[h][questions]
\end{center}

\noindent
\rule[2ex]{\textwidth}{2pt}

\begin{questions}

%\question 
%
%\begin{parts}
%\part[2] 
%\begin{solution}
%\end{solution}
%
%
%\end{parts}
%\addpoints

% \question[1] Indica a cuáles de los conjuntos
% $\mathbb{N}$, $\mathbb{Z}$, $\mathbb{Q}$, $\mathbb{R}$ pertenecen cada uno de los siguientes números:
% \begin{center}
% \begin{tabular}{|c |c |c |c |c|}\hline
% &$\mathbb{N}$& $\mathbb{Z}$& $\mathbb{Q}$&$\mathbb{R}$\\ 
% \hline
% $\frac{8}{16}$&&&&\\
% \hline
% $\sqrt[3]{-27}$&&&&\\
% \hline
% $3.0\wideparen{1}$&&&&\\
% \hline
% $-\frac{12}{4}$&&&&\\
% \hline
% $-\sqrt{25}$&&&&\\
% \hline
% $\sqrt{8}$&&&&\\
% \hline
% $4$&&&&\\
% \hline
% $\pi$&&&&\\
% \hline
% $\sqrt{-4}$&&&&\\
% \hline
% $\frac{39}{13}$&&&&\\
% \hline
% \end{tabular}

% \end{center}

%         \question[2]  Dados los conjuntos:\\
%          $ A=\left\{ x \in \mathbb{R}| 6 \leq x  < 8 \right\}, \\ B=\left(-\infty, -3\right) \cup \left(3, \infty\right)  \\  C=\left\{ x \in \mathbb{R}| \left|{x - 3}\right|\leq12 \right\}  $
         
%          \begin{parts}
%           % \part Indica razonadamente, si existen, los supremos, ínfimos, máximos y mínimos de los conjuntos $A$, $B$ y $C$
%           \part Calcula $A \cup  B$ , $A \cap B$ y $(A \cup B) \cap C$ , 
% y expresa los resultados en forma de Intervalos.
%          \end{parts}
         
% \begin{solution}  $ C=\left[-9, 15\right] \ \ A \cup  B = \left(-\infty, -3\right) \cup \left(3, \infty\right)  \\  A \cap B= \left[6, 8\right)   \\  (A \cup B) \cap C= \left[-9, -3\right) \cup \left(3, 15\right] $  \end{solution}

       
        
% \question[1\half] Halla el término general de las siguientes sucesiones. Indica cuáles de ellas son progresiones aritméticas y cuáles progresiones geométricas:
% \begin{parts}
%     \part \( 4, 9, 14, 19, 24, 29, \ldots \)
%     \part \( 2, -6, 18, -54, 162, \ldots \)
% \end{parts}
% \begin{solution}
% $\text{PA: 1. } a_n = 5n - 1; \, \text{SG: -3. } a_n = (-3)^{n-1}\cdot 2$\end{solution}


\question Resuelve las siguientes ecuaciones:

\begin{parts}
%     \part[2]$2 \sqrt{x + 4} - \sqrt{x - 1} = 4$
%     \begin{solution}
% 1. Aislamos uno de los radicales:
%    \[
%    2 \sqrt{x + 4} = \sqrt{x - 1} + 4
%    \]
% 2. Elevamos ambos lados al cuadrado:
%    \[
%    4(x + 4) = (\sqrt{x - 1} + 4)^2
%    \]
%    \[
%    4x + 16 = x - 1 + 8 \sqrt{x - 1} + 16
%    \]
% 3. Simplificamos y resolvemos la ecuación resultante.
% $3x+1=8 \sqrt{x - 1} \to x=13/9, x=5
%  $

% \end{solution}
%     \part[1\half] $\sqrt{2x+3}-\sqrt{x+2}=2$
%     \begin{solution}
%     $2x+3=4+4\sqrt{x+2}+x+2 \to x-3=4\sqrt{x+2} \to x^2-6x+9=16x+32\to x^2-22x-23=0$\\
%     $x=23$ sí, $x=-1$ No
% \end{solution}
%     \part[1\half] $\sqrt[3]{x + 6} = x$ \begin{solution}
% 1. Elevamos ambos lados al cubo:
%    \[
%    x + 6 = x^3
%    \]
% 2. Reorganizamos la ecuación:
%    \[
%    x^3 - x - 6 = 0
%    \]
% 3. Intentamos con divisores de 6 para encontrar posibles raíces.
% $$x^3 - x - 6=\left(x - 2\right) \left(x^{2} + 2 x + 3\right)\to x=2$$
% \end{solution}
%     \part[1\half] $9^x - 6 \cdot {3^{x + 1}} + 81 = 0 $
% \begin{solution}
%     $\rightarrow \left [ 2\right ]$
% \end{solution}
\part[1]  $ 2 \log x - \log (x + 6) = 3 \log 2 $  \begin{solution}  $ \left [ 12\right ] $  \end{solution}
    % \part[1] $|2x-4|=|x+2|$ \begin{solution} $\left[ \frac{2}{3}, \  6\right]$\end{solution}
\end{parts}


\question[1] Da una cota del error absoluto y otra del error relativo cometidos al aproximar el número aúreo $\left(\phi\right)$ a $1.62$

        \question[3] Resuelve, justificadamente, los siguientes sistemas de inecuaciones:
        \begin{multicols}{2}
        \begin{parts} 
         \part  $$ \left\{\begin{matrix}{( {x - 1} )^2} - {( {x + 3} )^2} \leq 0\\x - 3( {x - 1} ) \geq 3 \end{matrix}\right. $$  
         \begin{solution}  $ \left[-1, 0\right] $  \end{solution}
         % \part[2]  $$ \left\{\begin{matrix}x \geq 0 \\0 \leq y \leq 3\\x - 2y \leq 10\\x + y \geq 10\\\end{matrix}\right. $$  \begin{solution}    \end{solution}  

         \part  $$ \left\{\begin{matrix}x \geq 0 \\0 \leq y \leq 3 \\x + y \leq 5\\\end{matrix}\right. $$  \begin{solution}    \end{solution} 
        \end{parts}
                  

        
        % \question Calcula expresando el resultado en forma de fracción algebraica irreducible:
        % \begin{multicols}{1}
        % \begin{parts} \part[1]  $ \dfrac{2+{ \dfrac{1}{x}}}{{2 + \dfrac{1}{{1 + \dfrac{1}{x}}}}} $  \begin{solution}  $ \dfrac{2 x^{2} + 3 x + 1}{3 x^{2} + 2 x} $  \end{solution}
        % \end{parts}
        % \end{multicols}
            
        \end{multicols}

% \question[1] Resolver la ecuación
% \[
% 4x^3 - 6x^2 - x = -\frac{3}{2},
% \]
% sabiendo que una solución es \( \frac{1}{2} \).

% \begin{solution}
% 1. Llevamos todos los términos a un lado:
%    \[
%    4x^3 - 6x^2 - x + \frac{3}{2} = 0
%    \]
% 2. Multiplicamos por 2 para eliminar fracciones:
%    \[
%    8x^3 - 12x^2 - 2x + 3 = 0
%    \]
% 3. Factorizando el polinomio 
% $\frac{\left(2 x - 3\right) \left(2 x - 1\right) \left(2 x + 1\right)}{2} \to \left[ - \frac{1}{2}, \  \frac{1}{2}, \  \frac{3}{2}\right]
% $
% \end{solution}

% \question[1] Resolver la ecuación
% \[
% 2x^3 - 3x^2 = -\frac{1}{2},
% \]
% sabiendo que una de sus raíces es \( \frac{1}{2} \).

% \begin{solution}
% 1. Llevamos todos los términos a un lado:
%    \[
%    2x^3 - 3x^2 + \frac{1}{2} = 0
%    \]
% 2. Multiplicamos por 2 para eliminar fracciones:
%    \[
%    4x^3 - 6x^2 + 1 = 0
%    \]
% 3. $$\left(2 x - 1\right) \left(2 x^{2} - 2 x - 1\right) \to \left[ \frac{1}{2}, \  \frac{1}{2} - \frac{\sqrt{3}}{2}, \  \frac{1}{2} + \frac{\sqrt{3}}{2}\right]$$
% \end{solution}

% \question Resuelve la siguiente ecuación:
% $9^x - 6 \cdot {3^{x + 1}} + 81 = 0 $
% \begin{solution}
%     $\rightarrow \left [ 2\right ]$
% \end{solution}


% \question Resuelve los siguientes sistemas:
% \begin{parts}
%     \part[1\half] $\left\{\begin{matrix}{2^x} + {2^y} = 24\\ \\{2^{x + y}} = 128\end{matrix}\right. $

% \begin{solution}
%     $\rightarrow \left [ \left \{ x : 3, \quad y : 4\right \}, \quad \left \{ x : 4, \quad y : 3\right \}\right ]$
% \end{solution}

%     \part[1\half] $\left\{\begin{matrix}\log_{2}(x) + \log_{2}(x + y) = 4\\ \\x + y = 2 \end{matrix}\right. $
% \begin{solution}
%     $\rightarrow \left [ \left \{ x : 8, \quad y : -6\right \}\right ]$
% \end{solution}

% \end{parts}


% \question Resuelve:
% $$\left\{\begin{matrix}3\log x - 2\log y = 10\\\log x + 3\log y = 7\end{matrix}\right. $$

% \begin{solution}
%     $\rightarrow \left [ \left \{ x : 10000, \quad y : 10\right \}\right ]$
% \end{solution}

% \question[2] Resuelve el siguiente sistema de ecuaciones:
% \[
% \begin{cases}
% \dfrac{1}{x} + y = 3 \\ \\
% \dfrac{1}{x} - \dfrac{1}{y} = \dfrac{1}{2}
% \end{cases}
% \]

\begin{solution}
1. Llamamos \(u = \frac{1}{x}\) y \(v = y\), para obtener el sistema:
   \[
   u + v = 3
   \]
   \[
   u - \frac{1}{v} = \frac{1}{2}
   \]
2. Resolvemos este sistema para \(u\) y \(v\), y luego encontramos \(x\) y \(y\) a partir de \(u\) y \(v\).
\end{solution}

% \question Resuelve las siguientes ecuaciones:
% \begin{parts}
%     \part $|2x-4|=|x+2|$ \begin{solution} $\left[ \frac{2}{3}, \  6\right]$\end{solution}
% \end{parts}


% https://lareinadelasciencias.wordpress.com/wp-content/uploads/2014/01/problemas-ecuaciones-sistemas-no-lineales.pdf

%https://matematicasiesoja.wordpress.com/wp-content/uploads/2013/10/problemas-_metodo_gauss2.pdf

\question Resuelve mediante expresiones algebraicas y, en caso que se pueda, por Gauss:
\begin{parts}
    \part[1] Un padre ha comprado un jersey para cada uno de sus cinco hijos, gastándose en total
108,75 euros. Tres de los jerseys tenían un 15\% de descuento, y otro de ellos tenía un 20\%
de descuento. Sabiendo que inicialmente costaban lo mismo, ¿cuánto ha tenido que pagar
por cada jersey? 
\begin{solution}$3\cdot0.85x+0.8x+x=108.75 \to x=25 \to 21.25; 20; 25$\end{solution}

    \part[1] El área de un triángulo es 78 $cm^2$  y entre la base y la altura suman 25 $cm$. Calcula la base y la altura.
% \begin{solution}  $ \left\{\begin{matrix}\frac{xy}{2}=78\\ x+y=25\\ \end{matrix}\right.  \rightarrow  \\\left[\begin{matrix}1 & 1 & 25\\0 & \frac{x}{2} - \frac{y}{2} & - \frac{25 y}{2} + 78\end{matrix}\right] \rightarrow  \left [ \left \{ x : 12, \quad y : 13\right \}, \quad \left \{ x : 13, \quad y : 12\right \}\right ] $  \end{solution}

    \part[2] Miguel y Ana tienen un perro. Averigua el peso de cada uno de los tres sabiendo que Miguel y Ana pesan 50
kg juntos, y Ana y su perro 29 kg y, finalmente, Miguel y el perro 35 kg.  \begin{solution}  $ \left\{\begin{matrix}x+y = 50\\ y+z=29\\ x+z=35\\ \end{matrix}\right.  \rightarrow  \\\left[\begin{matrix}1 & 1 & 0 & 50\\0 & 1 & 1 & 29\\0 & 0 & 2 & 14\end{matrix}\right] \rightarrow  \left \{ x : 28, \quad y : 22, \quad z : 7\right \} $  \end{solution}
  
%     \part[2] En una residencia de estudiantes se compran semanalmente 110 helados de distintos
% sabores: vainilla, chocolate y nata. El presupuesto destinado para esta compra es de 540
% euros y el precio de cada helado es de 4 euros el de vainilla, 5 euros el de chocolate y 6
% euros el de nata. Conocidos los gustos de los estudiante, se sabe que entre helados de
% chocolate y de nata se han de comprar el 20\% más que de vainilla 
% \begin{solution}    
% \textbf{Resolución del problema:}

% Sea:
% \[
% x: \text{número de helados de vainilla}, \quad
% y: \text{número de helados de chocolate}, \quad
% z: \text{número de helados de nata}.
% \]

% Planteamos las ecuaciones:
% \[
% x + y + z = 110, \quad 4x + 5y + 6z = 540, \quad y + z = 1.2x.
% \]

% \textbf{Sistema matricial:}
% \[
% \begin{bmatrix}
% 1 & 1 & 1 \\ 
% 4 & 5 & 6 \\ 
% -1.2 & 1 & 1
% \end{bmatrix}
% \begin{bmatrix}
% x \\ 
% y \\ 
% z
% \end{bmatrix}
% =
% \begin{bmatrix}
% 110 \\ 
% 540 \\ 
% 0
% \end{bmatrix}.
% \]

% \textbf{Eliminación de Gauss:}

% Partimos de la matriz ampliada:
% \[
% \left[
% \begin{array}{ccc|c}
% 1 & 1 & 1 & 110 \\ 
% 4 & 5 & 6 & 540 \\ 
% -1.2 & 1 & 1 & 0
% \end{array}
% \right].
% \]

% Tras operaciones elementales:
% \[
% \left[
% \begin{array}{ccc|c}
% 1 & 1 & 1 & 110 \\ 
% 0 & 1 & 2 & 100 \\ 
% 0 & 0 & 1 & 40
% \end{array}
% \right].
% \]

% \textbf{Sustitución hacia atrás:}
% \[
% z = 40, \quad y + 2z = 100 \implies y = 20, \quad x + y + z = 110 \implies x = 50.
% \]

% \textbf{Solución final:}
% \[
% x = 50 \quad \text{(vainilla)}, \quad y = 20 \quad \text{(chocolate)}, \quad z = 40 \quad \text{(nata)}.
% \]
% \end{solution}


% \part[2] Una empresa envasadora ha comprado un total de 1500 cajas de pescado en tres mercados 
% diferentes, a un precio por caja 30, 20 y 40  euros respectivamente. el coste total de la operación ha sido 
% de 40500\euro. Calcular cuánto ha pagado la empresa en cada mercado, sabiendo que en el primero de ellos ha 
% comprado el 30\% de las cajas

% \begin{solution}
%     Llamamos $x$, $y$ y $z$ cajas de 30, 20 y 40\euro \  respectivamente.
%     El sistema a resolver es:$\left\{ \begin{matrix}x + y + z = 1500 \\ 30 x + 20 y + 40 z = 40500 \\ x = 450 \\ \end{matrix}\right.$ \\ 
%     \textbf{Discusión y resolución por Gauss:} Escalonando la matriz ampliada tenemos\\$A^*= \left(\begin{matrix}1 & 1 & 1 & 1500\\30 & 20 & 40 & 40500\\1 & 0 & 0 & 450\end{matrix}\right) \thicksim \left(\begin{matrix}1 & 1 & 1 & 1500\\0 & -10 & 10 & -4500\\0 & 0 & -2 & -600\end{matrix}\right)$. \\  De los valores de la última fila podemos concluir:\begin{itemize}\item S.C.D.\begin{itemize}\item $\left(\begin{matrix}0 & 0 & -2 & -600\end{matrix}\right) \to z = 300$\end{itemize}\begin{itemize}\item $\left(\begin{matrix}0 & -10 & 10 & -4500\end{matrix}\right) \to y = 750$\end{itemize}\begin{itemize}\item $\left(\begin{matrix}1 & 1 & 1 & 1500\end{matrix}\right) \to x = 450$\end{itemize}\end{itemize}  \textbf{Por rangos y determinantes:} \\$\left|A\right|=\left|\begin{matrix}1 & 1 & 1\\30 & 20 & 40\\1 & 0 & 0\end{matrix}\right|=20  \neq 0 $\begin{itemize} \item $rg(A)=3 \land rg(A^*)=3 \to $ S.C.D.  \\ \\ Por Cramer: \begin{itemize}\item $x=\frac{\left|\begin{matrix}1500 & 1 & 1\\40500 & 20 & 40\\450 & 0 & 0\end{matrix}\right|}{20}=\frac{9000}{20}=450$\item $y=\frac{\left|\begin{matrix}1 & 1500 & 1\\30 & 40500 & 40\\1 & 450 & 0\end{matrix}\right|}{20}=\frac{15000}{20}=750$\item $z=\frac{\left|\begin{matrix}1 & 1 & 1500\\30 & 20 & 40500\\1 & 0 & 450\end{matrix}\right|}{20}=\frac{6000}{20}=300$\end{itemize}\end{itemize}
%     \textbf{SOLUCIÓN: } 13500 (450x30), 15000 y 12000 \euro.
% \end{solution}

    
%     \part[1] En una reunión hay 22 personas, entre hombres, mujeres y niños. El doble del
% número de mujeres más el triple del número de niños, es igual al doble del número de 
% hombres. 
% \begin{solution}
%     Sea:
% \[
% x: \text{número de hombres}, \quad
% y: \text{número de mujeres}, \quad
% z: \text{número de niños}.
% \]

% Planteamos las ecuaciones:
% \[
% x + y + z = 22, \quad 2y + 3z = 2x.
% \]

% \textbf{Sistema matricial:}
% \[
% \begin{bmatrix}
% 1 & 1 & 1 \\ 
% -2 & 2 & 3
% \end{bmatrix}
% \begin{bmatrix}
% x \\ 
% y \\ 
% z
% \end{bmatrix}
% =
% \begin{bmatrix}
% 22 \\ 
% 0
% \end{bmatrix}.
% \]

% \textbf{Eliminación de Gauss:}

% Partimos de la matriz ampliada:
% \[
% \left[
% \begin{array}{ccc|c}
% 1 & 1 & 1 & 22 \\ 
% -2 & 2 & 3 & 0
% \end{array}
% \right].
% \]

% Operaciones elementales eliminan el coeficiente de \(x\) en la segunda fila:
% \[
% \left[
% \begin{array}{ccc|c}
% 1 & 1 & 1 & 22 \\ 
% 0 & 4 & 5 & 44
% \end{array}
% \right].
% \]

% Simplificamos la segunda fila:
% \[
% \left[
% \begin{array}{ccc|c}
% 1 & 1 & 1 & 22 \\ 
% 0 & 1 & \frac{5}{4} & 11
% \end{array}
% \right].
% \]

% \textbf{Sustitución hacia atrás:}

% De la segunda fila:
% \[
% y + \frac{5}{4}z = 11 \quad \implies \quad y = 11 - \frac{5}{4}z.
% \]

% De la primera fila:
% \[
% x + y + z = 22 \quad \implies \quad x = 22 - y - z.
% \]

% Sustituyendo \(y\):
% \[
% x = 22 - \left(11 - \frac{5}{4}z\right) - z \quad \implies \quad x = 11 - \frac{1}{4}z.
% \]

% Para que \(x, y, z\) sean números enteros, \(z = 4\):
% \[
% y = 11 - \frac{5}{4}(4) = 6, \quad x = 11 - \frac{1}{4}(4) = 10.
% \]

% \textbf{Solución final:}
% \[
% x = 10 \quad \text{(hombres)}, \quad y = 6 \quad \text{(mujeres)}, \quad z = 4 \quad \text{(niños)}.
% \]    
% \end{solution}


\end{parts}


% \question Resuelve los siguientes sistemas de ecuaciones no lineales

% \begin{parts}

% \part[1] 
% \[
% \begin{cases}
% 2x - y = 1 \\
% 3x - 2y^2 = 1
% \end{cases}
% \]

% \begin{solution}
% $x = \frac{5}{4}, \, y = \frac{3}{2}$ \quad y \quad $x = -\frac{1}{4}, \, y = \frac{1}{2}$
% \end{solution}

% \part[1] 
% \[
% \begin{cases}
% x^2 - 2y^2 = 1 \\
% x \cdot y = 6
% \end{cases}
% \]

% \begin{solution}
% $x = 3, \, y = 2$ \quad y \quad $x = -3, \, y = -2$
% \end{solution}

% \part[1] 
% \[
% \begin{cases}
% \sqrt{x + y} + y = 1 \\
% 2x + y = 2
% \end{cases}
% \]

% \begin{solution}
% $x = 1, \, y = 0$
% \end{solution}

% \part[1] 
% \[
% \begin{cases}
% 2x^2 + 3y^2 = 21 \\
% -5x^2 + 2y^2 = -43
% \end{cases}
% \]

% \begin{solution}
% $x = \pm 2, \, y = \pm 3$
% \end{solution}

% \part[1] 
% \[
% \begin{cases}
% \frac{y}{x} - \frac{3}{y} = \frac{1}{2} \\
% x - 3y = -5
% \end{cases}
% \]

% \begin{solution}
% $x = 1, \, y = 2$
% \end{solution}

% \end{parts}

% \question[2] Resuelve por Gauss el siguiente sistema de ecuaciones:
% \[
% \begin{cases}
% 3x + 2y - 2z = 4 \\
% 4x + y - z = 7 \\
% x + 4y - 4z = 0
% \end{cases}
% \]

% \begin{solution}
% $\left[\begin{matrix}3 & 2 & -2 & 4\\0 & - \frac{5}{3} & \frac{5}{3} & \frac{5}{3}\\0 & 0 & 0 & 2\end{matrix}\right]$ Luego S.I.
% \end{solution}



\question Resuelve las siguientes inecuaciones:
\begin{parts}
            \part[1]  $ \dfrac{{{x^3} - 5{x^2} + 2x + 8}}{1-x^2} \leq 0 $  \begin{solution}  $  \left(1, 2\right] \cup \left[4, \infty\right) $  \end{solution}
        % \part[1] $1-\dfrac{{x + 3}}{{x + 6}} \geq 0 $ \begin{solution} $\rightarrow \left(-6, \infty\right)$\end{solution}
\end{parts}



        

\addpoints

\end{questions}

\end{document}
\grid
