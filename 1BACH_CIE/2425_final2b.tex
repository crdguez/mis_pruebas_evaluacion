\documentclass[addpoints,spanish, 12pt,a4paper]{exam}
%\documentclass[answers, spanish, 12pt,a4paper]{exam}
% \printanswers
\renewcommand*\half{.5}
\pointpoints{punto}{puntos}
\hpword{Puntos:}
\vpword{Puntos:}
\htword{Total}
\vtword{Total}
\hsword{Resultado:}
\hqword{Ejercicio:}
\vqword{Ejercicio:}

\usepackage[utf8]{inputenc}
\usepackage[spanish]{babel}
\usepackage{eurosym}
%\usepackage[spanish,es-lcroman, es-tabla, es-noshorthands]{babel}


\usepackage[margin=1in]{geometry}
\usepackage{amsmath,amssymb}
\usepackage{multicol}
\usepackage{yhmath}

\pointsinrightmargin % Para poner las puntuaciones a la derecha. Se puede cambiar. Si se comenta, sale a la izquierda.
\extrawidth{-2.4cm} %Un poquito más de margen por si ponemos textos largos.
\marginpointname{ \emph{\points}}

\usepackage{graphicx}

\graphicspath{{../img/}} 

\newcommand{\class}{1º Bachillerato CIE}
\newcommand{\examdate}{\today}
\newcommand{\examnum}{Final 2ªEv.}
\newcommand{\tipo}{A}


\newcommand{\timelimit}{45 minutos}

\renewcommand{\solutiontitle}{\noindent\textbf{Solución:}\enspace}


\pagestyle{head}
\firstpageheader{\includegraphics[width=0.2\columnwidth]{header_left}}{\textbf{Departamento de Matemáticas\linebreak \class}\linebreak \examnum}{\includegraphics[width=0.1\columnwidth]{header_right}}
\runningheader{\class}{\examnum}{Página \thepage\ de \numpages}
\runningheadrule


\usepackage{pgf,tikz,pgfplots}
\pgfplotsset{compat=1.15}
\usepackage{mathrsfs}
\usetikzlibrary{arrows}


\begin{document}

\noindent
\begin{tabular*}{\textwidth}{l @{\extracolsep{\fill}} r @{\extracolsep{6pt}} }
\textbf{Nombre:} \makebox[3.5in]{\hrulefill} & \textbf{Fecha:}\makebox[1in]{\hrulefill} \\
 & \\
\textbf{Tiempo: \timelimit} & Tipo: \tipo 
\end{tabular*}
\rule[2ex]{\textwidth}{2pt}
Esta prueba tiene \numquestions\ ejercicios. La puntuación máxima es de \numpoints. 
La nota final de la prueba será la parte proporcional de la puntuación obtenida sobre la puntuación máxima. 

\begin{center}


\addpoints
 %\gradetable[h][questions]
	\pointtable[h][questions]
\end{center}

\noindent
\rule[2ex]{\textwidth}{2pt}

\begin{questions}

% \question Dados los vectores $\vec{u}=(\frac{1}{2},-1)$ y $\vec{v}=(0,-2)$ calcula:
% \begin{parts}
%     \part[1] $|\vec{u}|$
%     \part[1] $2\vec{u}-3\vec{v}$
%     \part[1] $\vec{u}\cdot \vec{v}$
% \end{parts}

\question[2] Expresa
$\vec{a} =(9,5)$
como combinación lineal de
 $\vec{u}=(1,3)$ y $\vec{v} =(3,-2)$
\begin{solution}
    $3\vec{u}+2\vec{v}$
\end{solution}

\question[1] Determinar el ángulo formado por las rectas: $ r\equiv2x-y-2=0\  y \ s\equiv3x+2y-4=0 $  \begin{solution}  $ 119.74488129694222 $  \end{solution}

\question Dada la recta $r\equiv x+y-3=0$ y el punto $P(-1,2)$, se pide:
\begin{parts}
    \part[1] Halla la ecuación general de la recta perpendicular a r que pasa por P
    \begin{solution}
        $x-y+3=0$
    \end{solution}
    \part[1] Calcula el punto de corte de la recta anterior y la recta r
    \begin{solution}
        $(0,3)$
    \end{solution}
    \part[1] Calcula el punto simétrico de P respecto del calculado en el apartado anterior
    \begin{solution}
        $(1,4)$
    \end{solution}
    \part[1]  Halla la ecuación general de la recta paralela a r que pasa por P
    \begin{solution}
        $x+y-1=0$
    \end{solution}
\end{parts}

% \question[2]  Comprueba analíticamente que los puntos A(-3,2), B(1,9) y C(4,-2) forman un triángulo rectángulo en A.
% \begin{solution}
%     $\vec{AB}=(4,7)$ , $\vec{AC}=(7,-4)$ y $\vec{AB}\cdot\vec{AC}=0$
% \end{solution}

\question[2] Calcula el vértice C de un triángulo isósceles (dos lados iguales) ABC, sabiendo que: 
$A(2, -3)$, $\ B(5, 2)$ y $C \in r\equiv y=4 $  \begin{solution}  $ \left [ \left \{ x : -4, \quad y : 4\right \}\right ] $  \end{solution}

% \question[2] Calcula el vértice C de un triángulo isósceles (dos lados iguales) ABC, sabiendo que: 
% $A(2, -3)$, $\ B(5, 2)$ y $C \in r\equiv -x+3y-16=0 $  \begin{solution}  $ \left [ \left \{ x : -4, \quad y : 4\right \}\right ] $  \end{solution}

% \question[1] Calcula k para que el triángulo ABC sea isósceles, siendo $A(2, -3)\ , \ B=(5, 2) \ y \ C(k,4)$  \begin{solution}  $ \left [ \left \{ x : -4, \quad y : 4\right \}\right ] $  \end{solution}

% \question[2] Calcula, sin usar la trigonometría, el área del triángulo de vértices $A\left(-1, 3\right)$, $B\left(6, 5\right)$, $C\left(2, 1\right)$.
% \begin{solution}
% $\mathtt{\text{altura}} : \frac{5 \sqrt{2}}{2}, \  \mathtt{\text{area}} : 10, \  \mathtt{\text{base}} : 4 \sqrt{2}$
% \end{solution}


\question[2] Calcula $(2+5i):(3+4i)$. Da el resultado en binómico
\begin{solution}
    $\frac{26}{25} + \frac{7 i}{25}$
\end{solution}

\end{questions}

\end{document}
\grid
