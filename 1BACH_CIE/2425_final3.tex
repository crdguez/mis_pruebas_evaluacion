\documentclass[addpoints,spanish, 12pt,a4paper]{exam}
%\documentclass[answers, spanish, 12pt,a4paper]{exam}
% \printanswers
\renewcommand*\half{.5}
\pointpoints{punto}{puntos}
\hpword{Puntos:}
\vpword{Puntos:}
\htword{Total}
\vtword{Total}
\hsword{Resultado:}
\hqword{Ejercicio:}
\vqword{Ejercicio:}

\usepackage[utf8]{inputenc}
\usepackage[spanish]{babel}
\usepackage{eurosym}
%\usepackage[spanish,es-lcroman, es-tabla, es-noshorthands]{babel}

\usepackage[margin=1in]{geometry}
\usepackage{amsmath,amssymb}
\usepackage{multicol}
\usepackage{yhmath}


\pointsinrightmargin % Para poner las puntuaciones a la derecha. Se puede cambiar. Si se comenta, sale a la izquierda.
\extrawidth{-2.4cm} %Un poquito más de margen por si ponemos textos largos.
\marginpointname{ \emph{\points}}

\usepackage{graphicx}

\graphicspath{{../img/}}

\newcommand{\class}{1º Bachillerato CIE}
\newcommand{\examdate}{\today}
\newcommand{\examnum}{Final 3ªEv.}
\newcommand{\tipo}{A}

\newcommand{\timelimit}{45 minutos}

\renewcommand{\solutiontitle}{\noindent\textbf{Solución:}\enspace}

\pagestyle{head}
\firstpageheader{\includegraphics[width=0.2\columnwidth]{header_left}}{\textbf{Departamento de Matemáticas\linebreak \class}\linebreak \examnum}{\includegraphics[width=0.1\columnwidth]{header_right}}
\runningheader{\class}{\examnum}{Página \thepage\ de \numpages}
\runningheadrule

\usepackage{pgf,tikz,pgfplots}
\pgfplotsset{compat=1.15}
\usepackage{mathrsfs}
\usetikzlibrary{arrows}

\begin{document}

\noindent
\begin{tabular*}{\textwidth}{l @{\extracolsep{\fill}} r @{\extracolsep{6pt}} }
\textbf{Nombre:} \makebox[3.5in]{\hrulefill} & \textbf{Fecha:}\makebox[1in]{\hrulefill} \\
 & \\
\textbf{Tiempo: \timelimit} & Tipo: \tipo 
\end{tabular*}
\rule[2ex]{\textwidth}{2pt}
Esta prueba tiene \numquestions\ ejercicios. La puntuación máxima es de \numpoints.
La nota final de la prueba será la parte proporcional de la puntuación obtenida sobre la puntuación máxima.

\begin{center}

\addpoints
 %\gradetable[h][questions]
\pointtable[h][questions]
\end{center}

\noindent
\rule[2ex]{\textwidth}{2pt}

\begin{questions}

\question[1\half] Calcula los límites de la siguiente función:

\begin{tikzpicture}[scale=0.4]
    % Configuración de ejes
    \draw[thick] (-8, 0) -- (8, 0) node[right] {$X$}; % Eje X
    \draw[thick] (0, -2) -- (0, 8) node[above] {$Y$}; % Eje Y

    % Cuadrícula
    \draw[help lines, color=gray!20] (-8, -2) grid (8, 8);

    % Etiquetas de cuadrícula
    \foreach \x in {-8, -6, -4, -2, 2, 4, 6, 8}
        \draw (\x, 0.1) -- (\x, -0.1) node[below] {\x};
    \foreach \y in {-2, 2, 4, 6, 8}
        \draw (0.1, \y) -- (-0.1, \y) node[left] {\y};

    % Primer trozo: Parábola que termina en (-4, 4)
    \draw[thick, red, domain=-8:-4, samples=50] plot (\x, {4+(0.25*(\x+4)*(\x+4))}) node[above] {};
    \filldraw[red] (-4, 4) circle (2pt); % Punto final en (-4, 4)

    % Segundo trozo: Recta de (-4, 1) a (-2, 3)
    \draw[thick, red] (-4, 1) -- (-2, 3);
    \filldraw[fill=white, draw=red] (-4, 1) circle (2pt); % Punto inicial
    % \filldraw[fill=white, draw=red] (-2, 3) circle (2pt); % Punto final

    % Tercer trozo: Recta de (-2, 3) a (1, 0)
    \draw[thick, red] (-2, 3) -- (1, 0);
    % \draw[red] (1, 0) circle (2pt); % Punto final en (1, 0)

    \filldraw[fill=white, draw=red] (-2, 3) circle (2pt); % Punto final
    \filldraw[red] (-2, 1) circle (2pt); % Punto final

    % Cuarto trozo: Proporcionalidad inversa creciente con mayor excentricidad
    \draw[thick, red, domain=1:2.883, samples=50] plot (\x, {-1*(\x-1)/(2*(\x-3))}) node[above] {};
    \draw[dashed] (3, 0) -- (3, 8); % Asíntota vertical en x = 3
    % \filldraw[red] (1, 0) circle (2pt); % Punto inicial en (1, 0)

    % Quinto trozo: Exponencial que pasa por (3, 5) y tiene asíntota horizontal y = 3
    \draw[thick, red, domain=3:8, samples=50] plot (\x, {3 + 2*exp(-0.5*(\x-3))});
    \draw[dashed] (0, 3) -- (8, 3); % Asíntota horizontal en y = 3
    \filldraw[fill=white, draw=red] (3, 5) circle (2pt); % Punto en (3, 5)

\end{tikzpicture}
\begin{multicols}{3}
\begin{parts}
    \part $\lim_{x \to -\infty} f(x)$
    \part $\lim_{x \to -4^-} f(x)$
    \part $\lim_{x \to -4^+} f(x)$
    \part $\lim_{x \to -4} f(x)$
    \part $\lim_{x \to -2} f(x)$
    \part $\lim_{x \to 1} f(x)$
    \part $\lim_{x \to -3^-} f(x)$
    \part $\lim_{x \to -3^+} f(x)$
    \part $\lim_{x \to +\infty} f(x)$
\end{parts}
\end{multicols}

\question Calcula los siguientes límites:
\begin{parts}
    \part[1] $$\lim_{x \to 2}\dfrac{2 x^{3} - 4 x^{2} - x + 2}{x^2-5x+6}$$

    \begin{solution}
    Primero factorizamos el denominador:
    \[
    x^2 - 5x + 6 = (x - 2)(x - 3)
    \]

    Ahora intentamos factorizar el numerador:
    \[
    2x^3 - 4x^2 - x + 2
    \]
    Agrupamos:
    \[
    (2x^3 - 4x^2) + (-x + 2) = 2x^2(x - 2) -1(x - 2)
    \]
    Sacamos factor común:
    \[
    = (x - 2)(2x^2 - 1)
    \]

    Sustituimos ambas factorizaciones en la fracción:
    \[
    \frac{(x - 2)(2x^2 - 1)}{(x - 2)(x - 3)}
    \]

    Simplificamos el factor común \( (x - 2) \):
    \[
    \frac{2x^2 - 1}{x - 3}
    \]

    Ahora evaluamos el límite:
    \[
    \lim_{x \to 2} \frac{2x^2 - 1}{x - 3} = \frac{2(2)^2 - 1}{2 - 3} = \frac{8 - 1}{-1} = \frac{7}{-1} = -7
    \]

    \textbf{Resultado final:} \fbox{\( \lim_{x \to 2} \frac{2x^3 - 4x^2 - x + 2}{x^2 - 5x + 6} = -7 \)}
    \end{solution}

    \part[1] {$$\lim_{x \to \infty} \dfrac{2x^3 - 4x^2 - x + 2}{5x^3 - 25x^2 + 30x}$$}

    \begin{solution}
    Dividimos numerador y denominador entre $x^3$:
    
    \[
    \lim_{x \to \infty} \frac{2x^3 - 4x^2 - x + 2}{5x^3 - 25x^2 + 30x}
    = \lim_{x \to \infty} \frac{2 - \frac{4}{x} - \frac{1}{x^2} + \frac{2}{x^3}}{5 - \frac{25}{x} + \frac{30}{x^2}}
    \]
    
    Al tomar el límite cuando $x \to \infty$, los términos con $1/x$, $1/x^2$ y $1/x^3$ tienden a cero:
    
    \[
    = \frac{2}{5}
    \]
    \end{solution}
\end{parts}

\question[2] Calcula las siguientes derivadas:
\begin{multicols}{2}
\begin{parts}
  \part $f(x)=$
  \part $f(x)=$
\end{parts}
\end{multicols}

  \question  Dada la función: $$\dfrac{x^2}{x-1}$$ 
\begin{parts}
    \part[1] Estudia el crecimiento y decrecimiento de la función.
    
    \begin{solution}
    La derivada de la función es:
    \[
    f'(x) = \frac{x(x - 2)}{(x - 1)^2}
    \]
    El signo de \( f'(x) \) depende del numerador \( x(x - 2) \), ya que el denominador \( (x - 1)^2 \) siempre es positivo.

    - \( f'(x) > 0 \) cuando \( x < 0 \) o \( x > 2 \), lo que indica que la función es **creciente** en los intervalos \( (-\infty, 0) \cup (2, \infty) \).
    - \( f'(x) < 0 \) cuando \( 0 < x < 2 \), lo que indica que la función es **decreciente** en el intervalo \( (0, 2) \).
    \end{solution}
    
    \part[2] Encuentra los extremos relativos.

    \begin{solution}
    Los puntos críticos se encuentran donde \( f'(x) = 0 \), es decir, cuando \( x(x - 2) = 0 \). Esto da los puntos \( x = 0 \) y \( x = 2 \).
    
    - En \( x = 0 \), \( f'(x) \) cambia de positivo a negativo, por lo que hay un **máximo relativo** en \( (0, 0) \).
    - En \( x = 2 \), \( f'(x) \) cambia de negativo a positivo, por lo que hay un **mínimo relativo** en \( (2, 4) \).
    
    Los valores de la función en estos puntos son:
    - \( f(0) = 0 \)
    - \( f(2) = 4 \)
    \end{solution}
\end{parts}

\question La obsolescencia tecnológica implica una disminución del valor de un producto con el tiempo. En cierto dispositivo, el valor $V(t)>0$, viene dado por la expresión $V(t)=200-\frac{100t}{10+2t}$ \euro, siendo t los años transcurridos desde la compra del dispositivo.
\begin{parts}
    \part[1] Calcule el valor inicial del producto y su valor en un horizonte infinito de tiempo
    \part[1] Calcule $V'(t)$ y justifique que $V(t )$ es decreciente. Utilice esta conclusión y los resultados del apartado anterior para argumentar que no será posible que el valor de $V(t)$ sea igual a 125\euro.
    \part[0\half] ¿Cuánto tiempo tiene que pasar para que el dispositivo tenga un valor de 175 \euro?
\end{parts}

\end{questions}

\end{document}

