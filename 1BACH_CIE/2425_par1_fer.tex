\documentclass[addpoints,spanish, 12pt,a4paper]{exam}
%\documentclass[answers, spanish, 12pt,a4paper]{exam}
\printanswers
\renewcommand*\half{.5}
\pointpoints{punto}{puntos}
\hpword{Puntos:}
\vpword{Puntos:}
\htword{Total}
\vtword{Total}
\hsword{Resultado:}
\hqword{Ejercicio:}
\vqword{Ejercicio:}

\usepackage[utf8]{inputenc}
\usepackage[spanish]{babel}
\usepackage{eurosym}
%\usepackage[spanish,es-lcroman, es-tabla, es-noshorthands]{babel}


\usepackage[margin=1in]{geometry}
\usepackage{amsmath,amssymb}
\usepackage{multicol}
\usepackage{yhmath}

\pointsinrightmargin % Para poner las puntuaciones a la derecha. Se puede cambiar. Si se comenta, sale a la izquierda.
\extrawidth{-2.4cm} %Un poquito más de margen por si ponemos textos largos.
\marginpointname{ \emph{\points}}

\usepackage{graphicx}

\graphicspath{{../img/}} 

\newcommand{\class}{1º Bachillerato CIE}
\newcommand{\examdate}{\today}
\newcommand{\examnum}{Parcial 1ªEv.}
\newcommand{\tipo}{A}


\newcommand{\timelimit}{45 minutos}

\renewcommand{\solutiontitle}{\noindent\textbf{Solución:}\enspace}


\pagestyle{head}
\firstpageheader{\includegraphics[width=0.2\columnwidth]{header_left}}{\textbf{Departamento de Matemáticas\linebreak \class}\linebreak \examnum}{\includegraphics[width=0.1\columnwidth]{header_right}}
\runningheader{\class}{\examnum}{Página \thepage\ de \numpages}
\runningheadrule


\usepackage{pgf,tikz,pgfplots}
\pgfplotsset{compat=1.15}
\usepackage{mathrsfs}
\usetikzlibrary{arrows}


\begin{document}

\noindent
\begin{tabular*}{\textwidth}{l @{\extracolsep{\fill}} r @{\extracolsep{6pt}} }
\textbf{Nombre:} \makebox[3.5in]{\hrulefill} & \textbf{Fecha:}\makebox[1in]{\hrulefill} \\
 & \\
\textbf{Tiempo: \timelimit} & Tipo: \tipo 
\end{tabular*}
\rule[2ex]{\textwidth}{2pt}
Esta prueba tiene \numquestions\ ejercicios. La puntuación máxima es de \numpoints. 
La nota final de la prueba será la parte proporcional de la puntuación obtenida sobre la puntuación máxima. 

\begin{center}


\addpoints
 %\gradetable[h][questions]
	\pointtable[h][questions]
\end{center}

\noindent
\rule[2ex]{\textwidth}{2pt}

\begin{questions}

%\question 
%
%\begin{parts}
%\part[2] 
%\begin{solution}
%\end{solution}
%
%
%\end{parts}
%\addpoints

% \question[1] Indica a cuáles de los conjuntos
% $\mathbb{N}$, $\mathbb{Z}$, $\mathbb{Q}$, $\mathbb{R}$ pertenecen cada uno de los siguientes números:
% \begin{center}
% \begin{tabular}{|c |c |c |c |c|}\hline
% &$\mathbb{N}$& $\mathbb{Z}$& $\mathbb{Q}$&$\mathbb{R}$\\ 
% \hline
% $\frac{8}{16}$&&&&\\
% \hline
% $\sqrt[3]{-27}$&&&&\\
% \hline
% $3.0\wideparen{1}$&&&&\\
% \hline
% $-\frac{12}{4}$&&&&\\
% \hline
% $-\sqrt{25}$&&&&\\
% \hline
% $\sqrt{8}$&&&&\\
% \hline
% $4$&&&&\\
% \hline
% $\pi$&&&&\\
% \hline
% $\sqrt{-4}$&&&&\\
% \hline
% $\frac{39}{13}$&&&&\\
% \hline
% \end{tabular}

% \end{center}

%         \question Sean los siguientes conjuntos $ A=\left\{ x \in \mathbb{R}| -2 \leq x  \leq 5 \right\}$, $  B=\left(-\infty, -1\right) \cup \left(1, \infty\right)$  y  $C=\left\{ x \in \mathbb{R}| \left|{x - 2}\right|\leq3 \right\}$:  
%         \begin{parts} 
%         \part[0\half] Represéntalos en la recta real
%         \part[1\half] Calcula $A \cup  B$ , $A \cap B$ y $A \cap B \cap C$, 
% expresando los resultados en forma de Intervalos. 
%         \end{parts}

%           \begin{solution}  $ A \cup  B = \left(-\infty, \infty\right)  \\  A \cap B= \left[-2, -1\right) \cup \left(1, 5\right]   \\  A \cap B  \cap C= \left(1, 5\right] $  \end{solution}
        
% \question[1\half] Halla el término general de las siguientes sucesiones. Indica cuáles de ellas son progresiones aritméticas y cuáles progresiones geométricas:
% \begin{parts}
%     \part \( 4, 9, 14, 19, 24, 29, \ldots \)
%     \part \( 2, -6, 18, -54, 162, \ldots \)
% \end{parts}
% \begin{solution}
% \text{PA: 1. } a_n = 5n - 1; \, \text{SG: -3. } a_n = (-3)^{n-1}\cdot2
% \end{solution}


\question Resuelve las siguientes ecuaciones:

\begin{parts}
%     \part[2]$2 \sqrt{x + 4} - \sqrt{x - 1} = 4$
%     \begin{solution}
% 1. Aislamos uno de los radicales:
%    \[
%    2 \sqrt{x + 4} = \sqrt{x - 1} + 4
%    \]
% 2. Elevamos ambos lados al cuadrado:
%    \[
%    4(x + 4) = (\sqrt{x - 1} + 4)^2
%    \]
%    \[
%    4x + 16 = x - 1 + 8 \sqrt{x - 1} + 16
%    \]
% 3. Simplificamos y resolvemos la ecuación resultante.
% $3x+1=8 \sqrt{x - 1} \to x=13/9, x=5
%  $

% \end{solution}
    \part[1\half] $\sqrt{2x+3}-\sqrt{x+2}=2$
    \begin{solution}
    $2x+3=4+4\sqrt{x+2}+x+2 \to x-3=4\sqrt{x+2} \to x^2-6x+9=16x+32\to x^2-22x-23=0$\\
    $x=23$ sí, $x=-1$ No
\end{solution}
    \part[1\half] $\sqrt[3]{x + 6} = x$
    \begin{solution}
1. Elevamos ambos lados al cubo:
   \[
   x + 6 = x^3
   \]
2. Reorganizamos la ecuación:
   \[
   x^3 - x - 6 = 0
   \]
3. Intentamos con divisores de 6 para encontrar posibles raíces.
$$x^3 - x - 6=\left(x - 2\right) \left(x^{2} + 2 x + 3\right)\to x=2$$
\end{solution}
    \part[1\half] $9^x - 6 \cdot {3^{x + 1}} + 81 = 0 $
\begin{solution}
    $\rightarrow \left [ 2\right ]$
\end{solution}



\end{parts}


% \question Resolver la ecuación
% \[
% 4x^3 - 6x^2 - x = -\frac{3}{2},
% \]
% sabiendo que una de sus raíces es \( \frac{1}{2} \).

% \begin{solution}
% 1. Llevamos todos los términos a un lado:
%    \[
%    4x^3 - 6x^2 - x + \frac{3}{2} = 0
%    \]
% 2. Multiplicamos por 2 para eliminar fracciones:
%    \[
%    8x^3 - 12x^2 - 2x + 3 = 0
%    \]
% 3. Factorizando el polinomio 
% $\frac{\left(2 x - 3\right) \left(2 x - 1\right) \left(2 x + 1\right)}{2} \to \left[ - \frac{1}{2}, \  \frac{1}{2}, \  \frac{3}{2}\right]
% $
% \end{solution}

% \question[1] Resolver la ecuación
% \[
% 2x^3 - 3x^2 = -\frac{1}{2},
% \]
% sabiendo que una de sus raíces es \( \frac{1}{2} \).

% \begin{solution}
% 1. Llevamos todos los términos a un lado:
%    \[
%    2x^3 - 3x^2 + \frac{1}{2} = 0
%    \]
% 2. Multiplicamos por 2 para eliminar fracciones:
%    \[
%    4x^3 - 6x^2 + 1 = 0
%    \]
% 3. $$\left(2 x - 1\right) \left(2 x^{2} - 2 x - 1\right) \to \left[ \frac{1}{2}, \  \frac{1}{2} - \frac{\sqrt{3}}{2}, \  \frac{1}{2} + \frac{\sqrt{3}}{2}\right]$$
% \end{solution}

% \question Resuelve la siguiente ecuación:
% $9^x - 6 \cdot {3^{x + 1}} + 81 = 0 $
% \begin{solution}
%     $\rightarrow \left [ 2\right ]$
% \end{solution}


\question Resuelve los siguientes sistemas:
\begin{parts}
    \part[1\half] $\left\{\begin{matrix}{2^x} + {2^y} = 24\\ \\{2^{x + y}} = 128\end{matrix}\right. $

\begin{solution}
    $\rightarrow \left [ \left \{ x : 3, \quad y : 4\right \}, \quad \left \{ x : 4, \quad y : 3\right \}\right ]$
\end{solution}

    \part[1\half] $\left\{\begin{matrix}\log_{2}(x) + \log_{2}(x + y) = 4\\ \\x + y = 2 \end{matrix}\right. $
\begin{solution}
    $\rightarrow \left [ \left \{ x : 8, \quad y : -6\right \}\right ]$
\end{solution}

\end{parts}


% \question Resuelve:
% $$\left\{\begin{matrix}3\log x - 2\log y = 10\\\log x + 3\log y = 7\end{matrix}\right. $$

% \begin{solution}
%     $\rightarrow \left [ \left \{ x : 10000, \quad y : 10\right \}\right ]$
% \end{solution}

% \question[2] Resuelve el siguiente sistema de ecuaciones:
% \[
% \begin{cases}
% \dfrac{1}{x} + y = 3 \\ \\
% \dfrac{1}{x} - \dfrac{1}{y} = \dfrac{1}{2}
% \end{cases}
% \]

\begin{solution}
1. Llamamos \(u = \frac{1}{x}\) y \(v = y\), para obtener el sistema:
   \[
   u + v = 3
   \]
   \[
   u - \frac{1}{v} = \frac{1}{2}
   \]
2. Resolvemos este sistema para \(u\) y \(v\), y luego encontramos \(x\) y \(y\) a partir de \(u\) y \(v\).
\end{solution}

\question[2] Resuelve por Gauss el siguiente sistema de ecuaciones:
\[
\begin{cases}
3x + 2y - 2z = 4 \\
4x + y - z = 7 \\
x + 4y - 4z = 0
\end{cases}
\]

\begin{solution}
$\left[\begin{matrix}3 & 2 & -2 & 4\\0 & - \frac{5}{3} & \frac{5}{3} & \frac{5}{3}\\0 & 0 & 0 & 2\end{matrix}\right]$ Luego S.I.
\end{solution}

\question[1\half] Resuelve la siguiente inecuación: $$1-\frac{{x + 3}}{{x + 6}} \geq 0 $$ \begin{solution} \rightarrow \left(-6, \infty\right)\end{solution}



        
        % \question Usando la definición y las propiedades de los números combinatorios, resolver las ecuaciones:
        % \begin{multicols}{1} 
        % \begin{parts} \part[1]  $ {\binom{x}{2}} = 28 $  \begin{solution}  $ \left\{8\right\} $  \end{solution}
        % \end{parts}
        % \end{multicols}
        % \question Calcula, sin hacer todo el desarrollo, el coeficiente del término asociado a:
        % \begin{multicols}{1} 
        % \begin{parts} \part[2]  $ P(x)=\left(2 x - \frac{3}{x}\right)^{8} \  \ y \ parte \ literal \ \frac{1}{x^{2}} $  \begin{solution}  $ -108864 $  \end{solution}
        % \end{parts}
        % \end{multicols}
        % \question Efectúa:
        % \begin{multicols}{1} 
        % \begin{parts} \part[1]  $ 3\sqrt [3] {16} - 2\sqrt [3] {250} + 5\sqrt [3] {250} - 4\sqrt [3] { 2}  $  \begin{solution}  $ 17 \sqrt[3]{2}\to17 \sqrt[3]{2} $  \end{solution} \part[1]  $ \frac{{2 - \sqrt {3} }}{{1 - \sqrt {3} }}-\frac{1}{2\sqrt {3}}+ \frac{3}{2-\sqrt{3} } $  \begin{solution}  $ \frac{14 \sqrt{3} + 39}{6}\to\frac{7 \sqrt{3}}{3} + \frac{13}{2} $  \end{solution}
        % \end{parts}
        % \end{multicols}
        % \question Calcula el valor de $k$ para que:
        % \begin{multicols}{1} 
        % \begin{parts} \part[1]  $ El \ resto \ de \ dividir \ P(x)=x^{25}-kx+3k-4 \ entre\  x+1 \ sea \ 11  $  \begin{solution}  $ 4 $  \end{solution}
        % \end{parts}
        % \end{multicols}
        % \question Halla el m.c.d. y el m.c.m. de los polinomios: 
        % \begin{multicols}{1} 
        % \begin{parts} \part[3]  $  A(x)= x^{5} - 6 x^{3} + 2 x^{2} + 9 x - 6\  y \\  B(x)= x^{5} + 3 x^{4} - 3 x^{3} - 13 x^{2} + 12 \\  $  \begin{solution}  $ Descomposici \acute on:(\left(x - 1\right)^{2} \left(x + 2\right) \left(x^{2} - 3\right) \ y  \ \left(x - 1\right) \left(x + 2\right)^{2} \left(x^{2} - 3\right))\\x^{4} + x^{3} - 5 x^{2} - 3 x + 6= \left(x - 1\right) \left(x + 2\right) \left(x^{2} - 3\right) \ MCD\ y \\ x^{6} + 2 x^{5} - 6 x^{4} - 10 x^{3} + 13 x^{2} + 12 x - 12= \left(x - 1\right)^{2} \left(x + 2\right)^{2} \left(x^{2} - 3\right) \ MCM $  \end{solution}
        % \end{parts}
        % \end{multicols}






\addpoints

\end{questions}

\end{document}
\grid
