\documentclass[addpoints,spanish, 12pt,a4paper]{exam}
%\documentclass[answers, spanish, 12pt,a4paper]{exam}
% \printanswers
\renewcommand*\half{.5}
\pointpoints{punto}{puntos}
\hpword{Puntos:}
\vpword{Puntos:}
\htword{Total}
\vtword{Total}
\hsword{Resultado:}
\hqword{Ejercicio:}
\vqword{Ejercicio:}

\usepackage[utf8]{inputenc}
\usepackage[spanish]{babel}
\usepackage{eurosym}
%\usepackage[spanish,es-lcroman, es-tabla, es-noshorthands]{babel}


\usepackage[margin=1in]{geometry}
\usepackage{amsmath,amssymb}
\usepackage{multicol}
\usepackage{yhmath}

\pointsinrightmargin % Para poner las puntuaciones a la derecha. Se puede cambiar. Si se comenta, sale a la izquierda.
\extrawidth{-2.4cm} %Un poquito más de margen por si ponemos textos largos.
\marginpointname{ \emph{\points}}

\usepackage{graphicx}

\graphicspath{{../img/}} 

\newcommand{\class}{1º Bachillerato CIE}
\newcommand{\examdate}{\today}
\newcommand{\examnum}{Parcial 1ªEv.}
\newcommand{\tipo}{A}


\newcommand{\timelimit}{45 minutos}

\renewcommand{\solutiontitle}{\noindent\textbf{Solución:}\enspace}


\pagestyle{head}
\firstpageheader{\includegraphics[width=0.2\columnwidth]{header_left}}{\textbf{Departamento de Matemáticas\linebreak \class}\linebreak \examnum}{\includegraphics[width=0.1\columnwidth]{header_right}}
\runningheader{\class}{\examnum}{Página \thepage\ de \numpages}
\runningheadrule


\usepackage{pgf,tikz,pgfplots}
\pgfplotsset{compat=1.15}
\usepackage{mathrsfs}
\usetikzlibrary{arrows}


\begin{document}

\noindent
\begin{tabular*}{\textwidth}{l @{\extracolsep{\fill}} r @{\extracolsep{6pt}} }
\textbf{Nombre:} \makebox[3.5in]{\hrulefill} & \textbf{Fecha:}\makebox[1in]{\hrulefill} \\
 & \\
\textbf{Tiempo: \timelimit} & Tipo: \tipo 
\end{tabular*}
\rule[2ex]{\textwidth}{2pt}
Esta prueba tiene \numquestions\ ejercicios. La puntuación máxima es de \numpoints. 
La nota final de la prueba será la parte proporcional de la puntuación obtenida sobre la puntuación máxima. 

\begin{center}


\addpoints
 %\gradetable[h][questions]
	\pointtable[h][questions]
\end{center}

\noindent
\rule[2ex]{\textwidth}{2pt}
\begin{questions}

\question Resuelve las siguientes ecuaciones trigonométricas:
\begin{parts}
    \part[1]$\cos^2{x}-\sen^2{x}=0$
    % \part[1]$\sqrt{2}\sen{x}=1$
\end{parts}

\question Dado $\alpha \in III$ y $\cos{\alpha}=-\frac{1}{2}$, hallar, utilizando la fórmula correspondiente (resultados simplificados y racionalizados; no vale utilizar decimales):
\begin{parts}
    % \part[1] $\sen{2\alpha}$
     \part[1] $\cos{\frac{\alpha}{2}}$
\end{parts}

\question Dado $z=\sqrt{2}-\sqrt{2}i$
\begin{parts}
    \part[1] calcula el opuesto y el conjugado y represéntalos en el plano complejo junto con $z$
    \part[1] calcular 
 $z^5\cdot\overline{z}$
 \begin{solution}
     $-64$
 \end{solution}
    
\end{parts}

 \question Resolver las siguientes ecuaciones: (soluciones complejas incluidas)
 \begin{parts}
     \part[2] $z^5=1$ \begin{solution}\end{solution}
     % \part[2] $x^4-16=0$ \begin{solution}$\pm 2, \pm 2i$\end{solution}
 \end{parts}


%     \question[2] Desde un determinado punto se ve la parte superior de una antena de comunicaciones bajo un ángulo de $50^\circ$ con respecto a la horizontal. Nos alejamos 10 m y la visual forma ahora $40^\circ$. Queremos saber la distancia de ambos puntos de observación a la cúspide de la antena, con el fin de fijar sendos cables de sujeción.
% \begin{solution}
%     \begin{tikzpicture}[scale=1.5, every node/.style={font=\small}]

%     % Ejes y antena
%     \draw[thick] (0,0) -- (0,5) node[above]{Cúspide de la antena};
%     \draw[dashed] (0,0) -- (6,0) node[below right]{Base de la antena};
    
%     % Puntos de observación
%     \draw[fill] (2,0) circle (0.05) node[below]{Primer punto};
%     \draw[fill] (4,0) circle (0.05) node[below]{Segundo punto};
    
%     % Líneas visuales
%     \draw[thick] (2,0) -- (0,5) node[midway,left]{Visual $50^\circ$};
%     \draw[thick] (4,0) -- (0,5) node[midway,left]{Visual $40^\circ$};
    
%     % Ángulos
%     \draw[->] (2,0) ++(-0.4,0) arc[start angle=180,end angle=130,radius=0.4] node[midway,left]{$50^\circ$};
%     \draw[->] (4,0) ++(-0.4,0) arc[start angle=180,end angle=140,radius=0.4] node[midway,left]{$40^\circ$};
    
%     % Distancias
%     \draw[<->,dashed] (2,-0.5) -- (4,-0.5) node[midway,below]{$10 \, \text{m}$};
    
%     \end{tikzpicture}

%     En el triángulo pequeño de la derecha se cumple que los ángulos son $180-50=130^\circ$ y $180-130-40=10^\circ$. Aplicando el teorema del seno:
%     $$\dfrac{10}{\sin{10}}=\dfrac{cable\ corto}{\sin{40}} \to cable\ corto=\dfrac{10\cdot\sin{40}}{\sin{10}}\approx 37.0166631359329
%     $$
%     Por el teorema del coseno:
%     $$cable\ largo^2=10^2 + cable\ corto^2 - 2\cdot 10\cdot cable \ corto \cdot \cos{130^\circ}$$
    
%     $$cable \ largo =\sqrt{10^2 + cable\ corto^2 - 2\cdot 10\cdot cable \ corto \cdot \cos{130^\circ}}\approx 44.1147412780977$$

% \end{solution}

\question[2]
Un grupo decide escalar una montaña de la que desconocen la altura. A la salida del pueblo han medido
el ángulo de elevación, que resulta ser 30º. A continuación han avanzado 100 m hacia la base de la
montaña y han vuelto a medir el ángulo de elevación, siendo ahora 45º. Calcular la altura de la montaña
\begin{solution}
    $136,60 m$
\end{solution}

\addpoints
\end{questions}
\end{document}
\grid
