\documentclass[addpoints,spanish, 12pt,a4paper]{exam}
%\documentclass[answers, spanish, 12pt,a4paper]{exam}
\printanswers
\renewcommand*\half{.5}
\pointpoints{punto}{puntos}
\hpword{Puntos:}
\vpword{Puntos:}
\htword{Total}
\vtword{Total}
\hsword{Resultado:}
\hqword{Ejercicio:}
\vqword{Ejercicio:}

\usepackage[utf8]{inputenc}
\usepackage[spanish]{babel}
\usepackage{eurosym}
%\usepackage[spanish,es-lcroman, es-tabla, es-noshorthands]{babel}


\usepackage[margin=1in]{geometry}
\usepackage{amsmath,amssymb}
\usepackage{multicol}
\usepackage{yhmath}

\pointsinrightmargin % Para poner las puntuaciones a la derecha. Se puede cambiar. Si se comenta, sale a la izquierda.
\extrawidth{-2.4cm} %Un poquito más de margen por si ponemos textos largos.
\marginpointname{ \emph{\points}}

\usepackage{graphicx}

\graphicspath{{../img/}} 

\newcommand{\class}{1º Bachillerato CIE}
\newcommand{\examdate}{\today}
\newcommand{\examnum}{Parcial 1ªEv.}
\newcommand{\tipo}{A}


\newcommand{\timelimit}{45 minutos}

\renewcommand{\solutiontitle}{\noindent\textbf{Solución:}\enspace}


\pagestyle{head}
\firstpageheader{\includegraphics[width=0.2\columnwidth]{header_left}}{\textbf{Departamento de Matemáticas\linebreak \class}\linebreak \examnum}{\includegraphics[width=0.1\columnwidth]{header_right}}
\runningheader{\class}{\examnum}{Página \thepage\ de \numpages}
\runningheadrule


\usepackage{pgf,tikz,pgfplots}
\pgfplotsset{compat=1.15}
\usepackage{mathrsfs}
\usetikzlibrary{arrows}


\begin{document}

\noindent
\begin{tabular*}{\textwidth}{l @{\extracolsep{\fill}} r @{\extracolsep{6pt}} }
\textbf{Nombre:} \makebox[3.5in]{\hrulefill} & \textbf{Fecha:}\makebox[1in]{\hrulefill} \\
 & \\
\textbf{Tiempo: \timelimit} & Tipo: \tipo 
\end{tabular*}
\rule[2ex]{\textwidth}{2pt}
Esta prueba tiene \numquestions\ ejercicios. La puntuación máxima es de \numpoints. 
La nota final de la prueba será la parte proporcional de la puntuación obtenida sobre la puntuación máxima. 

\begin{center}


\addpoints
 %\gradetable[h][questions]
	\pointtable[h][questions]
\end{center}

\noindent
\rule[2ex]{\textwidth}{2pt}
\begin{questions}

% \question Resuelve las siguientes ecuaciones trigonométricas:
% \begin{parts}
%     \part[1]$\cos^2{x}-\sen^2{x}=0$
%     % \part[1]$\sqrt{2}\sen{x}=1$
% \end{parts}

% \question Dado $\alpha \in III$ y $\cos{\alpha}=-\frac{1}{2}$, hallar, utilizando la fórmula correspondiente (resultados simplificados y racionalizados; no vale utilizar decimales):
% \begin{parts}
%     % \part[1] $\sen{2\alpha}$
%      \part[1] $\cos{\frac{\alpha}{2}}$
% \end{parts}

\question Resuelve las siguientes ecuaciones trigonométricas:
\begin{parts}
    \part[1]$\cos^2{x}-\sen^2{x}=0$
    \begin{solution}
        La ecuación es:
        \[
        \cos^2{x} - \sen^2{x} = 0.
        \]
        Usando la identidad trigonométrica \( \cos^2{x} - \sen^2{x} = \cos{2x} \), podemos reescribir la ecuación como:
        \[
        \cos{2x} = 0.
        \]
        Las soluciones de \( \cos{2x} = 0 \) son:
        \[
        2x = \frac{\pi}{2} + k\pi, \quad k \in \mathbb{Z}.
        \]
        Despejando \( x \), tenemos:
        \[
        x = \frac{\pi}{4} + \frac{k\pi}{2}, \quad k \in \mathbb{Z}.
        \]
    \end{solution}
\end{parts}

\question Dado \( \alpha \in III \) y \( \cos{\alpha} = -\frac{1}{2} \), hallar, utilizando la fórmula correspondiente (resultados simplificados y racionalizados; no vale utilizar decimales):
\begin{parts}
    \part[1] \( \cos{\frac{\alpha}{2}} \)
    \begin{solution}
        Sabemos que \( \cos{\alpha} = -\frac{1}{2} \) y que \( \alpha \in III \), lo que implica que \( \alpha \) está en el tercer cuadrante. Usamos la fórmula de mitad de ángulo para el coseno:
        \[
        \cos{\frac{\alpha}{2}} = \pm \sqrt{\frac{1 + \cos{\alpha}}{2}}.
        \]
        Sustituyendo \( \cos{\alpha} = -\frac{1}{2} \):
        \[
        \cos{\frac{\alpha}{2}} = \pm \sqrt{\frac{1 + \left(-\frac{1}{2}\right)}{2}} = \pm \sqrt{\frac{\frac{1}{2}}{2}} = \pm \sqrt{\frac{1}{4}} = \pm \frac{1}{2}.
        \]
        Como \( \alpha \in III \), entonces \( \frac{\alpha}{2} \) está en el segundo cuadrante, donde el coseno es negativo. Por lo tanto:
        \[
        \cos{\frac{\alpha}{2}} = -\frac{1}{2}.
        \]
    \end{solution}
\end{parts}

% \question Dado $z=\sqrt{2}-\sqrt{2}i$
% \begin{parts}
%     \part[1] calcula el opuesto y el conjugado y represéntalos en el plano complejo junto con $z$
%     \part[1] calcular 
%  $z^5\cdot\overline{z}$
%  \begin{solution}
%      $-64$
%  \end{solution}
    
% \end{parts}


\question Dado \( z = \sqrt{2} - \sqrt{2}i \)
\begin{parts}
    \part[1] Calcula el opuesto y el conjugado y represéntalos en el plano complejo junto con \( z \).
    \begin{solution}
        El opuesto de \( z \) es el número que tiene el signo opuesto en ambas partes, es decir:
        \[
        -z = -(\sqrt{2} - \sqrt{2}i) = -\sqrt{2} + \sqrt{2}i.
        \]
        El conjugado de \( z \), denotado por \( \overline{z} \), es el número que tiene el mismo valor en la parte real y el signo opuesto en la parte imaginaria, es decir:
        \[
        \overline{z} = \sqrt{2} + \sqrt{2}i.
        \]
        Para representar los tres números en el plano complejo, podemos ubicar \( z \) en el punto \( (\sqrt{2}, -\sqrt{2}) \), el opuesto \( -z \) en el punto \( (-\sqrt{2}, \sqrt{2}) \), y el conjugado \( \overline{z} \) en el punto \( (\sqrt{2}, \sqrt{2}) \).
    \end{solution}
    
    \part[1] Calcular \( z^5 \cdot \overline{z} \)
    \begin{solution}
        Sabemos que \( z = \sqrt{2} - \sqrt{2}i \) y \( \overline{z} = \sqrt{2} + \sqrt{2}i \). Primero, calculamos el módulo de \( z \):
        \[
        |z| = \sqrt{(\sqrt{2})^2 + (-\sqrt{2})^2} = \sqrt{2 + 2} = \sqrt{4} = 2.
        \]
        Ahora, utilizamos la fórmula para elevar un número complejo a una potencia. Expresamos \( z \) en forma polar:
        \[
        z = 2\left( \cos\left(-\frac{\pi}{4}\right) + i \sin\left(-\frac{\pi}{4}\right) \right).
        \]
        Usamos la fórmula de De Moivre para calcular \( z^5 \):
        \[
        z^5 = 2^5 \left( \cos\left( 5 \cdot -\frac{\pi}{4} \right) + i \sin\left( 5 \cdot -\frac{\pi}{4} \right) \right) = 32 \left( \cos\left( -\frac{5\pi}{4} \right) + i \sin\left( -\frac{5\pi}{4} \right) \right).
        \]
        Como \( \cos\left( -\frac{5\pi}{4} \right) = -\frac{1}{\sqrt{2}} \) y \( \sin\left( -\frac{5\pi}{4} \right) = -\frac{1}{\sqrt{2}} \), tenemos:
        \[
        z^5 = 32 \left( -\frac{1}{\sqrt{2}} - i \frac{1}{\sqrt{2}} \right) = -16\sqrt{2} - 16\sqrt{2}i.
        \]
        Ahora multiplicamos \( z^5 \) por el conjugado \( \overline{z} \):
        \[
        z^5 \cdot \overline{z} = \left( -16\sqrt{2} - 16\sqrt{2}i \right) \cdot \left( \sqrt{2} + \sqrt{2}i \right).
        \]
        Realizando la multiplicación:
        \[
        z^5 \cdot \overline{z} = (-16\sqrt{2})(\sqrt{2}) + (-16\sqrt{2})(\sqrt{2}i) + (-16\sqrt{2}i)(\sqrt{2}) + (-16\sqrt{2}i)(\sqrt{2}i).
        \]
        Simplificando:
        \[
        = -32 - 32i - 32i - 32 = -64 - 64i.
        \]
        Finalmente, la respuesta es:
        \[
        z^5 \cdot \overline{z} = -64.
        \]
    \end{solution}
\end{parts}


\question Resolver las siguientes ecuaciones: (soluciones complejas incluidas)
\begin{parts}
    \part[2] $z^5 = 1$
    \begin{solution}
        La ecuación \( z^5 = 1 \) es una ecuación polinómica de grado 5, cuya solución corresponde a las raíces quintas de la unidad. Estas raíces son los números complejos de la forma:
        \[
        z_k = e^{2\pi i k / 5}, \quad \text{para} \quad k = 0, 1, 2, 3, 4.
        \]
        Las soluciones son:
        \[
        z_0 = e^{2\pi i \cdot 0 / 5} = 1, \quad z_1 = e^{2\pi i / 5}, \quad z_2 = e^{4\pi i / 5}, \quad z_3 = e^{6\pi i / 5}, \quad z_4 = e^{8\pi i / 5}.
        \]
        En forma trigonométrica:
        \[
        z_0 = 1, \quad z_1 = \cos\left(\frac{2\pi}{5}\right) + i \sin\left(\frac{2\pi}{5}\right), \quad z_2 = \cos\left(\frac{4\pi}{5}\right) + i \sin\left(\frac{4\pi}{5}\right),
        \]
        \[
        z_3 = \cos\left(\frac{6\pi}{5}\right) + i \sin\left(\frac{6\pi}{5}\right), \quad z_4 = \cos\left(\frac{8\pi}{5}\right) + i \sin\left(\frac{8\pi}{5}\right).
        \]
    \end{solution}
\end{parts}


    \question[2] Desde un determinado punto se ve la parte superior de una antena de comunicaciones bajo un ángulo de $50^\circ$ con respecto a la horizontal. Nos alejamos 10 m y la visual forma ahora $40^\circ$. Queremos saber la distancia de ambos puntos de observación a la cúspide de la antena, con el fin de fijar sendos cables de sujeción.
\begin{solution}
    \begin{tikzpicture}[scale=1.5, every node/.style={font=\small}]

    % Ejes y antena
    \draw[thick] (0,0) -- (0,5) node[above]{Cúspide de la antena};
    \draw[dashed] (0,0) -- (6,0) node[below right]{Base de la antena};
    
    % Puntos de observación
    \draw[fill] (2,0) circle (0.05) node[below]{Primer punto};
    \draw[fill] (4,0) circle (0.05) node[below]{Segundo punto};
    
    % Líneas visuales
    \draw[thick] (2,0) -- (0,5) node[midway,left]{Visual $50^\circ$};
    \draw[thick] (4,0) -- (0,5) node[midway,left]{Visual $40^\circ$};
    
    % Ángulos
    \draw[->] (2,0) ++(-0.4,0) arc[start angle=180,end angle=130,radius=0.4] node[midway,left]{$50^\circ$};
    \draw[->] (4,0) ++(-0.4,0) arc[start angle=180,end angle=140,radius=0.4] node[midway,left]{$40^\circ$};
    
    % Distancias
    \draw[<->,dashed] (2,-0.5) -- (4,-0.5) node[midway,below]{$10 \, \text{m}$};
    
    \end{tikzpicture}

    En el triángulo pequeño de la derecha se cumple que los ángulos son $180-50=130^\circ$ y $180-130-40=10^\circ$. Aplicando el teorema del seno:
    $$\dfrac{10}{\sin{10}}=\dfrac{cable\ corto}{\sin{40}} \to cable\ corto=\dfrac{10\cdot\sin{40}}{\sin{10}}\approx 37.0166631359329
    $$
    Por el teorema del coseno:
    $$cable\ largo^2=10^2 + cable\ corto^2 - 2\cdot 10\cdot cable \ corto \cdot \cos{130^\circ}$$
    
    $$cable \ largo =\sqrt{10^2 + cable\ corto^2 - 2\cdot 10\cdot cable \ corto \cdot \cos{130^\circ}}\approx 44.1147412780977$$

\end{solution}

% \question[2]
% Un grupo decide escalar una montaña de la que desconocen la altura. A la salida del pueblo han medido
% el ángulo de elevación, que resulta ser 30º. A continuación han avanzado 100 m hacia la base de la
% montaña y han vuelto a medir el ángulo de elevación, siendo ahora 45º. Calcular la altura de la montaña
% \begin{solution}
%     $136,60 m$
% \end{solution}

\question[2] 
Un grupo decide escalar una montaña de la que desconocen la altura. A la salida del pueblo han medido el ángulo de elevación, que resulta ser 30º. A continuación han avanzado 100 m hacia la base de la montaña y han vuelto a medir el ángulo de elevación, siendo ahora 45º. Calcular la altura de la montaña.
\begin{solution}
    Sea \( h \) la altura de la montaña y \( x \) la distancia horizontal desde el primer punto de observación hasta la base de la montaña.

    En el primer punto de observación, el ángulo de elevación es 30º. Usando la tangente, que relaciona el ángulo de elevación con la altura y la distancia horizontal:
    \[
    \tan(30^\circ) = \frac{h}{x}.
    \]
    Sabemos que \( \tan(30^\circ) = \frac{1}{\sqrt{3}} \), por lo que:
    \[
    \frac{1}{\sqrt{3}} = \frac{h}{x}, \quad \text{lo que implica que} \quad h = \frac{x}{\sqrt{3}}. \tag{1}
    \]

    En el segundo punto de observación, después de avanzar 100 m hacia la base de la montaña, el ángulo de elevación es 45º. Ahora la distancia horizontal es \( x - 100 \), y usando la tangente nuevamente:
    \[
    \tan(45^\circ) = \frac{h}{x - 100}.
    \]
    Sabemos que \( \tan(45^\circ) = 1 \), por lo que:
    \[
    1 = \frac{h}{x - 100}, \quad \text{lo que implica que} \quad h = x - 100. \tag{2}
    \]

    Ahora, igualamos las dos expresiones para \( h \) obtenidas en las ecuaciones (1) y (2):
    \[
    \frac{x}{\sqrt{3}} = x - 100.
    \]
    Resolviendo esta ecuación para \( x \), multiplicamos ambos lados por \( \sqrt{3} \) para eliminar el denominador:
    \[
    x = \sqrt{3}(x - 100).
    \]
    Expandiendo:
    \[
    x = \sqrt{3}x - 100\sqrt{3}.
    \]
    Despejamos \( x \) en un solo lado:
    \[
    x - \sqrt{3}x = -100\sqrt{3},
    \]
    \[
    x(1 - \sqrt{3}) = -100\sqrt{3}.
    \]
    Finalmente, despejamos \( x \):
    \[
    x = \frac{-100\sqrt{3}}{1 - \sqrt{3}}.
    \]
    Para simplificar, multiplicamos numerador y denominador por el conjugado \( 1 + \sqrt{3} \):
    \[
    x = \frac{-100\sqrt{3}(1 + \sqrt{3})}{(1 - \sqrt{3})(1 + \sqrt{3})} = \frac{-100\sqrt{3}(1 + \sqrt{3})}{1 - 3} = \frac{-100\sqrt{3}(1 + \sqrt{3})}{-2}.
    \]
    Simplificando:
    \[
    x = 50\sqrt{3}(1 + \sqrt{3}).
    \]
    Ahora que tenemos el valor de \( x \), sustituimos en la ecuación (1) para encontrar \( h \):
    \[
    h = \frac{x}{\sqrt{3}} = \frac{50\sqrt{3}(1 + \sqrt{3})}{\sqrt{3}} = 50(1 + \sqrt{3}).
    \]
    Expresamos \( h \) de forma más simple:
    \[
    h = 50 + 50\sqrt{3}=136,60\  m.
    \]
    Este es el valor de la altura de la montaña.
\end{solution}


\addpoints
\end{questions}
\end{document}
\grid
