\documentclass[addpoints,spanish, 12pt,a4paper]{exam}
%\documentclass[answers, spanish, 12pt,a4paper]{exam}
\printanswers
\renewcommand*\half{.5}
\pointpoints{punto}{puntos}
\hpword{Puntos:}
\vpword{Puntos:}
\htword{Total}
\vtword{Total}
\hsword{Resultado:}
\hqword{Ejercicio:}
\vqword{Ejercicio:}

\usepackage[utf8]{inputenc}
\usepackage[spanish]{babel}
\usepackage{eurosym}
%\usepackage[spanish,es-lcroman, es-tabla, es-noshorthands]{babel}


\usepackage[margin=1in]{geometry}
\usepackage{amsmath,amssymb}
\usepackage{multicol}
\usepackage{yhmath}

\pointsinrightmargin % Para poner las puntuaciones a la derecha. Se puede cambiar. Si se comenta, sale a la izquierda.
\extrawidth{-2.4cm} %Un poquito más de margen por si ponemos textos largos.
\marginpointname{ \emph{\points}}

\usepackage{graphicx}

\graphicspath{{../img/}} 

\newcommand{\class}{1º Bachillerato CIE}
\newcommand{\examdate}{\today}
\newcommand{\examnum}{Parcial 3ªEv.}
\newcommand{\tipo}{A}


\newcommand{\timelimit}{45 minutos}

\renewcommand{\solutiontitle}{\noindent\textbf{Solución:}\enspace}


\pagestyle{head}
\firstpageheader{\includegraphics[width=0.2\columnwidth]{header_left}}{\textbf{Departamento de Matemáticas\linebreak \class}\linebreak \examnum}{\includegraphics[width=0.1\columnwidth]{header_right}}
\runningheader{\class}{\examnum}{Página \thepage\ de \numpages}
\runningheadrule


\usepackage{pgf,tikz,pgfplots}
\pgfplotsset{compat=1.15}
\usepackage{mathrsfs}
\usetikzlibrary{arrows}


\begin{document}

\noindent
\begin{tabular*}{\textwidth}{l @{\extracolsep{\fill}} r @{\extracolsep{6pt}} }
\textbf{Nombre:} \makebox[3.5in]{\hrulefill} & \textbf{Fecha:}\makebox[1in]{\hrulefill} \\
 & \\
\textbf{Tiempo: \timelimit} & Tipo: \tipo 
\end{tabular*}
\rule[2ex]{\textwidth}{2pt}
Esta prueba tiene \numquestions\ ejercicios. La puntuación máxima es de \numpoints. 
La nota final de la prueba será la parte proporcional de la puntuación obtenida sobre la puntuación máxima. 

\begin{center}


\addpoints
 %\gradetable[h][questions]
	\pointtable[h][questions]
\end{center}

\noindent
\rule[2ex]{\textwidth}{2pt}
\begin{questions}
\question[1\half] Calcula los límites de la siguiente función:

\begin{tikzpicture}[scale=0.6]
    % Configuración de ejes
    \draw[->] (-8, 0) -- (8, 0) node[right] {$X$}; % Eje X
    \draw[->] (0, -2) -- (0, 8) node[above] {$Y$}; % Eje Y

    % Cuadrícula
    \draw[help lines, color=gray!20] (-8, -2) grid (8, 8);

    % Etiquetas de cuadrícula
    \foreach \x in {-8, -6, -4, -2, 2, 4, 6, 8}
        \draw (\x, 0.1) -- (\x, -0.1) node[below] {\x};
    \foreach \y in {-2, 2, 4, 6, 8}
        \draw (0.1, \y) -- (-0.1, \y) node[left] {\y};

    % Primer trozo: Parábola que termina en (-4, 4)
    \draw[thick, red, domain=-8:-4, samples=50] plot (\x, {4+(0.25*(\x+4)*(\x+4))}) node[above] {};
    \filldraw[red] (-4, 4) circle (2pt); % Punto final en (-4, 4)

    % Segundo trozo: Recta de (-4, 1) a (-2, 3)
    \draw[thick, red] (-4, 1) -- (-2, 3);
    \filldraw[fill=white, draw=red] (-4, 1) circle (2pt); % Punto inicial
    % \filldraw[fill=white, draw=red] (-2, 3) circle (2pt); % Punto final

    % Tercer trozo: Recta de (-2, 3) a (1, 0)
    \draw[thick, red] (-2, 3) -- (1, 0);
    % \draw[red] (1, 0) circle (2pt); % Punto final en (1, 0)

    \filldraw[fill=white, draw=red] (-2, 3) circle (2pt); % Punto final
    \filldraw[red] (-2, 1) circle (2pt); % Punto final

    % Cuarto trozo: Proporcionalidad inversa creciente con mayor excentricidad
    \draw[thick, red, domain=1:2.883, samples=50] plot (\x, {-1*(\x-1)/(2*(\x-3))}) node[above] {};
    \draw[dashed] (3, 0) -- (3, 8); % Asíntota vertical en x = 3
    % \filldraw[red] (1, 0) circle (2pt); % Punto inicial en (1, 0)

    % Quinto trozo: Exponencial que pasa por (3, 5) y tiene asíntota horizontal y = 3
    \draw[thick, red, domain=3:8, samples=50] plot (\x, {3 + 2*exp(-0.5*(\x-3))});
    \draw[dashed] (0, 3) -- (8, 3); % Asíntota horizontal en y = 3
    \filldraw[fill=white, draw=red] (3, 5) circle (2pt); % Punto en (3, 5)

\end{tikzpicture}
\begin{multicols}{3}
\begin{parts}
    \part $\lim_{x \to -\infty} f(x)$
    \part $\lim_{x \to -4^-} f(x)$
    \part $\lim_{x \to -4^+} f(x)$
    \part $\lim_{x \to -4} f(x)$
    \part $\lim_{x \to -2} f(x)$
    \part $\lim_{x \to 1} f(x)$
    \part $\lim_{x \to -3^-} f(x)$
    \part $\lim_{x \to -3^+} f(x)$
    \part $\lim_{x \to +\infty} f(x)$
\end{parts}
\end{multicols}

        \question Determina el dominio de definición de las siguientes funciones:
        \begin{parts}
        % reduce_inequalities([(x**2-x)/(x+2)>=0]).as_set()
        \part[1] $f(x)=\sqrt{\dfrac{x^2 -x}{x+2}}$\begin{solution}  $\left(-2, 0\right] \cup \left[1, \infty\right)$\end{solution}
        \part[1] $f(x)=\dfrac{3x+2}{x^4-5x^2-36}$\begin{solution}$\left(-\infty, -3\right] \cup \left[3, \infty\right)$\end{solution}
        \end{parts}
        
        % \question Dada la función $f(x)=\dfrac{3 x - 2}{2}$:
        %         \begin{parts} \part[1] Calcula $f^{-1}(x)$, es decir la inversa de $f(x)$  \begin{solution}   $f^{-1}(x)=\frac{2 x}{3} + \frac{2}{3}$ \\ $f^{-1} \circ f(x)=x=x$ \\ 
        %        \end{solution}
        %        \part[1] Comprueba que efectivamente son inversas
        % % \part[1] $f(x)=\frac{x}{- x + 1}$  \begin{solution}   $f^{-1}(x)=\frac{x}{x + 1}$ \\ $f^{-1} \circ f(x)=\frac{x}{\left(- x + 1\right) \left(\frac{x}{- x + 1} + 1\right)}=x$   \end{solution}
        % \end{parts}
        
        
        \question Calcula los siguientes límites:

        \begin{parts} 
        \part[1] $$\lim_{x \to -1}\left(x^{2} - 3\right)$$  \begin{solution}   $\lim_{x \to -1^-}\left(x^{2} - 3\right)=-2$ y  \\ $\lim_{x \to -1^+}\left(x^{2} - 3\right)=-2$   \end{solution} 
        \part[1] $$\lim_{x \to 2}\left(\frac{x^{3} - 2 x^{2} + 2 x - 4}{3 x^{2} - 8 x + 4}\right)$$  \begin{solution}   $\frac{3}{2}$   \end{solution} 
        % \part[1] $$\lim_{x \to -\infty} e^{x - 1}$$  \begin{solution}   $0$   \end{solution} 
        \part[1] $$\lim_{x \to -1}\left(\frac{x^{3} + 1}{x^{2} + 2 x + 1}\right)$$  \begin{solution}   No existe el límite   \end{solution} 
        % \part[2] $$\lim_{x \to 3} \left(\frac{x^{2} - x}{x + 3}\right)^{\frac{1}{x - 3}}$$  \begin{solution}   $e^{\frac{2}{3}}$   \end{solution}
        \part[2] $$\lim_{x \to \infty}\left(\dfrac{x^{2} - 1}{x} - \dfrac{2 x^{2} + 1}{2 x - 1}\right)$$\begin{solution}$-\frac{1}{2}$\end{solution}
        
        \end{parts}
        
        
        
%        \question Dada la función:$f(x)=\frac{x^{2} - 2 x + 1}{2 x + 3}$, calcular:
%        \begin{multicols}{1}
%        \begin{parts} \part[1] Dominio de $f(x)$  \begin{solution}   $Dom(f)=\left(-\infty, - \frac{3}{2}\right) \cup \left(- \frac{3}{2}, \infty\right)$\\ \resizebox{0.4\textwidth}{!}{\includegraphics[width=1\columnwidth]{fin301-0}}   \end{solution} \part[2] Asíntotas verticales, horizontales y oblicuas, en caso que existan  \begin{solution}   Asíntotas:\\A.V. $x=-3/2$\\A.O. $y=\frac{x}{2} - \frac{7}{4}$ \\A.O. $y=\frac{x}{2} - \frac{7}{4}$ \\   \end{solution}
%        \end{parts}
%        \end{multicols}
%        \question Dada la función:$f(x)=\frac{- x^{2} - x + 3}{x^{2} + x - 2}$, calcular:
%        \begin{multicols}{1}
%        \begin{parts} \part[1] Dominio de $f(x)$  \begin{solution}   $Dom(f)=\left(-\infty, -2\right) \cup \left(-2, 1\right) \cup \left(1, \infty\right)$\\ \resizebox{0.4\textwidth}{!}{\includegraphics[width=1\columnwidth]{fin301-1}}   \end{solution} \part[2] Asíntotas verticales, horizontales y oblicuas, en caso que existan  \begin{solution}   Asíntotas:\\A.V. $x=-2$\\, A.V. $x=1$\\A.H. $y=-1$\\A.H. $y=-1$\\A.O. $y=-1$ \\A.O. $y=-1$ \\   \end{solution}
%        \end{parts}
%        \end{multicols}


        % \question Dada la función:$f(x)=\sqrt{\frac{x}{x - 1}}$, calcular:
        % \begin{multicols}{1}
        % \begin{parts} \part[1] Dominio de $f(x)$  \begin{solution}   $Dom(f)=\left(-\infty, 0\right] \cup \left(1, \infty\right)$\\ \resizebox{0.4\textwidth}{!}{\includegraphics[width=1\columnwidth]{fin301-2}}   \end{solution} \part[2] Asíntotas verticales, horizontales y oblicuas, en caso que existan  \begin{solution}   Asíntotas:\\A.V. $x=1$\\A.H. $y=1$\\A.H. $y=1$\\A.O. $y=1$ \\A.O. $y=1$ \\   \end{solution}
        % \end{parts}
        % \end{multicols}

       \question Estudia en qué puntos de $\mathbb{R}$ la función no es continua: 

       \begin{parts} \part[2] $f(x)=\begin{cases} \dfrac{2 x^{2} + 7 x}{x^{2} - 9} & \text{si}\: x \leq -2 \\\dfrac{x+2}{x^{2} + 2 x} & \text{si} \: x > -2\end{cases}$  \begin{solution}   Singularidades de las expresiones analíticas: $\left\{-3, 0\right\}$.\\ Posibles discontinuidades en los extremos de los trozos:-2.\\En -2 no es continua porque no existe límite. Límites laterales: $\frac{6}{5}$ y $- \frac{1}{2}$   \end{solution}
       \part \textbf{(2 puntos extra)} Indica, razonadamente, los tipos de discontinuidad que hay la función anterior
       \begin{solution}
       En $x=-3$ salto infinito \end{solution}
       \end{parts}
       

       % \question Estudia en qué puntos de $\mathbb{R}$ la función no es continua: 

       % \begin{parts} \part[2] $f(x)=\begin{cases} \dfrac{2 x^{2} + 7 x + 3}{x^{2} - 9} & \text{si}\: x \leq -2 \\\dfrac{\sqrt{x + 3} - 1}{x^{2} + 2 x} & \text{si} \: x > -2\end{cases}$  \begin{solution}   Singularidades de las expresiones analíticas: $\left\{-3, 0\right\}$.\\ Posibles discontinuidades en los extremos de los trozos:-2.\\En -2 no es continua porque no existe límite. Límites laterales: $\frac{3}{5}$ y $- \frac{1}{4}$   \end{solution}
       % \part \textbf{(2 puntos extra)} Indica, razonadamente, los tipos de discontinuidad que hay
       % \end{parts}

        
        
        % \question Estudia en qué puntos de $\mathbb{R}$ la función no es continua: 
        % \begin{multicols}{1}
        % \begin{parts} \part[2] $f(x)=\begin{cases} \frac{x^{2} - 4}{x^{2} - 3 x + 2} & \text{si}\: x < 2 \\4 & \text{si}\: 2\leq x < 5 \\e^{x - 5} + 3 & \text{si}\: x > 5 \end{cases}$  \begin{solution}   Singularidades de las expresiones analíticas: $\left\{1\right\}$.\\ Posibles discontinuidades en los extremos de los trozos:2, 5.\\En 2 es continua ya que hay límite y $\lim = f(2)=4$. \\En 5 es continua ya que hay límite y $\lim = f(5)=4$   \end{solution}
        % \end{parts}
        % \end{multicols}
        
       % \question Halla a y b de modo que las siguientes funciones sean continuas:
       % \begin{multicols}{1}
       % \begin{parts} \part[2] $$f(x)=\begin{cases} a + e^{x + 2} & \text{si}\: x \leq -2 \\\frac{x + 1}{3 - x} & \text{si}\: -2 < x < 1 \\b x + 3 & \text{si}\: x \geq 1 \end{cases}$$  \begin{solution}   $\left\{ a : - \frac{6}{5}, \  b : -2\right\}$   \end{solution}
       % \end{parts}
       % \end{multicols}
        
        % \question Halla a y b de modo que las siguientes funciones sean continuas:
        % \begin{multicols}{1}
        % \begin{parts} \part[2] $$f(x)=\begin{cases} a + e^{x + 3} & \text{si}\: x \leq -3 \\\frac{x + 2}{4 - x} & \text{si}\: -3 < x < 1 \\b x + 6 & \text{si}\: x \geq 1 \end{cases}$$  \begin{solution}   $\left\{ a : - \frac{8}{7}, \  b : -5\right\}$   \end{solution}
        % \end{parts}
        % \end{multicols}
        
        
        
%        \question Deriva las siguientes funciones (simplificando el resultado al máximo):
%        \begin{multicols}{1}
%        \begin{parts} \part[1] $y=\frac{3 x^{2} - 2 x + 1}{\left(x - 1\right)^{2}}$  \begin{solution}   $y'=- \frac{4 x}{x^{3} - 3 x^{2} + 3 x - 1}$   \end{solution} \part[1] $y=\sqrt{\sqrt{x} + 1}$  \begin{solution}   $y'=\frac{1}{4 \sqrt{x} \sqrt{\sqrt{x} + 1}}$   \end{solution} \part[1] $y=\frac{\log{\left(x^{2} \right)}}{x}$  \begin{solution}   $y'=\frac{2 - \log{\left(x^{2} \right)}}{x^{2}}$   \end{solution} \part[1] $y=3 \sin{\left(\cos{\left(2 x \right)} \right)}$  \begin{solution}   $y'=- 6 \sin{\left(2 x \right)} \cos{\left(\cos{\left(2 x \right)} \right)}$   \end{solution}
%        \end{parts}
%        \end{multicols}
        
        % \question Deriva las siguientes funciones (simplificando el resultado al máximo):
        % \begin{multicols}{1}
        % \begin{parts} \part[1] $y=\frac{2 x^{2} - 2 x + 1}{\left(x - 1\right)^{2}}$  \begin{solution}   $y'=- \frac{2 x}{x^{3} - 3 x^{2} + 3 x - 1}$   \end{solution} \part[1] $y=\sqrt{2 - \sqrt{x}}$  \begin{solution}   $y'=- \frac{1}{4 \sqrt{x} \sqrt{2 - \sqrt{x}}}$   \end{solution} \part[1] $y=\frac{\log{\left(x \right)}}{x}$  \begin{solution}   $y'=\frac{1 - \log{\left(x \right)}}{x^{2}}$   \end{solution} \part[1] $y=2 \cos{\left(\sin{\left(2 x \right)} \right)}$  \begin{solution}   $y'=- 4 \sin{\left(\sin{\left(2 x \right)} \right)} \cos{\left(2 x \right)}$   \end{solution}
        % \end{parts}
        % \end{multicols}


\addpoints
\end{questions}
\end{document}
\grid
