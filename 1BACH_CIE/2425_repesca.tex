\documentclass[addpoints,spanish, 12pt,a4paper]{exam}
%\documentclass[answers, spanish, 12pt,a4paper]{exam}
% \printanswers
\renewcommand*\half{.5}
\pointpoints{punto}{puntos}
\hpword{Puntos:}
\vpword{Puntos:}
\htword{Total}
\vtword{Total}
\hsword{Resultado:}
\hqword{Ejercicio:}
\vqword{Ejercicio:}

\usepackage[utf8]{inputenc}
\usepackage[spanish]{babel}
\usepackage{eurosym}
%\usepackage[spanish,es-lcroman, es-tabla, es-noshorthands]{babel}


\usepackage[margin=1in]{geometry}
\usepackage{amsmath,amssymb}
\usepackage{multicol}
\usepackage{yhmath}

\pointsinrightmargin % Para poner las puntuaciones a la derecha. Se puede cambiar. Si se comenta, sale a la izquierda.
\extrawidth{-2.4cm} %Un poquito más de margen por si ponemos extremostextos largos.
\marginpointname{ \emph{\points}}

\usepackage{graphicx}

\graphicspath{{../img/}} 

\newcommand{\class}{1º Bachillerato CIE}
\newcommand{\examdate}{\today}
\newcommand{\examnum}{Parcial 3ªEv.}
\newcommand{\tipo}{A}


\newcommand{\timelimit}{45 minutos}

\renewcommand{\solutiontitle}{\noindent\textbf{Solución:}\enspace}


\pagestyle{head}
\firstpageheader{\includegraphics[width=0.2\columnwidth]{header_left}}{\textbf{Departamento de Matemáticas\linebreak \class}\linebreak \examnum}{\includegraphics[width=0.1\columnwidth]{header_right}}
\runningheader{\class}{\examnum}{Página \thepage\ de \numpages}
\runningheadrule


\usepackage{pgf,tikz,pgfplots}
\pgfplotsset{compat=1.15}
\usepackage{mathrsfs}
\usetikzlibrary{arrows}


\begin{document}

\noindent
\begin{tabular*}{\textwidth}{l @{\extracolsep{\fill}} r @{\extracolsep{6pt}} }
\textbf{Nombre:} \makebox[3.5in]{\hrulefill} & \textbf{Fecha:}\makebox[1in]{\hrulefill} \\
 & \\
\textbf{Tiempo: \timelimit} & Tipo: \tipo 
\end{tabular*}
\rule[2ex]{\textwidth}{2pt}
Esta prueba tiene \numquestions\ ejercicios. La puntuación máxima es de \numpoints. 
La nota final de la prueba será la parte proporcional de la puntuación obtenida sobre la puntuación máxima. 

\begin{center}


\addpoints
 %\gradetable[h][questions]
	\pointtable[h][questions]
\end{center}

\noindent
\rule[2ex]{\textwidth}{2pt}
\begin{questions}
\question[2] Resuelve mediante expresiones algebraicas y, en caso que se pueda, por
Gauss:
\begin{parts}
    \part Tres amigas, Elena, Carmen y Cristina entran en una tienda de deportes en la que sólo hay
tres tipos de artículos. Carmen
se compra 1 par de zapatillas, 2 sudaderas y 2 pantalones. Elena se compra 2 pares de zapatillas, 1 sudadera y 1 pantalón, y Cristina se compra 2 pares de zapatillas
y 3 pantalones. Elena se ha gastado en total 70 euros, Carmen 80 euros y Cristina 70 euros. ¿Cuánto vale
cada artículo? 
\begin{solution}
    \[
\left[
\begin{array}{ccc|c}
1 & 2 & 2 & 80 \\
2 & 1 & 1 & 70 \\
2 & 0 & 3 & 70
\end{array}
\right]
\]

\[
\left[
\begin{array}{ccc|c}
1 & 2 & 2 & 80 \\
0 & 1 & 1 & 30 \\
0 & 0 & 1 & 10
\end{array}
\right]
\]

\[
\begin{aligned}
z &= 20 \quad \text{(zapatillas)} \\
s &= 20 \quad \text{(sudadera)} \\
p &= 10 \quad \text{(pantalón)}
\end{aligned}
\]

\end{solution}
\end{parts}

\question Resuelve las siguientes inecuaciones:
\begin{parts}
            \part[1]  $ \dfrac{{{x^3} - 5{x^2} + 2x + 8}}{1-x^2} \leq 0 $  \begin{solution}  $  \left(1, 2\right] \cup \left[4, \infty\right) $  \end{solution}
        \part[1] $1-\dfrac{{x + 3}}{{x + 6}} \geq 0 $ \begin{solution} $\rightarrow \left(-6, \infty\right)$\end{solution}
\end{parts}

\question[1] Responde a las siguientes cuestiones:
\begin{parts}
    \part Resuelve la ecuación $\cos{x} = \sen^2{x}-1$
    \begin{solution}
        $[-90, 90, 180]$
    \end{solution}
    \part Resuelve la ecuación $\sen{x} = \cos^2{x}-1$
    \begin{solution}
        $[0, 180, 270]$
    \end{solution}
    \part  Resuelve la ecuación $ 2\cos^2{x}-\sqrt{3}\cos{x}=0 $  \begin{solution}  $ \left [ 30, \quad 90, \quad 270, \quad 330\right ] $  \end{solution}
    \part Resuelve la ecuación  $ \cos{2x}-3\cos{x}+1=0 $  \begin{solution}  $ \left [ -90, \quad 90\right ] $  \end{solution}
    \part Desde un punto del suelo se ve la copa de un pino bajo un ángulo de 42°. Si nos alejamos 2,5 m hacia otro punto del suelo, alineado con el anterior y con el pie del pino, vemos la copa bajo un ángulo de 24°. Calcula la altura del pino. \begin{solution}
        $2,2 m$
        \end{solution}
\end{parts}

\question Resolver las siguientes ecuaciones:
\begin{parts} \part[1]  $ \cos{2x}-3\cos{x}+1=0 $  \begin{solution}  $ \left [ -90, \quad 90\right ] $  \end{solution} \part[1]  $ 2\cos^2{x}-\sqrt{3}\cos{x}=0 $  \begin{solution}  $ \left [ 30, \quad 90, \quad 270, \quad 330\right ] $  \end{solution}
\end{parts}

\question Calcula:
\begin{parts} \part[1]  $ \frac{(1+2i)i^7}{(3-2i)-(2+i)} $  \begin{solution}  $ \frac{1}{2} + \frac{i}{2} $  \end{solution}
\end{parts}

\question Escribe los siguientes números complejos en forma polar con el argumento en radianes:
\begin{parts} \part[1]  $ -4 $  \begin{solution}  $ 4_{\pi} $  \end{solution} \part[1]  $ 2i $  \begin{solution}  $ 2_{\frac{\pi}{2}} $  \end{solution} \part[1]  $ 2 - 2\sqrt {3}i $  \begin{solution}  $ 4_{- \frac{\pi}{3}} $  \end{solution}
\end{parts}

\question[1] Resuelve las siguientes cuestiones:
        \begin{parts} \part  Determina el ángulo formado por las rectas: $ r\equiv2x-y-2=0\  y \ s\equiv3x+2y-4=0 $  \begin{solution}  $ 119.74488129694222 $  \end{solution}
        \part Determina la ecuación de la recta perpendicular a la recta \( 3x - 4y = 12 \), que pase por el punto \( P(2, 3) \)
        \begin{solution}
            $4x+3y=17$
        \end{solution}
        \end{parts}

\addpoints
\end{questions}
\end{document}
\grid
