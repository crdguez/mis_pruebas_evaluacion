
\documentclass[spanish, 11pt]{exam}

\usepackage{array,epsfig}
\usepackage{amsmath, textcomp}
\usepackage{amsfonts}
\usepackage{amssymb}
\usepackage{amsxtra}
\usepackage{amsthm}
\usepackage{mathrsfs}
\usepackage{color}
\usepackage{multicol, xparse}
\usepackage{verbatim}
\usepackage{booktabs}

\usepackage[utf8]{inputenc}
\usepackage[spanish]{babel}
\usepackage{eurosym}

\usepackage{graphicx}
\graphicspath{{../img/}}
\usepackage{pgf}

\usepackage{pgf,tikz,pgfplots}
\pgfplotsset{compat=1.15}
\usepackage{mathrsfs}
\usetikzlibrary{arrows}

%\printanswers
\nopointsinmargin
\pointformat{}

\let\multicolmulticols\multicols
\let\endmulticolmulticols\endmulticols
\RenewDocumentEnvironment{multicols}{mO{}}
 {%
  \ifnum#1=1
    #2%
  \else
    \multicolmulticols{#1}[#2]
  \fi
 }
 {%
  \ifnum#1=1
  \else
    \endmulticolmulticols
  \fi
 }

\renewcommand{\solutiontitle}{\noindent\textbf{Sol:}\enspace}
\newcommand{\samedir}{\mathbin{\!/\mkern-5mu/\!}}

\newcommand{\class}{1º Bachillerato}
\newcommand{\examdate}{\today}
\newcommand{\tipo}{A}
\newcommand{\timelimit}{50 minutos}

\pagestyle{head}
\firstpageheader{Dep. Matemáticas}{Ejercicios de repaso 1 Bac. Ciencias}{IES Goya}
\runningheader{IES Goya}{Ejercicios de repaso}{Página \thepage\ de \numpages}
\runningheadrule

\begin{document}
\begin{questions}
% Preguntas extraídas de: 01_2425_par1.tex
%\question 
%
%\begin{parts}
%\part[2] 
%\begin{solution}
%\end{solution}
%
%
%\end{parts}
%\addpoints

\question[1] Indica a cuáles de los conjuntos
$\mathbb{N}$, $\mathbb{Z}$, $\mathbb{Q}$, $\mathbb{R}$ pertenecen cada uno de los siguientes números:
\begin{center}
\begin{tabular}{|c |c |c |c |c|}\hline
&$\mathbb{N}$& $\mathbb{Z}$& $\mathbb{Q}$&$\mathbb{R}$\\ 
\hline
$\frac{8}{16}$&&&&\\
\hline
$\sqrt[3]{-27}$&&&&\\
\hline
$3.0\wideparen{1}$&&&&\\
\hline
$-\frac{12}{4}$&&&&\\
\hline
$-\sqrt{25}$&&&&\\
\hline
$\sqrt{8}$&&&&\\
\hline
$4$&&&&\\
\hline
$\pi$&&&&\\
\hline
$\sqrt{-4}$&&&&\\
\hline
$\frac{39}{13}$&&&&\\
\hline
\end{tabular}

\end{center}

        \question Sean los siguientes conjuntos $ A=\left\{ x \in \mathbb{R}| -2 \leq x  \leq 5 \right\}$, $  B=\left(-\infty, -1\right) \cup \left(1, \infty\right)$  y  $C=\left\{ x \in \mathbb{R}| \left|{x - 2}\right|\leq3 \right\}$:  
        \begin{parts} 
        \part[0\half] Represéntalos en la recta real
        \part[1\half] Calcula $A \cup  B$ , $A \cap B$ y $A \cap B \cap C$, 
expresando los resultados en forma de Intervalos. 
        \end{parts}

          \begin{solution}  $ A \cup  B = \left(-\infty, \infty\right)  \\  A \cap B= \left[-2, -1\right) \cup \left(1, 5\right]   \\  A \cap B  \cap C= \left(1, 5\right] $  \end{solution}
        
\question[1\half] Halla el término general de las siguientes sucesiones. Indica cuáles de ellas son progresiones aritméticas y cuáles progresiones geométricas:
\begin{parts}
    \part \( 4, 9, 14, 19, 24, 29, \ldots \)
    \part \( 2, -6, 18, -54, 162, \ldots \)
\end{parts}
\begin{solution}
\text{PA: 1. } a_n = 5n - 1; \, \text{SG: -3. } a_n = (-3)^{n-1}\cdot2
\end{solution}


\question Resuelve las siguientes ecuaciones:

\begin{parts}
%     \part[2]$2 \sqrt{x + 4} - \sqrt{x - 1} = 4$
%     \begin{solution}
% 1. Aislamos uno de los radicales:
%    \[
%    2 \sqrt{x + 4} = \sqrt{x - 1} + 4
%    \]
% 2. Elevamos ambos lados al cuadrado:
%    \[
%    4(x + 4) = (\sqrt{x - 1} + 4)^2
%    \]
%    \[
%    4x + 16 = x - 1 + 8 \sqrt{x - 1} + 16
%    \]
% 3. Simplificamos y resolvemos la ecuación resultante.
% $3x+1=8 \sqrt{x - 1} \to x=13/9, x=5
%  $

% \end{solution}
    \part[1\half] $\sqrt{2x+3}-\sqrt{x+2}=2$
    \begin{solution}
    $2x+3=4+4\sqrt{x+2}+x+2 \to x-3=4\sqrt{x+2} \to x^2-6x+9=16x+32\to x^2-22x-23=0$\\
    $x=23$ sí, $x=-1$ No
\end{solution}
    \part[1\half] $\sqrt[3]{x + 6} = x$
    \begin{solution}
1. Elevamos ambos lados al cubo:
   \[
   x + 6 = x^3
   \]
2. Reorganizamos la ecuación:
   \[
   x^3 - x - 6 = 0
   \]
3. Intentamos con divisores de 6 para encontrar posibles raíces.
$$x^3 - x - 6=\left(x - 2\right) \left(x^{2} + 2 x + 3\right)\to x=2$$
\end{solution}
    \part[1\half] $9^x - 6 \cdot {3^{x + 1}} + 81 = 0 $
\begin{solution}
    $\rightarrow \left [ 2\right ]$
\end{solution}



\end{parts}


% \question Resolver la ecuación
% \[
% 4x^3 - 6x^2 - x = -\frac{3}{2},
% \]
% sabiendo que una de sus raíces es \( \frac{1}{2} \).

% \begin{solution}
% 1. Llevamos todos los términos a un lado:
%    \[
%    4x^3 - 6x^2 - x + \frac{3}{2} = 0
%    \]
% 2. Multiplicamos por 2 para eliminar fracciones:
%    \[
%    8x^3 - 12x^2 - 2x + 3 = 0
%    \]
% 3. Factorizando el polinomio 
% $\frac{\left(2 x - 3\right) \left(2 x - 1\right) \left(2 x + 1\right)}{2} \to \left[ - \frac{1}{2}, \  \frac{1}{2}, \  \frac{3}{2}\right]
% $
% \end{solution}

% \question[1] Resolver la ecuación
% \[
% 2x^3 - 3x^2 = -\frac{1}{2},
% \]
% sabiendo que una de sus raíces es \( \frac{1}{2} \).

% \begin{solution}
% 1. Llevamos todos los términos a un lado:
%    \[
%    2x^3 - 3x^2 + \frac{1}{2} = 0
%    \]
% 2. Multiplicamos por 2 para eliminar fracciones:
%    \[
%    4x^3 - 6x^2 + 1 = 0
%    \]
% 3. $$\left(2 x - 1\right) \left(2 x^{2} - 2 x - 1\right) \to \left[ \frac{1}{2}, \  \frac{1}{2} - \frac{\sqrt{3}}{2}, \  \frac{1}{2} + \frac{\sqrt{3}}{2}\right]$$
% \end{solution}

% \question Resuelve la siguiente ecuación:
% $9^x - 6 \cdot {3^{x + 1}} + 81 = 0 $
% \begin{solution}
%     $\rightarrow \left [ 2\right ]$
% \end{solution}


\question Resuelve los siguientes sistemas:
\begin{parts}
    \part[1\half] $\left\{\begin{matrix}{2^x} + {2^y} = 24\\ \\{2^{x + y}} = 128\end{matrix}\right. $

\begin{solution}
    $\rightarrow \left [ \left \{ x : 3, \quad y : 4\right \}, \quad \left \{ x : 4, \quad y : 3\right \}\right ]$
\end{solution}

    \part[1\half] $\left\{\begin{matrix}\log_{2}(x) + \log_{2}(x + y) = 4\\ \\x + y = 2 \end{matrix}\right. $
\begin{solution}
    $\rightarrow \left [ \left \{ x : 8, \quad y : -6\right \}\right ]$
\end{solution}

\end{parts}


% \question Resuelve:
% $$\left\{\begin{matrix}3\log x - 2\log y = 10\\\log x + 3\log y = 7\end{matrix}\right. $$

% \begin{solution}
%     $\rightarrow \left [ \left \{ x : 10000, \quad y : 10\right \}\right ]$
% \end{solution}

% \question[2] Resuelve el siguiente sistema de ecuaciones:
% \[
% \begin{cases}
% \dfrac{1}{x} + y = 3 \\ \\
% \dfrac{1}{x} - \dfrac{1}{y} = \dfrac{1}{2}
% \end{cases}
% \]

\begin{solution}
1. Llamamos \(u = \frac{1}{x}\) y \(v = y\), para obtener el sistema:
   \[
   u + v = 3
   \]
   \[
   u - \frac{1}{v} = \frac{1}{2}
   \]
2. Resolvemos este sistema para \(u\) y \(v\), y luego encontramos \(x\) y \(y\) a partir de \(u\) y \(v\).
\end{solution}

\question[2] Resuelve por Gauss el siguiente sistema de ecuaciones:
\[
\begin{cases}
3x + 2y - 2z = 4 \\
4x + y - z = 7 \\
x + 4y - 4z = 0
\end{cases}
\]

\begin{solution}
$\left[\begin{matrix}3 & 2 & -2 & 4\\0 & - \frac{5}{3} & \frac{5}{3} & \frac{5}{3}\\0 & 0 & 0 & 2\end{matrix}\right]$ Luego S.I.
\end{solution}

\question[1] Resuelve la siguiente inecuación: $$1-\frac{{x + 3}}{{x + 6}} \geq 0 $$ \begin{solution} $\rightarrow \left(-6, \infty\right)$\end{solution}



        
        % \question Usando la definición y las propiedades de los números combinatorios, resolver las ecuaciones:
        % \begin{multicols}{1} 
        % \begin{parts} \part[1]  $ {\binom{x}{2}} = 28 $  \begin{solution}  $ \left\{8\right\} $  \end{solution}
        % \end{parts}
        % \end{multicols}
        % \question Calcula, sin hacer todo el desarrollo, el coeficiente del término asociado a:
        % \begin{multicols}{1} 
        % \begin{parts} \part[2]  $ P(x)=\left(2 x - \frac{3}{x}\right)^{8} \  \ y \ parte \ literal \ \frac{1}{x^{2}} $  \begin{solution}  $ -108864 $  \end{solution}
        % \end{parts}
        % \end{multicols}
        % \question Efectúa:
        % \begin{multicols}{1} 
        % \begin{parts} \part[1]  $ 3\sqrt [3] {16} - 2\sqrt [3] {250} + 5\sqrt [3] {250} - 4\sqrt [3] { 2}  $  \begin{solution}  $ 17 \sqrt[3]{2}\to17 \sqrt[3]{2} $  \end{solution} \part[1]  $ \frac{{2 - \sqrt {3} }}{{1 - \sqrt {3} }}-\frac{1}{2\sqrt {3}}+ \frac{3}{2-\sqrt{3} } $  \begin{solution}  $ \frac{14 \sqrt{3} + 39}{6}\to\frac{7 \sqrt{3}}{3} + \frac{13}{2} $  \end{solution}
        % \end{parts}
        % \end{multicols}
        % \question Calcula el valor de $k$ para que:
        % \begin{multicols}{1} 
        % \begin{parts} \part[1]  $ El \ resto \ de \ dividir \ P(x)=x^{25}-kx+3k-4 \ entre\  x+1 \ sea \ 11  $  \begin{solution}  $ 4 $  \end{solution}
        % \end{parts}
        % \end{multicols}
        % \question Halla el m.c.d. y el m.c.m. de los polinomios: 
        % \begin{multicols}{1} 
        % \begin{parts} \part[3]  $  A(x)= x^{5} - 6 x^{3} + 2 x^{2} + 9 x - 6\  y \\  B(x)= x^{5} + 3 x^{4} - 3 x^{3} - 13 x^{2} + 12 \\  $  \begin{solution}  $ Descomposici \acute on:(\left(x - 1\right)^{2} \left(x + 2\right) \left(x^{2} - 3\right) \ y  \ \left(x - 1\right) \left(x + 2\right)^{2} \left(x^{2} - 3\right))\\x^{4} + x^{3} - 5 x^{2} - 3 x + 6= \left(x - 1\right) \left(x + 2\right) \left(x^{2} - 3\right) \ MCD\ y \\ x^{6} + 2 x^{5} - 6 x^{4} - 10 x^{3} + 13 x^{2} + 12 x - 12= \left(x - 1\right)^{2} \left(x + 2\right)^{2} \left(x^{2} - 3\right) \ MCM $  \end{solution}
        % \end{parts}
        % \end{multicols}






\addpoints

% Preguntas extraídas de: 02_2425_final1.tex
%\question 
%
%\begin{parts}
%\part[2] 
%\begin{solution}
%\end{solution}
%
%
%\end{parts}
%\addpoints

% \question[1] Indica a cuáles de los conjuntos
% $\mathbb{N}$, $\mathbb{Z}$, $\mathbb{Q}$, $\mathbb{R}$ pertenecen cada uno de los siguientes números:
% \begin{center}
% \begin{tabular}{|c |c |c |c |c|}\hline
% &$\mathbb{N}$& $\mathbb{Z}$& $\mathbb{Q}$&$\mathbb{R}$\\ 
% \hline
% $\frac{8}{16}$&&&&\\
% \hline
% $\sqrt[3]{-27}$&&&&\\
% \hline
% $3.0\wideparen{1}$&&&&\\
% \hline
% $-\frac{12}{4}$&&&&\\
% \hline
% $-\sqrt{25}$&&&&\\
% \hline
% $\sqrt{8}$&&&&\\
% \hline
% $4$&&&&\\
% \hline
% $\pi$&&&&\\
% \hline
% $\sqrt{-4}$&&&&\\
% \hline
% $\frac{39}{13}$&&&&\\
% \hline
% \end{tabular}

% \end{center}

%         \question[2]  Dados los conjuntos:\\
%          $ A=\left\{ x \in \mathbb{R}| 6 \leq x  < 8 \right\}, \\ B=\left(-\infty, -3\right) \cup \left(3, \infty\right)  \\  C=\left\{ x \in \mathbb{R}| \left|{x - 3}\right|\leq12 \right\}  $
         
%          \begin{parts}
%           % \part Indica razonadamente, si existen, los supremos, ínfimos, máximos y mínimos de los conjuntos $A$, $B$ y $C$
%           \part Calcula $A \cup  B$ , $A \cap B$ y $(A \cup B) \cap C$ , 
% y expresa los resultados en forma de Intervalos.
%          \end{parts}
         
% \begin{solution}  $ C=\left[-9, 15\right] \ \ A \cup  B = \left(-\infty, -3\right) \cup \left(3, \infty\right)  \\  A \cap B= \left[6, 8\right)   \\  (A \cup B) \cap C= \left[-9, -3\right) \cup \left(3, 15\right] $  \end{solution}

       
        
% \question[1\half] Halla el término general de las siguientes sucesiones. Indica cuáles de ellas son progresiones aritméticas y cuáles progresiones geométricas:
% \begin{parts}
%     \part \( 4, 9, 14, 19, 24, 29, \ldots \)
%     \part \( 2, -6, 18, -54, 162, \ldots \)
% \end{parts}
% \begin{solution}
% $\text{PA: 1. } a_n = 5n - 1; \, \text{SG: -3. } a_n = (-3)^{n-1}\cdot 2$\end{solution}


\question Resuelve las siguientes ecuaciones:

\begin{parts}
%     \part[2]$2 \sqrt{x + 4} - \sqrt{x - 1} = 4$
%     \begin{solution}
% 1. Aislamos uno de los radicales:
%    \[
%    2 \sqrt{x + 4} = \sqrt{x - 1} + 4
%    \]
% 2. Elevamos ambos lados al cuadrado:
%    \[
%    4(x + 4) = (\sqrt{x - 1} + 4)^2
%    \]
%    \[
%    4x + 16 = x - 1 + 8 \sqrt{x - 1} + 16
%    \]
% 3. Simplificamos y resolvemos la ecuación resultante.
% $3x+1=8 \sqrt{x - 1} \to x=13/9, x=5
%  $

% \end{solution}
%     \part[1\half] $\sqrt{2x+3}-\sqrt{x+2}=2$
%     \begin{solution}
%     $2x+3=4+4\sqrt{x+2}+x+2 \to x-3=4\sqrt{x+2} \to x^2-6x+9=16x+32\to x^2-22x-23=0$\\
%     $x=23$ sí, $x=-1$ No
% \end{solution}
%     \part[1\half] $\sqrt[3]{x + 6} = x$ \begin{solution}
% 1. Elevamos ambos lados al cubo:
%    \[
%    x + 6 = x^3
%    \]
% 2. Reorganizamos la ecuación:
%    \[
%    x^3 - x - 6 = 0
%    \]
% 3. Intentamos con divisores de 6 para encontrar posibles raíces.
% $$x^3 - x - 6=\left(x - 2\right) \left(x^{2} + 2 x + 3\right)\to x=2$$
% \end{solution}
%     \part[1\half] $9^x - 6 \cdot {3^{x + 1}} + 81 = 0 $
% \begin{solution}
%     $\rightarrow \left [ 2\right ]$
% \end{solution}
\part[1]  $ 2 \log x - \log (x + 6) = 3 \log 2 $  \begin{solution}  $ \left [ 12\right ] $  \end{solution}
    \part[1] $|2x-4|=|x+2|$ \begin{solution} $\left[ \frac{2}{3}, \  6\right]$\end{solution}
\end{parts}


\question[1] Da una cota del error absoluto y otra del error relativo cometidos al aproximar el número aúreo $\left(\phi\right)$ a $1.62$

        \question[3] Resuelve, justificadamente, los siguientes sistemas de inecuaciones:
        \begin{multicols}{2}
        \begin{parts} 
         \part  $$ \left\{\begin{matrix}{( {x - 1} )^2} - {( {x + 3} )^2} \leq 0\\x - 3( {x - 1} ) \geq 3 \end{matrix}\right. $$  
         \begin{solution}  $ \left[-1, 0\right] $  \end{solution}
         % \part[2]  $$ \left\{\begin{matrix}x \geq 0 \\0 \leq y \leq 3\\x - 2y \leq 10\\x + y \geq 10\\\end{matrix}\right. $$  \begin{solution}    \end{solution}  

         \part  $$ \left\{\begin{matrix}x \geq 0 \\0 \leq y \leq 3 \\x + y \leq 5\\\end{matrix}\right. $$  \begin{solution}    \end{solution} 
        \end{parts}
                  

        
        % \question Calcula expresando el resultado en forma de fracción algebraica irreducible:
        % \begin{multicols}{1}
        % \begin{parts} \part[1]  $ \dfrac{2+{ \dfrac{1}{x}}}{{2 + \dfrac{1}{{1 + \dfrac{1}{x}}}}} $  \begin{solution}  $ \dfrac{2 x^{2} + 3 x + 1}{3 x^{2} + 2 x} $  \end{solution}
        % \end{parts}
        % \end{multicols}
            
        \end{multicols}

% \question[1] Resolver la ecuación
% \[
% 4x^3 - 6x^2 - x = -\frac{3}{2},
% \]
% sabiendo que una solución es \( \frac{1}{2} \).

% \begin{solution}
% 1. Llevamos todos los términos a un lado:
%    \[
%    4x^3 - 6x^2 - x + \frac{3}{2} = 0
%    \]
% 2. Multiplicamos por 2 para eliminar fracciones:
%    \[
%    8x^3 - 12x^2 - 2x + 3 = 0
%    \]
% 3. Factorizando el polinomio 
% $\frac{\left(2 x - 3\right) \left(2 x - 1\right) \left(2 x + 1\right)}{2} \to \left[ - \frac{1}{2}, \  \frac{1}{2}, \  \frac{3}{2}\right]
% $
% \end{solution}

% \question[1] Resolver la ecuación
% \[
% 2x^3 - 3x^2 = -\frac{1}{2},
% \]
% sabiendo que una de sus raíces es \( \frac{1}{2} \).

% \begin{solution}
% 1. Llevamos todos los términos a un lado:
%    \[
%    2x^3 - 3x^2 + \frac{1}{2} = 0
%    \]
% 2. Multiplicamos por 2 para eliminar fracciones:
%    \[
%    4x^3 - 6x^2 + 1 = 0
%    \]
% 3. $$\left(2 x - 1\right) \left(2 x^{2} - 2 x - 1\right) \to \left[ \frac{1}{2}, \  \frac{1}{2} - \frac{\sqrt{3}}{2}, \  \frac{1}{2} + \frac{\sqrt{3}}{2}\right]$$
% \end{solution}

% \question Resuelve la siguiente ecuación:
% $9^x - 6 \cdot {3^{x + 1}} + 81 = 0 $
% \begin{solution}
%     $\rightarrow \left [ 2\right ]$
% \end{solution}


% \question Resuelve los siguientes sistemas:
% \begin{parts}
%     \part[1\half] $\left\{\begin{matrix}{2^x} + {2^y} = 24\\ \\{2^{x + y}} = 128\end{matrix}\right. $

% \begin{solution}
%     $\rightarrow \left [ \left \{ x : 3, \quad y : 4\right \}, \quad \left \{ x : 4, \quad y : 3\right \}\right ]$
% \end{solution}

%     \part[1\half] $\left\{\begin{matrix}\log_{2}(x) + \log_{2}(x + y) = 4\\ \\x + y = 2 \end{matrix}\right. $
% \begin{solution}
%     $\rightarrow \left [ \left \{ x : 8, \quad y : -6\right \}\right ]$
% \end{solution}

% \end{parts}


% \question Resuelve:
% $$\left\{\begin{matrix}3\log x - 2\log y = 10\\\log x + 3\log y = 7\end{matrix}\right. $$

% \begin{solution}
%     $\rightarrow \left [ \left \{ x : 10000, \quad y : 10\right \}\right ]$
% \end{solution}

% \question[2] Resuelve el siguiente sistema de ecuaciones:
% \[
% \begin{cases}
% \dfrac{1}{x} + y = 3 \\ \\
% \dfrac{1}{x} - \dfrac{1}{y} = \dfrac{1}{2}
% \end{cases}
% \]

\begin{solution}
1. Llamamos \(u = \frac{1}{x}\) y \(v = y\), para obtener el sistema:
   \[
   u + v = 3
   \]
   \[
   u - \frac{1}{v} = \frac{1}{2}
   \]
2. Resolvemos este sistema para \(u\) y \(v\), y luego encontramos \(x\) y \(y\) a partir de \(u\) y \(v\).
\end{solution}

% \question Resuelve las siguientes ecuaciones:
% \begin{parts}
%     \part $|2x-4|=|x+2|$ \begin{solution} $\left[ \frac{2}{3}, \  6\right]$\end{solution}
% \end{parts}


% https://lareinadelasciencias.wordpress.com/wp-content/uploads/2014/01/problemas-ecuaciones-sistemas-no-lineales.pdf

%https://matematicasiesoja.wordpress.com/wp-content/uploads/2013/10/problemas-_metodo_gauss2.pdf

\question Resuelve mediante expresiones algebraicas y, en caso que se pueda, por Gauss:
\begin{parts}
    \part[1] Un padre ha comprado un jersey para cada uno de sus cinco hijos, gastándose en total
108,75 euros. Tres de los jerseys tenían un 15\% de descuento, y otro de ellos tenía un 20\%
de descuento. Sabiendo que inicialmente costaban lo mismo, ¿cuánto ha tenido que pagar
por cada jersey? 
\begin{solution}$3\cdot0.85x+0.8x+x=108.75 \to x=25 \to 21.25; 20; 25$\end{solution}

    \part[1] El área de un triángulo es 78 $cm^2$  y entre la base y la altura suman 25 $cm$. Calcula la base y la altura.
% \begin{solution}  $ \left\{\begin{matrix}\frac{xy}{2}=78\\ x+y=25\\ \end{matrix}\right.  \rightarrow  \\\left[\begin{matrix}1 & 1 & 25\\0 & \frac{x}{2} - \frac{y}{2} & - \frac{25 y}{2} + 78\end{matrix}\right] \rightarrow  \left [ \left \{ x : 12, \quad y : 13\right \}, \quad \left \{ x : 13, \quad y : 12\right \}\right ] $  \end{solution}

    \part[1] Miguel y Ana tiene un perro. Averigua el peso de cada uno de los tres sabiendo que Miguel y Ana pesan 50
kg juntos, y Ana y su perro 29 kg y, finalmente, Miguel y el perro 35 kg.  \begin{solution}  $ \left\{\begin{matrix}x+y = 50\\ y+z=29\\ x+z=35\\ \end{matrix}\right.  \rightarrow  \\\left[\begin{matrix}1 & 1 & 0 & 50\\0 & 1 & 1 & 29\\0 & 0 & 2 & 14\end{matrix}\right] \rightarrow  \left \{ x : 28, \quad y : 22, \quad z : 7\right \} $  \end{solution}
  
    \part[2] En una residencia de estudiantes se compran semanalmente 110 helados de distintos
sabores: vainilla, chocolate y nata. El presupuesto destinado para esta compra es de 540
euros y el precio de cada helado es de 4 euros el de vainilla, 5 euros el de chocolate y 6
euros el de nata. Conocidos los gustos de los estudiante, se sabe que entre helados de
chocolate y de nata se han de comprar el 20\% más que de vainilla 
\begin{solution}    
\textbf{Resolución del problema:}

Sea:
\[
x: \text{número de helados de vainilla}, \quad
y: \text{número de helados de chocolate}, \quad
z: \text{número de helados de nata}.
\]

Planteamos las ecuaciones:
\[
x + y + z = 110, \quad 4x + 5y + 6z = 540, \quad y + z = 1.2x.
\]

\textbf{Sistema matricial:}
\[
\begin{bmatrix}
1 & 1 & 1 \\ 
4 & 5 & 6 \\ 
-1.2 & 1 & 1
\end{bmatrix}
\begin{bmatrix}
x \\ 
y \\ 
z
\end{bmatrix}
=
\begin{bmatrix}
110 \\ 
540 \\ 
0
\end{bmatrix}.
\]

\textbf{Eliminación de Gauss:}

Partimos de la matriz ampliada:
\[
\left[
\begin{array}{ccc|c}
1 & 1 & 1 & 110 \\ 
4 & 5 & 6 & 540 \\ 
-1.2 & 1 & 1 & 0
\end{array}
\right].
\]

Tras operaciones elementales:
\[
\left[
\begin{array}{ccc|c}
1 & 1 & 1 & 110 \\ 
0 & 1 & 2 & 100 \\ 
0 & 0 & 1 & 40
\end{array}
\right].
\]

\textbf{Sustitución hacia atrás:}
\[
z = 40, \quad y + 2z = 100 \implies y = 20, \quad x + y + z = 110 \implies x = 50.
\]

\textbf{Solución final:}
\[
x = 50 \quad \text{(vainilla)}, \quad y = 20 \quad \text{(chocolate)}, \quad z = 40 \quad \text{(nata)}.
\]
\end{solution}


\part[2] Una empresa envasadora ha comprado un total de 1500 cajas de pescado en tres mercados 
diferentes, a un precio por caja 30, 20 y 40  euros respectivamente. el coste total de la operación ha sido 
de 40500\euro. Calcular cuánto ha pagado la empresa en cada mercado, sabiendo que en el primero de ellos ha 
comprado el 30\% de las cajas

\begin{solution}
    Llamamos $x$, $y$ y $z$ cajas de 30, 20 y 40\euro \  respectivamente.
    El sistema a resolver es:$\left\{ \begin{matrix}x + y + z = 1500 \\ 30 x + 20 y + 40 z = 40500 \\ x = 450 \\ \end{matrix}\right.$ \\ 
    \textbf{Discusión y resolución por Gauss:} Escalonando la matriz ampliada tenemos\\$A^*= \left(\begin{matrix}1 & 1 & 1 & 1500\\30 & 20 & 40 & 40500\\1 & 0 & 0 & 450\end{matrix}\right) \thicksim \left(\begin{matrix}1 & 1 & 1 & 1500\\0 & -10 & 10 & -4500\\0 & 0 & -2 & -600\end{matrix}\right)$. \\  De los valores de la última fila podemos concluir:\begin{itemize}\item S.C.D.\begin{itemize}\item $\left(\begin{matrix}0 & 0 & -2 & -600\end{matrix}\right) \to z = 300$\end{itemize}\begin{itemize}\item $\left(\begin{matrix}0 & -10 & 10 & -4500\end{matrix}\right) \to y = 750$\end{itemize}\begin{itemize}\item $\left(\begin{matrix}1 & 1 & 1 & 1500\end{matrix}\right) \to x = 450$\end{itemize}\end{itemize}  \textbf{Por rangos y determinantes:} \\$\left|A\right|=\left|\begin{matrix}1 & 1 & 1\\30 & 20 & 40\\1 & 0 & 0\end{matrix}\right|=20  \neq 0 $\begin{itemize} \item $rg(A)=3 \land rg(A^*)=3 \to $ S.C.D.  \\ \\ Por Cramer: \begin{itemize}\item $x=\frac{\left|\begin{matrix}1500 & 1 & 1\\40500 & 20 & 40\\450 & 0 & 0\end{matrix}\right|}{20}=\frac{9000}{20}=450$\item $y=\frac{\left|\begin{matrix}1 & 1500 & 1\\30 & 40500 & 40\\1 & 450 & 0\end{matrix}\right|}{20}=\frac{15000}{20}=750$\item $z=\frac{\left|\begin{matrix}1 & 1 & 1500\\30 & 20 & 40500\\1 & 0 & 450\end{matrix}\right|}{20}=\frac{6000}{20}=300$\end{itemize}\end{itemize}
    \textbf{SOLUCIÓN: } 13500 (450x30), 15000 y 12000 \euro.
\end{solution}

    
%     \part[1] En una reunión hay 22 personas, entre hombres, mujeres y niños. El doble del
% número de mujeres más el triple del número de niños, es igual al doble del número de 
% hombres. 
% \begin{solution}
%     Sea:
% \[
% x: \text{número de hombres}, \quad
% y: \text{número de mujeres}, \quad
% z: \text{número de niños}.
% \]

% Planteamos las ecuaciones:
% \[
% x + y + z = 22, \quad 2y + 3z = 2x.
% \]

% \textbf{Sistema matricial:}
% \[
% \begin{bmatrix}
% 1 & 1 & 1 \\ 
% -2 & 2 & 3
% \end{bmatrix}
% \begin{bmatrix}
% x \\ 
% y \\ 
% z
% \end{bmatrix}
% =
% \begin{bmatrix}
% 22 \\ 
% 0
% \end{bmatrix}.
% \]

% \textbf{Eliminación de Gauss:}

% Partimos de la matriz ampliada:
% \[
% \left[
% \begin{array}{ccc|c}
% 1 & 1 & 1 & 22 \\ 
% -2 & 2 & 3 & 0
% \end{array}
% \right].
% \]

% Operaciones elementales eliminan el coeficiente de \(x\) en la segunda fila:
% \[
% \left[
% \begin{array}{ccc|c}
% 1 & 1 & 1 & 22 \\ 
% 0 & 4 & 5 & 44
% \end{array}
% \right].
% \]

% Simplificamos la segunda fila:
% \[
% \left[
% \begin{array}{ccc|c}
% 1 & 1 & 1 & 22 \\ 
% 0 & 1 & \frac{5}{4} & 11
% \end{array}
% \right].
% \]

% \textbf{Sustitución hacia atrás:}

% De la segunda fila:
% \[
% y + \frac{5}{4}z = 11 \quad \implies \quad y = 11 - \frac{5}{4}z.
% \]

% De la primera fila:
% \[
% x + y + z = 22 \quad \implies \quad x = 22 - y - z.
% \]

% Sustituyendo \(y\):
% \[
% x = 22 - \left(11 - \frac{5}{4}z\right) - z \quad \implies \quad x = 11 - \frac{1}{4}z.
% \]

% Para que \(x, y, z\) sean números enteros, \(z = 4\):
% \[
% y = 11 - \frac{5}{4}(4) = 6, \quad x = 11 - \frac{1}{4}(4) = 10.
% \]

% \textbf{Solución final:}
% \[
% x = 10 \quad \text{(hombres)}, \quad y = 6 \quad \text{(mujeres)}, \quad z = 4 \quad \text{(niños)}.
% \]    
% \end{solution}


\end{parts}


\question Resuelve los siguientes sistemas de ecuaciones no lineales

\begin{parts}

\part[1] 
\[
\begin{cases}
2x - y = 1 \\
3x - 2y^2 = 1
\end{cases}
\]

\begin{solution}
$x = \frac{5}{4}, \, y = \frac{3}{2}$ \quad y \quad $x = -\frac{1}{4}, \, y = \frac{1}{2}$
\end{solution}

\part[1] 
\[
\begin{cases}
x^2 - 2y^2 = 1 \\
x \cdot y = 6
\end{cases}
\]

\begin{solution}
$x = 3, \, y = 2$ \quad y \quad $x = -3, \, y = -2$
\end{solution}

\part[1] 
\[
\begin{cases}
\sqrt{x + y} + y = 1 \\
2x + y = 2
\end{cases}
\]

\begin{solution}
$x = 1, \, y = 0$
\end{solution}

\part[1] 
\[
\begin{cases}
2x^2 + 3y^2 = 21 \\
-5x^2 + 2y^2 = -43
\end{cases}
\]

\begin{solution}
$x = \pm 2, \, y = \pm 3$
\end{solution}

\part[1] 
\[
\begin{cases}
\frac{y}{x} - \frac{3}{y} = \frac{1}{2} \\
x - 3y = -5
\end{cases}
\]

\begin{solution}
$x = 1, \, y = 2$
\end{solution}

\end{parts}

\question[2] Resuelve por Gauss el siguiente sistema de ecuaciones:
\[
\begin{cases}
3x + 2y - 2z = 4 \\
4x + y - z = 7 \\
x + 4y - 4z = 0
\end{cases}
\]

\begin{solution}
$\left[\begin{matrix}3 & 2 & -2 & 4\\0 & - \frac{5}{3} & \frac{5}{3} & \frac{5}{3}\\0 & 0 & 0 & 2\end{matrix}\right]$ Luego S.I.
\end{solution}



\question Resuelve las siguientes inecuaciones:
\begin{parts}
            \part[1]  $ \dfrac{{{x^3} - 5{x^2} + 2x + 8}}{1-x^2} \leq 0 $  \begin{solution}  $  \left(1, 2\right] \cup \left[4, \infty\right) $  \end{solution}
        \part[1] $1-\dfrac{{x + 3}}{{x + 6}} \geq 0 $ \begin{solution} $\rightarrow \left(-6, \infty\right)$\end{solution}
\end{parts}



        

\addpoints

% Preguntas extraídas de: 03_2425_par2.tex
\question Resuelve las siguientes ecuaciones trigonométricas:
\begin{parts}
    % \part[1]$\cos^2{x}-\sen^2{x}=0$
    \part[1]$\sqrt{2}\sen{x}=1$
\end{parts}

\question Dado $\alpha \in III$ y $\cos{\alpha}=-\frac{1}{2}$, hallar, utilizando la fórmula correspondiente (resultados simplificados y racionalizados; no vale utilizar decimales):
\begin{parts}
    \part[1] $\sen{2\alpha}$
    % \part[1] $\cos{\frac{\alpha}{2}}$
\end{parts}

\question Dado $z=\sqrt{2}-\sqrt{2}i$
\begin{parts}
    \part[1] calcula el opuesto y el conjugado y represéntalos en el plano complejo junto con $z$
    \part[1] calcular 
 $z^5\cdot\overline{z}$
 \begin{solution}
     $64$
 \end{solution}
    
\end{parts}

 \question Resolver las siguientes ecuaciones: (soluciones complejas incluidas)
 \begin{parts}
     % \part[2] $z^5=1$ \begin{solution}\end{solution}
     \part[2] $x^4-16=0$ \begin{solution}$\pm 2, \pm 2i$\end{solution}
 \end{parts}


    \question[2] Desde un determinado punto se ve la parte superior de una antena de comunicaciones bajo un ángulo de $50^\circ$ con respecto a la horizontal. Nos alejamos 10 m y la visual forma ahora $40^\circ$. Queremos saber la distancia de ambos puntos de observación a la cúspide de la antena, con el fin de fijar sendos cables de sujeción.
\begin{solution}
    \begin{tikzpicture}[scale=1.5, every node/.style={font=\small}]

    % Ejes y antena
    \draw[thick] (0,0) -- (0,5) node[above]{Cúspide de la antena};
    \draw[dashed] (0,0) -- (6,0) node[below right]{Base de la antena};
    
    % Puntos de observación
    \draw[fill] (2,0) circle (0.05) node[below]{Primer punto};
    \draw[fill] (4,0) circle (0.05) node[below]{Segundo punto};
    
    % Líneas visuales
    \draw[thick] (2,0) -- (0,5) node[midway,left]{Visual $50^\circ$};
    \draw[thick] (4,0) -- (0,5) node[midway,left]{Visual $40^\circ$};
    
    % Ángulos
    \draw[-] (2,0) ++(-0.4,0) arc[start angle=180,end angle=130,radius=0.4] node[midway,left]{$50^\circ$};
    \draw[-] (4,0) ++(-0.4,0) arc[start angle=180,end angle=140,radius=0.4] node[midway,left]{$40^\circ$};
    
    % Distancias
    \draw[-,dashed] (2,-0.5) -- (4,-0.5) node[midway,below]{$10 \, \text{m}$};
    
    \end{tikzpicture}

    En el triángulo pequeño de la derecha se cumple que los ángulos son $180-50=130^\circ$ y $180-130-40=10^\circ$. Aplicando el teorema del seno:
    $$\dfrac{10}{\sin{10}}=\dfrac{cable\ corto}{\sin{40}} \to cable\ corto=\dfrac{10\cdot\sin{40}}{\sin{10}}\approx 37.0166631359329
    $$
    Por el teorema del coseno:
    $$cable\ largo^2=10^2 + cable\ corto^2 - 2\cdot 10\cdot cable \ corto \cdot \cos{130^\circ}$$
    
    $$cable \ largo =\sqrt{10^2 + cable\ corto^2 - 2\cdot 10\cdot cable \ corto \cdot \cos{130^\circ}}\approx 44.1147412780977$$

\end{solution}

\addpoints

% Preguntas extraídas de: 04_2425_final2.tex
\question Dados los vectores $\vec{u}=(\frac{1}{2},-1)$ y $\vec{v}=(0,-2)$ calcula:
\begin{parts}
    \part[1] $|\vec{u}|$
    \part[1] $2\vec{u}-3\vec{v}$
    \part[1] $\vec{u}\cdot \vec{v}$
\end{parts}

% \question[1] Expresa
% $\vec{a} =(9,5)$
% como combinación lineal de
%  $\vec{u}=(1,3)$ y $\vec{v} =(3,-2)$
% \begin{solution}
%     $3\vec{u}+2\vec{v}$
% \end{solution}

\question[1] Determinar el ángulo formado por las rectas: $ r\equiv2x-y-2=0\  y \ s\equiv3x+2y-4=0 $  \begin{solution}  $ 119.74488129694222 $  \end{solution}

\question Dada la recta $r\equiv x+y-3=0$ y el punto $P(-1,2)$, se pide:
\begin{parts}
    \part[1] Halla la ecuación general de la recta perpendicular a r que pasa por P
    \begin{solution}
        $x-y+3=0$
    \end{solution}
    \part[1] Calcula el punto de corte de la recta anterior y la recta r
    \begin{solution}
        $(0,3)$
    \end{solution}
    \part[1] Calcula el punto simétrico de P respecto del calculado en el apartado anterior
    \begin{solution}
        $(1,4)$
    \end{solution}
    \part[1]  Halla la ecuación general de la recta paralela a r que pasa por P
    \begin{solution}
        $x+y-1=0$
    \end{solution}
\end{parts}

\question[2]  Comprueba analíticamente que los puntos A(-3,2), B(1,9) y C(4,-2) forman un triángulo rectángulo en A.
\begin{solution}
    $\vec{AB}=(4,7)$ , $\vec{AC}=(7,-4)$ y $\vec{AB}\cdot\vec{AC}=0$
\end{solution}

% \question[2] Calcula el vértice C de un triángulo isósceles (dos lados iguales) ABC, sabiendo que: 
% $A(2, -3)$, $\ B(5, 2)$ y $C \in r\equiv y=4 $  \begin{solution}  $ \left [ \left \{ x : -4, \quad y : 4\right \}\right ] $  \end{solution}

% \question[2] Calcula el vértice C de un triángulo isósceles (dos lados iguales) ABC, sabiendo que: 
% $A(2, -3)$, $\ B(5, 2)$ y $C \in r\equiv -x+3y-16=0 $  \begin{solution}  $ \left [ \left \{ x : -4, \quad y : 4\right \}\right ] $  \end{solution}

% \question[1] Calcula k para que el triángulo ABC sea isósceles, siendo $A(2, -3)\ , \ B=(5, 2) \ y \ C(k,4)$  \begin{solution}  $ \left [ \left \{ x : -4, \quad y : 4\right \}\right ] $  \end{solution}

% \question[2] Calcula, sin usar la trigonometría, el área del triángulo de vértices $A\left(-1, 3\right)$, $B\left(6, 5\right)$, $C\left(2, 1\right)$.
% \begin{solution}
% $\mathtt{\text{altura}} : \frac{5 \sqrt{2}}{2}, \  \mathtt{\text{area}} : 10, \  \mathtt{\text{base}} : 4 \sqrt{2}$
% \end{solution}


\question[2] Calcula $(2+5i) (3+4i)$. Da el resultado en binómico y en polar
\begin{solution}
    $-14+23i$
\end{solution}

% Preguntas extraídas de: 05_2425_par3.tex
\question[1\half] Calcula los límites de la siguiente función:

\begin{tikzpicture}[scale=0.6]
    % Configuración de ejes
    \draw[-] (-8, 0) -- (8, 0) node[right] {$X$}; % Eje X
    \draw[-] (0, -2) -- (0, 8) node[above] {$Y$}; % Eje Y

    % Cuadrícula
    \draw[help lines, color=gray!20] (-8, -2) grid (8, 8);

    % Etiquetas de cuadrícula
    \foreach \x in {-8, -6, -4, -2, 2, 4, 6, 8}
        \draw (\x, 0.1) -- (\x, -0.1) node[below] {\x};
    \foreach \y in {-2, 2, 4, 6, 8}
        \draw (0.1, \y) -- (-0.1, \y) node[left] {\y};

    % Primer trozo: Parábola que termina en (-4, 4)
    \draw[thick, red, domain=-8:-4, samples=50] plot (\x, {4+(0.25*(\x+4)*(\x+4))}) node[above] {};
    \filldraw[red] (-4, 4) circle (2pt); % Punto final en (-4, 4)

    % Segundo trozo: Recta de (-4, 1) a (-2, 3)
    \draw[thick, red] (-4, 1) -- (-2, 3);
    \filldraw[fill=white, draw=red] (-4, 1) circle (2pt); % Punto inicial
    % \filldraw[fill=white, draw=red] (-2, 3) circle (2pt); % Punto final

    % Tercer trozo: Recta de (-2, 3) a (1, 0)
    \draw[thick, red] (-2, 3) -- (1, 0);
    % \draw[red] (1, 0) circle (2pt); % Punto final en (1, 0)

    \filldraw[fill=white, draw=red] (-2, 3) circle (2pt); % Punto final
    \filldraw[red] (-2, 1) circle (2pt); % Punto final

    % Cuarto trozo: Proporcionalidad inversa creciente con mayor excentricidad
    \draw[thick, red, domain=1:2.883, samples=50] plot (\x, {-1*(\x-1)/(2*(\x-3))}) node[above] {};
    \draw[dashed] (3, 0) -- (3, 8); % Asíntota vertical en x = 3
    % \filldraw[red] (1, 0) circle (2pt); % Punto inicial en (1, 0)

    % Quinto trozo: Exponencial que pasa por (3, 5) y tiene asíntota horizontal y = 3
    \draw[thick, red, domain=3:8, samples=50] plot (\x, {3 + 2*exp(-0.5*(\x-3))});
    \draw[dashed] (0, 3) -- (8, 3); % Asíntota horizontal en y = 3
    \filldraw[fill=white, draw=red] (3, 5) circle (2pt); % Punto en (3, 5)

\end{tikzpicture}
\begin{multicols}{3}
\begin{parts}
    \part $\lim_{x \to -\infty} f(x)$
    \part $\lim_{x \to -4^-} f(x)$
    \part $\lim_{x \to -4^+} f(x)$
    \part $\lim_{x \to -4} f(x)$
    \part $\lim_{x \to -2} f(x)$
    \part $\lim_{x \to 1} f(x)$
    \part $\lim_{x \to -3^-} f(x)$
    \part $\lim_{x \to -3^+} f(x)$
    \part $\lim_{x \to +\infty} f(x)$
\end{parts}
\end{multicols}

        \question Determina el dominio de definición de las siguientes funciones:
        \begin{parts}
        % reduce_inequalities([(x**2-x)/(x+2)>=0]).as_set()
        \part[1] $f(x)=\sqrt{\dfrac{x^2 -x}{x+2}}$\begin{solution}  $\left(-2, 0\right] \cup \left[1, \infty\right)$\end{solution}
        \part[1] $f(x)=\dfrac{3x+2}{x^4-5x^2-36}$\begin{solution}$\left(-\infty, -3\right] \cup \left[3, \infty\right)$\end{solution}
        \end{parts}
        
        \question Dada la función $f(x)=\dfrac{3 x - 2}{2}$:
                \begin{parts} \part[1] Calcula $f^{-1}(x)$, es decir la inversa de $f(x)$  \begin{solution}   $f^{-1}(x)=\frac{2 x}{3} + \frac{2}{3}$ \\ $f^{-1} \circ f(x)=x=x$ \\ 
               \end{solution}
               \part[1] Comprueba que efectivamente son inversas
        % \part[1] $f(x)=\frac{x}{- x + 1}$  \begin{solution}   $f^{-1}(x)=\frac{x}{x + 1}$ \\ $f^{-1} \circ f(x)=\frac{x}{\left(- x + 1\right) \left(\frac{x}{- x + 1} + 1\right)}=x$   \end{solution}
        \end{parts}
        
        
        \question Calcula los siguientes límites:

        \begin{parts} 
        \part[1] $$\lim_{x \to -1}\left(x^{2} - 3\right)$$  \begin{solution}   $\lim_{x \to -1^-}\left(x^{2} - 3\right)=-2$ y  \\ $\lim_{x \to -1^+}\left(x^{2} - 3\right)=-2$   \end{solution} 
        \part[1] $$\lim_{x \to 2}\left(\frac{x^{3} - 2 x^{2} + 2 x - 4}{3 x^{2} - 8 x + 4}\right)$$  \begin{solution}   $\frac{3}{2}$   \end{solution} 
        % \part[1] $$\lim_{x \to -\infty} e^{x - 1}$$  \begin{solution}   $0$   \end{solution} 
        \part[1] $$\lim_{x \to -1}\left(\frac{x^{3} + 1}{x^{2} + 2 x + 1}\right)$$  \begin{solution}   No existe el límite   \end{solution} 
        % \part[2] $$\lim_{x \to 3} \left(\frac{x^{2} - x}{x + 3}\right)^{\frac{1}{x - 3}}$$  \begin{solution}   $e^{\frac{2}{3}}$   \end{solution}
        \part[2] $$\lim_{x \to \infty}\left(\dfrac{x^{2} - 1}{x} - \dfrac{2 x^{2} + 1}{2 x - 1}\right)$$\begin{solution}$-\frac{1}{2}$\end{solution}
        
        \end{parts}
        
        
        
%        \question Dada la función:$f(x)=\frac{x^{2} - 2 x + 1}{2 x + 3}$, calcular:
%        \begin{multicols}{1}
%        \begin{parts} \part[1] Dominio de $f(x)$  \begin{solution}   $Dom(f)=\left(-\infty, - \frac{3}{2}\right) \cup \left(- \frac{3}{2}, \infty\right)$\\ \resizebox{0.4\textwidth}{!}{\includegraphics[width=1\columnwidth]{fin301-0}}   \end{solution} \part[2] Asíntotas verticales, horizontales y oblicuas, en caso que existan  \begin{solution}   Asíntotas:\\A.V. $x=-3/2$\\A.O. $y=\frac{x}{2} - \frac{7}{4}$ \\A.O. $y=\frac{x}{2} - \frac{7}{4}$ \\   \end{solution}
%        \end{parts}
%        \end{multicols}
%        \question Dada la función:$f(x)=\frac{- x^{2} - x + 3}{x^{2} + x - 2}$, calcular:
%        \begin{multicols}{1}
%        \begin{parts} \part[1] Dominio de $f(x)$  \begin{solution}   $Dom(f)=\left(-\infty, -2\right) \cup \left(-2, 1\right) \cup \left(1, \infty\right)$\\ \resizebox{0.4\textwidth}{!}{\includegraphics[width=1\columnwidth]{fin301-1}}   \end{solution} \part[2] Asíntotas verticales, horizontales y oblicuas, en caso que existan  \begin{solution}   Asíntotas:\\A.V. $x=-2$\\, A.V. $x=1$\\A.H. $y=-1$\\A.H. $y=-1$\\A.O. $y=-1$ \\A.O. $y=-1$ \\   \end{solution}
%        \end{parts}
%        \end{multicols}


        % \question Dada la función:$f(x)=\sqrt{\frac{x}{x - 1}}$, calcular:
        % \begin{multicols}{1}
        % \begin{parts} \part[1] Dominio de $f(x)$  \begin{solution}   $Dom(f)=\left(-\infty, 0\right] \cup \left(1, \infty\right)$\\ \resizebox{0.4\textwidth}{!}{\includegraphics[width=1\columnwidth]{fin301-2}}   \end{solution} \part[2] Asíntotas verticales, horizontales y oblicuas, en caso que existan  \begin{solution}   Asíntotas:\\A.V. $x=1$\\A.H. $y=1$\\A.H. $y=1$\\A.O. $y=1$ \\A.O. $y=1$ \\   \end{solution}
        % \end{parts}
        % \end{multicols}

       \question Estudia en qué puntos de $\mathbb{R}$ la función no es continua: 

       \begin{parts} \part[2] $f(x)=\begin{cases} \dfrac{2 x^{2} + 7 x}{x^{2} - 9} & \text{si}\: x \leq -2 \\\dfrac{x+2}{x^{2} + 2 x} & \text{si} \: x > -2\end{cases}$  \begin{solution}   Singularidades de las expresiones analíticas: $\left\{-3, 0\right\}$.\\ Posibles discontinuidades en los extremos de los trozos:-2.\\En -2 no es continua porque no existe límite. Límites laterales: $\frac{6}{5}$ y $- \frac{1}{2}$   \end{solution}
       \part  Indica, razonadamente, los tipos de discontinuidad que hay la función anterior
       \begin{solution}
       En $x=-3$ salto infinito \end{solution}
       \end{parts}
       

       \question Estudia en qué puntos de $\mathbb{R}$ la función no es continua: 

       \begin{parts} \part[2] $f(x)=\begin{cases} \dfrac{2 x^{2} + 7 x + 3}{x^{2} - 9} & \text{si}\: x \leq -2 \\\dfrac{\sqrt{x + 3} - 1}{x^{2} + 2 x} & \text{si} \: x > -2\end{cases}$  \begin{solution}   Singularidades de las expresiones analíticas: $\left\{-3, 0\right\}$.\\ Posibles discontinuidades en los extremos de los trozos:-2.\\En -2 no es continua porque no existe límite. Límites laterales: $\frac{3}{5}$ y $- \frac{1}{4}$   \end{solution}
       \part \textbf{(2 puntos extra)} Indica, razonadamente, los tipos de discontinuidad que hay
       \end{parts}

        
        
        \question Estudia en qué puntos de $\mathbb{R}$ la función no es continua: 
        \begin{multicols}{1}
        \begin{parts} \part[2] $f(x)=\begin{cases} \frac{x^{2} - 4}{x^{2} - 3 x + 2} & \text{si}\: x < 2 \\4 & \text{si}\: 2\leq x < 5 \\e^{x - 5} + 3 & \text{si}\: x > 5 \end{cases}$  \begin{solution}   Singularidades de las expresiones analíticas: $\left\{1\right\}$.\\ Posibles discontinuidades en los extremos de los trozos:2, 5.\\En 2 es continua ya que hay límite y $\lim = f(2)=4$. \\En 5 es continua ya que hay límite y $\lim = f(5)=4$   \end{solution}
        \end{parts}
        \end{multicols}
        
       \question Halla a y b de modo que las siguientes funciones sean continuas:
       \begin{multicols}{1}
       \begin{parts} \part[2] $$f(x)=\begin{cases} a + e^{x + 2} & \text{si}\: x \leq -2 \\\frac{x + 1}{3 - x} & \text{si}\: -2 < x < 1 \\b x + 3 & \text{si}\: x \geq 1 \end{cases}$$  \begin{solution}   $\left\{ a : - \frac{6}{5}, \  b : -2\right\}$   \end{solution}
       \end{parts}
       \end{multicols}
        
        \question Halla a y b de modo que las siguientes funciones sean continuas:
        \begin{multicols}{1}
        \begin{parts} \part[2] $$f(x)=\begin{cases} a + e^{x + 3} & \text{si}\: x \leq -3 \\\frac{x + 2}{4 - x} & \text{si}\: -3 < x < 1 \\b x + 6 & \text{si}\: x \geq 1 \end{cases}$$  \begin{solution}   $\left\{ a : - \frac{8}{7}, \  b : -5\right\}$   \end{solution}
        \end{parts}
        \end{multicols}
        
        
        
       \question Deriva las siguientes funciones (simplificando el resultado al máximo):
       \begin{multicols}{1}
       \begin{parts} \part[1] $y=\frac{3 x^{2} - 2 x + 1}{\left(x - 1\right)^{2}}$  \begin{solution}   $y'=- \frac{4 x}{x^{3} - 3 x^{2} + 3 x - 1}$   \end{solution} \part[1] $y=\sqrt{\sqrt{x} + 1}$  \begin{solution}   $y'=\frac{1}{4 \sqrt{x} \sqrt{\sqrt{x} + 1}}$   \end{solution} \part[1] $y=\frac{\log{\left(x^{2} \right)}}{x}$  \begin{solution}   $y'=\frac{2 - \log{\left(x^{2} \right)}}{x^{2}}$   \end{solution} \part[1] $y=3 \sin{\left(\cos{\left(2 x \right)} \right)}$  \begin{solution}   $y'=- 6 \sin{\left(2 x \right)} \cos{\left(\cos{\left(2 x \right)} \right)}$   \end{solution}
       \end{parts}
       \end{multicols}
        
        \question Deriva las siguientes funciones (simplificando el resultado al máximo):
        \begin{multicols}{1}
        \begin{parts} \part[1] $y=\frac{2 x^{2} - 2 x + 1}{\left(x - 1\right)^{2}}$  \begin{solution}   $y'=- \frac{2 x}{x^{3} - 3 x^{2} + 3 x - 1}$   \end{solution} \part[1] $y=\sqrt{2 - \sqrt{x}}$  \begin{solution}   $y'=- \frac{1}{4 \sqrt{x} \sqrt{2 - \sqrt{x}}}$   \end{solution} \part[1] $y=\frac{\log{\left(x \right)}}{x}$  \begin{solution}   $y'=\frac{1 - \log{\left(x \right)}}{x^{2}}$   \end{solution} \part[1] $y=2 \cos{\left(\sin{\left(2 x \right)} \right)}$  \begin{solution}   $y'=- 4 \sin{\left(\sin{\left(2 x \right)} \right)} \cos{\left(2 x \right)}$   \end{solution}
        \end{parts}
        \end{multicols}


\addpoints

% Preguntas extraídas de: 06_2425_final3.tex
% \question[1\half] Calcula los límites de la siguiente función:

% \begin{tikzpicture}[scale=0.4]
%     % Configuración de ejes
%     \draw[thick] (-8, 0) -- (8, 0) node[right] {$X$}; % Eje X
%     \draw[thick] (0, -2) -- (0, 8) node[above] {$Y$}; % Eje Y

%     % Cuadrícula
%     \draw[help lines, color=gray!20] (-8, -2) grid (8, 8);

%     % Etiquetas de cuadrícula
%     \foreach \x in {-8, -6, -4, -2, 2, 4, 6, 8}
%         \draw (\x, 0.1) -- (\x, -0.1) node[below] {\x};
%     \foreach \y in {-2, 2, 4, 6, 8}
%         \draw (0.1, \y) -- (-0.1, \y) node[left] {\y};

%     % Primer trozo: Parábola que termina en (-4, 4)
%     \draw[thick, red, domain=-8:-4, samples=50] plot (\x, {4+(0.25*(\x+4)*(\x+4))}) node[above] {};
%     \filldraw[red] (-4, 4) circle (2pt); % Punto final en (-4, 4)

%     % Segundo trozo: Recta de (-4, 1) a (-2, 3)
%     \draw[thick, red] (-4, 1) -- (-2, 3);
%     \filldraw[fill=white, draw=red] (-4, 1) circle (2pt); % Punto inicial
%     % \filldraw[fill=white, draw=red] (-2, 3) circle (2pt); % Punto final

%     % Tercer trozo: Recta de (-2, 3) a (1, 0)
%     \draw[thick, red] (-2, 3) -- (1, 0);
%     % \draw[red] (1, 0) circle (2pt); % Punto final en (1, 0)

%     \filldraw[fill=white, draw=red] (-2, 3) circle (2pt); % Punto final
%     \filldraw[red] (-2, 1) circle (2pt); % Punto final

%     % Cuarto trozo: Proporcionalidad inversa creciente con mayor excentricidad
%     \draw[thick, red, domain=1:2.883, samples=50] plot (\x, {-1*(\x-1)/(2*(\x-3))}) node[above] {};
%     \draw[dashed] (3, 0) -- (3, 8); % Asíntota vertical en x = 3
%     % \filldraw[red] (1, 0) circle (2pt); % Punto inicial en (1, 0)

%     % Quinto trozo: Exponencial que pasa por (3, 5) y tiene asíntota horizontal y = 3
%     \draw[thick, red, domain=3:8, samples=50] plot (\x, {3 + 2*exp(-0.5*(\x-3))});
%     \draw[dashed] (0, 3) -- (8, 3); % Asíntota horizontal en y = 3
%     \filldraw[fill=white, draw=red] (3, 5) circle (2pt); % Punto en (3, 5)

% \end{tikzpicture}
% \begin{multicols}{3}
% \begin{parts}
%     \part $\lim_{x \to -\infty} f(x)$
%     \part $\lim_{x \to -4^-} f(x)$
%     \part $\lim_{x \to -4^+} f(x)$
%     \part $\lim_{x \to -4} f(x)$
%     \part $\lim_{x \to -2} f(x)$
%     \part $\lim_{x \to 1} f(x)$
%     \part $\lim_{x \to -3^-} f(x)$
%     \part $\lim_{x \to -3^+} f(x)$
%     \part $\lim_{x \to +\infty} f(x)$
% \end{parts}
% \end{multicols}

\question[2] Calcula los siguientes límites:
\begin{multicols}{2}
\begin{parts}
    \part $$\lim_{x \to 2}\dfrac{2 x^{3} - 4 x^{2} - x + 2}{x^2-5x+6}$$

    \begin{solution}
    Primero factorizamos el denominador:
    \[
    x^2 - 5x + 6 = (x - 2)(x - 3)
    \]

    Ahora intentamos factorizar el numerador:
    \[
    2x^3 - 4x^2 - x + 2
    \]
    Agrupamos:
    \[
    (2x^3 - 4x^2) + (-x + 2) = 2x^2(x - 2) -1(x - 2)
    \]
    Sacamos factor común:
    \[
    = (x - 2)(2x^2 - 1)
    \]

    Sustituimos ambas factorizaciones en la fracción:
    \[
    \frac{(x - 2)(2x^2 - 1)}{(x - 2)(x - 3)}
    \]

    Simplificamos el factor común \( (x - 2) \):
    \[
    \frac{2x^2 - 1}{x - 3}
    \]

    Ahora evaluamos el límite:
    \[
    \lim_{x \to 2} \frac{2x^2 - 1}{x - 3} = \frac{2(2)^2 - 1}{2 - 3} = \frac{8 - 1}{-1} = \frac{7}{-1} = -7
    \]

    \textbf{Resultado final:} \fbox{\( \lim_{x \to 2} \frac{2x^3 - 4x^2 - x + 2}{x^2 - 5x + 6} = -7 \)}
    \end{solution}

    \part {$$\lim_{x \to \infty} \dfrac{2x^3 - 4x^2 - x + 2}{5x^3 - 25x^2 + 30x}$$}

    \begin{solution}
    Dividimos numerador y denominador entre $x^3$:
    
    \[
    \lim_{x \to \infty} \frac{2x^3 - 4x^2 - x + 2}{5x^3 - 25x^2 + 30x}
    = \lim_{x \to \infty} \frac{2 - \frac{4}{x} - \frac{1}{x^2} + \frac{2}{x^3}}{5 - \frac{25}{x} + \frac{30}{x^2}}
    \]
    
    Al tomar el límite cuando $x \to \infty$, los términos con $1/x$, $1/x^2$ y $1/x^3$ tienden a cero:
    
    \[
    = \frac{2}{5}
    \]
    \end{solution}
\end{parts}
    
\end{multicols}

\question[2] Calcula las derivadas de las siguientes funciones dando el resultado simplificado:
\begin{multicols}{2}
\begin{parts}
    \part $f(x)=(x^2 +x)(x^2 -x ) $
    \begin{solution}
        $f'(x)=4x^3-2x$
    \end{solution}
  % \part $f(x)=\dfrac{2x}{x^2+x+1}$
  % \begin{solution}
  %     $f'(x)=\dfrac{2-2x^2}{(x^2+x+1)^2}$
  % \end{solution}
  \part $f(x)=\ln\frac{1-x}{1+x}$
  \begin{solution}
      $f'(x)=\frac{2}{x^{2} - 1}$
  \end{solution}
\end{parts}
\end{multicols}

  \question  Dada la función: $$f(x)=\dfrac{x^2}{x-1}$$ 
\begin{parts}
    \part[0\half] Estudia la continuidad de la función
    \begin{solution}
        f es continua en todo su dominio. $Dom(f)=\mathrm{R}-\{1\}$
    \end{solution}
    \part[1] Estudia el crecimiento y decrecimiento de la función.
    
    \begin{solution}
    La derivada de la función es:
    \[
    f'(x) = \frac{x(x - 2)}{(x - 1)^2}
    \]
    El signo de \( f'(x) \) depende del numerador \( x(x - 2) \), ya que el denominador \( (x - 1)^2 \) siempre es positivo.

    - \( f'(x) > 0 \) cuando \( x < 0 \) o \( x > 2 \), lo que indica que la función es **creciente** en los intervalos \( (-\infty, 0) \cup (2, \infty) \).
    - \( f'(x) < 0 \) cuando \( 0 < x < 2 \), lo que indica que la función es **decreciente** en el intervalo \( (0, 2) \).
    \end{solution}
    
    \part[1] Encuentra los extremos relativos.


    \begin{solution}
    Los puntos críticos se encuentran donde \( f'(x) = 0 \), es decir, cuando \( x(x - 2) = 0 \). Esto da los puntos \( x = 0 \) y \( x = 2 \).
    
    - En \( x = 0 \), \( f'(x) \) cambia de positivo a negativo, por lo que hay un **máximo relativo** en \( (0, 0) \).
    - En \( x = 2 \), \( f'(x) \) cambia de negativo a positivo, por lo que hay un **mínimo relativo** en \( (2, 4) \).
    
    Los valores de la función en estos puntos son:
    - \( f(0) = 0 \)
    - \( f(2) = 4 \)
    \end{solution}
        \part[0\half] Encuentra los extremos absolutos.
        \begin{solution}
            $-\infty$, $+\infty$
        \end{solution}
\end{parts}

\question La obsolescencia tecnológica implica una disminución del valor de un producto con el tiempo. En cierto dispositivo, el valor $V(t)>0$, viene dado por la expresión $V(t)=200-\dfrac{100t}{10+2t}$ \euro, siendo t los años transcurridos desde la compra del dispositivo.
\begin{parts}
    \part[1] Calcule el valor inicial del producto y su valor en un horizonte infinito de tiempo
    \begin{solution}
        $V(0)=100$ y $\lim_{x \to \infty}V=200-\frac{100}{2}=150$
    \end{solution}
    \part[1] Calcule $V'(t)$ y justifique que $V(t )$ es decreciente. Utilice esta conclusión y los resultados del apartado anterior para argumentar que no será posible que el valor de $V(t)$ sea igual a 125\euro.
    \begin{solution}
        $V'(t)=- \frac{250}{\left(x + 5\right)^{2}}$ luego $V'(t)<0 \ 
\forall t \to$ V es decreciente
    \end{solution}
    \part[0\half] ¿Cuánto tiempo tiene que pasar para que el dispositivo tenga un valor de 175 \euro?
    \begin{solution}
        $175=200-\dfrac{100t}{10+2t} \to t=5$. 5 años
    \end{solution}
\end{parts}

\end{questions}
\end{document}
