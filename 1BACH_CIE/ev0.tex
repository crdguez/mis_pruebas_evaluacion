\documentclass{exam}
\usepackage{amsmath, amsthm, amssymb, yhmath} 
\printanswers

\usepackage{tikz}
\usetikzlibrary{angles,quotes} % Para los ángulos

\begin{document}

\begin{center}
\bfseries Prueba inicial
\end{center}
\textbf{Nombre:} \\
\textbf{Fecha:} \\
\textbf{Curso:} \\
\hline

\begin{questions}

\question Indica a qué conjuntos ($\mathbb{N}$, $\mathbb{Z}$,$\mathbb{Q}$, $\mathbb{I}$, $\mathbb{R}$) pertenecen los siguientes números: $\dfrac{4}{5}$,$ \dfrac{10}{5}$, $-2,\wideparen{3}$, $\sqrt{7}$, $\sqrt{36}$, $\dfrac{\pi}{3}$
        \vspace{70pt}

\question Escribe en forma de intervalos los siguientes conjuntos:
\begin{parts}
    \part $\{x\in \mathbb{R}| \mid x    +2 \mid \geq 3\}$
            \vspace{30pt}

    \part $E(1,3)$
            \vspace{20pt}

\end{parts}

\question Los valores de $A$, $B$ y $C$ son: $A=2.28\cdot 10^7$, $B=2\cdot 10^{-4}$ y $C=4,3\cdot 10^5$
\begin{parts}
\part Calcula $\dfrac{A}{B}+A\cdot C$, dando el resultado en notación científica:
\begin{solution}
    $=1,14\cdot 10^{11}+9,804\cdot 10^{12}=9,918\cdot 10^{12}$
\end{solution}
        \vspace{60pt}

\part Da una cota del error absoluto y otra del relativo cometidas al aproximar el  numero anterior a dos cifras significativas
\begin{solution}
    $EA < 0,05\cdot 10^{12}$ y $ER< \dfrac{0,05\cdot 10^{12}}{9,9\cdot 10^{12}}$
\end{solution}
        \vspace{30pt}

\end{parts}

\question Halla el valor de $x$ aplicando la definición de logaritmo: 
\begin{parts}
    \part $\log_x 16 = 4$
            \vspace{30pt}

    \part $\log_3 x =4$
            \vspace{30pt}

    \part $\log_2 64 = x$
            \vspace{30pt}

\end{parts}

\question Si sabemos que $\log k =0.9$, calcula: $\log \dfrac{k^3}{100}-\log\left( 100 \sqrt{k}\right)$

% \question Da una cota del error absoluto y otra del relativo cometidas al aproximar el número $\pi$ a las centésimas
% \begin{solution}
%     $0,005$ y $\frac{0,005}{3,14}$
% \end{solution}

% \question Escribe en forma de intervalo y representa en la recta real los siguientes conjuntos:
% \begin{parts}
%     \part $A=\left\{ x \in \mathbf{R}| -1 < x \leq 5 \right\}$
%         \vspace{50}

%     \part $B= \left\{ x \in \mathbf{R}| |x-3|<5\right\}$
%         \vspace{50}

% \end{parts}

% \question Dados los conjuntos $A=(1,8]$ y $B=(-\infty,5]$, calcula razonadamente:
% \begin{parts}
%     \part  $A \bigcap B=$
%         \vspace{30}

%     \part  $A \bigcup B=$
%         \vspace{30}

% \end{parts}

% \question Simplifica, de manera detallada:
% \begin{parts}
%     \part $\dfrac{\sqrt{9}}{\sqrt[3]{3}}=$ 
%     \vspace{40}
%     \part $\dfrac{\sqrt[5]{16}}{\sqrt{2}}=$
%         \vspace{40}

%     \part $\dfrac{\sqrt[4]{a^3b^5c}}{\sqrt{ab^3c^3}}=$
%         \vspace{40}

%     \part $\left(\sqrt{\sqrt{\sqrt{2}}}\right)^8=$
%         \vspace{40}

% \end{parts}

% \question Calcula razonadamente:\\
%     $\sqrt{18}+\sqrt{50}-\sqrt{2}-\sqrt{8}=$


% \question Resuelve las siguientes ecuaciones:
% \begin{parts}
%     \part $x^5-81x=0$
%     \begin{solution}
%                     La ecuación original es:
            
%             \[
%             x^5 - 81x = 0
%             \]
            
%             Primero, factorizamos:
            
%             \[
%             x(x^4 - 81) = 0
%             \]
            
%             Esto nos da dos ecuaciones:
            
%             \[
%             x = 0
%             \]
            
%             y
            
%             \[
%             x^4 - 81 = 0
%             \]
            
%             Resolviendo \(x^4 - 81 = 0\):
            
%             \[
%             x^4 = 81
%             \]
            
%             Tomando la raíz cuarta de ambos lados:
            
%             \[
%             x = \pm \sqrt[4]{81} = \pm \sqrt[4]{3^4} = \pm 3
%             \]
            
%             Por lo tanto, las soluciones son:
            
%             \[
%             x = 0, \quad x = 3, \quad x = -3
%             \]
            
%             En conjunto, las soluciones finales son:
            
%             \[
%             x = \{0, 3, -3\}
%             \]
%     \end{solution}
%     \part $\dfrac{1}{x}-\dfrac{x+1}{x-1}+\dfrac{5}{2}$
%     \begin{solution}
%         Para resolver la ecuación
            
%             \[
%             \frac{1}{x} - \frac{x+1}{x-1} + \frac{5}{2} = 0
%             \]
            
%             primero encontramos un denominador común \(x(x-1)\) y reescribimos cada término:
            
%             \[
%             \frac{1}{x} = \frac{x-1}{x(x-1)}
%             \]
            
%             \[
%             \frac{x+1}{x-1} = \frac{x(x+1)}{x(x-1)}
%             \]
            
%             \[
%             \frac{5}{2} = \frac{5x(x-1)}{2x(x-1)}
%             \]
            
%             Sustituimos en la ecuación original:
            
%             \[
%             \frac{x-1 - x(x+1) + \frac{5x(x-1)}{2}}{x(x-1)} = 0
%             \]
            
%             Multiplicamos todo por \(2x(x-1)\) para eliminar el denominador:
            
%             \[
%             2(x-1) - 2x(x+1) + 5x(x-1) = 0
%             \]
            
%             Expandiendo y simplificando:
            
%             \[
%             2x - 2 - 2x^2 - 2x + 5x^2 - 5x = 0
%             \]
            
%             \[
%             3x^2 - 5x - 2 = 0
%             \]
            
%             Resolviendo la ecuación cuadrática:
            
%             \[
%             x = \frac{-b \pm \sqrt{b^2 - 4ac}}{2a}
%             \]
            
%             donde \(a = 3\), \(b = -5\), y \(c = -2\):
            
%             \[
%             x = \frac{5 \pm \sqrt{25 + 24}}{6}
%             \]
            
%             \[
%             x = \frac{5 \pm 7}{6}
%             \]
            
%             \[
%             x = 2 \quad \text{y} \quad x = -\frac{1}{3}
%             \]
            
%             Las soluciones son:
            
%             \[
%             x = 2 \quad \text{y} \quad x = -\frac{1}{3}
%             \]
        
%     \end{solution}
% \end{parts}


% \question Un comerciante compra dos motocicletas por 3000 € y las vende por 3330 \euro. Calcula
% cuanto pagó por cada una si en la venta de la primera ganó un 25\% y en la de la segunda
% perdió un 10\%

%     \begin{solution}
%                     \begin{itemize}
%                   \item \( x \): precio de compra de la primera motocicleta.
%                   \item \( y \): precio de compra de la segunda motocicleta.
%                 \end{itemize}
                
%                 Entonces tenemos:
                
%                 \[
%                 x + y = 3000
%                 \]
                
%                 La primera motocicleta se vende con una ganancia del 25\%, por lo que el precio de venta es \( 1.25x \), y la segunda se vende con una pérdida del 10\%, por lo que el precio de venta es \( 0.9y \). La suma de los precios de venta es:
                
%                 \[
%                 1.25x + 0.9y = 3330
%                 \]
                
%                 Resolviendo el sistema de ecuaciones:
                
%                 \[
%                 \begin{cases}
%                 x + y = 3000 \\
%                 1.25x + 0.9y = 3330
%                 \end{cases}
%                 \]
                
%                 Despejamos \( y \) en términos de \( x \) de la primera ecuación:
                
%                 \[
%                 y = 3000 - x
%                 \]
                
%                 Sustituimos \( y \) en la segunda ecuación:
                
%                 \[
%                 1.25x + 0.9(3000 - x) = 3330
%                 \]
                
%                 \[
%                 1.25x + 2700 - 0.9x = 3330
%                 \]
                
%                 \[
%                 0.35x + 2700 = 3330
%                 \]
                
%                 \[
%                 0.35x = 630
%                 \]
                
%                 \[
%                 x = \frac{630}{0.35} = 1800
%                 \]
                
%                 Sustituimos \( x = 1800 \) en la ecuación para \( y \):
                
%                 \[
%                 y = 3000 - 1800 = 1200
%                 \]
                
%                 Por lo tanto, el comerciante pagó 1800 € por la primera motocicleta y 1200 € por la segunda.
    
%     \end{solution}

% \question Un rectángulo tiene 48 cm$^{2}$
% de superficie y su diagonal mide 10 cm. ¿Cuánto miden sus
% lados?
% \begin{solution}
    
% \end{solution}

% \question Una antena de radio está sujeta al suelo con dos tiradores de cable de acero, como muestra
% la figura: 

% \begin{tikzpicture}

% % Coordenadas de los vértices del triángulo
% \coordinate (A) at (0,0);
% \coordinate (B) at (5,0);
% \coordinate (C) at (2,3);

% % Dibuja el triángulo
% \draw[thick] (A) -- (B) -- (C) -- cycle;

% % Dibuja la altura desde C hasta el lado AB
% \draw[dashed] (C) -- (2,0);

% % Marca los vértices
% \node[below left] at (A) {A};
% \node[below right] at (B) {B};
% \node[above] at (C) {C};

% % Marca la intersección de la altura con el lado AB
% % \node[below] at (2,0) {H};

% % Añade el valor de la base (126 m)
% \draw[<->] (A) -- (B) node[midway, below] {126 m};

% % Añade los ángulos A = 60° y B = 45°
% \draw pic["$60^\circ$", draw=black, angle eccentricity=1.5, angle radius=1cm] {angle=B--A--C};
% \draw pic["$45^\circ$", draw=black, angle eccentricity=1.5, angle radius=1cm] {angle=C--B--A};

% \end{tikzpicture}
% Calcula:
% \begin{parts}
%     \part La altura de la antena.
%     \part La longitud de los cables.
% \end{parts}

%         \begin{solution}
        
%         ### Paso 1: Cálculo del ángulo en \( C \)
        
%         Sabemos que la suma de los ángulos interiores de un triángulo es \( 180^\circ \). Por lo tanto, el ángulo en \( C \) es:
%         \[
%         \angle C = 180^\circ - 60^\circ - 45^\circ = 75^\circ
%         \]
        
%         ### Paso 2: Cálculo de las longitudes \( AC \) y \( BC \) usando la ley de senos
        
%         La ley de senos nos da la relación:
%         \[
%         \frac{AB}{\sin(\angle C)} = \frac{AC}{\sin(\angle B)} = \frac{BC}{\sin(\angle A)}
%         \]
        
%         Por lo tanto, podemos calcular \( AC \) y \( BC \):
        
%         \[
%         AC = \frac{126 \cdot \sin(45^\circ)}{\sin(75^\circ)} = \frac{126 \cdot 0.707}{0.966} \approx 92.25 \, \text{m}
%         \]
%         \[
%         BC = \frac{126 \cdot \sin(60^\circ)}{\sin(75^\circ)} = \frac{126 \cdot 0.866}{0.966} \approx 112.85 \, \text{m}
%         \]
        
%         ### Paso 3: Cálculo de la altura \( HC \)
        
%         La altura \( HC \) es la proyección de \( BC \) sobre la perpendicular a la base \( AB \). Se calcula como:
%         \[
%         HC = BC \cdot \sin(\angle B) = 112.85 \cdot 0.707 \approx 79.77 \, \text{m}
%         \]
        
%         ### Resultados finales:
        
%         \[
%         AC \approx 92.25 \, \text{m}, \quad BC \approx 112.85 \, \text{m}, \quad HC \approx 79.77 \, \text{m}
%         \]
        
        
%         \end{solution}
        
\end{questions}

\end{document}
