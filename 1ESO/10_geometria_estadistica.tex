\documentclass[addpoints,spanish, 12pt,a4paper]{exam}
%\documentclass[answers, spanish, 12pt,a4paper]{exam}
% \printanswers
\renewcommand*\half{.5}

\pointpoints{punto}{puntos}
\hpword{Puntos:}
\vpword{Puntos:}
\htword{Total}
\vtword{Total}
\hsword{Resultado:}
\hqword{Ejercicio:}
\vqword{Ejercicio:}

\usepackage{booktabs} 
\usepackage[utf8]{inputenc}
\usepackage[spanish]{babel}
\usepackage{eurosym}
%\usepackage[spanish,es-lcroman, es-tabla, es-noshorthands]{babel}


\usepackage[margin=1in]{geometry}
\usepackage{amsmath,amssymb}
\usepackage{multicol}
\usepackage{yhmath}

\pointsinrightmargin % Para poner las puntuaciones a la derecha. Se puede cambiar. Si se comenta, sale a la izquierda.
\extrawidth{-2.4cm} %Un poquito más de margen por si ponemos textos largos.
\marginpointname{ \emph{\points}}

\usepackage{graphicx}

\graphicspath{{../img/}} 

\newcommand{\class}{1 ESO}
\newcommand{\examdate}{\today}
\newcommand{\examnum}{Geometría y estadística}
\newcommand{\tipo}{A}


\newcommand{\timelimit}{45 minutos}

\renewcommand{\solutiontitle}{\noindent\textbf{Solución:}\enspace}


\pagestyle{head}
\firstpageheader{\includegraphics[width=0.2\columnwidth]{header_left}}{\textbf{Departamento de Matemáticas\linebreak \class}\linebreak \examnum}{\includegraphics[width=0.1\columnwidth]{header_right}}
\runningheader{\class}{\examnum}{Página \thepage\ de \numpages}
\runningheadrule


\usepackage{pgf,tikz,pgfplots}
\pgfplotsset{compat=1.15}
\usepackage{mathrsfs}
\usetikzlibrary{arrows}


\begin{document}

\noindent
\begin{tabular*}{\textwidth}{l @{\extracolsep{\fill}} r @{\extracolsep{6pt}} }
\textbf{Nombre:} \makebox[3.5in]{\hrulefill} & \textbf{Fecha:}\makebox[1in]{\hrulefill} \\
 & \\
\textbf{Tiempo: \timelimit} & Tipo: \tipo 
\end{tabular*}
\rule[2ex]{\textwidth}{2pt}
\textbf{Instrucciones:} Justifica todos los
resultados.
Esta prueba tiene \numquestions\ ejercicios con una puntuación máxima de \numpoints. 
La nota del examen se calculará de manera proporcional a la puntuación obtenida. 
Está permitido el uso de calculadora y los ejercicios han de resolverse utilizando la notación algebraica adecuada


\begin{center}


\addpoints
 %\gradetable[h][questions]
	\pointtable[h][questions]
   
\end{center}
% \textbf{Nota:} Se tendrán en cuenta los ejercicios 1 y 2 para recuperar contenidos previos.

\noindent
\rule[2ex]{\textwidth}{2pt}



\begin{questions}

% \question[2] El lado mayor de un triángulo rectángulo mide 15 cm y uno de los dos lados menores mide 9 cm. ¿Cuánto mide el tercer lado? (se recomienda hacer el dibujo)
% \vspace{60pt}




% \question[2] Si los lados de un rectángulo miden, respectivamente, 16 cm y 30 cm, ¿cuánto mide su diagonal? (se recomienda hacer el dibujo)
% \vspace{60pt}

\question[3] Cálcula el área y el perímetro de las siguientes figuras:\\


\begin{tikzpicture}[scale=0.2]

  % Hexágono más grande visualmente, mismo etiquetado
  \begin{scope}[shift={(0,3)}]
    \draw[thick] (0:6.5) \foreach \x in {60,120,...,360} { -- (\x:6.5) };
    \draw[dashed] (0,0) -- (90:5.2); % apotema visual para etiqueta
    \node at (0,-4.5) {6 cm};
    \node at (0,2.6) {5,2 cm};
  \end{scope}

  % Rectángulo
  \draw[thick] (12,0) rectangle +(18,9);
  \node at (21,9.8) {18 cm};
  \node at (11.3,4.5) {9 cm};

  % Círculo
  \draw[thick] (40,4.5) circle (7);
  \draw[dashed] (40,4.5) -- (47,4.5);
  \node at (43.5,5.7) {7 cm};

\end{tikzpicture}
\vspace{60pt}

% \question[3] Calcula el área y el perímetro de las siguientes figuras \\


% \begin{tikzpicture}[scale=0.1]

% % PARALELOGRAMO
% \draw[thick] (0,0) -- (16,0) -- (13,8) -- (-3,8) -- cycle;
% \draw[dashed] (3,0) -- (3,8);
% \node[left] at (-3,4) {10 cm};
% \node[below] at (8,0) {16 cm};
% \node[left] at (10,4) {8 cm};

% % TRAPECIO MÁS ACHATADO (alineado horizontalmente con el paralelogramo)
% \begin{scope}[shift={(30,0)}] % desplazar a la derecha
%   \def\h{12} % altura más baja

%   \draw[thick] (0,0) -- (24,0) -- (19,\h) -- (5,\h) -- cycle;
%   \draw[dashed] (5,0) -- (5,\h);

%   % Etiquetas (mismas que antes, aunque no coincidan con escala)
%   \node[below] at (12,0) {92 cm};
%   \node[above] at (12,\h) {68 cm};
%   \node[left] at (5,\h/2) {35 cm};
%   \node[right] at (24,\h/2) {37 cm};
% \end{scope}

% \end{tikzpicture}
% \vspace{60pt}

% \question[1] El perímetro de un rombo es de 40 cm y una de sus diagonales mide 16 cm. ¿Cuánto mide la otra diagonal?
% \vspace{30pt}

\question[2]  Si en la figura siguiente $a=10$cm , calcula el área y el perímetro \\
\begin{tikzpicture}[scale=0.5]
  % Coordenadas clave
  \coordinate (A) at (0,0);        % Esquina inferior izquierda
  \coordinate (B) at (0,2);        % Esquina superior izquierda
  \coordinate (C) at (2,2);        % Punto de unión cuadrado-triángulo arriba
  \coordinate (D) at (4,0);        % Esquina inferior derecha
  \coordinate (E) at (2,0);        % Punto de unión cuadrado-triángulo abajo

  % Cuadrado y triángulo
  \draw[thick] (A) -- (B) -- (C) -- (D) -- cycle;
  % \draw[dashed] (C) -- (E);

  % Etiquetas
  \draw[-] (0,1) -- +(0,0) node[left] {\(a\)};
  \draw[-] (1,2) -- +(0,0) node[above] {\(a\)};
  % \draw[-] (3,0) -- +(0,0) node[below] {\(a\)};
  % \draw[-] (3.1,1.1) -- +(0,0) node[right] {\(b\)};
  \draw[-] (0,-0.4) -- (4,-0.4) node[midway,below] {\(2a\)};
  
\end{tikzpicture}

\vspace{80pt}

\question Calcula el perímetro y el área de las siguientes figuras:


\begin{parts}
    \part[2] \begin{tikzpicture}[scale=0.3]
  % Dibujo del polígono
  \fill[gray!50] (0,0) -- (0,7.5) -- (7,7.5) -- (7,4.5) -- (13,4.5) -- (13,0) -- cycle;
  \draw[thick] (0,0) -- (0,7.5) -- (7,7.5) -- (7,4.5) -- (13,4.5) -- (13,0) -- cycle;

  % Etiquetas de medidas (posiciones ajustadas a la escala)
  \node at (-1.8,3.75) {15 m};
  \node at (3.5,8) {14 m};
  \node at (10,5) {12 m};
  \node at (14.5,2.25) {12 m};
\end{tikzpicture}

\vspace{80pt}

    \part[2] \begin{tikzpicture}[scale=0.2]
  % Dibujo del polígono con hueco triangular
  \fill[gray!50]
    (0,0) -- (0,10) -- (23,10) -- (12,5) -- (23,0) -- cycle;
  \draw[thick]
    (0,0) -- (0,10) -- (23,10) -- (12,5) -- (23,0) -- cycle;

  % Etiquetas
  \node at (-1,5) {10 cm};
  \node at (11.5,-1) {23 cm};
  \draw[-] (12,10.8) -- (23,10.8);
  \node at (17.5,11.5) {11 cm};
\end{tikzpicture}
\end{parts}

\vspace{80pt}

% https://crdguez-taller-estadistica-main-2fsnec.streamlit.app/
\question Las notas en matemáticas de los alumnos de una clase han sido:

5 7 8 7 6 4 3 9 10 8 7 6 8 7 9 10 1 3 2 5

Queremos hacer un estudio estadístico sobre las notas de matemáticas. Responde a las siguientes cuestiones:
\begin{parts}
   \part[1] ¿Cuál es la variable estadística? ¿De qué tipo es (cualitativa o cuantitativa)? ¿Por qué?
   
\vspace{40pt}

   \part[1] Realiza una tabla de frecuencias que organice la información y un diagrama de barras
 
\vspace{250pt}

   \part[1] Calcula la nota media
   
\end{parts} 


\end{questions}



\end{document}
\grid
