\documentclass[addpoints,spanish, 12pt,a4paper]{exam}
%\documentclass[answers, spanish, 12pt,a4paper]{exam}
% \printanswers
\renewcommand*\half{.5}

\pointpoints{punto}{puntos}
\hpword{Puntos:}
\vpword{Puntos:}
\htword{Total}
\vtword{Total}
\hsword{Resultado:}
\hqword{Ejercicio:}
\vqword{Ejercicio:}

\usepackage[utf8]{inputenc}
\usepackage[spanish]{babel}
\usepackage{eurosym}
%\usepackage[spanish,es-lcroman, es-tabla, es-noshorthands]{babel}


\usepackage[margin=1in]{geometry}
\usepackage{amsmath,amssymb}
\usepackage{multicol}
\usepackage{yhmath}

\pointsinrightmargin % Para poner las puntuaciones a la derecha. Se puede cambiar. Si se comenta, sale a la izquierda.
\extrawidth{-2.4cm} %Un poquito más de margen por si ponemos textos largos.
\marginpointname{ \emph{\points}}

\usepackage{graphicx}

\graphicspath{{../img/}} 

\newcommand{\class}{2 ESO}
\newcommand{\examdate}{\today}
\newcommand{\examnum}{Naturales y enteros}
\newcommand{\tipo}{A}


\newcommand{\timelimit}{45 minutos}

\renewcommand{\solutiontitle}{\noindent\textbf{Solución:}\enspace}


\pagestyle{head}
\firstpageheader{\includegraphics[width=0.2\columnwidth]{header_left}}{\textbf{Departamento de Matemáticas\linebreak \class}\linebreak \examnum}{\includegraphics[width=0.1\columnwidth]{header_right}}
\runningheader{\class}{\examnum}{Página \thepage\ de \numpages}
\runningheadrule


\usepackage{pgf,tikz,pgfplots}
\pgfplotsset{compat=1.15}
\usepackage{mathrsfs}
\usetikzlibrary{arrows}


\begin{document}

\noindent
\begin{tabular*}{\textwidth}{l @{\extracolsep{\fill}} r @{\extracolsep{6pt}} }
\textbf{Nombre:} \makebox[3.5in]{\hrulefill} & \textbf{Fecha:}\makebox[1in]{\hrulefill} \\
 & \\
\textbf{Tiempo: \timelimit} & Tipo: \tipo 
\end{tabular*}
\rule[2ex]{\textwidth}{2pt}
\textbf{Instrucciones:} Prohibido el uso de calculadora. Justifica los
resultados.
Esta prueba tiene \numquestions\ ejercicios con una puntuación máxima de \numpoints. 
La nota del examen se calculará de manera proporcional a la puntuación obtenida. 

\begin{center}


\addpoints
 %\gradetable[h][questions]
	\pointtable[h][questions]
\end{center}

\noindent
\rule[2ex]{\textwidth}{2pt}

\begin{questions}

%\question 
%
%\begin{parts}
%\part[2] 
%\begin{solution}
%\end{solution}
%
%
%\end{parts}
%\addpoints

\question[1\half] Escribe con cifras:
\begin{parts}
\part Cinco millones y medio
\vspace{20pt}
\part Novecientos noventa y nueve millones
\vspace{20pt}
\part Dos millones dos mil dos
\vspace{20pt}
\end{parts}

\question[1\half] Escribe como se leen las siguientes cifras:
\begin{parts}
    \part 7 700 000 000
    \vspace{30pt}
    \part 6 000 000 006 000 000     \vspace{30pt}
    \part 9 675 000 850 000     \vspace{30pt}
\end{parts}
\addpoints

\question[1] Aproxima a las decenas de millar, por redondeo, los siguientes números: 
\begin{parts}
    \part 679 563
    \part 2 462 768
    \part 5 678 300
    \part  54 343 795
\end{parts}

\question[2] Calcula:
\begin{parts}
    \part \textbf{3 · 9 + 7 + 6 - 5 · 3=}\vspace{20pt}\vspace{20pt}\vspace{20pt}
    \part \textbf{5 · (2 + 6) + 7 - 4 · 3=}\vspace{20pt}\vspace{20pt}\vspace{20pt}
    \part \textbf{3 + 2 · {[}3 · (2 · 5 - 7 + 3){]} - 2 · 3=} \vspace{80pt}
    \part \textbf{5 · {[}3 + 1 + 2 · (19 - 4 · 3 + 2){]} + 2=}\vspace{80pt}
\end{parts}

\question[2] En un supermercado compramos tres productos para
  un restaurante: 6 kg de carne de cerdo a 4 \euro/kg, 4 kg de merluza a 4
  \euro/kg y 3 kg de carne de pollo al mismo precio que la carne y la
  merluza. Si pago con un billete de 100 \euro, ¿cuánto me devuelven si el
  encargado del supermercado me hace una rebaja de 5 \euro \ por cada producto
  comprado? Halla la solución mediante una expresión de operaciones
  combinadas. \vspace{100pt}

\question[2] En una granja se recogen los huevos dos días a la
  semana. El primer día se llenaron 12 cajas de 16 bandejas de huevos;
  el segundo día se completaron 22 cajas de 15 bandejas cada una. Si en
  cada bandeja entran dos docenas y media de huevos, ¿cuántos huevos se
  han recogido durante esta semana? Halla la solución mediante una
  expresión de operaciones combinadas


%%%
%%% OTRO EXAMEN
%%%
% \newpage
% \question[1\half] Escribe con cifras:
% \begin{parts}
% \part Seis millones setecientos cincuenta mil
% \vspace{20pt}
% \begin{solution}
% 6.750.000
% \end{solution}

% \part Doscientos cuarenta y cinco millones ochocientos doce mil
% \vspace{20pt}
% \begin{solution}
% 245.812.000
% \end{solution}

% \part Tres mil quinientos dieciséis
% \vspace{20pt}
% \begin{solution}
% 3.516
% \end{solution}
% \end{parts}

% \question[1\half] Escribe como se leen las siguientes cifras:
% \begin{parts}
%     \part 8 500 000 000
%     \vspace{30pt}
%     \begin{solution}
%     Ocho mil quinientos millones
%     \end{solution}
    
%     \part 5 600 000 010 000 000
%     \vspace{30pt}
%     \begin{solution}
%     Cinco billones seiscientos mil millones diez mil
%     \end{solution}
    
%     \part 7 385 000 765 000
%     \vspace{30pt}
%     \begin{solution}
%     Siete billones trescientos ochenta y cinco mil millones setecientos sesenta y cinco mil
%     \end{solution}
% \end{parts}

% \question[1] Aproxima a las decenas de millar, por redondeo, los siguientes números: 
% \begin{parts}
%     \part 728 345
%     \begin{solution}
%     730.000
%     \end{solution}
    
%     \part 1 902 678
%     \begin{solution}
%     1.900.000
%     \end{solution}
    
%     \part 8 543 210
%     \begin{solution}
%     8.540.000
%     \end{solution}
    
%     \part  32 567 890
%     \begin{solution}
%     32.570.000
%     \end{solution}
% \end{parts}

% \question[2] Calcula:
% \begin{parts}
%     \part \textbf{4 · 7 + 8 - 5 · 4 =}
%     \vspace{20pt}
%     \begin{solution}
%     4 · 7 + 8 - 5 · 4 = 28 + 8 - 20 = 16
%     \end{solution}

%     \part \textbf{6 · (3 + 4) - 8 + 7 · 2 =}
%     \vspace{20pt}
%     \begin{solution}
%     6 · (3 + 4) - 8 + 7 · 2 = 6 · 7 - 8 + 14 = 42 - 8 + 14 = 48
%     \end{solution}

%     \part \textbf{4 + 3 · {[}2 · (6 - 3 + 4){]} - 4 · 2 =} 
%     \vspace{80pt}
%     \begin{solution}
%     4 + 3 · {[}2 · (6 - 3 + 4){]} - 4 · 2 = 4 + 3 · (2 · 7) - 8 = 4 + 3 · 14 - 8 = 4 + 42 - 8 = 38
%     \end{solution}

%     \part \textbf{6 · {[}4 + 2 + 3 · (18 - 5 · 2 + 1){]} + 3 =}
%     \vspace{80pt}
%     \begin{solution}
%     6 · {[}4 + 2 + 3 · (18 - 10 + 1){]} + 3 = 6 · {[}4 + 2 + 3 · 9{]} + 3 = 6 · (4 + 2 + 27) + 3 = 6 · 33 + 3 = 198 + 3 = 201
%     \end{solution}
% \end{parts}

% \question[2] En una tienda de comestibles compramos tres productos: 5 kg de arroz a 2 \euro/kg, 3 kg de lentejas a 3 \euro/kg, y 4 kg de garbanzos al mismo precio que el arroz y las lentejas. Si pagamos con un billete de 50 \euro, ¿cuánto nos devuelven si el encargado de la tienda nos hace una rebaja de 3 \euro por cada producto comprado? Halla la solución mediante una expresión de operaciones combinadas. 
% \vspace{100pt}
% \begin{solution}
% El precio total sin rebajas sería:

% \[ 5 \text{kg} \cdot 2 \euro/\text{kg} + 3 \text{kg} \cdot 3 \euro/\text{kg} + 4 \text{kg} \cdot 3 \euro/\text{kg} = 10 \euro + 9 \euro + 12 \euro = 31 \euro \]

% Con la rebaja de 3 \euro por cada producto comprado (3 productos):

% \[ 31 \euro - (3 \euro \times 3) = 31 \euro - 9 \euro = 22 \euro \]

% Pagamos con 50 \euro, así que nos devuelven:

% \[ 50 \euro - 22 \euro = 28 \euro \]
% \end{solution}

% \question[2] En una finca se recogen manzanas tres días a la semana. El primer día se llenaron 10 cestas con 12 kilos de manzanas; el segundo día se completaron 15 cestas con 14 kilos cada una, y el tercer día se recogieron 8 cestas con 16 kilos cada una. ¿Cuántos kilos de manzanas se han recogido durante la semana? Halla la solución mediante una expresión de operaciones combinadas.
% \vspace{100pt}
% \begin{solution}
% Calculamos los kilos totales recogidos cada día y sumamos:

% Primer día: 
% \[ 10 \text{cestas} \cdot 12 \text{kg} = 120 \text{kg} \]

% Segundo día: 
% \[ 15 \text{cestas} \cdot 14 \text{kg} = 210 \text{kg} \]

% Tercer día: 
% \[ 8 \text{cestas} \cdot 16 \text{kg} = 128 \text{kg} \]

% El total de kilos recogidos durante la semana es:
% \[ 120 \text{kg} + 210 \text{kg} + 128 \text{kg} = 458 \text{kg} \]
% \end{solution}


\end{questions}

\end{document}
\grid
