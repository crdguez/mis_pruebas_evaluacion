\documentclass[addpoints,spanish, 12pt,a4paper]{exam}
%\documentclass[answers, spanish, 12pt,a4paper]{exam}
\printanswers
\renewcommand*\half{.5}

\pointpoints{punto}{puntos}
\hpword{Puntos:}
\vpword{Puntos:}
\htword{Total}
\vtword{Total}
\hsword{Resultado:}
\hqword{Ejercicio:}
\vqword{Ejercicio:}

\usepackage[utf8]{inputenc}
\usepackage[spanish]{babel}
\usepackage{eurosym}
%\usepackage[spanish,es-lcroman, es-tabla, es-noshorthands]{babel}


\usepackage[margin=1in]{geometry}
\usepackage{amsmath,amssymb}
\usepackage{multicol}
\usepackage{yhmath}

\pointsinrightmargin % Para poner las puntuaciones a la derecha. Se puede cambiar. Si se comenta, sale a la izquierda.
\extrawidth{-2.4cm} %Un poquito más de margen por si ponemos textos largos.
\marginpointname{ \emph{\points}}

\usepackage{graphicx}

\graphicspath{{../img/}} 

\newcommand{\class}{1 ESO}
\newcommand{\examdate}{\today}
\newcommand{\examnum}{Divisibilidad}
\newcommand{\tipo}{A}


\newcommand{\timelimit}{45 minutos}

\renewcommand{\solutiontitle}{\noindent\textbf{Solución:}\enspace}


\pagestyle{head}
\firstpageheader{\includegraphics[width=0.2\columnwidth]{header_left}}{\textbf{Departamento de Matemáticas\linebreak \class}\linebreak \examnum}{\includegraphics[width=0.1\columnwidth]{header_right}}
\runningheader{\class}{\examnum}{Página \thepage\ de \numpages}
\runningheadrule


\usepackage{pgf,tikz,pgfplots}
\pgfplotsset{compat=1.15}
\usepackage{mathrsfs}
\usetikzlibrary{arrows}


\begin{document}

\noindent
\begin{tabular*}{\textwidth}{l @{\extracolsep{\fill}} r @{\extracolsep{6pt}} }
\textbf{Nombre:} \makebox[3.5in]{\hrulefill} & \textbf{Fecha:}\makebox[1in]{\hrulefill} \\
 & \\
\textbf{Tiempo: \timelimit} & Tipo: \tipo 
\end{tabular*}
\rule[2ex]{\textwidth}{2pt}
\textbf{Instrucciones:} Prohibido el uso de calculadora. Justifica los
resultados.
Esta prueba tiene \numquestions\ ejercicios con una puntuación máxima de \numpoints. 
La nota del examen se calculará de manera proporcional a la puntuación obtenida. 

\begin{center}


\addpoints
 %\gradetable[h][questions]
	\pointtable[h][questions]
\end{center}

\noindent
\rule[2ex]{\textwidth}{2pt}

\begin{questions}

    \question[1] ¿Cuál o cuáles de estos números son múltiplos de 12? Explica por qué:
    \begin{parts}
        \part 96 \vspace{20pt}
        \part 54 \vspace{20pt}
        \part 84 \vspace{20pt}
        \part 144 \vspace{20pt}
    \end{parts}

    \question[0\half] Calcula todos los divisores de los siguientes números:
    \begin{parts}
        \part 48
        \part 36
    \end{parts}

    \question[1] Observa estos números y completa:

    12 ‒ 14 ‒ 21 ‒ 25 ‒ 36 ‒ 40 ‒ 42 ‒ 45 ‒ 70 ‒ 75

    \begin{parts}
        \part Múltiplos de 2: \vspace{20pt}
        \part Múltiplos de 3: \vspace{20pt}
        \part Múltiplos de 5: \vspace{20pt}
        \part Múltiplos de 10: \vspace{20pt}
    \end{parts}

    \question[1] Descompón en factores primos:
    \begin{parts}
        \part 80 \vspace{20pt}\vspace{20pt}\vspace{20pt}
        \part 450 \vspace{20pt}\vspace{20pt}\vspace{20pt}
    \end{parts}

    \question[1\half] Calcula descomponiendo en factores primos:
    \begin{parts}
        \part m.c.m. (12, 24, 36) \vspace{20pt}\vspace{20pt}
        \part m.c.d. (28, 36) \vspace{20pt}\vspace{20pt}
    \end{parts}

    \question[1\half] ¿De cuántas formas podemos empaquetar 45 libros si debe haber el mismo número de libros en cada paquete?\vspace{20pt}\vspace{20pt}\vspace{20pt}

    \question[2] Un carpintero dispone de tres listones de madera de 40, 60 y 90 cm de longitud, respectivamente. Desea dividirlos en trozos iguales y de la mayor medida posible, sin que sobre madera. ¿Qué longitud deben tener esos trozos? ¿Cuántos trozos iguales de cada listón se obtienen?\vspace{20pt}\vspace{20pt}\vspace{20pt}

    \question[1\half] Calcula los múltiplos de 18 comprendidos entre 315 y 420.

\end{questions}



\end{document}
\grid
