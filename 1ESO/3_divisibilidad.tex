\documentclass[addpoints,spanish, 12pt,a4paper]{exam}
%\documentclass[answers, spanish, 12pt,a4paper]{exam}
\printanswers
\renewcommand*\half{.5}

\pointpoints{punto}{puntos}
\hpword{Puntos:}
\vpword{Puntos:}
\htword{Total}
\vtword{Total}
\hsword{Resultado:}
\hqword{Ejercicio:}
\vqword{Ejercicio:}

\usepackage[utf8]{inputenc}
\usepackage[spanish]{babel}
\usepackage{eurosym}
%\usepackage[spanish,es-lcroman, es-tabla, es-noshorthands]{babel}


\usepackage[margin=1in]{geometry}
\usepackage{amsmath,amssymb}
\usepackage{multicol}
\usepackage{yhmath}

\pointsinrightmargin % Para poner las puntuaciones a la derecha. Se puede cambiar. Si se comenta, sale a la izquierda.
\extrawidth{-2.4cm} %Un poquito más de margen por si ponemos textos largos.
\marginpointname{ \emph{\points}}

\usepackage{graphicx}

\graphicspath{{../img/}} 

\newcommand{\class}{1 ESO}
\newcommand{\examdate}{\today}
\newcommand{\examnum}{Divisibilidad}
\newcommand{\tipo}{A}


\newcommand{\timelimit}{45 minutos}

\renewcommand{\solutiontitle}{\noindent\textbf{Solución:}\enspace}


\pagestyle{head}
\firstpageheader{\includegraphics[width=0.2\columnwidth]{header_left}}{\textbf{Departamento de Matemáticas\linebreak \class}\linebreak \examnum}{\includegraphics[width=0.1\columnwidth]{header_right}}
\runningheader{\class}{\examnum}{Página \thepage\ de \numpages}
\runningheadrule


\usepackage{pgf,tikz,pgfplots}
\pgfplotsset{compat=1.15}
\usepackage{mathrsfs}
\usetikzlibrary{arrows}


\begin{document}

\noindent
\begin{tabular*}{\textwidth}{l @{\extracolsep{\fill}} r @{\extracolsep{6pt}} }
\textbf{Nombre:} \makebox[3.5in]{\hrulefill} & \textbf{Fecha:}\makebox[1in]{\hrulefill} \\
 & \\
\textbf{Tiempo: \timelimit} & Tipo: \tipo 
\end{tabular*}
\rule[2ex]{\textwidth}{2pt}
\textbf{Instrucciones:} Prohibido el uso de calculadora. Justifica los
resultados.
Esta prueba tiene \numquestions\ ejercicios con una puntuación máxima de \numpoints. 
La nota del examen se calculará de manera proporcional a la puntuación obtenida. \textbf{Para recuperar el examen anterior se tendrán en cuenta las preguntas 1 a 3}\\


\begin{center}


\addpoints
 %\gradetable[h][questions]
	\pointtable[h][questions]

    
\end{center}

\noindent
\rule[2ex]{\textwidth}{2pt}



\begin{questions}

\question[1\half] Expresa como una única potencia:
\begin{multicols}{3}
\begin{parts}
    \part $\left(4^2\right)^5=$
    \vspace{10pt}
    \part $3^4\cdot 3^6=$
    \vspace{10pt}
    \part $m^7 : m^3=$
    \vspace{10pt}
\end{parts}
\end{multicols}

\question[2] Aplica las propiedades para expresar como una única potencia y después calcula su valor:
\begin{multicols}{2}
\begin{parts}
    \part $4^3\cdot 4^5 : 4^6=$\vspace{20pt}
    \part $30^4 : 15^4=$\vspace{20pt}
    \part $25^5 : (5^3 \cdot 5^2)=$\vspace{20pt}
    \part $\left[\left(18^2 : 9^2\right)^3 : \left(12^2 : 6^2\right)^2\right]^3=$\vspace{20pt}
\end{parts}
\end{multicols}

% Pregunta 4
\question[1\half] Calcula:

\begin{parts}
    % \part $3^2 + \sqrt{3^2 + 2 \cdot 2^3}=$
    % \vspace{30pt}
    \part $15 - 2 \cdot 2^2 + 5 \cdot \sqrt{49}=$
    \vspace{40pt}
    \part $8 + \sqrt{7 + 3\cdot 6} - 2 + 5\cdot\left(6-3\right)^2=$
    \vspace{40pt}
\end{parts}

    % \question[1] De los números 96, 54, 84, 144, ¿Cuál o cuáles de estos números son múltiplos de 12? Explica por qué


    % \question[0\half] Calcula todos los divisores de los siguientes números:
    % \begin{parts}
    %     \part 48
    %     \part 36
    % \end{parts}

% \question[1] Observa estos números y completa:
% 12 ‒ 14 ‒ 21 ‒ 25 ‒ 36 ‒ 40 ‒ 42 ‒ 45 ‒ 70 ‒ 75
% \begin{multicols}{2}
%     \begin{parts}
%         \part Múltiplos de 2: \vspace{10pt}
%         \part Múltiplos de 3: \vspace{10pt}
%         \part Múltiplos de 5: \vspace{10pt}
%         \part Múltiplos de 10: \vspace{10pt}
%     \end{parts}
% \end{multicols}

    \question[1\half] Descompón en factores primos:
    \begin{multicols}{2}
    \begin{parts}
        \part 80 \vspace{20pt}\vspace{20pt}\vspace{20pt}
        \part 450 \vspace{20pt}\vspace{20pt}\vspace{20pt}
    \end{parts}
    \end{multicols}

    \question[1\half] Calcula descomponiendo en factores primos:
    \begin{multicols}{2}
    \begin{parts}
        \part m.c.m. (12, 24, 36) \vspace{20pt}\vspace{70pt}
        \part m.c.d. (28, 36) \vspace{20pt}\vspace{70pt}
    \end{parts}
    
    \end{multicols}

    \question[1\half] ¿De cuántas formas podemos empaquetar 45 libros si debe haber el mismo número de libros en cada paquete?\vspace{20pt}\vspace{20pt}\vspace{50pt}

    \question[2] Un carpintero dispone de tres listones de madera de 40, 60 y 90 cm de longitud, respectivamente. Desea dividirlos en trozos iguales y de la mayor medida posible, sin que sobre madera. ¿Qué longitud deben tener esos trozos? ¿Cuántos trozos iguales de cada listón se obtienen?\vspace{20pt}\vspace{20pt}\vspace{50pt}

    \question[1\half] Calcula los múltiplos de 18 comprendidos entre 315 y 420.

\end{questions}



\end{document}
\grid
