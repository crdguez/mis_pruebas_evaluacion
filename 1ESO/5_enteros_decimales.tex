\documentclass[addpoints,spanish, 12pt,a4paper]{exam}
%\documentclass[answers, spanish, 12pt,a4paper]{exam}
\printanswers
\renewcommand*\half{.5}

\pointpoints{punto}{puntos}
\hpword{Puntos:}
\vpword{Puntos:}
\htword{Total}
\vtword{Total}
\hsword{Resultado:}
\hqword{Ejercicio:}
\vqword{Ejercicio:}

\usepackage[utf8]{inputenc}
\usepackage[spanish]{babel}
\usepackage{eurosym}
%\usepackage[spanish,es-lcroman, es-tabla, es-noshorthands]{babel}


\usepackage[margin=1in]{geometry}
\usepackage{amsmath,amssymb}
\usepackage{multicol}
\usepackage{yhmath}

\pointsinrightmargin % Para poner las puntuaciones a la derecha. Se puede cambiar. Si se comenta, sale a la izquierda.
\extrawidth{-2.4cm} %Un poquito más de margen por si ponemos textos largos.
\marginpointname{ \emph{\points}}

\usepackage{graphicx}

\graphicspath{{../img/}} 

\newcommand{\class}{1 ESO}
\newcommand{\examdate}{\today}
\newcommand{\examnum}{Enteros y decimales}
\newcommand{\tipo}{A}


\newcommand{\timelimit}{45 minutos}

\renewcommand{\solutiontitle}{\noindent\textbf{Solución:}\enspace}


\pagestyle{head}
\firstpageheader{\includegraphics[width=0.2\columnwidth]{header_left}}{\textbf{Departamento de Matemáticas\linebreak \class}\linebreak \examnum}{\includegraphics[width=0.1\columnwidth]{header_right}}
\runningheader{\class}{\examnum}{Página \thepage\ de \numpages}
\runningheadrule


\usepackage{pgf,tikz,pgfplots}
\pgfplotsset{compat=1.15}
\usepackage{mathrsfs}
\usetikzlibrary{arrows}


\begin{document}

\noindent
\begin{tabular*}{\textwidth}{l @{\extracolsep{\fill}} r @{\extracolsep{6pt}} }
\textbf{Nombre:} \makebox[3.5in]{\hrulefill} & \textbf{Fecha:}\makebox[1in]{\hrulefill} \\
 & \\
\textbf{Tiempo: \timelimit} & Tipo: \tipo 
\end{tabular*}
\rule[2ex]{\textwidth}{2pt}
\textbf{Instrucciones:} Prohibido el uso de calculadora. Justifica los
resultados.
Esta prueba tiene \numquestions\ ejercicios con una puntuación máxima de \numpoints. 
La nota del examen se calculará de manera proporcional a la puntuación obtenida. 
% \textbf{Para recuperar el examen anterior se tendrán en cuenta las preguntas 1 a 3}\\


\begin{center}


\addpoints
 %\gradetable[h][questions]
	\pointtable[h][questions]

    
\end{center}

\noindent
\rule[2ex]{\textwidth}{2pt}



\begin{questions}




    % \question[1] Calcula descomponiendo en factores primos:
    % \begin{multicols}{2}
    % \begin{parts}
    %     \part m.c.m. (10, 12) \vspace{20pt}\vspace{70pt}
    %     \part m.c.d. (20, 12) \vspace{20pt}\vspace{70pt}
    % \end{parts}
    
    % \end{multicols}

\question[2] Realiza las siguientes operaciones con enteros:
\begin{multicols}{2}
\begin{parts}
    \part $(-5) + 7=$
    \vspace{0pt}
    \part $11 - (-3)=$
    \vspace{0pt}
    \part $(-8) - (-3)=$
    \vspace{0pt}
    \part $(-4) + (-7)=$
    \vspace{0pt}
\end{parts}
\end{multicols}


\question[2] Realiza las siguientes operaciones con enteros:
\begin{multicols}{2}
\begin{parts}
    \part $(-4) \cdot 3=$
    \vspace{0pt}
    \part $18 : (-3)=$
    \vspace{10pt}
    \part $(-8) \cdot (-7)=$
    \vspace{0pt}
    \part $(-42) : 6=$
    \vspace{0pt}
\end{parts}
\end{multicols}


\question[4] Realiza las siguientes operaciones combinadas con enteros:

\begin{parts}
    \part $(-5) \cdot (3 - 8) + 3=$
    \vspace{20pt}
    \part $15 - (-6) \cdot 2 + 4 : (-2)=$
    \vspace{20pt}
    % \part $\left[(-3) + 7\right] \cdot (-2) - 5=$
    % \vspace{20pt}
    % \part $(-18 : 3) + 4 \cdot (-2) - (-10)=$
    % \vspace{30pt}
    % \part $(-7) \cdot (2 + 5) - 3 \cdot (4 - 9) + \left[ 6 : (3 - 5) \right] =$ \vspace{40pt} 
    \part $\left[ 3 + (8 : 4) \right] \cdot (2 - 3) + 5 \cdot (-4 + 6) =$ \vspace{50pt} 
\end{parts}


\question[1] Aproxima a las décimas:
\begin{multicols}
    {2}
\begin{parts}
    \part 5,37 $\approx$
    \part 4,21 $\approx$
    \part 6,393 $\approx$
    \part 0,824 $\approx$
\end{parts}
\end{multicols}

\question[3] Calcula:
\begin{parts}
    \part $20,54 + 13,6 - 3,12 =$ \vspace{60pt}
    \part $8,32 \cdot 7,5 =$ \vspace{70pt}
    \part $1,29 : 5 =$ \vspace{70pt}
    % \part $865,5 : 15 =$ \vspace{40pt} \vspace{40pt}
\end{parts}

\question[2] ¿Cuánto costará pintar las puertas y ventanas de un piso si tiene 9 ventanas y 8 puertas y el pintor cobra 10,5 euros por pintar una puerta y 7,35 euros por pintar una ventana?
\vspace{40pt}\vspace{90pt}


\question[2] Un depósito contiene 46,22 litros de agua que vamos a traspasar a botellas de litro y medio. Halla cuántas botellas
llenaremos e indica la cantidad de agua sobrante.
% \question[1\half] Fernando compra un pollo de 2 kg 200 g y un conejo de 0,760 kg. ¿Cuánto pesa la compra de Fernando?
% \vspace{40pt}

% \question[1\half] Una finca de 20,32 hm² tiene 15,67 ha de secano plantadas de cereal y 11,300 m² de huerta en regadío. El resto es terreno en barbecho. ¿Cuál es la superficie en barbecho?



\end{questions}



\end{document}
\grid
