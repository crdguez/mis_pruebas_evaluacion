\documentclass[addpoints,spanish, 12pt,a4paper]{exam}
%\documentclass[answers, spanish, 12pt,a4paper]{exam}
% \printanswers
\renewcommand*\half{.5}

\pointpoints{punto}{puntos}
\hpword{Puntos:}
\vpword{Puntos:}
\htword{Total}
\vtword{Total}
\hsword{Resultado:}
\hqword{Ejercicio:}
\vqword{Ejercicio:}

\usepackage{booktabs} 
\usepackage[utf8]{inputenc}
\usepackage[spanish]{babel}
\usepackage{eurosym}
%\usepackage[spanish,es-lcroman, es-tabla, es-noshorthands]{babel}


\usepackage[margin=1in]{geometry}
\usepackage{amsmath,amssymb}
\usepackage{multicol}
\usepackage{yhmath}

\pointsinrightmargin % Para poner las puntuaciones a la derecha. Se puede cambiar. Si se comenta, sale a la izquierda.
\extrawidth{-2.4cm} %Un poquito más de margen por si ponemos textos largos.
\marginpointname{ \emph{\points}}

\usepackage{graphicx}

\graphicspath{{../img/}} 

\newcommand{\class}{1 ESO}
\newcommand{\examdate}{\today}
\newcommand{\examnum}{Enteros, decimales y fracciones}
\newcommand{\tipo}{A}


\newcommand{\timelimit}{45 minutos}

\renewcommand{\solutiontitle}{\noindent\textbf{Solución:}\enspace}


\pagestyle{head}
\firstpageheader{\includegraphics[width=0.2\columnwidth]{header_left}}{\textbf{Departamento de Matemáticas\linebreak \class}\linebreak \examnum}{\includegraphics[width=0.1\columnwidth]{header_right}}
\runningheader{\class}{\examnum}{Página \thepage\ de \numpages}
\runningheadrule


\usepackage{pgf,tikz,pgfplots}
\pgfplotsset{compat=1.15}
\usepackage{mathrsfs}
\usetikzlibrary{arrows}


\begin{document}

\noindent
\begin{tabular*}{\textwidth}{l @{\extracolsep{\fill}} r @{\extracolsep{6pt}} }
\textbf{Nombre:} \makebox[3.5in]{\hrulefill} & \textbf{Fecha:}\makebox[1in]{\hrulefill} \\
 & \\
\textbf{Tiempo: \timelimit} & Tipo: \tipo 
\end{tabular*}
\rule[2ex]{\textwidth}{2pt}
\textbf{Instrucciones:} Prohibido el uso de calculadora. Justifica los
resultados.
Esta prueba tiene \numquestions\ ejercicios con una puntuación máxima de \numpoints. 
La nota del examen se calculará de manera proporcional a la puntuación obtenida. 
% \textbf{Para recuperar el examen anterior se tendrán en cuenta las preguntas 1 a 3}\\


\begin{center}


\addpoints
 %\gradetable[h][questions]
	\pointtable[h][questions]
   
\end{center}
\textbf{Nota:} Se tendrán en cuenta los ejercicios 1 y 2 para recuperar contenidos previos

\noindent
\rule[2ex]{\textwidth}{2pt}



\begin{questions}




    % \question[1] Calcula descomponiendo en factores primos:
    % \begin{multicols}{2}
    % \begin{parts}
    %     \part m.c.m. (10, 12) \vspace{20pt}\vspace{70pt}
    %     \part m.c.d. (20, 12) \vspace{20pt}\vspace{70pt}
    % \end{parts}
    
    % \end{multicols}

\question[2] Calcula:
\begin{multicols}{2}
\begin{parts}
    % \part $(-5) + 7=$
    % \vspace{0pt}
    \part $11 - (-3)=$
    \vspace{0pt}
    \part $(-8) - (-3)=$
    \vspace{0pt}
    \part $(-4) + (-7)=$
    \vspace{0pt}
    \part $(-4) \cdot 3=$
    \vspace{0pt}
    \part $18 : (-3)=$
    \vspace{10pt}
    \part $(-8) \cdot (-7)=$
    \vspace{0pt}
    % \part $(-42) : 6=$
    % \vspace{0pt}
\end{parts}
\end{multicols}


% \question[2] Realiza las siguientes operaciones con enteros:
% \begin{multicols}{2}
% \begin{parts}
%     \part $(-4) \cdot 3=$
%     \vspace{0pt}
%     \part $18 : (-3)=$
%     \vspace{10pt}
%     \part $(-8) \cdot (-7)=$
%     \vspace{0pt}
%     \part $(-42) : 6=$
%     \vspace{0pt}
% \end{parts}
% \end{multicols}


\question Calcula:

\begin{parts}
    \part[1] $(-5) \cdot (3 - 8) + 3=$
    \vspace{20pt}
    % \part $15 - (-6) \cdot 2 + 4 : (-2)=$
    % \vspace{20pt}
    % \part $\left[(-3) + 7\right] \cdot (-2) - 5=$
    % \vspace{20pt}
    % \part $(-18 : 3) + 4 \cdot (-2) - (-10)=$
    % \vspace{30pt}
    % \part $(-7) \cdot (2 + 5) - 3 \cdot (4 - 9) + \left[ 6 : (3 - 5) \right] =$ \vspace{40pt} 
    % \part $\left[ 3 + (8 : 4) \right] \cdot (2 - 3) + 5 \cdot (-4 + 6) =$ \vspace{50pt} 
    \part[1] $20,54 + 13,6 - 3,12 =$ \vspace{60pt}
    % \part $8,32 \cdot 7,5 =$ \vspace{70pt}
    \part[1] $1,29 : 5 =$ \vspace{50pt}
    % \part $865,5 : 15 =$ \vspace{40pt} \vspace{40pt}
\end{parts}


% \question[1] Aproxima a las décimas:
% \begin{multicols}
%     {2}
% \begin{parts}
%     \part 5,37 $\approx$
%     \part 4,21 $\approx$
%     \part 6,393 $\approx$
%     \part 0,824 $\approx$
% \end{parts}
% \end{multicols}



% \question[2] ¿Cuánto costará pintar las puertas y ventanas de un piso si tiene 9 ventanas y 8 puertas y el pintor cobra 10,5 euros por pintar una puerta y 7,35 euros por pintar una ventana?
% \vspace{40pt}\vspace{90pt}


% \question[2] Un depósito contiene 46,22 litros de agua que vamos a traspasar a botellas de litro y medio. Halla cuántas botellas
% llenaremos e indica la cantidad de agua sobrante.
% \question[1\half] Fernando compra un pollo de 2 kg 200 g y un conejo de 0,760 kg. ¿Cuánto pesa la compra de Fernando?
% \vspace{40pt}

% \question[1\half] Una finca de 20,32 hm² tiene 15,67 ha de secano plantadas de cereal y 11,300 m² de huerta en regadío. El resto es terreno en barbecho. ¿Cuál es la superficie en barbecho?

% \question Clasifica las siguientes dfracciones y represéntalas en una recta:  
% \( \dfrac{2}{3}, \dfrac{7}{4}, -\dfrac{4}{5} \).

% \question Ordena de menor a mayor las siguientes dfracciones:  
% \[
% \dfrac{7}{8}, \dfrac{11}{15}, \dfrac{6}{7}, \dfrac{3}{5}, \dfrac{2}{3}
% \]

\question[3] Simplifica las siguientes dfracciones y obtén la dfracción irreducible correspondiente:  
\[
\dfrac{18}{66}, \quad \dfrac{45}{100}, \quad  \dfrac{30}{24}
\]
\vspace{40pt}

\question Opera y simplifica:  
\begin{parts}
    \part[1] $\dfrac{1}{3} - \dfrac{1}{6}=$
    \vspace{20pt}
    \part[1] \( 7 - 4\left(\dfrac{1}{3} - \dfrac{1}{6}\right)= \) 
    \begin{solution}
        \(\dfrac{19}{3}\)
    \end{solution}\vspace{40pt}
    
    % \part \( \left(\dfrac{-2}{3} - \dfrac{3}{4}\right) \div \left(-1\right) \) 
    % \begin{solution}
    %     \(\dfrac{17}{12}\)
    % \end{solution}
    
    \part[1] \( \left(\dfrac{5}{3} - \dfrac{1}{4}\right) - \left(\dfrac{3}{5} - \dfrac{1}{3}\right) =\)
    \begin{solution}
        \(\dfrac{23}{20}\)
    \end{solution}
    \vspace{40pt}
    % \part \( \dfrac{5}{4} - \left(\dfrac{3}{5} - \dfrac{1}{3}\right) \)
    % \begin{solution}
    %     \(\dfrac{59}{60}\)
    % \end{solution}
\end{parts}

% \question Opera y simplifica:  
% \[
% \left(\dfrac{5}{3} - \dfrac{1}{4}\right) - \left(\dfrac{3}{5} - \dfrac{1}{3}\right)
% \]

% \question Opera y simplifica:  
% \[
% \dfrac{2}{5} - \left(\dfrac{1}{3} - \dfrac{1}{8}\right) + \dfrac{3}{7}, \quad \left(\dfrac{5}{3} - \dfrac{1}{4}\right) - \left(\dfrac{3}{5} - \dfrac{1}{3}\right)
% \]

\question[1] Esta lista expresa en forma de fracción los resultados que un grupo de un examen. ¿Qué fracción de alumnado ha superado el examen?  

% \renewcommand{\arraystretch}{3} % Aumenta el espacio entre filas
% \begin{center}
%     \begin{tabular}{|c|c|c|c|c|}
%         \hline
%         \textbf{Calificación} & \textbf{Sobresaliente} & \textbf{Notable} & \textbf{Bien} & \textbf{Suficiente} \\
%         \hline
%         \textbf{dfracción} & $\dfrac{1}{6}$ & $\dfrac{1}{4}$ & $\dfrac{1}{3}$ & $\dfrac{1}{4}$ \\
%         \hline
%     \end{tabular}
% \end{center}

\begin{center}
\scalebox{0.8}{
    \begin{tabular}{ccccc}
        \toprule
        \textbf{Calificación} & \textbf{Sobresaliente} & \textbf{Notable} & \textbf{Bien} & \textbf{Suficiente} \\
        \midrule
        \textbf{fracción} & $\dfrac{1}{6}$ & $\dfrac{1}{4}$ & $\dfrac{1}{3}$ & $\dfrac{1}{4}$ \\
        \bottomrule
    \end{tabular}
    }
\end{center}
\vspace{40pt}

\question Resuelve:
\begin{parts}
    \part[1] Roberto recorre \( \dfrac{2}{5} \) partes de un camino de 10 km. ¿Cuántos km le falta por llegar al final? \vspace{60pt}
    \part[1] Los 3/8 de una población son 6000 habitantes. ¿Cuántos habitantes tiene en total?\vspace{60pt}
    % \part[1] La tercera parte de la capacidad de un depósito son 150 $m^3$. Hallar la capacidad del depósito.\vspace{40pt}
\end{parts}


\question[2] Compramos 100 litros de refresco a 2 euros el litro. Los envasamos en botes de \( \dfrac{1}{3} \) de litro y los vendemos a 1 euro. ¿Cuánto dinero ganaremos?

\end{questions}



\end{document}
\grid
