\documentclass[addpoints,spanish, 12pt,a4paper]{exam}
%\documentclass[answers, spanish, 12pt,a4paper]{exam}
% \printanswers
\renewcommand*\half{.5}

\pointpoints{punto}{puntos}
\hpword{Puntos:}
\vpword{Puntos:}
\htword{Total}
\vtword{Total}
\hsword{Resultado:}
\hqword{Ejercicio:}
\vqword{Ejercicio:}

\usepackage{booktabs} 
\usepackage[utf8]{inputenc}
\usepackage[spanish]{babel}
\usepackage{eurosym}
%\usepackage[spanish,es-lcroman, es-tabla, es-noshorthands]{babel}


\usepackage[margin=1in]{geometry}
\usepackage{amsmath,amssymb}
\usepackage{multicol}
\usepackage{yhmath}

\pointsinrightmargin % Para poner las puntuaciones a la derecha. Se puede cambiar. Si se comenta, sale a la izquierda.
\extrawidth{-2.4cm} %Un poquito más de margen por si ponemos textos largos.
\marginpointname{ \emph{\points}}

\usepackage{graphicx}

\graphicspath{{../img/}} 

\newcommand{\class}{1 ESO}
\newcommand{\examdate}{\today}
\newcommand{\examnum}{Fracciones, proporcionalidad y porcentajes}
\newcommand{\tipo}{A}


\newcommand{\timelimit}{45 minutos}

\renewcommand{\solutiontitle}{\noindent\textbf{Solución:}\enspace}


\pagestyle{head}
\firstpageheader{\includegraphics[width=0.2\columnwidth]{header_left}}{\textbf{Departamento de Matemáticas\linebreak \class}\linebreak \examnum}{\includegraphics[width=0.1\columnwidth]{header_right}}
\runningheader{\class}{\examnum}{Página \thepage\ de \numpages}
\runningheadrule


\usepackage{pgf,tikz,pgfplots}
\pgfplotsset{compat=1.15}
\usepackage{mathrsfs}
\usetikzlibrary{arrows}


\begin{document}

\noindent
\begin{tabular*}{\textwidth}{l @{\extracolsep{\fill}} r @{\extracolsep{6pt}} }
\textbf{Nombre:} \makebox[3.5in]{\hrulefill} & \textbf{Fecha:}\makebox[1in]{\hrulefill} \\
 & \\
\textbf{Tiempo: \timelimit} & Tipo: \tipo 
\end{tabular*}
\rule[2ex]{\textwidth}{2pt}
\textbf{Instrucciones:} Prohibido el uso de calculadora. Justifica los
resultados.
Esta prueba tiene \numquestions\ ejercicios con una puntuación máxima de \numpoints. 
La nota del examen se calculará de manera proporcional a la puntuación obtenida. 
% \textbf{Para recuperar el examen anterior se tendrán en cuenta las preguntas 1 a 3}\\


\begin{center}


\addpoints
 %\gradetable[h][questions]
	\pointtable[h][questions]
   
\end{center}
\textbf{Nota:} Se tendrán en cuenta los ejercicios 1 y 2 para recuperar contenidos previos.

\noindent
\rule[2ex]{\textwidth}{2pt}



\begin{questions}



\question[1] Simplifica las siguientes fracciones y obtén la fracción irreducible correspondiente:  
\[
\dfrac{18}{66}, \quad \dfrac{45}{100}, \quad  \dfrac{30}{24}
\]
\vspace{30pt}

\question Opera y simplifica:  
\begin{parts}
    \part[1] $\dfrac{1}{3} - \dfrac{1}{6}=$
    \vspace{20pt}
    % \part[1] \( 7 - 4\left(\dfrac{1}{3} - \dfrac{1}{6}\right)= \) 
    % \begin{solution}
    %     \(\dfrac{19}{3}\)
    % \end{solution}\vspace{40pt}
    
    \part[2] \( \left(\dfrac{-2}{3} - \dfrac{3}{4}\right) : \left(-1\right)= \) 
    \begin{solution}
        \(\dfrac{17}{12}\)
    \end{solution}
    \vspace{40pt}
    
    % \part[1] \( \left(\dfrac{5}{3} - \dfrac{1}{4}\right) - \left(\dfrac{3}{5} - \dfrac{1}{3}\right) =\)
    % \begin{solution}
    %     \(\dfrac{23}{20}\)
    % \end{solution}
    % \vspace{40pt}
    \part[1] \( \dfrac{5}{4} - \left(\dfrac{3}{5} - \dfrac{1}{3}\right)= \)
    \begin{solution}
        \(\dfrac{59}{60}\)
    \end{solution}
      \vspace{40pt}
\end{parts}

% \question Opera y simplifica:  
% \[
% \left(\dfrac{5}{3} - \dfrac{1}{4}\right) - \left(\dfrac{3}{5} - \dfrac{1}{3}\right)
% \]

% \question Opera y simplifica:  
% \[
% \dfrac{2}{5} - \left(\dfrac{1}{3} - \dfrac{1}{8}\right) + \dfrac{3}{7}, \quad \left(\dfrac{5}{3} - \dfrac{1}{4}\right) - \left(\dfrac{3}{5} - \dfrac{1}{3}\right)
% \]


\question[1] Calcula el término que falta en cada par para que sean dos fracciones equivalentes:
\begin{multicols}{3}
\begin{parts}
    \part $\dfrac{5}{8} = \dfrac{15}{ }$
    
    \part $\dfrac{7}{9} = \dfrac{}{27}$
    
    \part $\dfrac{}{4} = \dfrac{24}{32}$
\end{parts}
\end{multicols}
\vspace{40pt}

\question[1\half] Una fuente da 208 litros de agua en 8 minutos. ¿Cuántos litros de agua dará en un cuarto de hora?

\vspace{3cm}

\question[1] Para descargar un camión de sacos de cemento, 4 obreros han empleado 9 horas. ¿Cuánto tiempo emplearán 6 obreros?

\vspace{3cm}

\question[1\half] Completa la siguiente tabla escribiendo el porcentaje, la fracción y el número decimal que corresponde en cada caso:
\begin{center}
    \begin{tabular}{|c|c|c|c|c|}
        \hline
        PORCENTAJE & 18\% & & & \\ \hline
        FRACCIÓN & & & $\frac{35}{100}$ & \\ \hline
        NÚMERO DECIMAL & & $0.24$ & & $0.08$ \\ \hline
    \end{tabular}
\end{center}

\question[1\half] Un barco pesquero ha capturado cuatro toneladas de pescado, de las que el 45\% es sardina, el 15\% es boquerón y el resto es jurel. ¿Cuántos kilos de cada tipo de pescado lleva el barco?

\vspace{3cm}

\question[1] De 40 lanzamientos de penalti que ha realizado David, ha metido 18. ¿Qué porcentaje de aciertos tiene David?

\vspace{3cm}

\question[1\half] El precio de un televisor ha subido un 25\% con relación al del año pasado. ¿Cuál es su precio actual si el año pasado era de 510,8 euros?

\vspace{3cm}

\end{questions}



\end{document}
\grid
