\documentclass[addpoints,spanish, 12pt,a4paper]{exam}
%\documentclass[answers, spanish, 12pt,a4paper]{exam}
% \printanswers
\renewcommand*\half{.5}

\pointpoints{punto}{puntos}
\hpword{Puntos:}
\vpword{Puntos:}
\htword{Total}
\vtword{Total}
\hsword{Resultado:}
\hqword{Ejercicio:}
\vqword{Ejercicio:}

\usepackage{booktabs} 
\usepackage[utf8]{inputenc}
\usepackage[spanish]{babel}
\usepackage{eurosym}
%\usepackage[spanish,es-lcroman, es-tabla, es-noshorthands]{babel}


\usepackage[margin=1in]{geometry}
\usepackage{amsmath,amssymb}
\usepackage{multicol}
\usepackage{yhmath}

\pointsinrightmargin % Para poner las puntuaciones a la derecha. Se puede cambiar. Si se comenta, sale a la izquierda.
\extrawidth{-2.4cm} %Un poquito más de margen por si ponemos textos largos.
\marginpointname{ \emph{\points}}

\usepackage{graphicx}

\graphicspath{{../img/}} 

\newcommand{\class}{1 ESO}
\newcommand{\examdate}{\today}
\newcommand{\examnum}{Geometría y álgebra}
\newcommand{\tipo}{A}


\newcommand{\timelimit}{45 minutos}

\renewcommand{\solutiontitle}{\noindent\textbf{Solución:}\enspace}


\pagestyle{head}
\firstpageheader{\includegraphics[width=0.2\columnwidth]{header_left}}{\textbf{Departamento de Matemáticas\linebreak \class}\linebreak \examnum}{\includegraphics[width=0.1\columnwidth]{header_right}}
\runningheader{\class}{\examnum}{Página \thepage\ de \numpages}
\runningheadrule


\usepackage{pgf,tikz,pgfplots}
\pgfplotsset{compat=1.15}
\usepackage{mathrsfs}
\usetikzlibrary{arrows}


\begin{document}

\noindent
\begin{tabular*}{\textwidth}{l @{\extracolsep{\fill}} r @{\extracolsep{6pt}} }
\textbf{Nombre:} \makebox[3.5in]{\hrulefill} & \textbf{Fecha:}\makebox[1in]{\hrulefill} \\
 & \\
\textbf{Tiempo: \timelimit} & Tipo: \tipo 
\end{tabular*}
\rule[2ex]{\textwidth}{2pt}
\textbf{Instrucciones:} Justifica todos los
resultados.
Esta prueba tiene \numquestions\ ejercicios con una puntuación máxima de \numpoints. 
La nota del examen se calculará de manera proporcional a la puntuación obtenida. 
\textbf{Para recuperar el examen anterior se tendrán en cuenta las preguntas 1 a 4}\\
Está permitido el uso de calculadora y los ejercicios han de resolverse utilizando la notación algebraica adecuada


\begin{center}


\addpoints
 %\gradetable[h][questions]
	\pointtable[h][questions]
   
\end{center}
% \textbf{Nota:} Se tendrán en cuenta los ejercicios 1 y 2 para recuperar contenidos previos.

\noindent
\rule[2ex]{\textwidth}{2pt}



\begin{questions}

% \question
% Expresa de forma algebraica los siguientes enunciados matemáticos:
% \begin{parts}
% \part[1] La suma de un número, \( a \), y su mitad.
% \vspace{30pt}
% \part[1] El triple de la mitad de un número, \( n \).
% \vspace{30pt}
% \part[1] El área de un cuadrado de lado \( a \).
% \vspace{30pt}
% \end{parts}

% \question[1]
% Rodea con un círculo aquellas expresiones algebraicas que sean monomios:
% \[
% 6x^3 + 3y^4 \quad\quad 6ab \quad\quad 5xyz \quad\quad 7y^5 + 4x^3 \quad\quad 2y^3
% \]
% \vspace{30pt}

\question[1]
Completa la tabla indicando el coeficiente, la parte literal y el grado de cada monomio:
\vspace{10pt}

\begin{tabular}{|c|c|c|c|}
\hline
\textbf{Monomio} & \textbf{Coeficiente} & \textbf{Parte literal} & \textbf{Grado} \\
\hline
2 & & & \\
\hline
$5x^2$ & & & \\
\hline
$\frac{2}{3}ab$ & & & \\
\hline
$-x^3y$ & & & \\
\hline
% $-ab^3x$ & & & \\
% \hline
\end{tabular}


% \question[1]
% Rodea con un círculo los monomios que sean semejantes:
% \[
% 3a^2b^3x \quad\quad 4a^2b^3x \quad\quad -2a^2b^3x \quad\quad 5a^2b^2x^2
% \]

\question[1]
Realiza las siguientes operaciones con monomios:
\begin{multicols}{2}
\begin{parts}
\part \( 3x^2 + 5x^2 \)
% \vspace{30pt}
% \part \( 7a^3 - 2a^3 \)
% \vspace{30pt}
\part \( (2x)(-4x^3) \)
% \vspace{30pt}
% \part $\frac{6x^5}{2x^2}$
% \vspace{30pt}
\end{parts}
\end{multicols}

% \question[1]
% Calcula el valor numérico de la siguiente expresión algebraica sabiendo que \( x = 2 \) y \( y = -3 \):
% \[
% 4x^2 - 2xy + y^2
% \]
% \vspace{30pt}

% \question
% Rodea, en cada caso, el valor de \( x \) que es solución de la ecuación:
% \begin{parts}
% \part[1] \( 2x + 5 = 25 \quad \Rightarrow \quad x = 5 \quad x = 10 \quad x = 15 \quad x = 20 \)
% \part[1] \( 3x - 4 = 14 \quad \Rightarrow \quad x = 2 \quad x = 4 \quad x = 6 \quad x = 8 \)
% \end{parts}

% \question
% Resuelve las siguientes ecuaciones:
% \begin{parts}
% \part[1] \( x + 6 = 15 \)
% \vspace{30pt}
% \part[1] \( x - 4 = 9 \)
% \vspace{30pt}
% \part[1] \( \frac{x}{3} = 2 \)
% \vspace{30pt}
% \part[1] \( \frac{x}{2} = 12 \)
% \vspace{30pt}
% \end{parts}

\question
Resuelve las siguientes ecuaciones:
\begin{parts}
\part[1] \( x + 13 = 2x + 11 \)
\vspace{40pt}
\part[1] \( x + 10 = 3x + 4 \)
\vspace{40pt}
\end{parts}

% \question
% Resuelve las siguientes ecuaciones:
% \begin{parts}
% \part[1] \( (5x - 3) + (7x + 6) = 13 \)
% \vspace{80pt}
% \part[1] \( (2x + 4) + (1 - x) = 14 \)
% \vspace{80pt}
% \end{parts}

\question[2]
La suma de tres números consecutivos es 42. ¿Cuáles son esos números?
\vspace{50pt}

% \question[2]
% Juan tiene 25 euros más que Mario y 30 euros menos que Enrique. ¿Cuánto tiene cada uno sabiendo que entre los tres tienen 140 euros?
% \vspace{40pt}

\question[2] Suma los siguientes ángulos:
 \begin{multicols}{2}
\begin{parts}
   
         \part $35^\circ 48' + 27^\circ 35'$
  \begin{solution}
    $35^\circ 48' + 27^\circ 35' = 63^\circ 83' = 64^\circ 23'$
  \end{solution}
  \vspace{30pt}
  \part $15^\circ 47' 30'' + 28^\circ 32' 50''$
  \begin{solution}
    Suma de segundos: $30'' + 50'' = 80'' = 1' 20''$  
    Suma de minutos: $47' + 32' + 1' = 80' = 1^\circ 20'$  
    Suma de grados: $15^\circ + 28^\circ + 1^\circ = 44^\circ$  
    Resultado final: $44^\circ 20' 20''$
  \end{solution}
  \vspace{30pt}
    \end{parts}
    \end{multicols}


\question[2] Resta los siguientes ángulos:
\begin{multicols}{2}
\begin{parts}
  \part $63^\circ 25' - 27^\circ 49'$
  \begin{solution}
    Como $25' < 49'$, tomamos prestado 1 grado:  
    $62^\circ 85' - 27^\circ 49' = 35^\circ 36'$
  \end{solution}
  \vspace{30pt}
  
  \part $52^\circ 40' 25'' - 18^\circ 52' 50''$
  \begin{solution}
    Convertimos para poder restar:  
    Tomamos 1' prestado: $52^\circ 39' 85''$  
    Luego, tomamos 1° prestado: $51^\circ 99' 85''$  
    Resta: $51^\circ 99' 85'' - 18^\circ 52' 50'' = 33^\circ 47' 35''$
  \end{solution}
  \vspace{30pt}
\end{parts}
\end{multicols}



\question[2] El lado mayor de un triángulo rectángulo mide 15 cm y uno de los dos lados menores mide 9 cm. ¿Cuánto mide el tercer lado? (se recomienda hacer el dibujo)
\vspace{60pt}




\question[2] Si los lados de un rectángulo miden, respectivamente, 16 cm y 30 cm, ¿cuánto mide su diagonal? (se recomienda hacer el dibujo)
\vspace{60pt}


% \question[1] El perímetro de un rombo es de 40 cm y una de sus diagonales mide 16 cm. ¿Cuánto mide la otra diagonal?
% \vspace{30pt}

\question[2]  Si en la figura siguiente $a=10$cm , calcula b \\
\begin{tikzpicture}[scale=1.2]
  % Coordenadas clave
  \coordinate (A) at (0,0);        % Esquina inferior izquierda
  \coordinate (B) at (0,2);        % Esquina superior izquierda
  \coordinate (C) at (2,2);        % Punto de unión cuadrado-triángulo arriba
  \coordinate (D) at (4,0);        % Esquina inferior derecha
  \coordinate (E) at (2,0);        % Punto de unión cuadrado-triángulo abajo

  % Cuadrado y triángulo
  \draw[thick] (A) -- (B) -- (C) -- (D) -- cycle;
  \draw[dashed] (C) -- (E);

  % Etiquetas
  \draw[-] (0,1) -- +(0,0) node[left] {\(a\)};
  \draw[-] (1,2) -- +(0,0) node[above] {\(a\)};
  \draw[-] (3,0) -- +(0,0) node[below] {\(a\)};
  \draw[-] (3.1,1.1) -- +(0,0) node[right] {\(b\)};
  \draw[-] (0,-0.4) -- (4,-0.4) node[midway,below] {\(2a\)};
  
\end{tikzpicture}





\end{questions}



\end{document}
\grid
