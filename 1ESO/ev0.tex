\documentclass{exam}
\usepackage{amsmath, amsthm, amssymb, yhmath} 
% \printanswers

\usepackage{tikz}
\usetikzlibrary{angles,quotes} % Para los ángulos

\begin{document}

\begin{center}
\bfseries Prueba inicial
\end{center}
\textbf{Nombre:} \\
\textbf{Fecha:} \\
\textbf{Curso:} \\
\hline

\begin{questions}

\question Escribe como se leen los siguientes números:
\begin{parts}
    \part $999999$

    \part $12356478$

\end{parts}

\question Aproxima los siguientes números:
\begin{parts}
    \part $53678$ a las centenas
    \part $53,678$ a las décimas
    \part $53,678$ a las unidades
\end{parts}

\question Responde a las siguientes cuestiones:
\begin{parts}
       \part Si un libro tiene 120 páginas y un estudiante lee 15 páginas al día, ¿cuántos días le llevará terminar el libro? 
           \vspace{20pt}
               \vspace{20pt}
       \part En una caja hay 250 lápices. Si cada paquete contiene 10 lápices, ¿cuántos paquetes hay en total?
           \vspace{20pt}
               \vspace{20pt}
       \part En una librería, hay 6 estantes, y cada estante contiene 25 libros. Si la librería recibe 3 cajas adicionales de libros, y cada caja contiene 15 libros, ¿cuántos libros hay en total en la librería después de recibir las cajas?
           \vspace{20pt}    \vspace{20pt} \vspace{20pt}
\end{parts}




\question Resuelve las siguientes operaciones combinadas. Asegúrate de seguir el orden de las operaciones:

\begin{parts}
    \part \( 8 + 3 \times 2 - 5 = \)
               \vspace{20pt}    \vspace{20pt}

    \part \( (12 - 4) \times 3 + 10 = \)
               \vspace{20pt}    \vspace{20pt}

    \part \( (6 : 2) \times 4 + 15 =\)
           \vspace{20pt}    \vspace{20pt}

    \part \( 2 + 5 \times 4  - 7 = \)
           \vspace{20pt}    \vspace{20pt}

    \part \( 20 - 5 + (8 : 2) \times 3 =\)

    \vspace{20pt}    \vspace{20pt}
    \vspace{20pt}    \vspace{20pt}

\end{parts}

\question Escribe y resuelve la operación combinada que de respuesta al siguiente problema: En una tienda, un cliente compra 3 paquetes de galletas que cuestan 4 euros cada uno y 2 botellas de refresco que cuestan 2 euros cada una. Si el cliente paga con un billete de 20 euros, ¿cuánto dinero le devolverán?

\end{questions}


\end{document}

