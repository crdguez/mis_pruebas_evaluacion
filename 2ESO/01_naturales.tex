\documentclass[addpoints,spanish, 12pt,a4paper]{exam}
%\documentclass[answers, spanish, 12pt,a4paper]{exam}
% \printanswers
\renewcommand*\half{.5}

\pointpoints{punto}{puntos}
\hpword{Puntos:}
\vpword{Puntos:}
\htword{Total}
\vtword{Total}
\hsword{Resultado:}
\hqword{Ejercicio:}
\vqword{Ejercicio:}

\usepackage[utf8]{inputenc}
\usepackage[spanish]{babel}
\usepackage{eurosym}
%\usepackage[spanish,es-lcroman, es-tabla, es-noshorthands]{babel}


\usepackage[margin=1in]{geometry}
\usepackage{amsmath,amssymb}
\usepackage{multicol}
\usepackage{yhmath}

\pointsinrightmargin % Para poner las puntuaciones a la derecha. Se puede cambiar. Si se comenta, sale a la izquierda.
\extrawidth{-2.4cm} %Un poquito más de margen por si ponemos textos largos.
\marginpointname{ \emph{\points}}

\usepackage{graphicx}

\graphicspath{{../img/}} 

\newcommand{\class}{2 ESO}
\newcommand{\examdate}{\today}
\newcommand{\examnum}{Naturales y enteros}
\newcommand{\tipo}{A}


\newcommand{\timelimit}{45 minutos}

\renewcommand{\solutiontitle}{\noindent\textbf{Solución:}\enspace}


\pagestyle{head}
\firstpageheader{\includegraphics[width=0.2\columnwidth]{header_left}}{\textbf{Departamento de Matemáticas\linebreak \class}\linebreak \examnum}{\includegraphics[width=0.1\columnwidth]{header_right}}
\runningheader{\class}{\examnum}{Página \thepage\ de \numpages}
\runningheadrule


\usepackage{pgf,tikz,pgfplots}
\pgfplotsset{compat=1.15}
\usepackage{mathrsfs}
\usetikzlibrary{arrows}


\begin{document}

\noindent
\begin{tabular*}{\textwidth}{l @{\extracolsep{\fill}} r @{\extracolsep{6pt}} }
\textbf{Nombre:} \makebox[3.5in]{\hrulefill} & \textbf{Fecha:}\makebox[1in]{\hrulefill} \\
 & \\
\textbf{Tiempo: \timelimit} & Tipo: \tipo 
\end{tabular*}
\rule[2ex]{\textwidth}{2pt}
\textbf{Instrucciones:} Prohibido el uso de calculadora. Justifica los
resultados.
Esta prueba tiene \numquestions\ ejercicios con una puntuación máxima de \numpoints. 
La nota del examen se calculará de manera proporcional a la puntuación obtenida. 

\begin{center}


\addpoints
 %\gradetable[h][questions]
	\pointtable[h][questions]
\end{center}

\noindent

\textbf{Pendientes:} Se tendrán en cuenta los apartados del ejercicio 1, 2 y 4
\rule[2ex]{\textwidth}{2pt}

\begin{questions}

%\question 
%
%\begin{parts}
%\part[2] 
%\begin{solution}
%\end{solution}
%
%
%\end{parts}
%\addpoints

\question Responde a las siguientes cuestiones:
\begin{parts}
\part[1] Los cuatro primeros múltiplos de 19. \vspace{40pt}
\begin{solution}
19, 38, 57, 76
\end{solution}

\part[2] Todos los divisores de 60. \vspace{60pt}
\begin{solution}
1, 2, 3, 4, 5, 6, 10, 12, 15, 20, 30, 60
\end{solution}

\part[2] El primer múltiplo de 23 después de 1000.\vspace{60pt}
\begin{solution}
\(1000 \div 23 \approx 43.48 \Rightarrow 23 \times 44 = 1012\)
\end{solution}

\part[1] Indica cuáles de estos números son múltiplos de 2, 3, 5 y 10:\\
842, 675, 1280, 3471, 5900, 7314, 9000 \vspace{60pt}
\begin{solution}
\begin{itemize}
  \item Múltiplos de 2: 842, 1280, 5900, 7314, 9000
  \item Múltiplos de 3: 675, 3471, 7314, 9000
  \item Múltiplos de 5: 675, 1280, 5900, 9000
  \item Múltiplos de 10: 1280, 5900, 9000
\end{itemize}
\end{solution}

\part[2] \( \text{max.c.d.}(36, 60) \)\vspace{60pt}
\begin{solution}
\(2^2 \cdot 3 = 12\)
\end{solution}

\part[2] \( \text{min.c.m.}(36, 60) \)\vspace{80pt}
\begin{solution}
\(2^3 \cdot 3 \cdot 5 = 180\)
\end{solution}
\end{parts}

\question[3] En un taller suena una alarma de seguridad cada 36 segundos, y una campana de aviso cada 48 segundos. Si ambos sonidos se escuchan al mismo tiempo ahora, ¿cuánto tardarán en coincidir de nuevo?\vspace{100pt}

\begin{solution}
$$
\text{m.c.m.}(36, 48) = 144 \Rightarrow \text{Coinciden cada } 144 \text{ segundos = 2 minutos y 24 segundos.}
$$
\end{solution}

\question[3] Una tienda de juguetes pone a la venta una partida de peluches, todos del mismo precio. El primer día vende varios por valor de 198 \euro y el segundo día, más por valor de 234 \euro. ¿Cuál crees que es el precio de un peluche?\vspace{100pt}
\begin{solution}
\[
198 = 18 \cdot 11,\quad 234 = 21 \cdot 11 \Rightarrow \text{m.c.d.} = 11
\]
El precio de un peluche es **11 €**.
\end{solution}


\question Calcula:
\begin{parts}
    \part[1] \textbf{3 · 9 + 7 + 6 - 5 · 3=}\vspace{20pt}\vspace{20pt}
    \part[1] \textbf{5 · (2 + 6) + 7 - 4 · 3=}\vspace{20pt}\vspace{20pt}    % \part \textbf{3 + 2 · {[}3 · (2 · 5 - 7 + 3){]} - 2 · 3=} \vspace{80pt}
    % \part \textbf{5 · {[}3 + 1 + 2 · (19 - 4 · 3 + 2){]} + 2=}\vspace{80pt}
\end{parts}


\question Calcular, paso a paso, las siguientes operaciones combinadas:

\begin{parts}

\part[2] \( 6 - (-3) \cdot [ -4 - 2 \cdot (1 - 6)] = \)

\begin{solution}
\[
6 - (-3) \cdot [ -4 - 2 \cdot (1 - 6)] = 6 - (-3) \cdot [ -4 - 2 \cdot (-5)]  = 6 + 18 = 24
\]
\end{solution} \vspace{40pt}

\part[2] \( (-5) \cdot [10 + 2 \cdot (7 - 9)] = \)

\begin{solution}
\[
(-5) \cdot [10 + 2 \cdot (7 - 9)] = (-5) \cdot [10 + 2 \cdot (-2)] = (-5) \cdot [10 - 4] = (-5) \cdot 6 = -30
\]
\end{solution} \vspace{40pt}

\part[2] \( 7 \cdot (-3) - 4 \cdot (-6) = \)

\begin{solution}
\[
7 \cdot (-3) - 4 \cdot (-6) = -21 + 24 = 3
\]
\end{solution} \vspace{40pt}

\part[2] \( 5 - (-2) + (-3) \cdot [ (1) + (-4) - 3] = \)

\begin{solution}
\[
5 - (-2) + (-3) \cdot [1 - 4 - 3] = 5 + 2 + (-3) \cdot (-6) = 7 + 18 = 25
\]
\end{solution} \vspace{40pt}

% \part[2] \( 10 - [ (4 - 9) + (1 - 4) - (6 - 8)] = \)

% \begin{solution}
% \[
% 10 - [ -5 + (-3) - (-2)] = 10 - [-5 - 3 + 2] = 10 - (-6) = 16
% \]
% \end{solution} \vspace{40pt}

\part[2] \( (15 - 18) - [ (3 - 7 + 10) - 2 \cdot (4 - 6)] = \)

\begin{solution}
\[
(15 - 18) - [ (3 - 7 + 10) - 2 \cdot (-2)] = (-3) - [6 + 4] = -3 - 10 = -13
\]
\end{solution} \vspace{80pt}

\part[2] \( \left[3 \cdot (6 - 8) \cdot (-2) - (9 - 3) : (-3)\right] \cdot (-2) = \)

\begin{solution}
\[
\left[3 \cdot (-2) \cdot (-2) - 6 : (-3)\right] \cdot (-2)  = 14 \cdot (-2) = -28
\]
\end{solution} \vspace{80pt}

\end{parts}






% \question[2] En un supermercado compramos tres productos para
%   un restaurante: 6 kg de carne de cerdo a 4 \euro/kg, 4 kg de merluza a 4
%   \euro/kg y 3 kg de carne de pollo al mismo precio que la carne y la
%   merluza. Si pago con un billete de 100 \euro, ¿cuánto me devuelven si el
%   encargado del supermercado me hace una rebaja de 5 \euro \ por cada producto
%   comprado? Halla la solución mediante una expresión de operaciones
%   combinadas. \vspace{100pt}

% \question[2] En una granja se recogen los huevos dos días a la
%   semana. El primer día se llenaron 12 cajas de 16 bandejas de huevos;
%   el segundo día se completaron 22 cajas de 15 bandejas cada una. Si en
%   cada bandeja entran dos docenas y media de huevos, ¿cuántos huevos se
%   han recogido durante esta semana? Halla la solución mediante una
%   expresión de operaciones combinadas


%%%
%%% OTRO EXAMEN
%%%
% \newpage
% \question[1\half] Escribe con cifras:
% \begin{parts}
% \part Seis millones setecientos cincuenta mil
% \vspace{20pt}
% \begin{solution}
% 6.750.000
% \end{solution}

% \part Doscientos cuarenta y cinco millones ochocientos doce mil
% \vspace{20pt}
% \begin{solution}
% 245.812.000
% \end{solution}

% \part Tres mil quinientos dieciséis
% \vspace{20pt}
% \begin{solution}
% 3.516
% \end{solution}
% \end{parts}

% \question[1\half] Escribe como se leen las siguientes cifras:
% \begin{parts}
%     \part 8 500 000 000
%     \vspace{30pt}
%     \begin{solution}
%     Ocho mil quinientos millones
%     \end{solution}
    
%     \part 5 600 000 010 000 000
%     \vspace{30pt}
%     \begin{solution}
%     Cinco billones seiscientos mil millones diez mil
%     \end{solution}
    
%     \part 7 385 000 765 000
%     \vspace{30pt}
%     \begin{solution}
%     Siete billones trescientos ochenta y cinco mil millones setecientos sesenta y cinco mil
%     \end{solution}
% \end{parts}

% \question[1] Aproxima a las decenas de millar, por redondeo, los siguientes números: 
% \begin{parts}
%     \part 728 345
%     \begin{solution}
%     730.000
%     \end{solution}
    
%     \part 1 902 678
%     \begin{solution}
%     1.900.000
%     \end{solution}
    
%     \part 8 543 210
%     \begin{solution}
%     8.540.000
%     \end{solution}
    
%     \part  32 567 890
%     \begin{solution}
%     32.570.000
%     \end{solution}
% \end{parts}

% \question[2] Calcula:
% \begin{parts}
%     \part \textbf{4 · 7 + 8 - 5 · 4 =}
%     \vspace{20pt}
%     \begin{solution}
%     4 · 7 + 8 - 5 · 4 = 28 + 8 - 20 = 16
%     \end{solution}

%     \part \textbf{6 · (3 + 4) - 8 + 7 · 2 =}
%     \vspace{20pt}
%     \begin{solution}
%     6 · (3 + 4) - 8 + 7 · 2 = 6 · 7 - 8 + 14 = 42 - 8 + 14 = 48
%     \end{solution}

%     \part \textbf{4 + 3 · {[}2 · (6 - 3 + 4){]} - 4 · 2 =} 
%     \vspace{80pt}
%     \begin{solution}
%     4 + 3 · {[}2 · (6 - 3 + 4){]} - 4 · 2 = 4 + 3 · (2 · 7) - 8 = 4 + 3 · 14 - 8 = 4 + 42 - 8 = 38
%     \end{solution}

%     \part \textbf{6 · {[}4 + 2 + 3 · (18 - 5 · 2 + 1){]} + 3 =}
%     \vspace{80pt}
%     \begin{solution}
%     6 · {[}4 + 2 + 3 · (18 - 10 + 1){]} + 3 = 6 · {[}4 + 2 + 3 · 9{]} + 3 = 6 · (4 + 2 + 27) + 3 = 6 · 33 + 3 = 198 + 3 = 201
%     \end{solution}
% \end{parts}

% \question[2] En una tienda de comestibles compramos tres productos: 5 kg de arroz a 2 \euro/kg, 3 kg de lentejas a 3 \euro/kg, y 4 kg de garbanzos al mismo precio que el arroz y las lentejas. Si pagamos con un billete de 50 \euro, ¿cuánto nos devuelven si el encargado de la tienda nos hace una rebaja de 3 \euro por cada producto comprado? Halla la solución mediante una expresión de operaciones combinadas. 
% \vspace{100pt}
% \begin{solution}
% El precio total sin rebajas sería:

% \[ 5 \text{kg} \cdot 2 \euro/\text{kg} + 3 \text{kg} \cdot 3 \euro/\text{kg} + 4 \text{kg} \cdot 3 \euro/\text{kg} = 10 \euro + 9 \euro + 12 \euro = 31 \euro \]

% Con la rebaja de 3 \euro por cada producto comprado (3 productos):

% \[ 31 \euro - (3 \euro \times 3) = 31 \euro - 9 \euro = 22 \euro \]

% Pagamos con 50 \euro, así que nos devuelven:

% \[ 50 \euro - 22 \euro = 28 \euro \]
% \end{solution}

% \question[2] En una finca se recogen manzanas tres días a la semana. El primer día se llenaron 10 cestas con 12 kilos de manzanas; el segundo día se completaron 15 cestas con 14 kilos cada una, y el tercer día se recogieron 8 cestas con 16 kilos cada una. ¿Cuántos kilos de manzanas se han recogido durante la semana? Halla la solución mediante una expresión de operaciones combinadas.
% \vspace{100pt}
% \begin{solution}
% Calculamos los kilos totales recogidos cada día y sumamos:

% Primer día: 
% \[ 10 \text{cestas} \cdot 12 \text{kg} = 120 \text{kg} \]

% Segundo día: 
% \[ 15 \text{cestas} \cdot 14 \text{kg} = 210 \text{kg} \]

% Tercer día: 
% \[ 8 \text{cestas} \cdot 16 \text{kg} = 128 \text{kg} \]

% El total de kilos recogidos durante la semana es:
% \[ 120 \text{kg} + 210 \text{kg} + 128 \text{kg} = 458 \text{kg} \]
% \end{solution}

% BORRAR


% % ----------------------
% % PREGUNTA 1
% % ----------------------
% \question Escribe:
% \begin{parts}
% \part Los cuatro primeros múltiplos de 17.
% \begin{solution}
% 17, 34, 51, 68
% \end{solution}

% \part Todos los divisores de 72.
% \begin{solution}
% 1, 2, 3, 4, 6, 8, 9, 12, 18, 24, 36, 72
% \end{solution}
% \end{parts}

% % ----------------------
% % PREGUNTA 2
% % ----------------------
% \question Busca:
% \begin{parts}
% \part El primer múltiplo de 17 después de 1000.
% \begin{solution}
% \(1000 \div 17 \approx 58.82 \Rightarrow 17 \times 59 = 1003\)
% \end{solution}

% \part Un número de dos cifras que sea divisor de 415.
% \begin{solution}
% Ejemplos: 5 y 83
% \end{solution}
% \end{parts}

% % ----------------------
% % PREGUNTA 3
% % ----------------------
% \question Escribe los números primos comprendidos entre 20 y 40.
% \begin{solution}
% 23, 29, 31, 37
% \end{solution}

% % ----------------------
% % PREGUNTA 4
% % ----------------------
% \question Indica cuáles de estos números son múltiplos de 2, 3, 5 y 10:\\
% 897, 765, 990, 2713, 6077, 6324, 7005
% \begin{solution}
% \begin{itemize}
%   \item Múltiplos de 2: 990, 6324
%   \item Múltiplos de 3: 897, 765, 990, 6324, 7005
%   \item Múltiplos de 5: 765, 990, 7005
%   \item Múltiplos de 10: 990
% \end{itemize}
% \end{solution}

% % ----------------------
% % PREGUNTA 5
% % ----------------------
% \question Copia en tu cuaderno y descompón en factores primos los números 150 y 225.
% \begin{solution}
% \[
% 150 = 2 \cdot 3 \cdot 5^2,\quad 225 = 3^2 \cdot 5^2
% \]
% \end{solution}

% % ----------------------
% % PREGUNTA 6
% % ----------------------
% \question Calcula:
% \begin{parts}
% \part \( \text{m.c.d.}(150, 225) \)
% \begin{solution}
% \(3 \cdot 5^2 = 75\)
% \end{solution}

% \part \( \text{m.c.m.}(150, 225) \)
% \begin{solution}
% \(2 \cdot 3^2 \cdot 5^2 = 450\)
% \end{solution}
% \end{parts}

% % ----------------------
% % PREGUNTA 7
% % ----------------------
% \question Calcula:
% \begin{parts}
% \part \(6 - 11 + (9 - 13)\)
% \begin{solution}
% \(-5 + (-4) = -9\)
% \end{solution}

% \part \(2 - (5 - 8)\)
% \begin{solution}
% \(2 - (-3) = 5\)
% \end{solution}

% \part \((7 - 15) - (6 - 2)\)
% \begin{solution}
% \(-8 - 4 = -12\)
% \end{solution}

% \part \(5 - [2 - (3 - 2)]\)
% \begin{solution}
% \(5 - [2 - 1] = 5 - 1 = 4\)
% \end{solution}
% \end{parts}

% % ----------------------
% % PREGUNTA 8
% % ----------------------
% \question Calcula:
% \begin{parts}
% \part \(4 \cdot 5 - 3 \cdot (-2) + 5 \cdot (-8) - 4 \cdot (-3)\)
% \begin{solution}
% \(20 + 6 - 40 + 12 = -2\)
% \end{solution}

% \part \((10 - 3 \cdot 6) - 2 \cdot [5 + 3 \cdot (4 - 7)]\)
% \begin{solution}
% \(-8 - (-8) = 0\)
% \end{solution}

% \part \(10 - 10 \cdot [-6 + 5 \cdot (-4 + 7 - 3)]\)
% \begin{solution}
% \(10 - (-60) = 70\)
% \end{solution}
% \end{parts}

% % ----------------------
% % PREGUNTA 9
% % ----------------------
% \question Reduce a una sola potencia:
% \begin{parts}
% \part \(a^3 : b^3\)
% \begin{solution}
% \(\left(\dfrac{a}{b}\right)^3\)
% \end{solution}

% \part \(a^5 : b^5\)
% \begin{solution}
% \(\left(\dfrac{a}{b}\right)^5\)
% \end{solution}

% \part \(a^4 \cdot a^2\)
% \begin{solution}
% \(a^6\)
% \end{solution}

% \part \(x^6 \cdot x^4\)
% \begin{solution}
% \(x^{10}\)
% \end{solution}

% \part \((x^3)^3\)
% \begin{solution}
% \(x^9\)
% \end{solution}

% \part \((-5)^7 : (-5)^5\)
% \begin{solution}
% \((-5)^2 = 25\)
% \end{solution}
% \end{parts}

% % ----------------------
% % PREGUNTA 10
% % ----------------------
% \question Una tienda de ropa pone a la venta una partida de camisetas, todas del mismo precio. El primer día vende unas cuantas por valor de 221 \euro y el segundo día, unas cuantas más por valor de 272 \euro. \\
% ¿Cuál crees que es el precio de una camiseta?
% \begin{solution}
% \[
% 221 = 13 \cdot 17,\quad 272 = 16 \cdot 17 \Rightarrow \text{m.c.d.} = 17
% \]
% El precio de una camiseta es **17 €**.
% \end{solution}

% % ----------------------
% % PREGUNTA 11
% % ----------------------
% \question Se desea dividir un terreno rectangular, de 100 m de ancho por 120 m de largo, en parcelas cuadradas lo más grandes que sea posible. ¿Cuánto debe medir el lado de cada parcela?
% \begin{solution}
% $$
% \text{m.c.d.}(100, 120) = 20 \Rightarrow \text{Cada parcela mide } 20 \text{ m de lado.}
% $$\end{solution}

% % ----------------------
% % PREGUNTA 12
% % ----------------------
% \question En una fábrica se oye el escape de una válvula de gas cada 45 segundos, y el golpe de un martillo pilón cada 60 segundos. Si se acaban de oír ambos sonidos simultáneamente, ¿cuánto tardarán en coincidir de nuevo?
% \begin{solution}
% $$
% \text{m.c.m.}(45, 60) = 180 \Rightarrow \text{Coinciden cada } 180 \text{ segundos = 3 minutos.}
% $$
% \end{solution}

% % ----------------------
% % PREGUNTA 13
% % ----------------------
% \question Se van apilando, en una torre, cubos de 45 cm de arista y, al lado, en otra, cubos de 60 cm de arista. ¿A qué altura coinciden por tercera vez las cimas de ambas torres?
% \begin{solution}
% $$
% \text{m.c.m.}(45, 60) = 180,\quad 180 \times 3 = 540 \Rightarrow \text{Altura: } 540 \text{ cm}
% $$
% \end{solution}

% % ----------------------
% % PREGUNTA 14
% % ----------------------
% \question La suma de dos números enteros es 4, y la suma de sus valores absolutos, 16. ¿Qué números son?
% \begin{solution}
% Supongamos \( x + y = 4 \) y \( |x| + |y| = 16 \).\\
% Prueba con \( x = -6 \), \( y = 10 \):\\
% \[
% -6 + 10 = 4,\quad 6 + 10 = 16 \Rightarrow \text{Correcto.}
% \]
% Los números son \(-6\) y \(10\).
% \end{solution}


% \question[8] Calcular, paso a paso, las siguientes operaciones combinadas:

% \begin{parts}

% \part \( 4 - (-2) \cdot [ -8 - 3 \cdot (5 - 7)] \)=

% \begin{solution}
% \[
% 4 - (-2) \cdot [ -8 - 3 \cdot (5 - 7)] = 4 - (-2) \cdot [ -8 - 3 \cdot (-2)] = 4 - (-2) \cdot (-2) = 4 - 4 = 0
% \]
% \end{solution} \vspace{20pt}
% \part \( (-4) \cdot [12 + 3 \cdot (5 - 8)] = \)

% \begin{solution}
% \[
% (-4) \cdot [12 + 3 \cdot (5 - 8)] = (-4) \cdot [12 + 3 \cdot (-3)] = (-4) \cdot [12 - 9] = (-4) \cdot 3 = -12
% \]
% \end{solution}
%  \vspace{20pt}
% \part \( 4 \cdot (-6) - 8 \cdot (-4)= \)
% \begin{solution}
% \[
% 4 \cdot (-6) - 8 \cdot (-4) = -24 + 32 = -56
% \]
% \end{solution}
%  \vspace{20pt}
% \part \( 6 - (-3) + (-2) \cdot [ (2) + (-3) - 5] = \)

% \begin{solution}
% \[
% 6 - (-3) + (-2) \cdot [ (2) + (-3) - 5] = 6 + 3 + (-2) \cdot (-6) = 9 + 12 = 21
% \]
% \end{solution}
%  \vspace{20pt}
% \part \( 8 - [ (6-11) + (2-5) - (7-10)] = \)

% \begin{solution}
% \[
% 8 - [ (6-11) + (2-5) - (7-10)] = 8 - [ -5 + (-3) + 3] = 8 - (-5) = 13
% \]
% \end{solution}
%  \vspace{20pt}
% \part \( (13 - 21) - [ (2 - 5 + 8) - 2 \cdot (6 - 9)] = \)

% \begin{solution}
% \[
% (13 - 21) - [ (2 - 5 + 8) - 2 \cdot (6 - 9)] = (-8) - [5 - 2 \cdot (-3)] = -8 - [5 + 6] = -8 - 11 = -19
% \]
% \end{solution}
%  \vspace{20pt}
% \part \( [2 \cdot (-4) - (11 + 5)] : [ 2 \cdot (5 + 9 + 6) - (7 - 10)] = \)

% \begin{solution}
% \[
% [2 \cdot (-4) - (11 + 5)] : [ 2 \cdot (5 + 9 + 6) - (7 - 10)] = [-8 - 16] : [2 \cdot 20 - (-3)] = (-24) : [40 + 3] = (-24) : 43 = -\frac{24}{43}
% \]
% \end{solution}
%  \vspace{20pt}
% \part \( \left[2 \cdot (4 - 5) \cdot (-3) - (8 - 2) : (-3)\right] \cdot (-4) = \)

% \begin{solution}
% \[
% \left[2 \cdot (4 - 5) \cdot (-3) - (8 - 2) : (-3)\right] \cdot (-4) = \left[2 \cdot (-1) \cdot (-3) - 6 : (-3)\right] \cdot (-4) = \left[6 - (-2)\right] \cdot (-4) = [6 + 2] \cdot (-4) = 8 \cdot (-4) = -32
% \]
% \end{solution}
%  \vspace{20pt}
% \end{parts}

\newpage

\question[] (Solo si has acabado) Encuentra las siguientes palabras: múltiplo, divisor, natural, entero, divisible, negativo, opuesto \\ \newline 
\scalebox{0.9}{
\begin{tabular}{|c|c|c|c|c|c|c|c|c|c|c|}
\hline
P & D & Y & M & G & Y & K & H & C & H & T \\ \hline
N & D & I & V & I & S & O & R & K & D & L \\ \hline
M & E & Q & N & W & W & O & U & I & C & E \\ \hline
B & Ú & G & Q & W & Q & T & V & N & Q & N \\ \hline
Y & I & L & A & T & C & I & P & A & P & T \\ \hline
D & A & F & T & T & S & Z & Z & T & C & E \\ \hline
N & K & E & M & I & I & W & L & U & G & R \\ \hline
Y & V & Z & B & G & P & V & A & R & G & O \\ \hline
M & O & L & U & X & N & L & O & A & F & B \\ \hline
L & E & Q & S & P & E & A & O & L & X & I \\ \hline
K & T & S & V & O & P & U & E & S & T & O \\ \hline
\end{tabular}
}

\begin{solution}   
\begin{tabular}{|c|c|c|c|c|c|c|c|c|c|c|}
\hline
P & D & Y & M & G & Y & K & H & C & H & T \\ \hline
\textbf{N} & \textbf{D} & \textbf{I} & \textbf{V} & \textbf{I} & \textbf{S} & \textbf{O} & \textbf{R} & K & \textbf{D} & L \\ \hline
\textbf{M} & \textbf{E} & Q & N & W & W & O & U & \textbf{I} & C & \textbf{E} \\ \hline
B & \textbf{Ú} & \textbf{G} & Q & W & Q & T & \textbf{V} & \textbf{N} & Q & \textbf{N} \\ \hline
Y & I & \textbf{L} & \textbf{A} & T & C & \textbf{I} & P & \textbf{A} & P & \textbf{T} \\ \hline
D & A & F & \textbf{T} & \textbf{T} & \textbf{S} & Z & Z & \textbf{T} & C & \textbf{E} \\ \hline
N & K & E & M & \textbf{I} & \textbf{I} & W & L & \textbf{U} & G & \textbf{R} \\ \hline
Y & V & Z & \textbf{B} & G & \textbf{P} & \textbf{V} & A & \textbf{R} & G & \textbf{O} \\ \hline
M & O & \textbf{L} & U & X & N & \textbf{L} & \textbf{O} & \textbf{A} & F & B \\ \hline
L & \textbf{E} & Q & S & P & E & A & \textbf{O} & \textbf{L} & X & I \\ \hline
K & T & S & V & \textbf{O} & \textbf{P} & \textbf{U} & \textbf{E} & \textbf{S} & \textbf{T} & \textbf{O} \\ \hline
\end{tabular}\end{solution}


\end{questions}

\end{document}
\grid
