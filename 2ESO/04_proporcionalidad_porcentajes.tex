\documentclass[addpoints,spanish, 12pt,a4paper]{exam}
%\documentclass[answers, spanish, 12pt,a4paper]{exam}
\printanswers
\renewcommand*\half{.5}

\pointpoints{punto}{puntos}
\hpword{Puntos:}
\vpword{Puntos:}
\htword{Total}
\vtword{Total}
\hsword{Resultado:}
\hqword{Ejercicio:}
\vqword{Ejercicio:}

\usepackage[utf8]{inputenc}
\usepackage[spanish]{babel}
\usepackage{eurosym}
%\usepackage[spanish,es-lcroman, es-tabla, es-noshorthands]{babel}


\usepackage[margin=1in]{geometry}
\usepackage{amsmath,amssymb}
\usepackage{multicol}
\usepackage{yhmath}

\pointsinrightmargin % Para poner las puntuaciones a la derecha. Se puede cambiar. Si se comenta, sale a la izquierda.
\extrawidth{-2.4cm} %Un poquito más de margen por si ponemos textos largos.
\marginpointname{ \emph{\points}}

\usepackage{graphicx}

\graphicspath{{../img/}} 

\newcommand{\class}{2 ESO}
\newcommand{\examdate}{\today}
\newcommand{\examnum}{Proporcionalidad y porcentajes}
\newcommand{\tipo}{A}


\newcommand{\timelimit}{45 minutos}

\renewcommand{\solutiontitle}{\noindent\textbf{Solución:}\enspace}


\pagestyle{head}
\firstpageheader{\includegraphics[width=0.2\columnwidth]{header_left}}{\textbf{Departamento de Matemáticas\linebreak \class}\linebreak \examnum}{\includegraphics[width=0.1\columnwidth]{header_right}}
\runningheader{\class}{\examnum}{Página \thepage\ de \numpages}
\runningheadrule


\usepackage{pgf,tikz,pgfplots}
\pgfplotsset{compat=1.15}
\usepackage{mathrsfs}
\usetikzlibrary{arrows}


\begin{document}

\noindent
\begin{tabular*}{\textwidth}{l @{\extracolsep{\fill}} r @{\extracolsep{6pt}} }
\textbf{Nombre:} \makebox[3.5in]{\hrulefill} & \textbf{Fecha:}\makebox[1in]{\hrulefill} \\
 & \\
\textbf{Tiempo: \timelimit} & Tipo: \tipo 
\end{tabular*}
\rule[2ex]{\textwidth}{2pt}
\textbf{Instrucciones:}  Justifica los
resultados.
Esta prueba tiene \numquestions\ ejercicios con una puntuación máxima de \numpoints. 
La nota del examen se calculará de manera proporcional a la puntuación obtenida. 

\begin{center}


\addpoints
 %\gradetable[h][questions]
	\pointtable[h][questions]
\end{center}

\noindent

\textbf{Pendientes:} Se tendrán en cuenta los apartados del ejercicio  2, 3, 4 y 9
\textbf{Tema anterior:} Se tendrán en cuenta los apartados del ejercicio 1
\rule[2ex]{\textwidth}{2pt}

\begin{questions}

%\question 
%
%\begin{parts}
%\part[2] 
%\begin{solution}
%\end{solution}
%
%
%\end{parts}
%\addpoints









% \question Resuelve las siguientes operaciones:
% \begin{parts}
%     \part[1] $3^4:\sqrt{1+\left(20+6\cdot10\right)}=$
%     \vspace{50pt}
%     \part[1] $\left(6 + 2 \cdot 3^2 + 3 \cdot 2^2 \right) : \left(3 - \sqrt{81} \right)^2 =
% $ \begin{solution} $1$ \end{solution}\vspace{90pt}
% %     \part[1] $\left[ 9 - \sqrt{25} \cdot (-2)^3 \right] : \left[ (-3 - 1)^2 - 9 \right] =
% % $\begin{solution}$7$\end{solution}\vspace{100pt}


% %     \part[1\half] $\dfrac{\left[(-3)^3\right]^2 \cdot \left[3 \cdot (-9)\right]^6}{81^5} =
% % $\begin{solution}
% %     $3^4$
% % \end{solution} \vspace{100pt}
% \end{parts}



\question
Calcula y simplifica:
\begin{parts}
  \part[2] \( \left( 2 - \dfrac{1}{5} \right) \) + \( 7 - \dfrac{5}{12} =\) \vspace{50pt}
  % \part[2] \( \dfrac{3}{5} - \dfrac{2}{5} \cdot \left( 1 - \dfrac{1}{3} \right) - 3 \cdot \dfrac{2}{9}=\)\vspace{50pt}
  \part[2] \( \dfrac{7}{5} : \left[ \dfrac{3}{5} - 2 \left( 1 - \dfrac{4}{5} \right) \right]= \)\vspace{50pt}
  \part[2] $\dfrac{\left[ (-2)^4 \right]^6 : \left( (2^2 \cdot 8)^4 \right)}{\left( \dfrac{4}{3} \right)^8 : \left( \dfrac{4}{3} \right)^6 \cdot (-1)^8}=$ \vspace{50pt}

\end{parts}

\question[1] Escribe tres parejas de números cuya razón sea \( \tfrac{5}{7} \).
\begin{solution}
Algunas posibles parejas son: \((5,7), (10,14), (15,21)\).
\end{solution}
\vspace{50pt}

\question[1] Halla el término desconocido en cada proporción.
\begin{parts}

    \part \( \frac{9}{12} = \frac{27}{x} \)
    \begin{solution}
    \( x = 36 \)
    \end{solution}
    \vspace{50pt}

    \part \( \frac{32}{x} = \frac{8}{5} \)
    \begin{solution}
    \( x = 20 \)
    \end{solution}
    \vspace{50pt}

    % \part \( \frac{x}{18} = \frac{14}{21} \)
    % \begin{solution}
    % \( x = 12 \)
    % \end{solution}

\end{parts}

\question Completa la siguiente tabla.

\[
\begin{array}{c|cccc}
\text{MAGNITUD M} & 2 & 3 & 5 & 8 \\ \hline
\text{MAGNITUD N} & 30 & \, \ \, & \, \ \, & \, \ \
\end{array}
\]

\begin{parts}

    \part[1] Suponiendo que las magnitudes \( M \) y \( N \) son directamente proporcionales.
    \begin{solution}
    Constante de proporcionalidad \( k = \frac{30}{2} = 15 \).  
    Entonces: \(N = (30, 45, 75, 120)\).
    \end{solution}
    \vspace{50pt}

    \part[1] Suponiendo que las magnitudes \( M \) y \( N \) son inversamente proporcionales.
    \begin{solution}
    Producto constante: \( M \cdot N = 2 \cdot 30 = 60 \).  
    Entonces: \(N = (30, 20, 12, 7.5)\).
    \end{solution}
    \vspace{50pt}

\end{parts}

% \question Identifica si la siguiente tabla representa proporcionalidad directa o inversa.

% \[
% \begin{array}{c|ccccc}
% \text{MAGNITUD M} & 2 & 4 & 6 & 8 & 12 \\ \hline
% \text{MAGNITUD N} & 90 & 45 & 30 & 22.5 & 15
% \end{array}
% \]

% \begin{solution}
% El producto \( M \cdot N = 180 \) es constante.  
% Por tanto, es una **proporcionalidad inversa**.
% \end{solution}

% \question Completa la tabla, sabiendo que \( A \) y \( B \) son directamente proporcionales.

% \[
% \begin{array}{c|cccccc}
% A & 2 & 4 & 6 & 10 & 12 & 20 \\ \hline
% B & 5 & \,?\, & \,?\, & \,?\, & \,?\, & \,?\, 
% \end{array}
% \]

% \begin{solution}
% La constante de proporcionalidad es \( k = \frac{5}{2} = 2.5 \).  
% Así, \( B = (5, 10, 15, 25, 30, 50) \).
% \end{solution}

% \question Copia y completa esta tabla de proporcionalidad, sabiendo que la constante es \(0.6\).

% \[
% \begin{array}{c|ccc}
% \text{Magnitud A} & 10 & 25 & 40 \\ \hline
% \text{Magnitud B} & \,?\, & \,?\, & \,?\,
% \end{array}
% \]

% \begin{solution}
% Se multiplica cada valor de \(A\) por \(0.6\):  
% \( B = (6, 15, 24) \).
% \end{solution}


\question Resuelve por reducción a la unidad.

\begin{parts}

\part[2] Un manantial arroja 240 L de agua en 8 minutos.  
¿Cuántos litros arrojará en 20 minutos?

\begin{solution}
Primero hallamos los litros por minuto:  
\[
\frac{240}{8} = 30 \text{ L/min}
\]  
En 20 minutos arrojará:  
\[
30 \times 20 = 600 \text{ L}
\]
\end{solution}
\vspace{70pt}

\part[2] Abriendo 5 grifos, un depósito se vacía en 40 minutos.  
¿Cuánto tardará en vaciarse abriendo solo 3 grifos?

\begin{solution}
El tiempo es inversamente proporcional al número de grifos.  
Producto constante:  
\[
5 \times 40 = 200
\]  
Por tanto:  
\[
t = \frac{200}{3} \approx 66.7 \text{ min}
\]
\end{solution}
\vspace{70pt}

\end{parts}


\question Reparte una cantidad en forma proporcional.

\begin{parts}

\part[3] Tres socios (A, B y C) invierten 4 000 €, 6 000 € y 10 000 € respectivamente.  
Si al final obtienen un beneficio total de 12 000 €, ¿cuánto recibe cada uno?

\begin{solution}
La suma de las inversiones es:  
\[
4000 + 6000 + 10000 = 20000
\]

La proporción de cada socio es:

\[
\text{A: } \frac{4000}{20000} = 0.2,\qquad
\text{B: } \frac{6000}{20000} = 0.3,\qquad
\text{C: } \frac{10000}{20000} = 0.5
\]

Reparto del beneficio:

\[
\text{A: } 12000 \times 0.2 = 2400\ \text{€}
\]
\[
\text{B: } 12000 \times 0.3 = 3600\ \text{€}
\]
\[
\text{C: } 12000 \times 0.5 = 6000\ \text{€}
\]
\end{solution}
\vspace{60pt}

\end{parts}


% \question[1] Completa la siguiente tabla.

% \begin{solution}
% \[
% \begin{array}{|c|c|c|}
% \hline
% \textbf{Porcentaje} & \textbf{Fracción} & \textbf{Decimal} \\
% \hline
% 20\% & \frac{1}{5} & 0.2 \\
% \hline
% 75\% & \frac{3}{4} & 0.75 \\
% \hline
% \end{array}
% \]
% \end{solution}

\question[2] Calcula lo siguiente:

\begin{parts}
    \part 80\% de 150 
    \begin{solution}
    80\% de 150 es:
    \[
    0.80 \times 150 = 120
    \]
    \end{solution}
    \vspace{40pt}

    \part 5\% de 2000
    \begin{solution}
    5\% de 2000 es:
    \[
    0.05 \times 2000 = 100
    \]
    \end{solution}
    \vspace{40pt}

    % \part 25\% de 120
    % \begin{solution}
    % 25\% de 120 es:
    % \[
    % 0.25 \times 120 = 30
    % \]
    % \end{solution}

    \part 130\% de 45
    \begin{solution}
    130\% de 45 es:
    \[
    1.30 \times 45 = 58.5
    \]
    \end{solution}
    \vspace{40pt}

    \part 0,25\% de 400
    \begin{solution}
    0,25\% de 400 es:
    \[
    0.0025 \times 400 = 1
    \]
    \end{solution}
    \vspace{40pt}

    % \part 3,5\% de 60
    % \begin{solution}
    % 3,5\% de 60 es:
    % \[
    % 0.035 \times 60 = 2.1
    % \]
    % \end{solution}
\end{parts}

% \question Calcula \( x \) en cada caso:

% \begin{parts}
%     \part 60\% de \( x \) = 18
%     \begin{solution}
%     Para encontrar \( x \), se resuelve la ecuación:
%     \[
%     0.60 \times x = 18 \quad \Rightarrow \quad x = \frac{18}{0.60} = 30
%     \]
%     \end{solution}

%     \part 20\% de \( x \) = 10
%     \begin{solution}
%     Para encontrar \( x \), se resuelve la ecuación:
%     \[
%     0.20 \times x = 10 \quad \Rightarrow \quad x = \frac{10}{0.20} = 50
%     \]
%     \end{solution}

%     \part 30\% de \( x \) = 21
%     \begin{solution}
%     Para encontrar \( x \), se resuelve la ecuación:
%     \[
%     0.30 \times x = 21 \quad \Rightarrow \quad x = \frac{21}{0.30} = 70
%     \]
%     \end{solution}

%     \part 80\% de \( x \) = 16
%     \begin{solution}
%     Para encontrar \( x \), se resuelve la ecuación:
%     \[
%     0.80 \times x = 16 \quad \Rightarrow \quad x = \frac{16}{0.80} = 20
%     \]
%     \end{solution}

%     \part 10\% de \( x \) = 3
%     \begin{solution}
%     Para encontrar \( x \), se resuelve la ecuación:
%     \[
%     0.10 \times x = 3 \quad \Rightarrow \quad x = \frac{3}{0.10} = 30
%     \]
%     \end{solution}

%     \part 50\% de \( x \) = 25
%     \begin{solution}
%     Para encontrar \( x \), se resuelve la ecuación:
%     \[
%     0.50 \times x = 25 \quad \Rightarrow \quad x = \frac{25}{0.50} = 50
%     \]
%     \end{solution}
% \end{parts}

\question[2] La entrada de cine costaba 8,00 \euro, y hoy ha subido un 10\%. ¿Cuánto cuesta ahora?

\begin{solution}
Para calcular el nuevo precio después de la subida, multiplicamos el precio original por el incremento del 10\%:

\[
8,00 \, \text{\euro} \times (1 + 0,10) = 8,00 \, \text{\euro} \times 1,10 = 8,80 \, \text{€}
\]

Por lo tanto, el nuevo precio es 8,80 €.
\end{solution}
\vspace{50pt}

\question[2] Un juguete vale en una juguetería 40 euros. Durante las fiestas navideñas sube un 22\% y una vez que éstas han pasado, baja un 9\%. Calcular su precio final y el porcentaje de subida respecto al precio inicial.

\begin{solution}
Primero, calculamos el precio después de la subida del 22\%:
\[
\text{Precio después de la subida} = 40 \, \text{€} \times (1 + 0.22) = 40 \, \text{€} \times 1.22 = 48.8 \, \text{€}
\]

Luego, calculamos el precio después de la bajada del 9\%:
\[
\text{Precio después de la bajada} = 48.8 \, \text{€} \times (1 - 0.09) = 48.8 \, \text{€} \times 0.91 = 44.368 \, \text{€}
\]

Por lo tanto, el precio final del juguete es 44,37 €.

Ahora, calculamos el porcentaje de subida respecto al precio inicial. El aumento fue de:
\[
\text{Aumento} = 44.368 \, \text{€} - 40 \, \text{€} = 4.368 \, \text{€}
\]
El porcentaje de subida es:
\[
\frac{4.368 \, \text{€}}{40 \, \text{€}} \times 100 = 10.92\%
\]
Por lo tanto, el porcentaje de subida respecto al precio inicial es del 10.92\%.
\end{solution}
\vspace{70pt}

\question[2] Al rebajar el precio de un ordenador ha pasado de 1100 euros a 957 euros. ¿Qué tanto por ciento ha bajado?

\begin{solution}
La disminución en el precio es:
\[
\text{Disminución} = 1100 \, \text{€} - 957 \, \text{€} = 143 \, \text{€}
\]
El porcentaje de bajada es:
\[
\frac{143 \, \text{€}}{1100 \, \text{€}} \times 100 = 13\%
\]
Por lo tanto, el precio del ordenador ha bajado un 13\%.
\end{solution}
\vspace{70pt}

\question[2] Seis grifos, tardan 10 horas en llenar un depósito de 400 m³ de capacidad. ¿Cuántas horas tardarán cuatro grifos en llenar 2 depósitos de 500 m³ cada uno?

\begin{solution}
Primero, calculamos el trabajo que realiza un grifo en una hora. Si 6 grifos llenan 400 m³ en 10 horas, el trabajo de un solo grifo en una hora es:
\[
\text{Trabajo por grifo por hora} = \frac{400 \, \text{m}^3}{6 \times 10} = \frac{400 \, \text{m}^3}{60} = 6.67 \, \text{m}^3/\text{hora}
\]

Ahora, para 4 grifos, el trabajo combinado por hora es:
\[
\text{Trabajo total por hora} = 4 \times 6.67 \, \text{m}^3/\text{hora} = 26.68 \, \text{m}^3/\text{hora}
\]

El volumen total que deben llenar 4 grifos es:
\[
\text{Volumen total} = 2 \times 500 \, \text{m}^3 = 1000 \, \text{m}^3
\]

Finalmente, calculamos el tiempo necesario para llenar el volumen total:
\[
\text{Tiempo} = \frac{1000 \, \text{m}^3}{26.68 \, \text{m}^3/\text{hora}} \approx 37.5 \, \text{horas}
\]

Por lo tanto, 4 grifos tardarán aproximadamente 37.5 horas en llenar 2 depósitos de 500 m³ cada uno.
\end{solution}
\vspace{70pt}

\newpage


\question[] Encuentra las siguientes palabras: porcentaje, razon, proporcion, directa, inversa, compuesta \\ \newline 
\begin{tabular}{|c|c|c|c|c|c|c|c|c|c|c|c|}
\hline
A & R & R & K & W & I & W & H & O & I & C & T \\ \hline
F & K & M & C & S & M & O & R & F & N & F & P \\ \hline
I & H & A & R & M & R & S & W & Y & V & V & R \\ \hline
Z & J & L & M & C & L & P & C & B & E & D & O \\ \hline
I & Y & M & G & G & A & F & V & X & R & Y & P \\ \hline
E & R & A & Z & O & N & W & H & A & S & T & O \\ \hline
W & K & P & W & A & A & I & N & X & A & Y & R \\ \hline
P & O & R & C & E & N & T & A & J & E & U & C \\ \hline
Z & C & D & I & R & E & C & T & A & L & F & I \\ \hline
A & O & Z & V & V & R & K & C & L & N & K & O \\ \hline
E & I & N & Y & I & Q & Z & C & U & D & G & N \\ \hline
Y & P & X & C & O & M & P & U & E & S & T & A \\ \hline
\end{tabular} \begin{solution}  \newline 
\begin{tabular}{|c|c|c|c|c|c|c|c|c|c|c|c|}
\hline
A & R & R & K & W & I & W & H & O & \textbf{I} & C & T \\ \hline
F & K & M & C & S & M & O & R & F & \textbf{N} & F & \textbf{P} \\ \hline
I & H & A & R & M & R & S & W & Y & \textbf{V} & V & \textbf{R} \\ \hline
Z & J & L & M & C & L & P & C & B & \textbf{E} & D & \textbf{O} \\ \hline
I & Y & M & G & G & A & F & V & X & \textbf{R} & Y & \textbf{P} \\ \hline
E & \textbf{R} & \textbf{A} & \textbf{Z} & \textbf{O} & \textbf{N} & W & H & A & \textbf{S} & T & \textbf{O} \\ \hline
W & K & P & W & A & A & I & N & X & \textbf{A} & Y & \textbf{R} \\ \hline
\textbf{P} & \textbf{O} & \textbf{R} & \textbf{C} & \textbf{E} & \textbf{N} & \textbf{T} & \textbf{A} & \textbf{J} & \textbf{E} & U & \textbf{C} \\ \hline
Z & C & \textbf{D} & \textbf{I} & \textbf{R} & \textbf{E} & \textbf{C} & \textbf{T} & \textbf{A} & L & F & \textbf{I} \\ \hline
A & O & Z & V & V & R & K & C & L & N & K & \textbf{O} \\ \hline
E & I & N & Y & I & Q & Z & C & U & D & G & \textbf{N} \\ \hline
Y & P & X & \textbf{C} & \textbf{O} & \textbf{M} & \textbf{P} & \textbf{U} & \textbf{E} & \textbf{S} & \textbf{T} & \textbf{A} \\ \hline
\end{tabular}\end{solution}



\end{questions}

\end{document}
\grid
