
\documentclass[spanish, 11pt]{exam}

\usepackage{array,epsfig}
\usepackage{amsmath, textcomp}
\usepackage{amsfonts}
\usepackage{amssymb}
\usepackage{amsxtra}
\usepackage{amsthm}
\usepackage{mathrsfs}
\usepackage{color}
\usepackage{multicol, xparse}
\usepackage{verbatim}
\usepackage{booktabs}

\usepackage[utf8]{inputenc}
\usepackage[spanish]{babel}
\usepackage{eurosym}

\usepackage{graphicx}
\graphicspath{{../img/}}
\usepackage{pgf}

\usepackage{pgf,tikz,pgfplots}
\pgfplotsset{compat=1.15}
\usepackage{mathrsfs}
\usetikzlibrary{arrows}

%\printanswers
\nopointsinmargin
\pointformat{}

\let\multicolmulticols\multicols
\let\endmulticolmulticols\endmulticols
\RenewDocumentEnvironment{multicols}{mO{}}
 {%
  \ifnum#1=1
    #2%
  \else
    \multicolmulticols{#1}[#2]
  \fi
 }
 {%
  \ifnum#1=1
  \else
    \endmulticolmulticols
  \fi
 }

\renewcommand{\solutiontitle}{\noindent\textbf{Sol:}\enspace}
\newcommand{\samedir}{\mathbin{\!/\mkern-5mu/\!}}

\newcommand{\class}{1º Bachillerato}
\newcommand{\examdate}{\today}
\newcommand{\tipo}{A}
\newcommand{\timelimit}{50 minutos}

\pagestyle{head}
\firstpageheader{Dep. Matemáticas}{Repaso trimestre 1
}{IES Goya}
\runningheader{IES Goya}{Preparación de prueba global}{Página \thepage\ de \numpages}
\runningheadrule

\begin{document}
\begin{questions}
% Preguntas extraídas de: 01_naturales.tex
%\question 
%
%\begin{parts}
%\part[2] 
%\begin{solution}
%\end{solution}
%
%
%\end{parts}
%\addpoints

\question Responde a las siguientes cuestiones:
\begin{parts}
\part[1] Los cuatro primeros múltiplos de 19. 
\begin{solution}
19, 38, 57, 76
\end{solution}

\part[2] Todos los divisores de 60. 
\begin{solution}
1, 2, 3, 4, 5, 6, 10, 12, 15, 20, 30, 60
\end{solution}

\part[2] El primer múltiplo de 23 después de 1000.
\begin{solution}
\(1000 \div 23 \approx 43.48 \Rightarrow 23 \times 44 = 1012\)
\end{solution}

\part[1] Indica cuáles de estos números son múltiplos de 2, 3, 5 y 10:\\
842, 675, 1280, 3471, 5900, 7314, 9000 
\begin{solution}
\begin{itemize}
  \item Múltiplos de 2: 842, 1280, 5900, 7314, 9000
  \item Múltiplos de 3: 675, 3471, 7314, 9000
  \item Múltiplos de 5: 675, 1280, 5900, 9000
  \item Múltiplos de 10: 1280, 5900, 9000
\end{itemize}
\end{solution}

\part[2] \( \text{max.c.d.}(36, 60) \)
\begin{solution}
\(2^2 \cdot 3 = 12\)
\end{solution}

\part[2] \( \text{min.c.m.}(36, 60) \)
\begin{solution}
\(2^3 \cdot 3 \cdot 5 = 180\)
\end{solution}
\end{parts}

\question[3] En un taller suena una alarma de seguridad cada 36 segundos, y una campana de aviso cada 48 segundos. Si ambos sonidos se escuchan al mismo tiempo ahora, ¿cuánto tardarán en coincidir de nuevo?

\begin{solution}
$$
\text{m.c.m.}(36, 48) = 144 \Rightarrow \text{Coinciden cada } 144 \text{ segundos = 2 minutos y 24 segundos.}
$$
\end{solution}

\question[3] Una tienda de juguetes pone a la venta una partida de peluches, todos del mismo precio. El primer día vende varios por valor de 198 \euro y el segundo día, más por valor de 234 \euro. ¿Cuál crees que es el precio de un peluche?
\begin{solution}
\[
198 = 18 \cdot 11,\quad 234 = 2 \cdot 3^2\cdot  11 \Rightarrow \text{m.c.d.} = 18
\]
El precio de un peluche es **18 €**.
\end{solution}


\question Calcula:
\begin{parts}
    \part[1] \textbf{3 · 9 + 7 + 6 - 5 · 3=}
    \part[1] \textbf{5 · (2 + 6) + 7 - 4 · 3=}    % \part \textbf{3 + 2 · {[}3 · (2 · 5 - 7 + 3){]} - 2 · 3=} 
    % \part \textbf{5 · {[}3 + 1 + 2 · (19 - 4 · 3 + 2){]} + 2=}
\end{parts}


\question Calcular, paso a paso, las siguientes operaciones combinadas:

\begin{parts}

\part[2] \( 6 - (-3) \cdot [ -4 - 2 \cdot (1 - 6)] = \)

\begin{solution}
\[
6 - (-3) \cdot [ -4 - 2 \cdot (1 - 6)] = 6 - (-3) \cdot [ -4 - 2 \cdot (-5)]  = 6 + 18 = 24
\]
\end{solution} 

\part[2] \( (-5) \cdot [10 + 2 \cdot (7 - 9)] = \)

\begin{solution}
\[
(-5) \cdot [10 + 2 \cdot (7 - 9)] = (-5) \cdot [10 + 2 \cdot (-2)] = (-5) \cdot [10 - 4] = (-5) \cdot 6 = -30
\]
\end{solution} 

\part[2] \( 7 \cdot (-3) - 4 \cdot (-6) = \)

\begin{solution}
\[
7 \cdot (-3) - 4 \cdot (-6) = -21 + 24 = 3
\]
\end{solution} 

\part[2] \( 5 - (-2) + (-3) \cdot [ (1) + (-4) - 3] = \)

\begin{solution}
\[
5 - (-2) + (-3) \cdot [1 - 4 - 3] = 5 + 2 + (-3) \cdot (-6) = 7 + 18 = 25
\]
\end{solution} 

% \part[2] \( 10 - [ (4 - 9) + (1 - 4) - (6 - 8)] = \)

% \begin{solution}
% \[
% 10 - [ -5 + (-3) - (-2)] = 10 - [-5 - 3 + 2] = 10 - (-6) = 16
% \]
% \end{solution} 

\part[2] \( (15 - 18) - [ (3 - 7 + 10) - 2 \cdot (4 - 6)] = \)

\begin{solution}
\[
(15 - 18) - [ (3 - 7 + 10) - 2 \cdot (-2)] = (-3) - [6 + 4] = -3 - 10 = -13
\]
\end{solution} 

\part[2] \( \left[3 \cdot (6 - 8) \cdot (-2) - (9 - 3) : (-3)\right] \cdot (-2) = \)

\begin{solution}
\[
\left[3 \cdot (-2) \cdot (-2) - 6 : (-3)\right] \cdot (-2)  = 14 \cdot (-2) = -28
\]
\end{solution} 

\end{parts}






% \question[2] En un supermercado compramos tres productos para
%   un restaurante: 6 kg de carne de cerdo a 4 \euro/kg, 4 kg de merluza a 4
%   \euro/kg y 3 kg de carne de pollo al mismo precio que la carne y la
%   merluza. Si pago con un billete de 100 \euro, ¿cuánto me devuelven si el
%   encargado del supermercado me hace una rebaja de 5 \euro \ por cada producto
%   comprado? Halla la solución mediante una expresión de operaciones
%   combinadas. 

% \question[2] En una granja se recogen los huevos dos días a la
%   semana. El primer día se llenaron 12 cajas de 16 bandejas de huevos;
%   el segundo día se completaron 22 cajas de 15 bandejas cada una. Si en
%   cada bandeja entran dos docenas y media de huevos, ¿cuántos huevos se
%   han recogido durante esta semana? Halla la solución mediante una
%   expresión de operaciones combinadas


%%%
%%% OTRO EXAMEN
%%%
% % \question[1\half] Escribe con cifras:
% \begin{parts}
% \part Seis millones setecientos cincuenta mil
% 
% \begin{solution}
% 6.750.000
% \end{solution}

% \part Doscientos cuarenta y cinco millones ochocientos doce mil
% 
% \begin{solution}
% 245.812.000
% \end{solution}

% \part Tres mil quinientos dieciséis
% 
% \begin{solution}
% 3.516
% \end{solution}
% \end{parts}

% \question[1\half] Escribe como se leen las siguientes cifras:
% \begin{parts}
%     \part 8 500 000 000
%     
%     \begin{solution}
%     Ocho mil quinientos millones
%     \end{solution}
    
%     \part 5 600 000 010 000 000
%     
%     \begin{solution}
%     Cinco billones seiscientos mil millones diez mil
%     \end{solution}
    
%     \part 7 385 000 765 000
%     
%     \begin{solution}
%     Siete billones trescientos ochenta y cinco mil millones setecientos sesenta y cinco mil
%     \end{solution}
% \end{parts}

% \question[1] Aproxima a las decenas de millar, por redondeo, los siguientes números: 
% \begin{parts}
%     \part 728 345
%     \begin{solution}
%     730.000
%     \end{solution}
    
%     \part 1 902 678
%     \begin{solution}
%     1.900.000
%     \end{solution}
    
%     \part 8 543 210
%     \begin{solution}
%     8.540.000
%     \end{solution}
    
%     \part  32 567 890
%     \begin{solution}
%     32.570.000
%     \end{solution}
% \end{parts}

% \question[2] Calcula:
% \begin{parts}
%     \part \textbf{4 · 7 + 8 - 5 · 4 =}
%     
%     \begin{solution}
%     4 · 7 + 8 - 5 · 4 = 28 + 8 - 20 = 16
%     \end{solution}

%     \part \textbf{6 · (3 + 4) - 8 + 7 · 2 =}
%     
%     \begin{solution}
%     6 · (3 + 4) - 8 + 7 · 2 = 6 · 7 - 8 + 14 = 42 - 8 + 14 = 48
%     \end{solution}

%     \part \textbf{4 + 3 · {[}2 · (6 - 3 + 4){]} - 4 · 2 =} 
%     
%     \begin{solution}
%     4 + 3 · {[}2 · (6 - 3 + 4){]} - 4 · 2 = 4 + 3 · (2 · 7) - 8 = 4 + 3 · 14 - 8 = 4 + 42 - 8 = 38
%     \end{solution}

%     \part \textbf{6 · {[}4 + 2 + 3 · (18 - 5 · 2 + 1){]} + 3 =}
%     
%     \begin{solution}
%     6 · {[}4 + 2 + 3 · (18 - 10 + 1){]} + 3 = 6 · {[}4 + 2 + 3 · 9{]} + 3 = 6 · (4 + 2 + 27) + 3 = 6 · 33 + 3 = 198 + 3 = 201
%     \end{solution}
% \end{parts}

% \question[2] En una tienda de comestibles compramos tres productos: 5 kg de arroz a 2 \euro/kg, 3 kg de lentejas a 3 \euro/kg, y 4 kg de garbanzos al mismo precio que el arroz y las lentejas. Si pagamos con un billete de 50 \euro, ¿cuánto nos devuelven si el encargado de la tienda nos hace una rebaja de 3 \euro por cada producto comprado? Halla la solución mediante una expresión de operaciones combinadas. 
% 
% \begin{solution}
% El precio total sin rebajas sería:

% \[ 5 \text{kg} \cdot 2 \euro/\text{kg} + 3 \text{kg} \cdot 3 \euro/\text{kg} + 4 \text{kg} \cdot 3 \euro/\text{kg} = 10 \euro + 9 \euro + 12 \euro = 31 \euro \]

% Con la rebaja de 3 \euro por cada producto comprado (3 productos):

% \[ 31 \euro - (3 \euro \times 3) = 31 \euro - 9 \euro = 22 \euro \]

% Pagamos con 50 \euro, así que nos devuelven:

% \[ 50 \euro - 22 \euro = 28 \euro \]
% \end{solution}

% \question[2] En una finca se recogen manzanas tres días a la semana. El primer día se llenaron 10 cestas con 12 kilos de manzanas; el segundo día se completaron 15 cestas con 14 kilos cada una, y el tercer día se recogieron 8 cestas con 16 kilos cada una. ¿Cuántos kilos de manzanas se han recogido durante la semana? Halla la solución mediante una expresión de operaciones combinadas.
% 
% \begin{solution}
% Calculamos los kilos totales recogidos cada día y sumamos:

% Primer día: 
% \[ 10 \text{cestas} \cdot 12 \text{kg} = 120 \text{kg} \]

% Segundo día: 
% \[ 15 \text{cestas} \cdot 14 \text{kg} = 210 \text{kg} \]

% Tercer día: 
% \[ 8 \text{cestas} \cdot 16 \text{kg} = 128 \text{kg} \]

% El total de kilos recogidos durante la semana es:
% \[ 120 \text{kg} + 210 \text{kg} + 128 \text{kg} = 458 \text{kg} \]
% \end{solution}

% BORRAR


% % ----------------------
% % PREGUNTA 1
% % ----------------------
% \question Escribe:
% \begin{parts}
% \part Los cuatro primeros múltiplos de 17.
% \begin{solution}
% 17, 34, 51, 68
% \end{solution}

% \part Todos los divisores de 72.
% \begin{solution}
% 1, 2, 3, 4, 6, 8, 9, 12, 18, 24, 36, 72
% \end{solution}
% \end{parts}

% % ----------------------
% % PREGUNTA 2
% % ----------------------
% \question Busca:
% \begin{parts}
% \part El primer múltiplo de 17 después de 1000.
% \begin{solution}
% \(1000 \div 17 \approx 58.82 \Rightarrow 17 \times 59 = 1003\)
% \end{solution}

% \part Un número de dos cifras que sea divisor de 415.
% \begin{solution}
% Ejemplos: 5 y 83
% \end{solution}
% \end{parts}

% % ----------------------
% % PREGUNTA 3
% % ----------------------
% \question Escribe los números primos comprendidos entre 20 y 40.
% \begin{solution}
% 23, 29, 31, 37
% \end{solution}

% % ----------------------
% % PREGUNTA 4
% % ----------------------
% \question Indica cuáles de estos números son múltiplos de 2, 3, 5 y 10:\\
% 897, 765, 990, 2713, 6077, 6324, 7005
% \begin{solution}
% \begin{itemize}
%   \item Múltiplos de 2: 990, 6324
%   \item Múltiplos de 3: 897, 765, 990, 6324, 7005
%   \item Múltiplos de 5: 765, 990, 7005
%   \item Múltiplos de 10: 990
% \end{itemize}
% \end{solution}

% % ----------------------
% % PREGUNTA 5
% % ----------------------
% \question Copia en tu cuaderno y descompón en factores primos los números 150 y 225.
% \begin{solution}
% \[
% 150 = 2 \cdot 3 \cdot 5^2,\quad 225 = 3^2 \cdot 5^2
% \]
% \end{solution}

% % ----------------------
% % PREGUNTA 6
% % ----------------------
% \question Calcula:
% \begin{parts}
% \part \( \text{m.c.d.}(150, 225) \)
% \begin{solution}
% \(3 \cdot 5^2 = 75\)
% \end{solution}

% \part \( \text{m.c.m.}(150, 225) \)
% \begin{solution}
% \(2 \cdot 3^2 \cdot 5^2 = 450\)
% \end{solution}
% \end{parts}

% % ----------------------
% % PREGUNTA 7
% % ----------------------
% \question Calcula:
% \begin{parts}
% \part \(6 - 11 + (9 - 13)\)
% \begin{solution}
% \(-5 + (-4) = -9\)
% \end{solution}

% \part \(2 - (5 - 8)\)
% \begin{solution}
% \(2 - (-3) = 5\)
% \end{solution}

% \part \((7 - 15) - (6 - 2)\)
% \begin{solution}
% \(-8 - 4 = -12\)
% \end{solution}

% \part \(5 - [2 - (3 - 2)]\)
% \begin{solution}
% \(5 - [2 - 1] = 5 - 1 = 4\)
% \end{solution}
% \end{parts}

% % ----------------------
% % PREGUNTA 8
% % ----------------------
% \question Calcula:
% \begin{parts}
% \part \(4 \cdot 5 - 3 \cdot (-2) + 5 \cdot (-8) - 4 \cdot (-3)\)
% \begin{solution}
% \(20 + 6 - 40 + 12 = -2\)
% \end{solution}

% \part \((10 - 3 \cdot 6) - 2 \cdot [5 + 3 \cdot (4 - 7)]\)
% \begin{solution}
% \(-8 - (-8) = 0\)
% \end{solution}

% \part \(10 - 10 \cdot [-6 + 5 \cdot (-4 + 7 - 3)]\)
% \begin{solution}
% \(10 - (-60) = 70\)
% \end{solution}
% \end{parts}

% % ----------------------
% % PREGUNTA 9
% % ----------------------
% \question Reduce a una sola potencia:
% \begin{parts}
% \part \(a^3 : b^3\)
% \begin{solution}
% \(\left(\dfrac{a}{b}\right)^3\)
% \end{solution}

% \part \(a^5 : b^5\)
% \begin{solution}
% \(\left(\dfrac{a}{b}\right)^5\)
% \end{solution}

% \part \(a^4 \cdot a^2\)
% \begin{solution}
% \(a^6\)
% \end{solution}

% \part \(x^6 \cdot x^4\)
% \begin{solution}
% \(x^{10}\)
% \end{solution}

% \part \((x^3)^3\)
% \begin{solution}
% \(x^9\)
% \end{solution}

% \part \((-5)^7 : (-5)^5\)
% \begin{solution}
% \((-5)^2 = 25\)
% \end{solution}
% \end{parts}

% % ----------------------
% % PREGUNTA 10
% % ----------------------
% \question Una tienda de ropa pone a la venta una partida de camisetas, todas del mismo precio. El primer día vende unas cuantas por valor de 221 \euro y el segundo día, unas cuantas más por valor de 272 \euro. \\
% ¿Cuál crees que es el precio de una camiseta?
% \begin{solution}
% \[
% 221 = 13 \cdot 17,\quad 272 = 16 \cdot 17 \Rightarrow \text{m.c.d.} = 17
% \]
% El precio de una camiseta es **17 €**.
% \end{solution}

% % ----------------------
% % PREGUNTA 11
% % ----------------------
% \question Se desea dividir un terreno rectangular, de 100 m de ancho por 120 m de largo, en parcelas cuadradas lo más grandes que sea posible. ¿Cuánto debe medir el lado de cada parcela?
% \begin{solution}
% $$
% \text{m.c.d.}(100, 120) = 20 \Rightarrow \text{Cada parcela mide } 20 \text{ m de lado.}
% $$\end{solution}

% % ----------------------
% % PREGUNTA 12
% % ----------------------
% \question En una fábrica se oye el escape de una válvula de gas cada 45 segundos, y el golpe de un martillo pilón cada 60 segundos. Si se acaban de oír ambos sonidos simultáneamente, ¿cuánto tardarán en coincidir de nuevo?
% \begin{solution}
% $$
% \text{m.c.m.}(45, 60) = 180 \Rightarrow \text{Coinciden cada } 180 \text{ segundos = 3 minutos.}
% $$
% \end{solution}

% % ----------------------
% % PREGUNTA 13
% % ----------------------
% \question Se van apilando, en una torre, cubos de 45 cm de arista y, al lado, en otra, cubos de 60 cm de arista. ¿A qué altura coinciden por tercera vez las cimas de ambas torres?
% \begin{solution}
% $$
% \text{m.c.m.}(45, 60) = 180,\quad 180 \times 3 = 540 \Rightarrow \text{Altura: } 540 \text{ cm}
% $$
% \end{solution}

% % ----------------------
% % PREGUNTA 14
% % ----------------------
% \question La suma de dos números enteros es 4, y la suma de sus valores absolutos, 16. ¿Qué números son?
% \begin{solution}
% Supongamos \( x + y = 4 \) y \( |x| + |y| = 16 \).\\
% Prueba con \( x = -6 \), \( y = 10 \):\\
% \[
% -6 + 10 = 4,\quad 6 + 10 = 16 \Rightarrow \text{Correcto.}
% \]
% Los números son \(-6\) y \(10\).
% \end{solution}


% \question[8] Calcular, paso a paso, las siguientes operaciones combinadas:

% \begin{parts}

% \part \( 4 - (-2) \cdot [ -8 - 3 \cdot (5 - 7)] \)=

% \begin{solution}
% \[
% 4 - (-2) \cdot [ -8 - 3 \cdot (5 - 7)] = 4 - (-2) \cdot [ -8 - 3 \cdot (-2)] = 4 - (-2) \cdot (-2) = 4 - 4 = 0
% \]
% \end{solution} 
% \part \( (-4) \cdot [12 + 3 \cdot (5 - 8)] = \)

% \begin{solution}
% \[
% (-4) \cdot [12 + 3 \cdot (5 - 8)] = (-4) \cdot [12 + 3 \cdot (-3)] = (-4) \cdot [12 - 9] = (-4) \cdot 3 = -12
% \]
% \end{solution}
%  
% \part \( 4 \cdot (-6) - 8 \cdot (-4)= \)
% \begin{solution}
% \[
% 4 \cdot (-6) - 8 \cdot (-4) = -24 + 32 = -56
% \]
% \end{solution}
%  
% \part \( 6 - (-3) + (-2) \cdot [ (2) + (-3) - 5] = \)

% \begin{solution}
% \[
% 6 - (-3) + (-2) \cdot [ (2) + (-3) - 5] = 6 + 3 + (-2) \cdot (-6) = 9 + 12 = 21
% \]
% \end{solution}
%  
% \part \( 8 - [ (6-11) + (2-5) - (7-10)] = \)

% \begin{solution}
% \[
% 8 - [ (6-11) + (2-5) - (7-10)] = 8 - [ -5 + (-3) + 3] = 8 - (-5) = 13
% \]
% \end{solution}
%  
% \part \( (13 - 21) - [ (2 - 5 + 8) - 2 \cdot (6 - 9)] = \)

% \begin{solution}
% \[
% (13 - 21) - [ (2 - 5 + 8) - 2 \cdot (6 - 9)] = (-8) - [5 - 2 \cdot (-3)] = -8 - [5 + 6] = -8 - 11 = -19
% \]
% \end{solution}
%  
% \part \( [2 \cdot (-4) - (11 + 5)] : [ 2 \cdot (5 + 9 + 6) - (7 - 10)] = \)

% \begin{solution}
% \[
% [2 \cdot (-4) - (11 + 5)] : [ 2 \cdot (5 + 9 + 6) - (7 - 10)] = [-8 - 16] : [2 \cdot 20 - (-3)] = (-24) : [40 + 3] = (-24) : 43 = -\frac{24}{43}
% \]
% \end{solution}
%  
% \part \( \left[2 \cdot (4 - 5) \cdot (-3) - (8 - 2) : (-3)\right] \cdot (-4) = \)

% \begin{solution}
% \[
% \left[2 \cdot (4 - 5) \cdot (-3) - (8 - 2) : (-3)\right] \cdot (-4) = \left[2 \cdot (-1) \cdot (-3) - 6 : (-3)\right] \cdot (-4) = \left[6 - (-2)\right] \cdot (-4) = [6 + 2] \cdot (-4) = 8 \cdot (-4) = -32
% \]
% \end{solution}
%  
% \end{parts}

% \question[] (Solo si has acabado) Encuentra las siguientes palabras: múltiplo, divisor, natural, entero, divisible, negativo, opuesto \\ \newline 
% \scalebox{0.9}{
% \begin{tabular}{|c|c|c|c|c|c|c|c|c|c|c|}
% \hline
% P & D & Y & M & G & Y & K & H & C & H & T \\ \hline
% N & D & I & V & I & S & O & R & K & D & L \\ \hline
% M & E & Q & N & W & W & O & U & I & C & E \\ \hline
% B & Ú & G & Q & W & Q & T & V & N & Q & N \\ \hline
% Y & I & L & A & T & C & I & P & A & P & T \\ \hline
% D & A & F & T & T & S & Z & Z & T & C & E \\ \hline
% N & K & E & M & I & I & W & L & U & G & R \\ \hline
% Y & V & Z & B & G & P & V & A & R & G & O \\ \hline
% M & O & L & U & X & N & L & O & A & F & B \\ \hline
% L & E & Q & S & P & E & A & O & L & X & I \\ \hline
% K & T & S & V & O & P & U & E & S & T & O \\ \hline
% \end{tabular}
% }

% \begin{solution}   
% \begin{tabular}{|c|c|c|c|c|c|c|c|c|c|c|}
% \hline
% P & D & Y & M & G & Y & K & H & C & H & T \\ \hline
% \textbf{N} & \textbf{D} & \textbf{I} & \textbf{V} & \textbf{I} & \textbf{S} & \textbf{O} & \textbf{R} & K & \textbf{D} & L \\ \hline
% \textbf{M} & \textbf{E} & Q & N & W & W & O & U & \textbf{I} & C & \textbf{E} \\ \hline
% B & \textbf{Ú} & \textbf{G} & Q & W & Q & T & \textbf{V} & \textbf{N} & Q & \textbf{N} \\ \hline
% Y & I & \textbf{L} & \textbf{A} & T & C & \textbf{I} & P & \textbf{A} & P & \textbf{T} \\ \hline
% D & A & F & \textbf{T} & \textbf{T} & \textbf{S} & Z & Z & \textbf{T} & C & \textbf{E} \\ \hline
% N & K & E & M & \textbf{I} & \textbf{I} & W & L & \textbf{U} & G & \textbf{R} \\ \hline
% Y & V & Z & \textbf{B} & G & \textbf{P} & \textbf{V} & A & \textbf{R} & G & \textbf{O} \\ \hline
% M & O & \textbf{L} & U & X & N & \textbf{L} & \textbf{O} & \textbf{A} & F & B \\ \hline
% L & \textbf{E} & Q & S & P & E & A & \textbf{O} & \textbf{L} & X & I \\ \hline
% K & T & S & V & \textbf{O} & \textbf{P} & \textbf{U} & \textbf{E} & \textbf{S} & \textbf{T} & \textbf{O} \\ \hline
% \end{tabular}\end{solution}

% Preguntas extraídas de: 02_potencias_raices.tex
%\question 
%
%\begin{parts}
%\part[2] 
%\begin{solution}
%\end{solution}
%
%
%\end{parts}
%\addpoints

\question Responde a las siguientes cuestiones:
\begin{parts}
\part[1] Indica cuáles de estos números son múltiplos de 2, 3, 5 y 10:\\
842, 675, 1280, 3471, 5900, 7314, 9000 
\begin{solution}
\begin{itemize}
  \item Múltiplos de 2: 842, 1280, 5900, 7314, 9000
  \item Múltiplos de 3: 675, 3471, 7314, 9000
  \item Múltiplos de 5: 675, 1280, 5900, 9000
  \item Múltiplos de 10: 1280, 5900, 9000
\end{itemize}
\end{solution}

\part[1] \( \text{max.c.d.}(32, 60) \)
\begin{solution}
$4$
\end{solution}

\part[1] \( \text{min.c.m.}(32, 60) \)
\begin{solution}
\(480\)
\end{solution}
\end{parts}


\question Calcula:
\begin{parts}
    % \part[1] \textbf{3 · 9 + 7 + 6 - 5 · 3=}
    % \part[1] \textbf{5 · (2 + 6) + 7 - 4 · 3=}    
    \part[1] \textbf{3 + 2 · {[}3 · (2 · 5 - 7 + 3){]} - 2 · 3=} 
    \part[1] \textbf{5 · {[}3 + 1 + 2 · (19 - 4 · 3 + 2){]} + 2=}
\end{parts}

\question[1\half] Expresa como una única potencia:
\begin{multicols}{3}
\begin{parts}
    \part $\left(6^3\right)^4=$
    
    \part $5^2\cdot 5^3=$
    
    \part $n^4 : n^2=$
    
\end{parts}
\end{multicols}



\question[2] Aplica las propiedades para expresar como una única potencia y después calcula su valor:
\begin{multicols}{2}
\begin{parts}
    \part $3^2\cdot 3^3 : 3^5=$
    % \part $\left(2^2\right)^3\cdot\left(5^2\right)^3=$
    \part $20^{6} :10^6=$
    \part $18^4:\left(2^4\cdot3^4\right)=$
    % \part $24^5:\left(2^5\cdot6^5\right)=$
    \part $\left[\left(12^3:4^3\right)^2:\left(15^2:5^2\right)^3\right]^8=$
\end{parts}
\end{multicols}

\question[2] Cristina ha utilizado el ordenador durante 8 h 37 min, de lunes a viernes. ¿Cuánto tiempo ha estado funcionando a diario el ordenador?(Da el resultado en horas, minutos y segundos)

\begin{solution}
$$
8\cdot60+37=517 \to 517:5=103.4 \to 1h43m + 0.4m \to 1h43m24s
$$
\end{solution}











% \question[2] En un supermercado compramos tres productos para
%   un restaurante: 6 kg de carne de cerdo a 4 \euro/kg, 4 kg de merluza a 4
%   \euro/kg y 3 kg de carne de pollo al mismo precio que la carne y la
%   merluza. Si pago con un billete de 100 \euro, ¿cuánto me devuelven si el
%   encargado del supermercado me hace una rebaja de 5 \euro \ por cada producto
%   comprado? Halla la solución mediante una expresión de operaciones
%   combinadas. 

\question[2] En una granja se recogen los huevos dos días a la
  semana. El primer día se llenaron 12 cajas de 16 bandejas de huevos;
  el segundo día se completaron 22 cajas de 15 bandejas cada una. Si en
  cada bandeja entran dos docenas y media de huevos, ¿cuántos huevos se
  han recogido durante esta semana? Halla la solución mediante una
  expresión de operaciones combinadas


\question Resuelve las siguientes operaciones:
\begin{parts}
    \part[1] $3^4:\sqrt{1+\left(20+6\cdot10\right)}=$
    
    \part[1] $\left(6 + 2 \cdot 3^2 + 3 \cdot 2^2 \right) : \left(3 - \sqrt{81} \right)^2 =
$ \begin{solution} $1$ \end{solution}
    \part[1] $\left[ 9 - \sqrt{25} \cdot (-2)^3 \right] : \left[ (-3 - 1)^2 - 9 \right] =
$\begin{solution}$7$\end{solution}


    \part[1\half] $\dfrac{\left[(-3)^3\right]^2 \cdot \left[3 \cdot (-9)\right]^6}{81^5} =
$\begin{solution}
    $3^4$
\end{solution} 
\end{parts}

%%%
%%% OTRO EXAMEN
%%%
% % \question[1\half] Escribe con cifras:
% \begin{parts}
% \part Seis millones setecientos cincuenta mil
% 
% \begin{solution}
% 6.750.000
% \end{solution}

% \part Doscientos cuarenta y cinco millones ochocientos doce mil
% 
% \begin{solution}
% 245.812.000
% \end{solution}

% \part Tres mil quinientos dieciséis
% 
% \begin{solution}
% 3.516
% \end{solution}
% \end{parts}

% \question[1\half] Escribe como se leen las siguientes cifras:
% \begin{parts}
%     \part 8 500 000 000
%     
%     \begin{solution}
%     Ocho mil quinientos millones
%     \end{solution}
    
%     \part 5 600 000 010 000 000
%     
%     \begin{solution}
%     Cinco billones seiscientos mil millones diez mil
%     \end{solution}
    
%     \part 7 385 000 765 000
%     
%     \begin{solution}
%     Siete billones trescientos ochenta y cinco mil millones setecientos sesenta y cinco mil
%     \end{solution}
% \end{parts}

% \question[1] Aproxima a las decenas de millar, por redondeo, los siguientes números: 
% \begin{parts}
%     \part 728 345
%     \begin{solution}
%     730.000
%     \end{solution}
    
%     \part 1 902 678
%     \begin{solution}
%     1.900.000
%     \end{solution}
    
%     \part 8 543 210
%     \begin{solution}
%     8.540.000
%     \end{solution}
    
%     \part  32 567 890
%     \begin{solution}
%     32.570.000
%     \end{solution}
% \end{parts}

% \question[2] Calcula:
% \begin{parts}
%     \part \textbf{4 · 7 + 8 - 5 · 4 =}
%     
%     \begin{solution}
%     4 · 7 + 8 - 5 · 4 = 28 + 8 - 20 = 16
%     \end{solution}

%     \part \textbf{6 · (3 + 4) - 8 + 7 · 2 =}
%     
%     \begin{solution}
%     6 · (3 + 4) - 8 + 7 · 2 = 6 · 7 - 8 + 14 = 42 - 8 + 14 = 48
%     \end{solution}

%     \part \textbf{4 + 3 · {[}2 · (6 - 3 + 4){]} - 4 · 2 =} 
%     
%     \begin{solution}
%     4 + 3 · {[}2 · (6 - 3 + 4){]} - 4 · 2 = 4 + 3 · (2 · 7) - 8 = 4 + 3 · 14 - 8 = 4 + 42 - 8 = 38
%     \end{solution}

%     \part \textbf{6 · {[}4 + 2 + 3 · (18 - 5 · 2 + 1){]} + 3 =}
%     
%     \begin{solution}
%     6 · {[}4 + 2 + 3 · (18 - 10 + 1){]} + 3 = 6 · {[}4 + 2 + 3 · 9{]} + 3 = 6 · (4 + 2 + 27) + 3 = 6 · 33 + 3 = 198 + 3 = 201
%     \end{solution}
% \end{parts}

% \question[2] En una tienda de comestibles compramos tres productos: 5 kg de arroz a 2 \euro/kg, 3 kg de lentejas a 3 \euro/kg, y 4 kg de garbanzos al mismo precio que el arroz y las lentejas. Si pagamos con un billete de 50 \euro, ¿cuánto nos devuelven si el encargado de la tienda nos hace una rebaja de 3 \euro por cada producto comprado? Halla la solución mediante una expresión de operaciones combinadas. 
% 
% \begin{solution}
% El precio total sin rebajas sería:

% \[ 5 \text{kg} \cdot 2 \euro/\text{kg} + 3 \text{kg} \cdot 3 \euro/\text{kg} + 4 \text{kg} \cdot 3 \euro/\text{kg} = 10 \euro + 9 \euro + 12 \euro = 31 \euro \]

% Con la rebaja de 3 \euro por cada producto comprado (3 productos):

% \[ 31 \euro - (3 \euro \times 3) = 31 \euro - 9 \euro = 22 \euro \]

% Pagamos con 50 \euro, así que nos devuelven:

% \[ 50 \euro - 22 \euro = 28 \euro \]
% \end{solution}

% \question[2] En una finca se recogen manzanas tres días a la semana. El primer día se llenaron 10 cestas con 12 kilos de manzanas; el segundo día se completaron 15 cestas con 14 kilos cada una, y el tercer día se recogieron 8 cestas con 16 kilos cada una. ¿Cuántos kilos de manzanas se han recogido durante la semana? Halla la solución mediante una expresión de operaciones combinadas.
% 
% \begin{solution}
% Calculamos los kilos totales recogidos cada día y sumamos:

% Primer día: 
% \[ 10 \text{cestas} \cdot 12 \text{kg} = 120 \text{kg} \]

% Segundo día: 
% \[ 15 \text{cestas} \cdot 14 \text{kg} = 210 \text{kg} \]

% Tercer día: 
% \[ 8 \text{cestas} \cdot 16 \text{kg} = 128 \text{kg} \]

% El total de kilos recogidos durante la semana es:
% \[ 120 \text{kg} + 210 \text{kg} + 128 \text{kg} = 458 \text{kg} \]
% \end{solution}

% BORRAR


% % ----------------------
% % PREGUNTA 1
% % ----------------------
% \question Escribe:
% \begin{parts}
% \part Los cuatro primeros múltiplos de 17.
% \begin{solution}
% 17, 34, 51, 68
% \end{solution}

% \part Todos los divisores de 72.
% \begin{solution}
% 1, 2, 3, 4, 6, 8, 9, 12, 18, 24, 36, 72
% \end{solution}
% \end{parts}

% % ----------------------
% % PREGUNTA 2
% % ----------------------
% \question Busca:
% \begin{parts}
% \part El primer múltiplo de 17 después de 1000.
% \begin{solution}
% \(1000 \div 17 \approx 58.82 \Rightarrow 17 \times 59 = 1003\)
% \end{solution}

% \part Un número de dos cifras que sea divisor de 415.
% \begin{solution}
% Ejemplos: 5 y 83
% \end{solution}
% \end{parts}

% % ----------------------
% % PREGUNTA 3
% % ----------------------
% \question Escribe los números primos comprendidos entre 20 y 40.
% \begin{solution}
% 23, 29, 31, 37
% \end{solution}

% % ----------------------
% % PREGUNTA 4
% % ----------------------
% \question Indica cuáles de estos números son múltiplos de 2, 3, 5 y 10:\\
% 897, 765, 990, 2713, 6077, 6324, 7005
% \begin{solution}
% \begin{itemize}
%   \item Múltiplos de 2: 990, 6324
%   \item Múltiplos de 3: 897, 765, 990, 6324, 7005
%   \item Múltiplos de 5: 765, 990, 7005
%   \item Múltiplos de 10: 990
% \end{itemize}
% \end{solution}

% % ----------------------
% % PREGUNTA 5
% % ----------------------
% \question Copia en tu cuaderno y descompón en factores primos los números 150 y 225.
% \begin{solution}
% \[
% 150 = 2 \cdot 3 \cdot 5^2,\quad 225 = 3^2 \cdot 5^2
% \]
% \end{solution}

% % ----------------------
% % PREGUNTA 6
% % ----------------------
% \question Calcula:
% \begin{parts}
% \part \( \text{m.c.d.}(150, 225) \)
% \begin{solution}
% \(3 \cdot 5^2 = 75\)
% \end{solution}

% \part \( \text{m.c.m.}(150, 225) \)
% \begin{solution}
% \(2 \cdot 3^2 \cdot 5^2 = 450\)
% \end{solution}
% \end{parts}

% % ----------------------
% % PREGUNTA 7
% % ----------------------
% \question Calcula:
% \begin{parts}
% \part \(6 - 11 + (9 - 13)\)
% \begin{solution}
% \(-5 + (-4) = -9\)
% \end{solution}

% \part \(2 - (5 - 8)\)
% \begin{solution}
% \(2 - (-3) = 5\)
% \end{solution}

% \part \((7 - 15) - (6 - 2)\)
% \begin{solution}
% \(-8 - 4 = -12\)
% \end{solution}

% \part \(5 - [2 - (3 - 2)]\)
% \begin{solution}
% \(5 - [2 - 1] = 5 - 1 = 4\)
% \end{solution}
% \end{parts}

% % ----------------------
% % PREGUNTA 8
% % ----------------------
% \question Calcula:
% \begin{parts}
% \part \(4 \cdot 5 - 3 \cdot (-2) + 5 \cdot (-8) - 4 \cdot (-3)\)
% \begin{solution}
% \(20 + 6 - 40 + 12 = -2\)
% \end{solution}

% \part \((10 - 3 \cdot 6) - 2 \cdot [5 + 3 \cdot (4 - 7)]\)
% \begin{solution}
% \(-8 - (-8) = 0\)
% \end{solution}

% \part \(10 - 10 \cdot [-6 + 5 \cdot (-4 + 7 - 3)]\)
% \begin{solution}
% \(10 - (-60) = 70\)
% \end{solution}
% \end{parts}

% % ----------------------
% % PREGUNTA 9
% % ----------------------
% \question Reduce a una sola potencia:
% \begin{parts}
% \part \(a^3 : b^3\)
% \begin{solution}
% \(\left(\dfrac{a}{b}\right)^3\)
% \end{solution}

% \part \(a^5 : b^5\)
% \begin{solution}
% \(\left(\dfrac{a}{b}\right)^5\)
% \end{solution}

% \part \(a^4 \cdot a^2\)
% \begin{solution}
% \(a^6\)
% \end{solution}

% \part \(x^6 \cdot x^4\)
% \begin{solution}
% \(x^{10}\)
% \end{solution}

% \part \((x^3)^3\)
% \begin{solution}
% \(x^9\)
% \end{solution}

% \part \((-5)^7 : (-5)^5\)
% \begin{solution}
% \((-5)^2 = 25\)
% \end{solution}
% \end{parts}

% % ----------------------
% % PREGUNTA 10
% % ----------------------
% \question Una tienda de ropa pone a la venta una partida de camisetas, todas del mismo precio. El primer día vende unas cuantas por valor de 221 \euro y el segundo día, unas cuantas más por valor de 272 \euro. \\
% ¿Cuál crees que es el precio de una camiseta?
% \begin{solution}
% \[
% 221 = 13 \cdot 17,\quad 272 = 16 \cdot 17 \Rightarrow \text{m.c.d.} = 17
% \]
% El precio de una camiseta es **17 €**.
% \end{solution}

% % ----------------------
% % PREGUNTA 11
% % ----------------------
% \question Se desea dividir un terreno rectangular, de 100 m de ancho por 120 m de largo, en parcelas cuadradas lo más grandes que sea posible. ¿Cuánto debe medir el lado de cada parcela?
% \begin{solution}
% $$
% \text{m.c.d.}(100, 120) = 20 \Rightarrow \text{Cada parcela mide } 20 \text{ m de lado.}
% $$\end{solution}

% % ----------------------
% % PREGUNTA 12
% % ----------------------
% \question En una fábrica se oye el escape de una válvula de gas cada 45 segundos, y el golpe de un martillo pilón cada 60 segundos. Si se acaban de oír ambos sonidos simultáneamente, ¿cuánto tardarán en coincidir de nuevo?
% \begin{solution}
% $$
% \text{m.c.m.}(45, 60) = 180 \Rightarrow \text{Coinciden cada } 180 \text{ segundos = 3 minutos.}
% $$
% \end{solution}

% % ----------------------
% % PREGUNTA 13
% % ----------------------
% \question Se van apilando, en una torre, cubos de 45 cm de arista y, al lado, en otra, cubos de 60 cm de arista. ¿A qué altura coinciden por tercera vez las cimas de ambas torres?
% \begin{solution}
% $$
% \text{m.c.m.}(45, 60) = 180,\quad 180 \times 3 = 540 \Rightarrow \text{Altura: } 540 \text{ cm}
% $$
% \end{solution}

% % ----------------------
% % PREGUNTA 14
% % ----------------------
% \question La suma de dos números enteros es 4, y la suma de sus valores absolutos, 16. ¿Qué números son?
% \begin{solution}
% Supongamos \( x + y = 4 \) y \( |x| + |y| = 16 \).\\
% Prueba con \( x = -6 \), \( y = 10 \):\\
% \[
% -6 + 10 = 4,\quad 6 + 10 = 16 \Rightarrow \text{Correcto.}
% \]
% Los números son \(-6\) y \(10\).
% \end{solution}


% \question[8] Calcular, paso a paso, las siguientes operaciones combinadas:

% \begin{parts}

% \part \( 4 - (-2) \cdot [ -8 - 3 \cdot (5 - 7)] \)=

% \begin{solution}
% \[
% 4 - (-2) \cdot [ -8 - 3 \cdot (5 - 7)] = 4 - (-2) \cdot [ -8 - 3 \cdot (-2)] = 4 - (-2) \cdot (-2) = 4 - 4 = 0
% \]
% \end{solution} 
% \part \( (-4) \cdot [12 + 3 \cdot (5 - 8)] = \)

% \begin{solution}
% \[
% (-4) \cdot [12 + 3 \cdot (5 - 8)] = (-4) \cdot [12 + 3 \cdot (-3)] = (-4) \cdot [12 - 9] = (-4) \cdot 3 = -12
% \]
% \end{solution}
%  
% \part \( 4 \cdot (-6) - 8 \cdot (-4)= \)
% \begin{solution}
% \[
% 4 \cdot (-6) - 8 \cdot (-4) = -24 + 32 = -56
% \]
% \end{solution}
%  
% \part \( 6 - (-3) + (-2) \cdot [ (2) + (-3) - 5] = \)

% \begin{solution}
% \[
% 6 - (-3) + (-2) \cdot [ (2) + (-3) - 5] = 6 + 3 + (-2) \cdot (-6) = 9 + 12 = 21
% \]
% \end{solution}
%  
% \part \( 8 - [ (6-11) + (2-5) - (7-10)] = \)

% \begin{solution}
% \[
% 8 - [ (6-11) + (2-5) - (7-10)] = 8 - [ -5 + (-3) + 3] = 8 - (-5) = 13
% \]
% \end{solution}
%  
% \part \( (13 - 21) - [ (2 - 5 + 8) - 2 \cdot (6 - 9)] = \)

% \begin{solution}
% \[
% (13 - 21) - [ (2 - 5 + 8) - 2 \cdot (6 - 9)] = (-8) - [5 - 2 \cdot (-3)] = -8 - [5 + 6] = -8 - 11 = -19
% \]
% \end{solution}
%  
% \part \( [2 \cdot (-4) - (11 + 5)] : [ 2 \cdot (5 + 9 + 6) - (7 - 10)] = \)

% \begin{solution}
% \[
% [2 \cdot (-4) - (11 + 5)] : [ 2 \cdot (5 + 9 + 6) - (7 - 10)] = [-8 - 16] : [2 \cdot 20 - (-3)] = (-24) : [40 + 3] = (-24) : 43 = -\frac{24}{43}
% \]
% \end{solution}
%  
% \part \( \left[2 \cdot (4 - 5) \cdot (-3) - (8 - 2) : (-3)\right] \cdot (-4) = \)

% \begin{solution}
% \[
% \left[2 \cdot (4 - 5) \cdot (-3) - (8 - 2) : (-3)\right] \cdot (-4) = \left[2 \cdot (-1) \cdot (-3) - 6 : (-3)\right] \cdot (-4) = \left[6 - (-2)\right] \cdot (-4) = [6 + 2] \cdot (-4) = 8 \cdot (-4) = -32
% \]
% \end{solution}
%  
% \end{parts}

% Encuentra las siguientes palabras: raíces, potencias, decimales, dividendo, divisor, cociente, resto \\ \newline 
% \begin{tabular}{|c|c|c|c|c|c|c|c|c|c|c|}
% \hline
% T & M & C & T & R & A & I & C & E & S & P \\ \hline
% K & G & V & N & H & S & E & X & H & O & G \\ \hline
% T & D & E & L & U & P & Y & U & T & I & C \\ \hline
% D & I & V & I & S & O & R & E & H & O & O \\ \hline
% C & V & J & M & X & P & N & D & C & A & A \\ \hline
% T & I & I & T & M & C & P & I & B & P & G \\ \hline
% W & D & D & H & I & Z & E & K & Y & Y & S \\ \hline
% R & E & Y & A & A & N & S & N & X & C & R \\ \hline
% N & N & S & Q & T & E & O & Y & X & E & M \\ \hline
% R & D & D & E & C & I & M & A & L & E & S \\ \hline
% J & O & U & X & R & E & S & T & O & X & C \\ \hline
% \end{tabular} \begin{solution}  \newline 
% \begin{tabular}{|c|c|c|c|c|c|c|c|c|c|c|}
% \hline
% T & M & C & T & \textbf{R} & \textbf{A} & \textbf{I} & \textbf{C} & \textbf{E} & \textbf{S} & \textbf{P} \\ \hline
% K & G & V & N & H & S & E & X & H & \textbf{O} & G \\ \hline
% T & \textbf{D} & E & L & U & P & Y & U & \textbf{T} & I & \textbf{C} \\ \hline
% \textbf{D} & \textbf{I} & \textbf{V} & \textbf{I} & \textbf{S} & \textbf{O} & \textbf{R} & \textbf{E} & H & \textbf{O} & O \\ \hline
% C & \textbf{V} & J & M & X & P & \textbf{N} & D & \textbf{C} & A & A \\ \hline
% T & \textbf{I} & I & T & M & \textbf{C} & P & \textbf{I} & B & P & G \\ \hline
% W & \textbf{D} & D & H & \textbf{I} & Z & \textbf{E} & K & Y & Y & S \\ \hline
% R & \textbf{E} & Y & \textbf{A} & A & \textbf{N} & S & N & X & C & R \\ \hline
% N & \textbf{N} & \textbf{S} & Q & \textbf{T} & E & O & Y & X & E & M \\ \hline
% R & \textbf{D} & \textbf{D} & \textbf{E} & \textbf{C} & \textbf{I} & \textbf{M} & \textbf{A} & \textbf{L} & \textbf{E} & \textbf{S} \\ \hline
% J & \textbf{O} & U & X & \textbf{R} & \textbf{E} & \textbf{S} & \textbf{T} & \textbf{O} & X & C \\ \hline
% \end{tabular}\end{solution}

% Preguntas extraídas de: 03_fracciones.tex
%\question 
%
%\begin{parts}
%\part[2] 
%\begin{solution}
%\end{solution}
%
%
%\end{parts}
%\addpoints





% \question[1\half] Expresa como una única potencia:
% \begin{multicols}{3}
% \begin{parts}
%     \part $\left(6^3\right)^4=$
%     
%     \part $5^2\cdot 5^3=$
%     
%     \part $n^4 : n^2=$
%     
% \end{parts}
% \end{multicols}



\question[2] Aplica las propiedades para expresar como una única potencia y después calcula su valor:
\begin{multicols}{2}
\begin{parts}
    % \part $3^2\cdot 3^3 : 3^5=$
    \part $\left(2^2\right)^3\cdot\left(5^2\right)^3=$
    \part $20^{6} :10^6=$
    % \part $18^4:\left(2^4\cdot3^4\right)=$
    \part $24^5:\left(2^5\cdot6^5\right)=$
    \part $\left[\left(12^3:4^3\right)^2:\left(15^2:5^2\right)^3\right]^8=$
\end{parts}
\end{multicols}




% \question Resuelve las siguientes operaciones:
% \begin{parts}
%     \part[1] $3^4:\sqrt{1+\left(20+6\cdot10\right)}=$
    
%     \part[1] $\left(6 + 2 \cdot 3^2 + 3 \cdot 2^2 \right) : \left(3 - \sqrt{81} \right)^2 =
% $ \begin{solution} $1$ \end{solution}
% %     \part[1] $\left[ 9 - \sqrt{25} \cdot (-2)^3 \right] : \left[ (-3 - 1)^2 - 9 \right] =
% % $\begin{solution}$7$\end{solution}


% %     \part[1\half] $\dfrac{\left[(-3)^3\right]^2 \cdot \left[3 \cdot (-9)\right]^6}{81^5} =
% % $\begin{solution}
% %     $3^4$
% % \end{solution} 
% \end{parts}

% \question[1]
%     Comprobar si son equivalentes las siguientes fracciones:
%     \begin{multicols}{2}
        
    
% \begin{parts}
%   \part \( \dfrac{84}{60} \) y \( \dfrac{14}{10} \)
%   \part \( \dfrac{14}{35} \) y \( \dfrac{26}{10} \)
% \end{parts}

% \end{multicols}
% 

\question[1]
Hallar \( x \) para que las siguientes fracciones sean equivalentes:
\begin{multicols}{2}    
\begin{parts}
  \part \( \dfrac{22}{14} \) = \( \dfrac{x}{91} \)
  \part \( \dfrac{21}{15} \) = \( \dfrac{25}{x} \)
\end{parts}
\end{multicols}



\question[2]
Simplifica las siguientes fracciones:
\begin{parts}
  \part \( \dfrac{756}{198} \)
  \part \( \dfrac{1225}{455} \)
\end{parts}

\question[3]
Ordena de mayor a menor, utilizando fracciones, los siguientes números:
$$1,4; \ 1,2\wideparen{3}; \ 1,36; \  \dfrac{43}{30} ; \ \dfrac{15}{22}$$



\question
Calcula y simplifica:
\begin{parts}
  \part[2] \( \left( 2 - \dfrac{1}{5} \right) \) + \( 7 - \dfrac{5}{12} =\) 
  \part[2] \( \dfrac{3}{5} - \dfrac{2}{5} \cdot \left( 1 - \dfrac{1}{3} \right) - 3 \cdot \dfrac{2}{9}=\)
  \part[2] \( \dfrac{7}{5} : \left[ \dfrac{3}{5} - 2 \left( 1 - \dfrac{4}{5} \right) \right]= \)
  \part[3] $\dfrac{\left[ (-2)^4 \right]^6 : \left( (2^2 \cdot 8)^4 \right)}{\left( \dfrac{4}{3} \right)^8 : \left( \dfrac{4}{3} \right)^6 \cdot (-1)^8}=$ 

\end{parts}

\question
Una cisterna está llena de agua. Se sacan los \( \dfrac{3}{5} \) de su contenido y después los \( \dfrac{3}{4} \) del resto:
\begin{parts}
  \part[1] ¿Qué fracción de la capacidad de la cisterna se ha sacado?
  \part[2] Quedan en la cisterna 12 litros. ¿Cuál es su capacidad?
\end{parts}

\question[3]
Un comerciante tiene una deuda, paga primero \( \dfrac{1}{4} \) de ella, y luego \( \dfrac{1}{6} \). Todavía le quedan por pagar 700 €. ¿A cuánto ascendía la deuda?

\question[2]
Con el contenido de un bidón de agua se han llenado 40 botellas de \( \dfrac{3}{4} \) de litro. ¿Cuántos litros de agua había en el bidón?



% \question[] Encuentra las siguientes palabras: numerador, denominador, generatriz, periódico \\ \newline 
% \begin{tabular}{|c|c|c|c|c|c|c|c|c|c|c|c|c|}
% \hline
% G & E & N & E & R & A & T & R & I & Z & V & I & T \\ \hline
% O & R & B & C & L & X & N & K & Z & W & D & S & V \\ \hline
% D & E & N & O & M & I & N & A & D & O & R & T & U \\ \hline
% U & M & G & S & G & J & F & T & R & D & C & V & R \\ \hline
% Y & X & Y & Y & N & S & P & E & S & Y & X & J & S \\ \hline
% J & U & G & C & X & U & C & J & A & S & K & I & O \\ \hline
% J & X & C & B & G & U & M & I & R & T & B & Q & H \\ \hline
% M & N & E & W & G & C & H & E & R & O & X & D & E \\ \hline
% I & I & V & C & Q & C & L & U & R & T & H & G & L \\ \hline
% S & L & F & O & J & Z & J & Y & F & A & N & Q & Z \\ \hline
% K & H & T & R & O & U & J & V & B & U & D & X & H \\ \hline
% U & P & E & R & I & O & D & I & C & O & A & O & G \\ \hline
% X & R & J & G & E & B & C & F & T & B & H & Y & R \\ \hline
% \end{tabular} \begin{solution}  \newline 
% \begin{tabular}{|c|c|c|c|c|c|c|c|c|c|c|c|c|}
% \hline
% \textbf{G} & \textbf{E} & \textbf{N} & \textbf{E} & \textbf{R} & \textbf{A} & \textbf{T} & \textbf{R} & \textbf{I} & \textbf{Z} & V & I & T \\ \hline
% O & R & B & C & L & X & N & K & Z & W & D & S & V \\ \hline
% \textbf{D} & \textbf{E} & \textbf{N} & \textbf{O} & \textbf{M} & \textbf{I} & \textbf{N} & \textbf{A} & \textbf{D} & \textbf{O} & \textbf{R} & T & U \\ \hline
% U & M & G & S & G & J & F & T & R & D & C & V & R \\ \hline
% Y & X & Y & Y & \textbf{N} & S & P & E & S & Y & X & J & S \\ \hline
% J & U & G & C & X & \textbf{U} & C & J & A & S & K & I & O \\ \hline
% J & X & C & B & G & U & \textbf{M} & I & R & T & B & Q & H \\ \hline
% M & N & E & W & G & C & H & \textbf{E} & R & O & X & D & E \\ \hline
% I & I & V & C & Q & C & L & U & \textbf{R} & T & H & G & L \\ \hline
% S & L & F & O & J & Z & J & Y & F & \textbf{A} & N & Q & Z \\ \hline
% K & H & T & R & O & U & J & V & B & U & \textbf{D} & X & H \\ \hline
% U & \textbf{P} & \textbf{E} & \textbf{R} & \textbf{I} & \textbf{O} & \textbf{D} & \textbf{I} & \textbf{C} & \textbf{O} & A & \textbf{O} & G \\ \hline
% X & R & J & G & E & B & C & F & T & B & H & Y & \textbf{R} \\ \hline
% \end{tabular}\end{solution}

% Preguntas extraídas de: 04_proporcionalidad_porcentajes.tex
%\question 
%
%\begin{parts}
%\part[2] 
%\begin{solution}
%\end{solution}
%
%
%\end{parts}
%\addpoints









% \question Resuelve las siguientes operaciones:
% \begin{parts}
%     \part[1] $3^4:\sqrt{1+\left(20+6\cdot10\right)}=$
%     
%     \part[1] $\left(6 + 2 \cdot 3^2 + 3 \cdot 2^2 \right) : \left(3 - \sqrt{81} \right)^2 =
% $ \begin{solution} $1$ \end{solution}
% %     \part[1] $\left[ 9 - \sqrt{25} \cdot (-2)^3 \right] : \left[ (-3 - 1)^2 - 9 \right] =
% % $\begin{solution}$7$\end{solution}


% %     \part[1\half] $\dfrac{\left[(-3)^3\right]^2 \cdot \left[3 \cdot (-9)\right]^6}{81^5} =
% % $\begin{solution}
% %     $3^4$
% % \end{solution} 
% \end{parts}



\question
Calcula y simplifica:
\begin{parts}
  \part[2] \( \left( 2 - \dfrac{1}{5} \right) \) + \( 7 - \dfrac{5}{12} =\) 
  % \part[2] \( \dfrac{3}{5} - \dfrac{2}{5} \cdot \left( 1 - \dfrac{1}{3} \right) - 3 \cdot \dfrac{2}{9}=\)
  \part[2] \( \dfrac{7}{5} : \left[ \dfrac{3}{5} - 2 \left( 1 - \dfrac{4}{5} \right) \right]= \)
  \part[2] $\dfrac{\left[ (-2)^4 \right]^6 : \left( (2^2 \cdot 8)^4 \right)}{\left( \dfrac{4}{3} \right)^8 : \left( \dfrac{4}{3} \right)^6 \cdot (-1)^8}=$ 

\end{parts}

\question[1] Escribe tres parejas de números cuya razón sea \( \tfrac{5}{7} \).
\begin{solution}
Algunas posibles parejas son: \((5,7), (10,14), (15,21)\).
\end{solution}


\question[1] Halla el término desconocido en cada proporción.
\begin{parts}

    \part \( \frac{9}{12} = \frac{27}{x} \)
    \begin{solution}
    \( x = 36 \)
    \end{solution}
    

    \part \( \frac{32}{x} = \frac{8}{5} \)
    \begin{solution}
    \( x = 20 \)
    \end{solution}
    

    % \part \( \frac{x}{18} = \frac{14}{21} \)
    % \begin{solution}
    % \( x = 12 \)
    % \end{solution}

\end{parts}

\question Completa la siguiente tabla.

\[
\begin{array}{c|cccc}
\text{MAGNITUD M} & 2 & 3 & 5 & 8 \\ \hline
\text{MAGNITUD N} & 30 & \, \ \, & \, \ \, & \, \ \
\end{array}
\]

\begin{parts}

    \part[1] Suponiendo que las magnitudes \( M \) y \( N \) son directamente proporcionales.
    \begin{solution}
    Constante de proporcionalidad \( k = \frac{30}{2} = 15 \).  
    Entonces: \(N = (30, 45, 75, 120)\).
    \end{solution}
    

    \part[1] Suponiendo que las magnitudes \( M \) y \( N \) son inversamente proporcionales.
    \begin{solution}
    Producto constante: \( M \cdot N = 2 \cdot 30 = 60 \).  
    Entonces: \(N = (30, 20, 12, 7.5)\).
    \end{solution}
    

\end{parts}

% \question Identifica si la siguiente tabla representa proporcionalidad directa o inversa.

% \[
% \begin{array}{c|ccccc}
% \text{MAGNITUD M} & 2 & 4 & 6 & 8 & 12 \\ \hline
% \text{MAGNITUD N} & 90 & 45 & 30 & 22.5 & 15
% \end{array}
% \]

% \begin{solution}
% El producto \( M \cdot N = 180 \) es constante.  
% Por tanto, es una **proporcionalidad inversa**.
% \end{solution}

% \question Completa la tabla, sabiendo que \( A \) y \( B \) son directamente proporcionales.

% \[
% \begin{array}{c|cccccc}
% A & 2 & 4 & 6 & 10 & 12 & 20 \\ \hline
% B & 5 & \,?\, & \,?\, & \,?\, & \,?\, & \,?\, 
% \end{array}
% \]

% \begin{solution}
% La constante de proporcionalidad es \( k = \frac{5}{2} = 2.5 \).  
% Así, \( B = (5, 10, 15, 25, 30, 50) \).
% \end{solution}

% \question Copia y completa esta tabla de proporcionalidad, sabiendo que la constante es \(0.6\).

% \[
% \begin{array}{c|ccc}
% \text{Magnitud A} & 10 & 25 & 40 \\ \hline
% \text{Magnitud B} & \,?\, & \,?\, & \,?\,
% \end{array}
% \]

% \begin{solution}
% Se multiplica cada valor de \(A\) por \(0.6\):  
% \( B = (6, 15, 24) \).
% \end{solution}


\question Resuelve por reducción a la unidad.

\begin{parts}

\part[2] Un manantial arroja 240 L de agua en 8 minutos.  
¿Cuántos litros arrojará en 20 minutos?

\begin{solution}
Primero hallamos los litros por minuto:  
\[
\frac{240}{8} = 30 \text{ L/min}
\]  
En 20 minutos arrojará:  
\[
30 \times 20 = 600 \text{ L}
\]
\end{solution}


\part[2] Abriendo 5 grifos, un depósito se vacía en 40 minutos.  
¿Cuánto tardará en vaciarse abriendo solo 3 grifos?

\begin{solution}
El tiempo es inversamente proporcional al número de grifos.  
Producto constante:  
\[
5 \times 40 = 200
\]  
Por tanto:  
\[
t = \frac{200}{3} \approx 66.7 \text{ min}
\]
\end{solution}


\end{parts}


\question Reparte una cantidad en forma proporcional.

\begin{parts}

\part[3] Tres socios (A, B y C) invierten 4 000 €, 6 000 € y 10 000 € respectivamente.  
Si al final obtienen un beneficio total de 12 000 €, ¿cuánto recibe cada uno?

\begin{solution}
La suma de las inversiones es:  
\[
4000 + 6000 + 10000 = 20000
\]

La proporción de cada socio es:

\[
\text{A: } \frac{4000}{20000} = 0.2,\qquad
\text{B: } \frac{6000}{20000} = 0.3,\qquad
\text{C: } \frac{10000}{20000} = 0.5
\]

Reparto del beneficio:

\[
\text{A: } 12000 \times 0.2 = 2400\ \text{€}
\]
\[
\text{B: } 12000 \times 0.3 = 3600\ \text{€}
\]
\[
\text{C: } 12000 \times 0.5 = 6000\ \text{€}
\]
\end{solution}


\end{parts}


% \question[1] Completa la siguiente tabla.

% \begin{solution}
% \[
% \begin{array}{|c|c|c|}
% \hline
% \textbf{Porcentaje} & \textbf{Fracción} & \textbf{Decimal} \\
% \hline
% 20\% & \frac{1}{5} & 0.2 \\
% \hline
% 75\% & \frac{3}{4} & 0.75 \\
% \hline
% \end{array}
% \]
% \end{solution}

\question[2] Calcula lo siguiente:

\begin{parts}
    \part 80\% de 150 
    \begin{solution}
    80\% de 150 es:
    \[
    0.80 \times 150 = 120
    \]
    \end{solution}
    

    \part 5\% de 2000
    \begin{solution}
    5\% de 2000 es:
    \[
    0.05 \times 2000 = 100
    \]
    \end{solution}
    

    % \part 25\% de 120
    % \begin{solution}
    % 25\% de 120 es:
    % \[
    % 0.25 \times 120 = 30
    % \]
    % \end{solution}

    \part 130\% de 45
    \begin{solution}
    130\% de 45 es:
    \[
    1.30 \times 45 = 58.5
    \]
    \end{solution}
    

    \part 0,25\% de 400
    \begin{solution}
    0,25\% de 400 es:
    \[
    0.0025 \times 400 = 1
    \]
    \end{solution}
    

    % \part 3,5\% de 60
    % \begin{solution}
    % 3,5\% de 60 es:
    % \[
    % 0.035 \times 60 = 2.1
    % \]
    % \end{solution}
\end{parts}

% \question Calcula \( x \) en cada caso:

% \begin{parts}
%     \part 60\% de \( x \) = 18
%     \begin{solution}
%     Para encontrar \( x \), se resuelve la ecuación:
%     \[
%     0.60 \times x = 18 \quad \Rightarrow \quad x = \frac{18}{0.60} = 30
%     \]
%     \end{solution}

%     \part 20\% de \( x \) = 10
%     \begin{solution}
%     Para encontrar \( x \), se resuelve la ecuación:
%     \[
%     0.20 \times x = 10 \quad \Rightarrow \quad x = \frac{10}{0.20} = 50
%     \]
%     \end{solution}

%     \part 30\% de \( x \) = 21
%     \begin{solution}
%     Para encontrar \( x \), se resuelve la ecuación:
%     \[
%     0.30 \times x = 21 \quad \Rightarrow \quad x = \frac{21}{0.30} = 70
%     \]
%     \end{solution}

%     \part 80\% de \( x \) = 16
%     \begin{solution}
%     Para encontrar \( x \), se resuelve la ecuación:
%     \[
%     0.80 \times x = 16 \quad \Rightarrow \quad x = \frac{16}{0.80} = 20
%     \]
%     \end{solution}

%     \part 10\% de \( x \) = 3
%     \begin{solution}
%     Para encontrar \( x \), se resuelve la ecuación:
%     \[
%     0.10 \times x = 3 \quad \Rightarrow \quad x = \frac{3}{0.10} = 30
%     \]
%     \end{solution}

%     \part 50\% de \( x \) = 25
%     \begin{solution}
%     Para encontrar \( x \), se resuelve la ecuación:
%     \[
%     0.50 \times x = 25 \quad \Rightarrow \quad x = \frac{25}{0.50} = 50
%     \]
%     \end{solution}
% \end{parts}

\question[2] La entrada de cine costaba 8,00 \euro, y hoy ha subido un 10\%. ¿Cuánto cuesta ahora?

\begin{solution}
Para calcular el nuevo precio después de la subida, multiplicamos el precio original por el incremento del 10\%:

\[
8,00 \, \text{\euro} \times (1 + 0,10) = 8,00 \, \text{\euro} \times 1,10 = 8,80 \, \text{€}
\]

Por lo tanto, el nuevo precio es 8,80 €.
\end{solution}


\question[2] Un juguete vale en una juguetería 40 euros. Durante las fiestas navideñas sube un 22\% y una vez que éstas han pasado, baja un 9\%. Calcular su precio final y el porcentaje de subida respecto al precio inicial.

\begin{solution}
Primero, calculamos el precio después de la subida del 22\%:
\[
\text{Precio después de la subida} = 40 \, \text{€} \times (1 + 0.22) = 40 \, \text{€} \times 1.22 = 48.8 \, \text{€}
\]

Luego, calculamos el precio después de la bajada del 9\%:
\[
\text{Precio después de la bajada} = 48.8 \, \text{€} \times (1 - 0.09) = 48.8 \, \text{€} \times 0.91 = 44.368 \, \text{€}
\]

Por lo tanto, el precio final del juguete es 44,37 €.

Ahora, calculamos el porcentaje de subida respecto al precio inicial. El aumento fue de:
\[
\text{Aumento} = 44.368 \, \text{€} - 40 \, \text{€} = 4.368 \, \text{€}
\]
El porcentaje de subida es:
\[
\frac{4.368 \, \text{€}}{40 \, \text{€}} \times 100 = 10.92\%
\]
Por lo tanto, el porcentaje de subida respecto al precio inicial es del 10.92\%.
\end{solution}


\question[2] Al rebajar el precio de un ordenador ha pasado de 1100 euros a 957 euros. ¿Qué tanto por ciento ha bajado?

\begin{solution}
La disminución en el precio es:
\[
\text{Disminución} = 1100 \, \text{€} - 957 \, \text{€} = 143 \, \text{€}
\]
El porcentaje de bajada es:
\[
\frac{143 \, \text{€}}{1100 \, \text{€}} \times 100 = 13\%
\]
Por lo tanto, el precio del ordenador ha bajado un 13\%.
\end{solution}


\question[2] Seis grifos, tardan 10 horas en llenar un depósito de 400 m³ de capacidad. ¿Cuántas horas tardarán cuatro grifos en llenar 2 depósitos de 500 m³ cada uno?

\begin{solution}
Primero, calculamos el trabajo que realiza un grifo en una hora. Si 6 grifos llenan 400 m³ en 10 horas, el trabajo de un solo grifo en una hora es:
\[
\text{Trabajo por grifo por hora} = \frac{400 \, \text{m}^3}{6 \times 10} = \frac{400 \, \text{m}^3}{60} = 6.67 \, \text{m}^3/\text{hora}
\]

Ahora, para 4 grifos, el trabajo combinado por hora es:
\[
\text{Trabajo total por hora} = 4 \times 6.67 \, \text{m}^3/\text{hora} = 26.68 \, \text{m}^3/\text{hora}
\]

El volumen total que deben llenar 4 grifos es:
\[
\text{Volumen total} = 2 \times 500 \, \text{m}^3 = 1000 \, \text{m}^3
\]

Finalmente, calculamos el tiempo necesario para llenar el volumen total:
\[
\text{Tiempo} = \frac{1000 \, \text{m}^3}{26.68 \, \text{m}^3/\text{hora}} \approx 37.5 \, \text{horas}
\]

Por lo tanto, 4 grifos tardarán aproximadamente 37.5 horas en llenar 2 depósitos de 500 m³ cada uno.
\end{solution}


% \question[] Encuentra las siguientes palabras: porcentaje, razon, proporcion, directa, inversa, compuesta \\ \newline 
% \begin{tabular}{|c|c|c|c|c|c|c|c|c|c|c|c|}
% \hline
% A & R & R & K & W & I & W & H & O & I & C & T \\ \hline
% F & K & M & C & S & M & O & R & F & N & F & P \\ \hline
% I & H & A & R & M & R & S & W & Y & V & V & R \\ \hline
% Z & J & L & M & C & L & P & C & B & E & D & O \\ \hline
% I & Y & M & G & G & A & F & V & X & R & Y & P \\ \hline
% E & R & A & Z & O & N & W & H & A & S & T & O \\ \hline
% W & K & P & W & A & A & I & N & X & A & Y & R \\ \hline
% P & O & R & C & E & N & T & A & J & E & U & C \\ \hline
% Z & C & D & I & R & E & C & T & A & L & F & I \\ \hline
% A & O & Z & V & V & R & K & C & L & N & K & O \\ \hline
% E & I & N & Y & I & Q & Z & C & U & D & G & N \\ \hline
% Y & P & X & C & O & M & P & U & E & S & T & A \\ \hline
% \end{tabular} \begin{solution}  \newline 
% \begin{tabular}{|c|c|c|c|c|c|c|c|c|c|c|c|}
% \hline
% A & R & R & K & W & I & W & H & O & \textbf{I} & C & T \\ \hline
% F & K & M & C & S & M & O & R & F & \textbf{N} & F & \textbf{P} \\ \hline
% I & H & A & R & M & R & S & W & Y & \textbf{V} & V & \textbf{R} \\ \hline
% Z & J & L & M & C & L & P & C & B & \textbf{E} & D & \textbf{O} \\ \hline
% I & Y & M & G & G & A & F & V & X & \textbf{R} & Y & \textbf{P} \\ \hline
% E & \textbf{R} & \textbf{A} & \textbf{Z} & \textbf{O} & \textbf{N} & W & H & A & \textbf{S} & T & \textbf{O} \\ \hline
% W & K & P & W & A & A & I & N & X & \textbf{A} & Y & \textbf{R} \\ \hline
% \textbf{P} & \textbf{O} & \textbf{R} & \textbf{C} & \textbf{E} & \textbf{N} & \textbf{T} & \textbf{A} & \textbf{J} & \textbf{E} & U & \textbf{C} \\ \hline
% Z & C & \textbf{D} & \textbf{I} & \textbf{R} & \textbf{E} & \textbf{C} & \textbf{T} & \textbf{A} & L & F & \textbf{I} \\ \hline
% A & O & Z & V & V & R & K & C & L & N & K & \textbf{O} \\ \hline
% E & I & N & Y & I & Q & Z & C & U & D & G & \textbf{N} \\ \hline
% Y & P & X & \textbf{C} & \textbf{O} & \textbf{M} & \textbf{P} & \textbf{U} & \textbf{E} & \textbf{S} & \textbf{T} & \textbf{A} \\ \hline
% \end{tabular}\end{solution}

\end{questions}
\end{document}
