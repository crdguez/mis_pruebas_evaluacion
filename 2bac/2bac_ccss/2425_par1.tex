\documentclass[addpoints,spanish, 12pt,a4paper]{exam}
%\documentclass[answers, spanish, 12pt,a4paper]{exam}
\printanswers
\renewcommand*\half{.5}
\pointpoints{punto}{puntos}
\hpword{Puntos:}
\vpword{Puntos:}
\htword{Total}
\vtword{Total}
\hsword{Resultado:}
\hqword{Ejercicio:}
\vqword{Ejercicio:}

\usepackage[utf8]{inputenc}
\usepackage[spanish]{babel}
\usepackage{eurosym}
%\usepackage[spanish,es-lcroman, es-tabla, es-noshorthands]{babel}


\usepackage[margin=1in]{geometry}
\usepackage{amsmath,amssymb}
\usepackage{multicol}
\usepackage{yhmath}

\pointsinrightmargin % Para poner las puntuaciones a la derecha. Se puede cambiar. Si se comenta, sale a la izquierda.
\extrawidth{-2.4cm} %Un poquito más de margen por si ponemos textos largos.
\marginpointname{ \emph{\points}}

\usepackage{graphicx}

\graphicspath{{../../img/}} 

\newcommand{\class}{2º Bachillerato CCSS}
\newcommand{\examdate}{\today}
\newcommand{\examnum}{Parcial 1ªEv.}
\newcommand{\tipo}{A}


\newcommand{\timelimit}{45 minutos}

\renewcommand{\solutiontitle}{\noindent\textbf{Solución:}\enspace}


\pagestyle{head}
\firstpageheader{\includegraphics[width=0.2\columnwidth]{header_left}}{\textbf{Departamento de Matemáticas\linebreak \class}\linebreak \examnum}{\includegraphics[width=0.1\columnwidth]{header_right}}
\runningheader{\class}{\examnum}{Página \thepage\ de \numpages}
\runningheadrule


\usepackage{pgf,tikz,pgfplots}
\pgfplotsset{compat=1.15}
\usepackage{mathrsfs}
\usetikzlibrary{arrows}


\begin{document}

\noindent
\begin{tabular*}{\textwidth}{l @{\extracolsep{\fill}} r @{\extracolsep{6pt}} }
\textbf{Nombre:} \makebox[3.5in]{\hrulefill} & \textbf{Fecha:}\makebox[1in]{\hrulefill} \\
 & \\
\textbf{Tiempo: \timelimit} & Tipo: \tipo 
\end{tabular*}
\rule[2ex]{\textwidth}{2pt}
Esta prueba tiene \numquestions\ ejercicios. La puntuación máxima es de \numpoints. 
La nota final de la prueba será la parte proporcional de la puntuación obtenida sobre la puntuación máxima. 

\begin{center}


\addpoints
 %\gradetable[h][questions]
	\pointtable[h][questions]
\end{center}

\noindent
\rule[2ex]{\textwidth}{2pt}

\begin{questions}

%\question 
%
%\begin{parts}
%\part[2] 
%\begin{solution}
%\end{solution}
%
%
%\end{parts}
%\addpoints



\question Se considera la matriz $A$ dada por:
$$A=\left(\begin{matrix}0 & 0 & 6\\\frac{1}{2} & 0 & 0\\0 & \frac{1}{3} & 0\end{matrix}\right)$$
% A= Matrix(3,3,[nsimplify(i) for i in [0,0,6,1/2,0,0,0,1/3,0]])
\begin{parts}
    \part[1] Determina razonadamente $A^3$ y $A^{2023}$
    \begin{solution}
        $A^2=\left(\begin{matrix}0 & 2 & 0\\0 & 0 & 3\\\frac{1}{6} & 0 & 0\end{matrix}\right)$, $A^3=\left(\begin{matrix}1 & 0 & 0\\0 & 1 & 0\\0 & 0 & 1\end{matrix}\right)$, $A^4=\left(\begin{matrix}0 & 0 & 6\\\frac{1}{2} & 0 & 0\\0 & \frac{1}{3} & 0\end{matrix}\right)$,
        $2023=3\cdot674+1 \to A^{2023}=A=\left(\begin{matrix}0 & 0 & 6\\\frac{1}{2} & 0 & 0\\0 & \frac{1}{3} & 0\end{matrix}\right)$
    \end{solution}
    % \part[1] Determina justificadamente, sin calcularla, si la matriz A tiene inversa. 
    % \begin{solution}
    %     $|A|=1 \to \left(\begin{matrix}0 & 2 & 0\\0 & 0 & 3\\\frac{1}{6} & 0 & 0\end{matrix}\right)$
    % \end{solution}
\end{parts}

\question Dadas las matrices  $A=\left(\begin{matrix}-2 & -1 & 1\\-1 & 0 & 1\end{matrix}\right)$ y $B=\left(\begin{matrix}1 & -1\\2 & 0\\-2 & 1\end{matrix}\right)$
%\noaddpoints % to omit double points count

\begin{parts}
\part[1] Calcula $C=B\cdot A - A^t \cdot B^t$
\begin{solution}
$B\cdot A=\left(\begin{matrix}-1 & -1 & 0\\-4 & -2 & 2\\3 & 2 & -1\end{matrix}\right)$ y $A^t\cdot B^t=\left(\begin{matrix}-1 & -4 & 3\\-1 & -2 & 2\\0 & 2 & -1\end{matrix}\right)$.Luego $C=B\cdot A - A^t \cdot B^t=\left(\begin{matrix}0 & 3 & -3\\-3 & 0 & 0\\3 & 0 & 0\end{matrix}\right)$.
\end{solution}

%\part[1] Sea $D=A\cdot B$. Halla $D^{-1}$
%\part[1] Halle la matriz $X$ siendo $D \cdot X =\left(\begin{matrix}4\\2\end{matrix}\right)$
\part[2] Halle la matriz $X$ siendo $A\cdot B \cdot X =\left(\begin{matrix}4\\2\end{matrix}\right)$
 \begin{solution}
$A \cdot B = \left(\begin{matrix}-6 & 3\\-3 & 2\end{matrix}\right)$ y $(A \cdot B)^{-1}=\left(\begin{matrix}- \frac{2}{3} & 1\\-1 & 2\end{matrix}\right)\xrightarrow{X} X=\left(\begin{matrix}- \frac{2}{3} & 1\\-1 & 2\end{matrix}\right)\cdot \left(\begin{matrix}4\\2\end{matrix}\right)=\left(\begin{matrix}- \frac{2}{3}\\0\end{matrix}\right)$
 \end{solution}

\end{parts}
\addpoints

% \question Dadas las matrices  $A=\left(\begin{matrix}1 & 0 & -1\\-3 & 2 & 3\\-1 & 3 & 0\end{matrix}\right)$, $B=\left(\begin{matrix}2 & 1\\0 & -1\\1 & 2\end{matrix}\right)$ y $C=\left(\begin{matrix}1 & -1 & 0\\1 & 0 & 1\end{matrix}\right)$.
% %\noaddpoints % to omit double points count
% \begin{parts}
% \part[1] Encontrar si existe una matriz $X$ tal que $3\cdot X + 2\cdot A = B \cdot C$. 
% \begin{solution}
% $X = \frac{1}{3}\cdot (B \cdot C - 2 \cdot A)=\frac{1}{3}\left(\left(\begin{matrix}3 & -2 & 1\\-1 & 0 & -1\\3 & -1 & 2\end{matrix}\right)-\left(\begin{matrix}2 & 0 & -2\\-6 & 4 & 6\\-2 & 6 & 0\end{matrix}\right)\right)=\left(\begin{matrix}\frac{1}{3} & - \frac{2}{3} & 1\\\frac{5}{3} & - \frac{4}{3} & - \frac{7}{3}\\\frac{5}{3} & - \frac{7}{3} & \frac{2}{3}\end{matrix}\right)$ 
% \end{solution}


% \part[1] Encontrar si existe la matriz inversa de $A$, por determinantes
% \begin{solution}
% $\left(\begin{matrix}1 & 0 & -1\\-3 & 2 & 3\\-1 & 3 & 0\end{matrix}\right)\xrightarrow{traspuesta}\left(\begin{matrix}1 & -3 & -1\\0 & 2 & 3\\-1 & 3 & 0\end{matrix}\right)\xrightarrow{adjunta}\left(\begin{matrix}-9 & -3 & 2\\-3 & -1 & 0\\-7 & -3 & 2\end{matrix}\right)\xrightarrow{inversa}\left(\begin{matrix}\frac{9}{2} & \frac{3}{2} & -1\\\frac{3}{2} & \frac{1}{2} & 0\\\frac{7}{2} & \frac{3}{2} & -1\end{matrix}\right)$.
% \end{solution}



% \end{parts}
% \addpoints



% \question Dada: $$A=\left(\begin{matrix}2 & 0 & 1\\0 & 0 & 0\\1 & 0 & 2\end{matrix}\right)$$ 
% %\noaddpoints % to omit double points count

% \begin{parts}
% \part[1] Calcula $$A-2\cdot I$$ siendo $I$ la matriz identidad de orden 3. 
% \begin{solution}
% $A-2\cdot I=\left(\begin{matrix}0 & 0 & 1\\0 & -2 & 0\\1 & 0 & 0\end{matrix}\right)$.  
% \end{solution}
% \part[2] Determina los valores del parámetro $k$ para que tenga inversa la matriz: $$A-k\cdot I$$
% \begin{solution}
% $A-k\cdot I =\left(\begin{matrix}2 & 0 & 1\\0 & 0 & 0\\1 & 0 & 2\end{matrix}\right)-k \cdot \left(\begin{matrix}1 & 0 & 0\\0 & 1 & 0\\0 & 0 & 1\end{matrix}\right)=\left(\begin{matrix}2 - k & 0 & 1\\0 & - k & 0\\1 & 0 & 2 - k\end{matrix}\right)$$\to \left|\begin{matrix}2 - k & 0 & 1\\0 & - k & 0\\1 & 0 & 2 - k\end{matrix}\right|=- k \left(k - 3\right) \left(k - 1\right)$$\to k \neq\left( 0, \  1, \  3\right)$.
% \end{solution}

% \part[2] Encuentra la matriz $X$ que verifica que:$$(A-2\cdot I )\cdot X = 2\cdot I$$.
% \begin{solution}

% $(A-2\cdot I )\cdot X = 2\cdot I \to X=(A-2\cdot I)^{-1}\cdot 2I$\\Obtengamos las matrices que necesitamos: $(A-2\cdot I )=\left(\begin{matrix}0 & 0 & 1\\0 & -2 & 0\\1 & 0 & 0\end{matrix}\right)$ y $\left|\begin{matrix}0 & 0 & 1\\0 & -2 & 0\\1 & 0 & 0\end{matrix}\right|=2$.$\rightarrow$ $\left(\begin{matrix}0 & 0 & 1\\0 & -2 & 0\\1 & 0 & 0\end{matrix}\right)\xrightarrow{traspuesta}\left(\begin{matrix}0 & 0 & 1\\0 & -2 & 0\\1 & 0 & 0\end{matrix}\right)\xrightarrow{adjunta}\left(\begin{matrix}0 & 0 & 2\\0 & -1 & 0\\2 & 0 & 0\end{matrix}\right)\xrightarrow{inversa}\left(\begin{matrix}0 & 0 & 1\\0 & - \frac{1}{2} & 0\\1 & 0 & 0\end{matrix}\right)$.\\Por tanto $X=\left(\begin{matrix}0 & 0 & 1\\0 & - \frac{1}{2} & 0\\1 & 0 & 0\end{matrix}\right)\cdot\left(\begin{matrix}2 & 0 & 0\\0 & 2 & 0\\0 & 0 & 2\end{matrix}\right)=\left(\begin{matrix}0 & 0 & 2\\0 & -1 & 0\\2 & 0 & 0\end{matrix}\right)$
% \end{solution}
% \end{parts}

% \addpoints






\question[2\half] Dada la matriz: $$A=\left(\begin{matrix}k & 0 & k\\0 & k + 2 & 0\\1 & 1 & k + 2\end{matrix}\right)$$ 

\begin{parts}


\part Determine el rango de A según los diferentes valores de k  \begin{solution} El único menor es el propio determinante $\to\left|\begin{matrix}k & 0 & k\\0 & k + 2 & 0\\1 & 1 & k + 2\end{matrix}\right|=k \left(k + 1\right) \left(k + 2\right)\to k\neq\left( -2, \  -1, \  0\right)$ rango será 3 \\ 
\\ caso k=-2: \\ $B=\left(\begin{matrix}-2 & 0 & -2\\0 & 0 & 0\\1 & 1 & 0\end{matrix}\right) \to ran(B)=2$. 
\\ caso k=-1: \\ $B=\left(\begin{matrix}-1 & 0 & -1\\0 & 1 & 0\\1 & 1 & 1\end{matrix}\right) \to ran(B)=2$. 
\\ caso k=0: \\ $B=\left(\begin{matrix}0 & 0 & 0\\0 & 2 & 0\\1 & 1 & 2\end{matrix}\right) \to ran(B)=2$.  \end{solution}
% \part[] Determine la inversa de A para k=1  
% \begin{solution} $\left(\begin{matrix}1 & 0 & 1\\0 & 3 & 0\\1 & 1 & 3\end{matrix}\right)\xrightarrow{traspuesta}\left(\begin{matrix}1 & 0 & 1\\0 & 3 & 1\\1 & 0 & 3\end{matrix}\right)\xrightarrow{adjunta}\left(\begin{matrix}9 & 1 & -3\\0 & 2 & 0\\-3 & -1 & 3\end{matrix}\right)\xrightarrow{inversa}\left(\begin{matrix}\frac{3}{2} & \frac{1}{6} & - \frac{1}{2}\\0 & \frac{1}{3} & 0\\- \frac{1}{2} & - \frac{1}{6} & \frac{1}{2}\end{matrix}\right)$. \end{solution}
\end{parts}

\question Sabiendo que $\left|\begin{matrix}x & -3 & 1\\y & 0 & 1\\z & 7 & 1\end{matrix}\right|=6$, calcula usando las propiedades de los determinantes:
\begin{parts}
\part[1] $\left|\begin{matrix}\frac{z}{2} & z+7 & 3\\\frac{y}{2} & y & 3\\\frac{x}{2} & x-3 & 3\end{matrix}\right|$
\begin{solution}
$\left|\begin{matrix}\frac{z}{2} & z+7 & 3\\\frac{y}{2} & y+0 & 3\\\frac{x}{2} & x-3 & 3\end{matrix}\right|=
\frac{3}{2}\cdot\left|\begin{matrix}z & z+7 & 1\\y & y+0 & 1\\x & x-3 & 1\end{matrix}\right|=
\frac{3}{2}\cdot\left|\begin{matrix}z & z & 1\\y & y & 1\\x & x & 1\end{matrix}\right|+\frac{3}{2}\cdot\left|\begin{matrix}z & 7 & 1\\y & 0 & 1\\x & -3 & 1\end{matrix}\right|=\frac{3}{2}\cdot 0+\frac{3}{2}\cdot\left|\begin{matrix}z & 7 & 1\\y & 0 & 1\\x & -3 & 1\end{matrix}\right|=-\frac{3}{2}\cdot\left|\begin{matrix}x & -3 & 1\\y & 0 & 1\\z & 7 & 1\end{matrix}\right|=-\frac{3}{2}\cdot 6=-9
$
\end{solution}
% \part[1] $\left|\begin{matrix}x & -3 & 1 & 2\\ y & 0 & 1 &2\\z & 7 & 1&2 \\ 0&6&0&2\end{matrix}\right|$
% \begin{solution}
% Desarrollando por los ajuntos de la última fila: \\
% $\left|\begin{matrix}x & -3 & 1 & 2\\ y & 0 & 1 &2\\z & 7 & 1&2 \\ 0&6&0&2\end{matrix}\right|=
% 0\cdot A_{41}+6\cdot A_{42}+0\cdot A_{43}+2\cdot A_{44}=6\cdot \left|\begin{matrix}x&1&2\\y&1&2\\z&1&2 \end{matrix}\right|+2\cdot \left|\begin{matrix}x&-3&1\\y&0&1\\z&7&1 \end{matrix}\right|=6\cdot 0+2\cdot 6=12
% $
% \end{solution}
\end{parts}

% \question[1\half] Un arquitecto está diseñando un pequeño centro comercial con tres tiendas principales.
% Para asegurar una buena distribución de productos y servicios, ha identificado tres tipos de
% tiendas:
% \begin{itemize}
%     \item A: Ropa
%     \item B:Electrónica
%     \item C:Alimentos
% \end{itemize}
% El arquitecto tiene que decidir cómo asignar metros cuadrados a cada tipo de tienda en tres
% plantas diferentes del edificio. Los metros cuadrados disponibles en cada planta están
% organizados en la siguiente tabla:
% \begin{table}[h!]
%     \centering
%     \begin{tabular}{|c|c|c|c|}
%         \hline
%         \textbf{Planta} & \textbf{Tienda A (Ropa)} & \textbf{Tienda B (Electrónica)} & \textbf{Tienda C (Alimentos)} \\
%         \hline
%         Planta 1 & 200 m\(^2\) & 100 m\(^2\) & 150 m\(^2\) \\
%         \hline
%         Planta 2 & 180 m\(^2\) & 120 m\(^2\) & 130 m\(^2\) \\
%         \hline
%         Planta 3 & 220 m\(^2\) & 140 m\(^2\) & 170 m\(^2\) \\
%         \hline
%     \end{tabular}
%     \caption{Distribución de metros cuadrados por planta y tipo de tienda}
% \end{table}

% Además, las rentas por metro cuadrado varían según el tipo de tienda:

% \begin{itemize}
%     \item Tienda A (Ropa): 12 \euro/m\(^2\)
%     \item Tienda B (Electrónica): 15 \euro/m\(^2\)
%     \item Tienda C (Alimentos): 10 \euro/m\(^2\)
% \end{itemize}

% \begin{parts}
%     \part Representa los metros cuadrados asignados a cada tipo de tienda en cada planta
% mediante una matriz de distribución.
%     \part Crea una matriz de rentas, donde cada elemento corresponde a la renta por metro
% cuadrado de cada tipo de tienda. Luego, utilizando la multiplicación de matrices, calcula los
% ingresos totales que generaría el centro comercial en función de los metros cuadrados
% asignados y las rentas por metro cuadrado.
%     \part Si se decide aumentar en un 10\% el área asignada a la Tienda A en cada planta y reducir
% en un 5\% el área asignada a la Tienda B en la Planta 3, ¿cómo quedaría la nueva matriz de
% distribución?
% \end{parts}
\newpage
\question Un arquitecto está diseñando un pequeño centro comercial. Para asegurar una buena distribución de productos y servicios, ha identificado tres tipos de tiendas:
\begin{itemize}
    \item A: Ropa
    \item B: Electrónica
    \item C: Alimentos
\end{itemize}
El arquitecto decide asignar una tienda de cada tipo en cada una de las 3 plantas del centro comercial. Los metros cuadrados asignados por tipo de tienda en cada planta están organizados en la tabla siguiente:

\begin{table}[h!]
    \centering
    \begin{tabular}{|c|c|c|c|}
        \hline
        \textbf{Planta} & \textbf{Tienda A (Ropa)} & \textbf{Tienda B (Electrónica)} & \textbf{Tienda C (Alimentos)} \\
        \hline
        Planta 1 & 200 m\(^2\) & 100 m\(^2\) & 150 m\(^2\) \\
        \hline
        Planta 2 & 180 m\(^2\) & 120 m\(^2\) & 130 m\(^2\) \\
        \hline
        Planta 3 & 220 m\(^2\) & 140 m\(^2\) & 170 m\(^2\) \\
        \hline
    \end{tabular}
    \caption{Distribución de metros cuadrados por planta y tipo de tienda}
\end{table}

Además, la renta (ingresos) por metro cuadrado varían según el tipo de tienda:
\begin{itemize}
    \item Tienda A (Ropa): 12 \euro/m\(^2\)
    \item Tienda B (Electrónica): 15 \euro/m\(^2\)
    \item Tienda C (Alimentos): 10 \euro/m\(^2\)
\end{itemize}

\begin{parts}
    \part[0\half] Representa los metros cuadrados asignados a cada tipo de tienda en cada planta mediante una matriz
    
    \begin{solution}
    La matriz de distribución de los metros cuadrados asignados a cada tipo de tienda en cada planta es:

    \[
    D = \begin{pmatrix}
    200 & 100 & 150 \\
    180 & 120 & 130 \\
    220 & 140 & 170
    \end{pmatrix}
    \]
    \end{solution}

    \part[0\half] Crea una matriz de rentas, donde cada elemento corresponda a la renta por metro cuadrado de cada tipo de tienda. Luego, utilizando la multiplicación de matrices, calcula los ingresos por planta y el total que generaría el centro comercial en función de los metros cuadrados asignados y las rentas por metro cuadrado.

    \begin{solution}
    La matriz de rentas se puede expresar como:

    \[
    R = \begin{pmatrix}
    12  \\
    15 \\
    10
    \end{pmatrix}
    \]

    Calculamos los ingresos totales multiplicando \( D \) por \( R \):

    \[
    I = D \cdot R = \begin{pmatrix}
    200 & 100 & 150 \\
    180 & 120 & 130 \\
    220 & 140 & 170
    \end{pmatrix}
    \cdot
    \begin{pmatrix}
    12  \\
    15 \\
    10
    \end{pmatrix}
    \]

    Multiplicamos las matrices:

    \[
    I = \begin{pmatrix}5400\\5260\\6440\end{pmatrix}
    \]

    Los ingresos totales de cada planta son:
    \begin{itemize}
        \item Planta 1: \( 2400 + 1500 + 1500 = 5400 \, \euro \)
        \item Planta 2: \( 2160 + 1800 + 1300 = 5260 \, \euro \)
        \item Planta 3: \( 2640 + 2100 + 1700 = 6440 \, \euro \)
    \end{itemize}

    Por lo tanto, los ingresos totales del centro comercial son:

    \[
    5400 + 5260 + 6440 = 17100 \, \euro
    \]
    \end{solution}

    \part[0\half] Si se decide aumentar en un 10\% el área asignada a la Tienda A en cada planta y reducir en un 5\% el área asignada a la Tienda B en la Planta 3, ¿cómo quedaría la matriz del primer apartado?

    \begin{solution}
    Aumentamos en un 10\% el área asignada a la Tienda A:
    \begin{itemize}
        \item Planta 1: \( 200 \, m^2 \times 1.10 = 220 \, m^2 \)
        \item Planta 2: \( 180 \, m^2 \times 1.10 = 198 \, m^2 \)
        \item Planta 3: \( 220 \, m^2 \times 1.10 = 242 \, m^2 \)
    \end{itemize}

    Reducimos en un 5\% el área asignada a la Tienda B en la Planta 3:
    \begin{itemize}
        \item Planta 3: \( 140 \, m^2 \times 0.95 = 133 \, m^2 \)
    \end{itemize}

    La nueva matriz de distribución es:

    \[
    D_{\text{nuevo}} = \begin{pmatrix}
    220 & 100 & 150 \\
    198 & 120 & 130 \\
    242 & 133 & 170
    \end{pmatrix}
    \]
    \end{solution}

\end{parts}



\addpoints

\end{questions}

\end{document}
\grid
