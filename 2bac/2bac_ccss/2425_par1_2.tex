\documentclass[addpoints,spanish, 12pt,a4paper]{exam}
%\documentclass[answers, spanish, 12pt,a4paper]{exam}
% \printanswers
\renewcommand*\half{.5}
\pointpoints{punto}{puntos}
\hpword{Puntos:}
\vpword{Puntos:}
\htword{Total}
\vtword{Total}
\hsword{Resultado:}
\hqword{Ejercicio:}
\vqword{Ejercicio:}

\usepackage[utf8]{inputenc}
\usepackage[spanish]{babel}
\usepackage{eurosym}
%\usepackage[spanish,es-lcroman, es-tabla, es-noshorthands]{babel}


\usepackage[margin=1in]{geometry}
\usepackage{amsmath,amssymb}
\usepackage{multicol}
\usepackage{yhmath}

\pointsinrightmargin % Para poner las puntuaciones a la derecha. Se puede cambiar. Si se comenta, sale a la izquierda.
\extrawidth{-2.4cm} %Un poquito más de margen por si ponemos textos largos.
\marginpointname{ \emph{\points}}

\usepackage{graphicx}

\graphicspath{{../../img/}} 

\newcommand{\class}{2º Bachillerato CCSS}
\newcommand{\examdate}{\today}
\newcommand{\examnum}{Parcial 1ªEv.}
\newcommand{\tipo}{A}


\newcommand{\timelimit}{45 minutos}

\renewcommand{\solutiontitle}{\noindent\textbf{Solución:}\enspace}


\pagestyle{head}
\firstpageheader{\includegraphics[width=0.2\columnwidth]{header_left}}{\textbf{Departamento de Matemáticas\linebreak \class}\linebreak \examnum}{\includegraphics[width=0.1\columnwidth]{header_right}}
\runningheader{\class}{\examnum}{Página \thepage\ de \numpages}
\runningheadrule


\usepackage{pgf,tikz,pgfplots}
\pgfplotsset{compat=1.15}
\usepackage{mathrsfs}
\usetikzlibrary{arrows}


\begin{document}

\noindent
\begin{tabular*}{\textwidth}{l @{\extracolsep{\fill}} r @{\extracolsep{6pt}} }
\textbf{Nombre:} \makebox[3.5in]{\hrulefill} & \textbf{Fecha:}\makebox[1in]{\hrulefill} \\
 & \\
\textbf{Tiempo: \timelimit} & Tipo: \tipo 
\end{tabular*}
\rule[2ex]{\textwidth}{2pt}
Esta prueba tiene \numquestions\ ejercicios. La puntuación máxima es de \numpoints. 
La nota final de la prueba será la parte proporcional de la puntuación obtenida sobre la puntuación máxima. 

\begin{center}


\addpoints
 %\gradetable[h][questions]
	\pointtable[h][questions]
\end{center}

\noindent
\rule[2ex]{\textwidth}{2pt}

\begin{questions}

%\question 
%
%\begin{parts}
%\part[2] 
%\begin{solution}
%\end{solution}
%
%
%\end{parts}
%\addpoints


\question Sea $A=\left(\begin{matrix}
    3 & m \\
    6& 3
\end{matrix}\right)$.
\begin{parts}
    \part[1] Determina para qué valores de $m$ la matriz $A$ es inversible (tiene inversa)
    \begin{solution}
        $|A|=9-6m \neq 0 \to m\neq \frac{3}{2} $
    \end{solution}
    \part[1] Calcula $A^{-1}$ para $m=2$
    \begin{solution}
        $A^{-1}=\dfrac{\left(\begin{matrix}
    3 & -6 \\
    -2& 3
\end{matrix}\right)^t}{-3}=\left(\begin{matrix}
    -1 & \frac{2}{3} \\
    2& -1
\end{matrix}\right)$
    \end{solution}
\end{parts}

\question Sea el sistema de ecuaciones lineales dependientes del parámetro \( a \):
\begin{align*}
    \begin{cases}
        x + y + (a + 1)z = 9 \\
        3x - 2y + z = 20a \\
        x + y + 2az = 9
    \end{cases}
\end{align*}

\begin{parts}
    \part[2] Discutir el sistema para los diferentes valores del parámetro \( a \)
    \begin{solution}
    
$\left\{ \begin{matrix}x + y + z \left(a + 1\right) = 9 \\ 3 x - 2 y + z = 20 a \\ 2 a z + x + y = 9 \\ \end{matrix}\right.$\\$|A|=\left|\begin{matrix}1 & 1 & a + 1\\3 & -2 & 1\\1 & 1 & 2 a\end{matrix}\right|=- 5 \left(a - 1\right)$\\* Si $a \neq 1\to |A| \neq 0 \to \exists A^{-1}$.\\Como $rg(A)=rg(A^*)=3$ --> S.C. (El sistema tiene solución) \\ Como además $rg(A)=3$ coincide con el número de incógnitas --> S.C.D -- > Se puede resolver por Gauss, Matriz inversa o Cramer \\ \textbf{Resolución por Gauss} \\ $A^*=\left(\begin{matrix}1 & 1 & a + 1 & 9\\3 & -2 & 1 & 20 a\\1 & 1 & 2 a & 9\end{matrix}\right)\sim\left(\begin{matrix}1 & 1 & a + 1 & 9\\0 & -5 & - 3 a - 2 & 20 a - 27\\0 & 0 & a - 1 & 0\end{matrix}\right)\to x=4 a + \frac{18}{5}, y=\frac{27}{5} - 4 a, z=0$ \\ \textbf{Método de la matríz inversa} \\ $A^{-1}=\left(\begin{matrix}\frac{4 a + 1}{5 a - 5} & \frac{1}{5} & \frac{- 2 a - 3}{5 a - 5}\\\frac{6 a - 1}{5 a - 5} & - \frac{1}{5} & \frac{- 3 a - 2}{5 a - 5}\\- \frac{1}{a - 1} & 0 & \frac{1}{a - 1}\end{matrix}\right) \to  X=A^{-1}\cdot b =\left(\begin{matrix}\frac{4 a + 1}{5 a - 5} & \frac{1}{5} & \frac{- 2 a - 3}{5 a - 5}\\\frac{6 a - 1}{5 a - 5} & - \frac{1}{5} & \frac{- 3 a - 2}{5 a - 5}\\- \frac{1}{a - 1} & 0 & \frac{1}{a - 1}\end{matrix}\right)\cdot \left(\begin{matrix}9\\20 a\\9\end{matrix}\right) =\left(\begin{matrix}4 a + \frac{18}{5}\\\frac{27}{5} - 4 a\\0\end{matrix}\right)$ \\ \textbf{Método de Cramer} \\ $x=\frac{\left|\begin{matrix}9 & 1 & a + 1\\20 a & -2 & 1\\9 & 1 & 2 a\end{matrix}\right|}{5 - 5 a}=\frac{- 20 a^{2} + 2 a + 18}{5 - 5 a}=\frac{- 20 a^{2} + 2 a + 18}{5 - 5 a}$ \\ $y=\frac{\left|\begin{matrix}1 & 9 & a + 1\\3 & 20 a & 1\\1 & 9 & 2 a\end{matrix}\right|}{5 - 5 a}=\frac{20 a^{2} - 47 a + 27}{5 - 5 a}=\frac{20 a^{2} - 47 a + 27}{5 - 5 a}$ \\ $z=\frac{\left|\begin{matrix}1 & 1 & 9\\3 & -2 & 20 a\\1 & 1 & 9\end{matrix}\right|}{5 - 5 a}=\frac{0}{5 - 5 a}=0$\\* Si $a=1\to |A| = 0 \to \nexists A^{-1}$\\Como $rg(A)=rg(A^*)=2$ --> S.C. (El sistema tiene solución) \\ Como además $rg(A)=2$ no coincide con el número de incógnitas--> S.C.I --> Solo se puede resolver por Gauss \\ \textbf{Resolución por Gauss} \\ $A^*=\left(\begin{matrix}1 & 1 & 2 & 9\\3 & -2 & 1 & 20\\1 & 1 & 2 & 9\end{matrix}\right)\sim\left(\begin{matrix}1 & 1 & 2 & 9\\0 & -5 & -5 & -7\\0 & 0 & 0 & 0\end{matrix}\right)\to x=\frac{38}{5} - \lambda, y=\frac{7}{5} - \lambda, z=\lambda$

    \end{solution}
    \part[1\half] Resolver el sistema en el caso de que tenga infinitas soluciones
    \part[2] Resolver el sisteman para $a =0$ usando técnicas matriciales
\end{parts}

\question Dadas las matrices
\[
    A = \begin{pmatrix} 2 & 1 & -1 \end{pmatrix}, \quad
    B = \begin{pmatrix} 3 \\ -2 \\ 1 \end{pmatrix}, \quad
    X = \begin{pmatrix} x \\ y \\ z \end{pmatrix}, \quad
    C = \begin{pmatrix} 4 \\ -2 \\ 0 \end{pmatrix}
\]

\begin{parts}
    \part[0\half] Calcular las matrices \( M = A \cdot B \) y \( N = B \cdot A \).
    \begin{solution}
        $A\cdot B = \left(3\right)$ \\
        $N=B\cdot A= \left(\begin{matrix}6 & 3 & -3\\-4 & -2 & 2\\2 & 1 & -1\end{matrix}\right)
$
    \end{solution}
    % \part[1\half] Calcular \( P^{-1} \), siendo \( P = (N - I) \), donde \( I \) representa la matriz identidad.
        \part[0\half] Calcular  $P$, siendo \( P = (N - I) \), donde \( I \) representa la matriz identidad.
    \begin{solution}
        $P=\left(\begin{matrix}6 & 3 & -3\\-4 & -2 & 2\\2 & 1 & -1\end{matrix}\right)
$
\\
$P^{-1}=\left(\begin{matrix}2 & \frac{3}{2} & - \frac{3}{2}\\-2 & -2 & 1\\1 & \frac{1}{2} & - \frac{3}{2}\end{matrix}\right)$

    \end{solution}
    % \part[1] Resuelve la ecuación \( P \cdot X - B = C \) despejando $X$ previamente
    \part[0\half] Despeja $X$ en la ecuación \( P \cdot X - B = C \). \textbf{Nota:} No hay que calcular $X$
    \begin{solution}
        $X=P^{-1} \cdot (B+C)=\left(\begin{matrix}6 & 3 & -3\\-4 & -2 & 2\\2 & 1 & -1\end{matrix}\right) \cdot \left(\begin{matrix}7\\-4\\1\end{matrix}\right)= \left(\begin{matrix}\frac{13}{2}\\-5\\\frac{7}{2}\end{matrix}\right)
 $
    \end{solution}
\end{parts}

% \question[3] En la bodega de Antonio hay botellas de vino blanco, de vino tinto y de vino rosado. Si sumamos las botellas de vino blanco con las de tinto obtenemos el triple de las botellas de rosado. La suma de las botellas de tinto con las de rosado supera en 40 unidades a las botellas de blanco. Además. sabemos que Antonio tiene en su bodega 280 botellas. Plantea y resuelve un sistema de ecuaciones que nos permita averiguar cuántas botellas hay de cada tipo de vino.





\addpoints

\end{questions}

\end{document}
\grid
