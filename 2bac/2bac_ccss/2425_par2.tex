\documentclass[addpoints,spanish, 12pt,a4paper]{exam}
%\documentclass[answers, spanish, 12pt,a4paper]{exam}
\printanswers
\renewcommand*\half{.5}
\pointpoints{punto}{puntos}
\hpword{Puntos:}
\vpword{Puntos:}
\htword{Total}
\vtword{Total}
\hsword{Resultado:}
\hqword{Ejercicio:}
\vqword{Ejercicio:}

\usepackage[utf8]{inputenc}
\usepackage[spanish]{babel}
\usepackage{eurosym}
%\usepackage[spanish,es-lcroman, es-tabla, es-noshorthands]{babel}


\usepackage[margin=1in]{geometry}
\usepackage{amsmath,amssymb}
\usepackage{multicol}
\usepackage{yhmath}

\pointsinrightmargin % Para poner las puntuaciones a la derecha. Se puede cambiar. Si se comenta, sale a la izquierda.
\extrawidth{-2.4cm} %Un poquito más de margen por si ponemos textos largos.
\marginpointname{ \emph{\points}}

\usepackage{graphicx}

\graphicspath{{../../img/}} 

\newcommand{\class}{2º Bachillerato CCSS}
\newcommand{\examdate}{\today}
\newcommand{\examnum}{Parcial 2ªEv.}
\newcommand{\tipo}{A}


\newcommand{\timelimit}{45 minutos}

\renewcommand{\solutiontitle}{\noindent\textbf{Solución:}\enspace}


\pagestyle{head}
\firstpageheader{\includegraphics[width=0.2\columnwidth]{header_left}}{\textbf{Departamento de Matemáticas\linebreak \class}\linebreak \examnum}{\includegraphics[width=0.1\columnwidth]{header_right}}
\runningheader{\class}{\examnum}{Página \thepage\ de \numpages}
\runningheadrule


\usepackage{pgf,tikz,pgfplots}
\pgfplotsset{compat=1.15}
\usepackage{mathrsfs}
\usetikzlibrary{arrows}


\begin{document}

\noindent
\begin{tabular*}{\textwidth}{l @{\extracolsep{\fill}} r @{\extracolsep{6pt}} }
\textbf{Nombre:} \makebox[3.5in]{\hrulefill} & \textbf{Fecha:}\makebox[1in]{\hrulefill} \\
 & \\
\textbf{Tiempo: \timelimit} & Tipo: \tipo 
\end{tabular*}
\rule[2ex]{\textwidth}{2pt}
Esta prueba tiene \numquestions\ ejercicios. La puntuación máxima es de \numpoints. 
La nota final de la prueba será la parte proporcional de la puntuación obtenida sobre la puntuación máxima. 

\begin{center}


\addpoints
 %\gradetable[h][questions]
	\pointtable[h][questions]
\end{center}

\noindent
\rule[2ex]{\textwidth}{2pt}

\begin{questions}

% Pregunta 1
\question Sea la función \( f(x) = \dfrac{-x^2+1}{x^2+5x-6} \). Se pide:

\begin{parts}
    \part[1] Especificar su dominio de definición.
    \part[2] Estudiar su continuidad (dónde es continua y dónde no indicando los tipos de discontinuidad).
\end{parts}

% \vspace{0.5cm}

% Pregunta 2
% \question[2] Se considera la función real de variable real:
% \[
% f(x) =
% \begin{cases}
%     \dfrac{x + 7}{x^2 - 9} & \text{si } x < 2 \\
%     \dfrac{x - 11}{7} & \text{si } 2 < x \leq 5 \\
%     2x+3 & \text{si } x > 5.
% \end{cases}
% \]
% Estudiar la continuidad de la función.

% \vspace{0.5cm}

% Pregunta 3
\question[2] Determina \( a \) y \( b \) para que \( f(x) \) sea continua en todo \( \mathbb{R} \):
\[
f(x) =
\begin{cases}
    x + a & \text{si } x \leq 2 \\
    3x - 2 & \text{si } 2 < x < 5 \\
    ax + b & \text{si } x \geq 5
\end{cases}
\]

% \vspace{0.5cm}

% Pregunta 4
\question Calcula los siguientes límites:

\begin{parts}
    \part[0\half] \[ \lim_{x \to \infty} \frac{6x^2 - 3x + 5}{2x^2 + 4} \]
    \part[1] \[ \lim_{x \to 3} \frac{x^2 - 2x - 3}{x^2 - 9} \]
    \part[0\half] \[ \lim_{x \to 1} \frac{x + 5}{x - 1} \]
    \part[1\half] \[ \lim_{x \to \infty} \left( \sqrt{x^2 - 4x} - \sqrt{x^2 + 1} \right) \]
\end{parts}

% \vspace{0.5cm}

\question[1\half] Calcula los límites de la siguiente función:

\begin{tikzpicture}[scale=0.7]
    % Configuración de ejes
    \draw[->] (-8, 0) -- (8, 0) node[right] {$X$}; % Eje X
    \draw[->] (0, -2) -- (0, 8) node[above] {$Y$}; % Eje Y

    % Cuadrícula
    \draw[help lines, color=gray!20] (-8, -2) grid (8, 8);

    % Etiquetas de cuadrícula
    \foreach \x in {-8, -6, -4, -2, 2, 4, 6, 8}
        \draw (\x, 0.1) -- (\x, -0.1) node[below] {\x};
    \foreach \y in {-2, 2, 4, 6, 8}
        \draw (0.1, \y) -- (-0.1, \y) node[left] {\y};

    % Primer trozo: Parábola que termina en (-4, 4)
    \draw[thick, red, domain=-8:-4, samples=50] plot (\x, {4+(0.25*(\x+4)*(\x+4))}) node[above] {};
    \filldraw[red] (-4, 4) circle (2pt); % Punto final en (-4, 4)

    % Segundo trozo: Recta de (-4, 1) a (-2, 3)
    \draw[thick, red] (-4, 1) -- (-2, 3);
    \filldraw[fill=white, draw=red] (-4, 1) circle (2pt); % Punto inicial
    % \filldraw[fill=white, draw=red] (-2, 3) circle (2pt); % Punto final

    % Tercer trozo: Recta de (-2, 3) a (1, 0)
    \draw[thick, red] (-2, 3) -- (1, 0);
    % \draw[red] (1, 0) circle (2pt); % Punto final en (1, 0)

    \filldraw[fill=white, draw=red] (-2, 3) circle (2pt); % Punto final
    \filldraw[red] (-2, 1) circle (2pt); % Punto final

    % Cuarto trozo: Proporcionalidad inversa creciente con mayor excentricidad
    \draw[thick, red, domain=1:2.883, samples=50] plot (\x, {-1*(\x-1)/(2*(\x-3))}) node[above] {};
    \draw[dashed] (3, 0) -- (3, 8); % Asíntota vertical en x = 3
    % \filldraw[red] (1, 0) circle (2pt); % Punto inicial en (1, 0)

    % Quinto trozo: Exponencial que pasa por (3, 5) y tiene asíntota horizontal y = 3
    \draw[thick, red, domain=3:8, samples=50] plot (\x, {3 + 2*exp(-0.5*(\x-3))});
    \draw[dashed] (0, 3) -- (8, 3); % Asíntota horizontal en y = 3
    \filldraw[fill=white, draw=red] (3, 5) circle (2pt); % Punto en (3, 5)

\end{tikzpicture}
\begin{multicols}{3}
\begin{parts}
    \part $\lim_{x \to -\infty} f(x)$
    \part $\lim_{x \to -4^-} f(x)$
    \part $\lim_{x \to -4^+} f(x)$
    \part $\lim_{x \to -4} f(x)$
    \part $\lim_{x \to -2} f(x)$
    \part $\lim_{x \to 1} f(x)$
    \part $\lim_{x \to -3^-} f(x)$
    \part $\lim_{x \to -3^+} f(x)$
    \part $\lim_{x \to +\infty} f(x)$
\end{parts}
    
\end{multicols}

% Pregunta 5 (Extra)
\question Pregunta extra \textbf{(hasta un 10\% de la nota adicional)}. Calcula el siguiente límite:
\[
\lim_{x \to \infty} \left(\dfrac{ 2x - 3 }{2x + 5}\right)^{6x-3}
\]

\end{questions}

\end{document}

\addpoints

\end{questions}

\end{document}
\grid
