\documentclass[addpoints,spanish, 12pt,a4paper]{exam}
%\documentclass[answers, spanish, 12pt,a4paper]{exam}
%\printanswers
\renewcommand*\half{.5}
\pointpoints{punto}{puntos}
\hpword{Puntos:}
\vpword{Puntos:}
\htword{Total}
\vtword{Total}
\hsword{Resultado:}
\hqword{Ejercicio:}
\vqword{Ejercicio:}

\usepackage[utf8]{inputenc}
\usepackage[spanish]{babel}
\usepackage{eurosym}
%\usepackage[spanish,es-lcroman, es-tabla, es-noshorthands]{babel}

\usepackage[margin=1in]{geometry}
\usepackage{amsmath,amssymb}
\usepackage{multicol}
\usepackage{yhmath}

\pointsinrightmargin % Para poner las puntuaciones a la derecha. Se puede cambiar. Si se comenta, sale a la izquierda.
\extrawidth{-2.4cm} %Un poquito más de margen por si ponemos textos largos.
\marginpointname{ \emph{\points}}

\usepackage{graphicx}

\graphicspath{{../../img/}}

\newcommand{\class}{2ª Bachillerato Sociales}
\newcommand{\examdate}{\today}
\newcommand{\examnum}{Probabilidad}
\newcommand{\tipo}{A}

\newcommand{\timelimit}{45 minutos}

\renewcommand{\solutiontitle}{\noindent\textbf{Solución:}\enspace}

\pagestyle{head}
\firstpageheader{\includegraphics[width=0.2\columnwidth]{header_left}}{\textbf{Departamento de Matemáticas\linebreak \class}\linebreak \examnum}{\includegraphics[width=0.1\columnwidth]{header_right}}
\runningheader{\class}{\examnum}{Página \thepage\ de \numpages}
\runningheadrule

\usepackage{pgf,tikz,pgfplots}
\pgfplotsset{compat=1.15}
\usepackage{mathrsfs}
\usetikzlibrary{arrows}

\begin{document}

\noindent
\begin{tabular*}{\textwidth}{l @{\extracolsep{\fill}} r @{\extracolsep{6pt}} }
\textbf{Nombre:} \makebox[3.5in]{\hrulefill} & \textbf{Fecha:}\makebox[1in]{\hrulefill} \\
 & \\
\textbf{Tiempo: \timelimit} & Tipo: \tipo 
\end{tabular*}
\rule[2ex]{\textwidth}{2pt}
Esta prueba tiene \numquestions\ ejercicios. La puntuación máxima es de \numpoints.
La nota final de la prueba será la parte proporcional de la puntuación obtenida sobre la puntuación máxima.

\begin{center}

\addpoints
 %\gradetable[h][questions]
\pointtable[h][questions]
\end{center}

\noindent
\rule[2ex]{\textwidth}{2pt}

\begin{questions}
\question En un grupo de estudiantes, un 10\% sabe inglés y alemán, un 50\% sabe inglés pero no alemán y, entre los que saben alemán, un 40\% sabe inglés.
\begin{parts}
  \part[1] ¿Qué porcentaje de estudiantes sabe inglés?
    \begin{solution}
     $ \begin{matrix} 
           & I & \overline{I} &  \\
          A &10 & & \\
          \overline{A} &50 & & \\
                       &60&40&100
      \end{matrix} $
      \\
      $50+10=60$\% de los estudiantes sabe inglés.

    \end{solution}
  \part[1] ¿Qué porcentaje sabe alemán?
  \begin{solution}$P(I|A)=0.4 \to \dfrac{P(I\cap A)}{P(A)}=0.4\to P(A)=\dfrac{0.1}{0.4}=\dfrac{1}{4}=0.25=25$\% sabe alemán \\
     $ \begin{matrix} 
           & I & \overline{I} &  \\
          A &10 & 15& 25\\
          \overline{A} &50 &25 &75 \\
                       &60&40&100
      \end{matrix} $
  \end{solution}
  

  \part[1] ¿Qué porcentaje sabe alguno de los dos idiomas?
  \begin{solution}
    $1-P(Ninguno)=1-0.25=0.75=75$\% sabe alguno
  \end{solution}
\end{parts}

\question[2] Dados dos sucesos A y B de un mismo experimento aleatorio, se sabe que 
  $P(A)=0.5$ , $P(B)=0.3$ y $P(A\cup B)=0.6$. 
  ¿Son A y B independientes?
  \begin{solution}
  $P(A\cup B)=P(A)+P(B)-P(A\cap B) 
  \to P(A\cap B)=P(A)+P(B)-P(A\cup B)=0.5+0.3-0.6=0.2\neq 0.15=0.5 \cdot 0.3=P(A) \cdot P(B)$ \\
  Por tanto no son independientes  
  \end{solution}

%\question La probabilidad de que un autobús llegue con retraso a una parada es $0.2$.  Si pasa cuatro veces a lo largo del día por la parada, calcular la probabilidad de que:  
% \begin{parts}
%   \part[0\half]  No llegue con retraso ninguna de las veces.
%     \begin{solution}
%         \[
%         P(X = 0) = (0.8)^4 = \mathbf{0.4096}
%         \]
%     \end{solution}

%     \part[0\half] Llegue con retraso al menos una vez.
%     \begin{solution}
%         \[
%         P(X \geq 1) = 1 - P(X = 0) = \mathbf{0.5904}
%         \]
%     \end{solution}

%     \part[0\half] Al menos tres veces llegue con retraso.
%     \begin{solution}
%         \[
%         P(X \geq 3) = P(X = 3) + P(X = 4) = \mathbf{0.0272}
%         \]
%     \end{solution}

%     \part[0\half] Llegue con retraso exactamente dos veces consecutivas.
%     \begin{solution}
%         \[
%         P(\text{exactamente 2 retrasos consecutivos}) = \mathbf{0.0768}
%         \]
%     \end{solution}
% \end{parts}

\question Para efectuar cierto diagnóstico, un hospital dispone de dos escáneres, A y B.
el 65\% de las pruebas se realizan con el escáner A y el resto con B.
Se sabe que el escáner A tiene un 5\% de diagnósticos erróneos, mientras que el escáner B tiene un 8\% de diagnósticos erróneos.

Calcular la probabilidad de que:

\begin{parts}
  \part[1] El diagnóstico efectuado a un paciente en ese hospital sea erróneo.
    \begin{solution}
        \[
        P(E) = (0.05 \times 0.65) + (0.08 \times 0.35) = 0.0325 + 0.028 = \mathbf{0.0605}
        \]
    \end{solution}

    \part[1] El diagnóstico se haya efectuado usando el escáner A, sabiendo que ha resultado erróneo.
    \begin{solution}
        \[
        P(A | E) = \frac{0.05 \times 0.65}{0.0605} = \frac{0.0325}{0.0605} \approx \mathbf{0.5376}
        \]
    \end{solution}
\end{parts}


\end{questions}
\end{document}

