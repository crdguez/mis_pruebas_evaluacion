\documentclass[addpoints,spanish, 12pt,a4paper]{exam}
%\documentclass[answers, spanish, 12pt,a4paper]{exam}
\printanswers
\renewcommand*\half{.5}
\pointpoints{punto}{puntos}
\hpword{Puntos:}
\vpword{Puntos:}
\htword{Total}
\vtword{Total}
\hsword{Resultado:}
\hqword{Ejercicio:}
\vqword{Ejercicio:}

\usepackage[utf8]{inputenc}
\usepackage[spanish]{babel}
\usepackage{eurosym}
%\usepackage[spanish,es-lcroman, es-tabla, es-noshorthands]{babel}


\usepackage[margin=1in]{geometry}
\usepackage{amsmath,amssymb}
\usepackage{multicol}
\usepackage{yhmath}

\pointsinrightmargin % Para poner las puntuaciones a la derecha. Se puede cambiar. Si se comenta, sale a la izquierda.
\extrawidth{-2.4cm} %Un poquito más de margen por si ponemos textos largos.
\marginpointname{ \emph{\points}}

\usepackage{graphicx}

\graphicspath{{../../img/}} 

\newcommand{\class}{2º Bachillerato CCSS}
\newcommand{\examdate}{\today}
\newcommand{\examnum}{Recuperación de Álgebra}
\newcommand{\tipo}{A}


\newcommand{\timelimit}{45 minutos}

\renewcommand{\solutiontitle}{\noindent\textbf{Solución:}\enspace}


\pagestyle{head}
\firstpageheader{\includegraphics[width=0.2\columnwidth]{header_left}}{\textbf{Departamento de Matemáticas\linebreak \class}\linebreak \examnum}{\includegraphics[width=0.1\columnwidth]{header_right}}
\runningheader{\class}{\examnum}{Página \thepage\ de \numpages}
\runningheadrule


\usepackage{pgf,tikz,pgfplots}
\pgfplotsset{compat=1.15}
\usepackage{mathrsfs}
\usetikzlibrary{arrows}


\begin{document}

\noindent
\begin{tabular*}{\textwidth}{l @{\extracolsep{\fill}} r @{\extracolsep{6pt}} }
\textbf{Nombre:} \makebox[3.5in]{\hrulefill} & \textbf{Fecha:}\makebox[1in]{\hrulefill} \\
 & \\
\textbf{Tiempo: \timelimit} & Tipo: \tipo 
\end{tabular*}
\rule[2ex]{\textwidth}{2pt}
Esta prueba tiene \numquestions\ ejercicios. La puntuación máxima es de \numpoints. 
La nota final de la prueba será la parte proporcional de la puntuación obtenida sobre la puntuación máxima. 

\begin{center}


\addpoints
 %\gradetable[h][questions]
	\pointtable[h][questions]
\end{center}

\noindent
\rule[2ex]{\textwidth}{2pt}

\begin{questions}

\question Dadas las matrices:
\[
A = \begin{pmatrix}
2 & -1  \\
0 & 0  \\
-1 & 0 
\end{pmatrix}, \quad
B = \begin{pmatrix}
1 & 2 & 0 \\
-2 & -4 & 1
\end{pmatrix}, \quad
C = \begin{pmatrix}
1 & 2 \\
-1 & 2 \\
1 & -3
\end{pmatrix}.
\]

\begin{parts}
    \part[1] Calcular \(B' + 2C\).
    \part[2] Hallar la matriz \(X = \begin{pmatrix}
    a & b \\
    c & d
    \end{pmatrix}\) que cumple \(AX = B' + 2C\).
\end{parts}

\question Dado el sistema: 
\[
\begin{cases}
x + ay + z = 1 \\
2y + az = 2 \\
x + y + z = 1
\end{cases}
\]
\begin{parts}
    \part[2] Discutir el sistema lineal de ecuaciones en función de los valores del parámetro \(a\)
    \part[1] Resolverlo para \(a = 3\).
\end{parts} 

\question[2] La repoblación forestal de un bosque quemado en un gran incendio se va a llevar a cabo por dos empresas diferentes de jardinería. Hay que repoblar con pinos, eucaliptos y chopos. La primera empresa es capaz de plantar, en una semana, 30 pinos, 20 eucaliptos y 20 chopos. La segunda empresa planta 20 pinos, 30 eucaliptos y 20 chopos. El coste semanal se estima en 33.000\euro \  para la primera empresa de jardinería y de 35.000\euro \  para la segunda. Se necesita plantar un mínimo de 60 pinos, 120 eucaliptos y 100 chopos. ¿Cuántas semanas deberá trabajar cada grupo para finalizar el proyecto con el mínimo coste?


\end{questions}

\end{document}

\addpoints

\end{questions}

\end{document}
\grid
