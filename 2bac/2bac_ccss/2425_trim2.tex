\documentclass[addpoints,spanish, 12pt,a4paper]{exam}
%\documentclass[answers, spanish, 12pt,a4paper]{exam}
\printanswers
\renewcommand*\half{.5}
\pointpoints{punto}{puntos}
\hpword{Puntos:}
\vpword{Puntos:}
\htword{Total}
\vtword{Total}
\hsword{Resultado:}
\hqword{Ejercicio:}
\vqword{Ejercicio:}

\usepackage[utf8]{inputenc}
\usepackage[spanish]{babel}
\usepackage{eurosym}
%\usepackage[spanish,es-lcroman, es-tabla, es-noshorthands]{babel}


\usepackage[margin=1in]{geometry}
\usepackage{amsmath,amssymb}
\usepackage{multicol}
\usepackage{yhmath}

\pointsinrightmargin % Para poner las puntuaciones a la derecha. Se puede cambiar. Si se comenta, sale a la izquierda.
\extrawidth{-2.4cm} %Un poquito más de margen por si ponemos textos largos.
\marginpointname{ \emph{\points}}

\usepackage{graphicx}

\graphicspath{{../../img/}} 

\newcommand{\class}{2º Bachillerato CCSS}
\newcommand{\examdate}{\today}
\newcommand{\examnum}{Examen Final 2ªEv.}
\newcommand{\tipo}{A}


\newcommand{\timelimit}{90 minutos}

\renewcommand{\solutiontitle}{\noindent\textbf{Solución:}\enspace}


\pagestyle{head}
\firstpageheader{\includegraphics[width=0.2\columnwidth]{header_left}}{\textbf{Departamento de Matemáticas\linebreak \class}\linebreak \examnum}{\includegraphics[width=0.1\columnwidth]{header_right}}
\runningheader{\class}{\examnum}{Página \thepage\ de \numpages}
\runningheadrule


\usepackage{pgf,tikz,pgfplots}
\pgfplotsset{compat=1.15}
\usepackage{mathrsfs}
\usetikzlibrary{arrows}

% % ##################### si queremos decimales distintos de 0.5 pero ojo, nos cargamos la tabla. Solo funciona  \gradetable 
% \usepackage{fp}
% \usepackage{numprint}
% \npdecimalsign{.}
% \nprounddigits{2}
% \usepackage{etoolbox}
% \makeatletter
% % points printed at each question
% \patchcmd{\point@block}{\@points}{\FPdiv\pointdiv{\@points}{100}\numprint{\pointdiv}}{}{}
% % points printed for each question in grade table
% \patchcmd{\do@oneline@v}{\pointsof@index{pq@index}}{\FPdiv\pointsdiv{\pointsof@index{pq@index}}{100}\numprint{\pointsdiv}}{}{}
% % total number of points in grade table
% \patchcmd{\prt@tablepoints}{\prt@hlfcntr{tbl@points}}{\FPdiv\pointsdiv{\prt@hlfcntr{tbl@points}}{100}\numprint{\pointsdiv}}{}{}
% % patching needed in many other places
% \makeatother
% % ##################### fin de si queremos decimales

\begin{document}

\noindent
\begin{tabular*}{\textwidth}{l @{\extracolsep{\fill}} r @{\extracolsep{6pt}} }
\textbf{Nombre:} \makebox[3.5in]{\hrulefill} & \textbf{Fecha:}\makebox[1in]{\hrulefill} \\
 & \\
\textbf{Tiempo: \timelimit} & Tipo: \tipo 
\end{tabular*}
% \rule[2ex]{\textwidth}{2pt}
% Esta prueba tiene \numquestions\ ejercicios. La puntuación máxima es de \numpoints. 
% La nota final de la prueba será la parte proporcional de la puntuación obtenida sobre la puntuación máxima. 

% \begin{center}


% \addpoints
%  % \gradetable[h][questions]
%  % \gradetable[h]
% 	% \pointtable[h][questions]
%     \pointtable[h]
    
% \end{center}

\noindent
\rule[2ex]{\textwidth}{2pt}

\begin{questions}

%\question 
%
%\begin{parts}
%\part[2] 
%\begin{solution}
%\end{solution}
%
%
%\end{parts}
%\addpoints

\question[1] Calcule los valores de \( a \) y \( b \) para que la siguiente función  sea continua en $\mathbb{R}$


\[
f(x) = \left\{
\begin{matrix}
2x^2 - a & \text{si} & x \leq -1 \\ 
-3x^2 + b & \text{si} & -1 < x < 1 \\ 
\ln x + a  & \text{si} & x \geq 1 
\end{matrix} \right.
\]
\begin{solution}
    $\left\{\begin{matrix}
        2-a=-3b\\
        -3+b=a
    \end{matrix}\right. \to a=1 \land b=4$
\end{solution}


\question[2] La función $B(x) = \dfrac{-x^2 + 9x - 16}{x}$ representa, en miles de euros, el beneficio neto de un proceso de venta, siendo \( x \) el número de artículos vendidos. Calcule el número de artículos que deben venderse para obtener el beneficio máximo y determine dicho beneficio máximo.
\begin{solution}
    $B'(x)=-1 + \frac{16}{x^{2}} \to x=4 \to B(4)=1 \to 1000 \euro$
\end{solution}

\question Se considera la función real:
\[
f(x) = x^3 + ax^2 + bx \quad a, b \in \mathbb{R}
\]

\begin{parts}
\part[1] ¿Qué valores deben tomar \( a \) y \( b \) para que \( f(x) \) tenga un máximo relativo en el punto \( P(1,4) \)?
\begin{solution}
    $f'(x)=2 a x + b + 3 x^{2} \to \left\{\begin{matrix}
        f'(1)=0\\
        f(1)=4
    \end{matrix}\right. \to \left\{\begin{matrix}
        2 a + b + 3\\
        a + b - 3 = 0
    \end{matrix}\right. \to \left\{ a : -6, \  b : 9\right\}
$
\end{solution}

\part[1] Para \( a = -2, b = -8 \), determine los puntos de corte de la gráfica de \( f(x) \) con los ejes de coordenadas y determine los puntos de inflexión de dicha gráfica.
\end{parts}
\begin{solution}
    $f(x)=x^{3} - 2 x^{2} - 8 \to \left(x - 4\right) \left(x + 2\right)
 \to (-2,0),(0,0),(-4,0)$ \\
 $f''(x)=0 \to 6x-4=0 \to x=\frac{2}{3} \land f'''(x)=6\neq 0 \land f(\frac{2}{3})=- \frac{160}{27} \to (\frac{2}{3},- \frac{160}{27})$
    
    
\end{solution}

\question Se considera la función real de variable real definida por  
\[
f(x) = \frac{x^2}{x - 1}
\]

\begin{parts}
\part[1\half] Determine sus asíntotas y sitúe la función respecto a ellas.
\begin{solution}
    A.H: $\lim_{x \to \pm\infty}\left(\frac{x^{2}}{x - 1}\right)=\pm\infty \to \nexists$
    \\
    A.V:
    $ \lim_{x \to 1^+}\left(\frac{x^{2}}{x - 1}\right)
 \infty  \land \lim_{x \to 1^-}\left(\frac{x^{2}}{x - 1}\right)
=-\infty \to x=1$
    \\
    A.O: $\lim_{x \to \infty}\left(\frac{x}{x - 1}\right)=1 \to m=1$ \\
    $\lim_{x \to \infty}\left(\frac{x^{2}}{x - 1} - x\right)=1 \to n=1 \to y=x+1$\\
\end{solution}

\part[1\half] Calcule sus máximos y mínimos locales y estudie el crecimiento y el decrecimiento de la función.
\begin{solution}\\
    $x=1 \notin Dom(f)$ 

    
    $f'(x)=\frac{x \left(x - 2\right)}{x^{2} - 2 x + 1} \to f'(x)=0 \iff x=0 \land x=2$

        \begin{tabular}{c|c|c|c|c}
            & $(-\infty,0)$ & $(0,1)$ & $(1,2)$ & $(2,\infty)$ \\
            \hline
           $x$ & - & + & + & + \\
           $x-2$ & - & - & - & + \\
           \hline
           $\frac{x \left(x - 2\right)}{x^{2} - 2 x + 1}$ & + & - & - & + \\
        \end{tabular}\\ \\
    $x=0 \to$ Máximo $(0,0)$ \\
    $x=2 \to$ Mínimo $(2,4)$\\
\end{solution}
\end{parts}

\question[2] Derive y simplifique al máximo la siguiente función  
\[
f(x) = \ln \left( \frac{x^2 + 1}{x^2 - 1} \right)
\]
\begin{solution}
    $f(x)=\ln(x^2+1) - \ln(x^2-1)$\\
    $f'(x)=\dfrac{2x}{x^2+1}-\dfrac{2x}{x^2-1}=- \dfrac{4 x}{x^{4} - 1}$
\end{solution}

\addpoints

\end{questions}
% \gradetable \\
% \gradetable[h] \\
% \pointtable
\end{document}
\grid
