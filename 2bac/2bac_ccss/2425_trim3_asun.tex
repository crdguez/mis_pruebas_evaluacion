\documentclass[addpoints,spanish, 12pt,a4paper]{exam}
%\documentclass[answers, spanish, 12pt,a4paper]{exam}
% \printanswers
\renewcommand*\half{.5}
\pointpoints{punto}{puntos}
\hpword{Puntos:}
\vpword{Puntos:}
\htword{Total}
\vtword{Total}
\hsword{Resultado:}
\hqword{Ejercicio:}
\vqword{Ejercicio:}

\usepackage[utf8]{inputenc}
\usepackage[spanish]{babel}
\usepackage{eurosym}
%\usepackage[spanish,es-lcroman, es-tabla, es-noshorthands]{babel}

\usepackage[margin=1in]{geometry}
\usepackage{amsmath,amssymb}
\usepackage{multicol}
\usepackage{yhmath}

\pointsinrightmargin % Para poner las puntuaciones a la derecha. Se puede cambiar. Si se comenta, sale a la izquierda.
\extrawidth{-2.4cm} %Un poquito más de margen por si ponemos textos largos.
\marginpointname{ \emph{\points}}

\usepackage{graphicx}

\graphicspath{{../../img/}}

\newcommand{\class}{2º Bachillerato Sociales}
\newcommand{\examdate}{\today}
\newcommand{\examnum}{Final 3ªEv.}
\newcommand{\tipo}{A}

\newcommand{\timelimit}{90 minutos}

\renewcommand{\solutiontitle}{\noindent\textbf{Solución:}\enspace}

\pagestyle{head}
\firstpageheader{\includegraphics[width=0.2\columnwidth]{header_left}}{\textbf{Departamento de Matemáticas\linebreak \class}\linebreak \examnum}{\includegraphics[width=0.1\columnwidth]{header_right}}
\runningheader{\class}{\examnum}{Página \thepage\ de \numpages}
\runningheadrule

\usepackage{pgf,tikz,pgfplots}
\pgfplotsset{compat=1.15}
\usepackage{mathrsfs}
\usetikzlibrary{arrows}

\begin{document}

\noindent
\begin{tabular*}{\textwidth}{l @{\extracolsep{\fill}} r @{\extracolsep{6pt}} }
\textbf{Nombre:} \makebox[3.5in]{\hrulefill} & \textbf{Fecha:}\makebox[1in]{\hrulefill} \\
 & \\
\textbf{Tiempo: \timelimit} & Tipo: \tipo 
\end{tabular*}
% \rule[2ex]{\textwidth}{2pt}
% Esta prueba tiene \numquestions\ ejercicios. La puntuación máxima es de \numpoints.
% La nota final de la prueba será la parte proporcional de la puntuación obtenida sobre la puntuación máxima.

% \begin{center}

% \addpoints
%  %\gradetable[h][questions]
% \pointtable[h][questions]
% \end{center}

% \noindent
\rule[2ex]{\textwidth}{2pt}

\begin{questions}

\question[2\half] Se ha estimado que el consumo medio de gasolina de los automóviles de un concesionario se distribuye según una distribución normal con una desviación típica de 0,5 litros. Se han probado 10 automóviles, elegidos aleatoriamente, de este concesionario por conductores con la misma forma de conducir y en carreteras similares, obteniendo un consumo medio de 6,5 litros por cada 100km.
\begin{parts}
    \part Determine un intervalo de confianza, al 95\% de confianza, para la media del gasto de gasolina de estos vehículos.
    \begin{solution}
        $X \sim N(\mu,0.5) \to  \overline{X}\sim N(\mu, \dfrac{0.5}{\sqrt{n}})$ \\
        $\overline{x}=6.5$ \\
        $P(Z<z_{\alpha/2}) = 0.975  \to z_{\alpha/2}\approx 1.96$ \\
        $E = z_{\alpha / 2}\cdot \frac{\sigma}{\sqrt{n}} \approx 0.31$
        El intervalo de confianza será  $$ \left( \overline{x} - z_{\alpha / 2}\cdot \frac{\sigma}{\sqrt{n}} ,  \overline{x} + z_{\alpha / 2}\cdot \frac{\sigma}{\sqrt{n}}\right)\approx(6.18994140625; 6.81005859375)$$
    \end{solution}
    \part Hallar el tamaño mínimo que debe tener la muestra para que, con un nivel de confianza del 95\%, el error cometido del consumo de gasolina sea inferior a 0,2.
    \begin{solution}
         $E = z_{\alpha / 2}\cdot \frac{\sigma}{\sqrt{n}} \to 0.2= \dfrac{1.96 \cdot 0.5}{\sqrt{n}} \to n = \left(\dfrac{1.96 \cdot 0.5}{0.2}\right)^2\approx 24.009999999999994 \to $\ al menos hay que tomar una muestra de 25 automóviles 
    \end{solution}
\end{parts}

% \question[2\half] En una empresa multinacional el 60\% de las reuniones se realizan a través de videoconferencia. El 40\% de los empleados que asisten a estas videoconferencias son de países de la Unión Europea, mientras que en las reuniones presenciales solo el 20\% son trabajadores que no pertenecen a la Unión Europea. Si elegimos un trabajador al azar:
% \begin{parts}
%     \part Calcule la probabilidad de que pertenezca a la Unión Europea.
%     \begin{solution}

%     % $0.4=P(E|V)=\dfrac{P(E\cap V)}{P(V)} \to $ \\
%     %     $ \begin{matrix} 
%     %        & V & \overline{V} &  \\
%     %       E & & & \\
%     %       \overline{E} & & & \\
%     %                    &60& &100
%     %   \end{matrix} $

%     \tikzstyle{bag} = [text width=4em, text centered]
% \tikzstyle{end} = [circle, minimum width=3pt,fill, inner sep=0pt]
% \tikzstyle{level 1} = [level distance=3.5cm, sibling distance=3.5cm]
% \tikzstyle{level 2} = [level distance=3.5cm, sibling distance=2cm]
% \begin{tikzpicture}[grow=right, sloped, scale=0.7]
% \node[bag] {}
%     child {
%         node[bag] {$\overline{V}$}        
%             child {
%                 node[end, label=right:
%                     {$\overline{E} \to P(\overline{V}\cap \overline{E})=0.4 \cdot 0.2 = 0.08$}] {}
%                 edge from parent
%                 node[above] {$ 0.2$}
%                 node[below]  {$ $}
%             }
%             child {
%                 node[end, label=right:
%                     {$E \to P(\overline{V}\cap E)=0.4 \cdot 0.8 = 0.32$}] {}
%                 edge from parent
%                 node[above] {$ 0.8 $}
%                 node[below] {$ $}
%             }
%             edge from parent 
%             node[above] {$ $}
%             node[below]  {$0.4$}
%     }
%     child {
%         node[bag] {$V$}        
%         child {
%                 node[end, label=right:
%                     {$\overline{E} \to P(V\cap \overline{E})=0.6 \cdot 0.6 = 0.36$}] {}
%                 edge from parent
%                 node[above] {$0.6$}
%                 node[below]  {$ $}
%             }
%             child {
%                 node[end, label=right:
%                     {$E \to P(V\cap E)=0.6\cdot 0.4 = 0.24$}] {}
%                 edge from parent
%                 node[above] {$0.4$}
%                 node[below]  {$ $}
%             }
%         edge from parent         
%             node[above] {$ $}
%             node[below]  {$0.6$}
%     };
% \end{tikzpicture}

% Por la probabilidad total:
% $$P(E)=P(V\cap E)+P(\overline{V}\cap E)=0.24+0.32=0.56$$
%     \end{solution}
     
%     \part Sabiendo que el trabajador es de la Unión Europea, ¿Cuál es la probabilidad de que haya asistido a la reunión por videoconferencia?
%     \begin{solution}
%         $$P(V|E)=\dfrac{P(V \cap E)}{P(E)}=\dfrac{0.24}{0.56}=0.4285714285714285$$
%     \end{solution}
% \end{parts}

\question[2\half] Un jugador de baloncesto tiene una probabilidad de 0.8 de encestar un tiro libre. Si en un partido lanza 6 tiros libres, halle la probabilidad de que enceste: 
\begin{parts}
    \part Exactamente cuatro tiros libres. 
    \begin{solution}
        $$X\sim B(6, 0.8)$$
        $$0: 6.39999999999999e-5, 1: 0.001536, 2: 0.01536, $$
        $$3: 0.08192, 4: 0.24576, 5: 0
.393216, 6: 0.262144
$$
        $P(X=4)=0.24576$
    \end{solution}
    \part Al menos cuatro tiros. 
    \begin{solution}
        $P(X\geq4)=P(X=4)+P(X=5)+P(X=6)=0.90112$
    \end{solution}
    \part Ninguno de ellos.
    \begin{solution}
    $P(x=0)=6.39999999999999e-5$    
    \end{solution}
    
    \part Alguno de ellos.
    \begin{solution}
    $P(X>=1)=1-P(X=0)=0.999936$    
    \end{solution}
    
\end{parts}

\question[2\half] La duración de las baterías de un determinado modelo de teléfono móvil tiene una distribución normal de media 34.5 horas y una desviación típica de 6.9 horas. Se toma una muestra aleatoria simple de 36 teléfonos móviles.
\begin{parts}
    \part ¿Cuál es la probabilidad de que la duración media de las baterías de la muestra este comprendida entre 32 y 33.5 horas?.
    \begin{solution}
        $$\overline{X}\sim N(\mu, \frac{\sigma}{\sqrt{n}}) \to \overline{X}\sim N(34.5; \frac{6.9}{\sqrt{36}}) \to \overline{X}\sim N(34.5; 1.15) $$

        $P(32<\overline{X}<33.5)=0.177413193354335$
    \end{solution}
    \part ¿Y de que sea mayor de 38 horas?.
    \begin{solution}
        $P(\overline{X}>38)=0.00116930168584491
$
    \end{solution}
\end{parts}

% \question[2\half] La duración de las baterías de un determinado modelo de un fabricante de teléfonos móviles tiene una distribución normal de media 34.5 horas y una desviación típica de 1.15 horas. Si hemos comprado uno de esos móviles:
% \begin{parts}
%     \part ¿Cuál es la probabilidad de que la duración de la batería del móvil este comprendida entre 32 y 33.5 horas?.
%     \begin{solution}
%         $$\overline{X}\sim N(\mu, \frac{\sigma}{\sqrt{n}}) \to \overline{X}\sim N(34.5; \frac{6.9}{\sqrt{36}}) \to \overline{X}\sim N(34.5; 1.15) $$

%         $P(32<\overline{X}<33.5)=0.177413193354335$
%     \end{solution}
%     \part ¿Y de que sea mayor de 38 horas?.
%     \begin{solution}
%         $P(\overline{X}>38)=0.00116930168584491
% $
%     \end{solution}
% \end{parts}


\question[2\half] En cierta población cercana a una estación de esquí se quiere estimar con un nivel de confianza del 95\% la proporción de habitantes que practican el esquí. Se toma una muestra de 400 habitantes de la población de la que 240 afirman que practican este deporte. Determinar el correspondiente intervalo de confianza. Explicar los pasos seguidos para obtener la respuesta.
\begin{solution}
 El intervalo de confinaza es:
    $$ \left( \widehat{p} - z_{\alpha / 2}\cdot \sqrt{\frac{\widehat{p}\cdot\left(1-\overline{p} \right)}{n}} ,  \widehat{p} + z_{\alpha / 2}\cdot \sqrt{\frac{\widehat{p}\cdot\left(1-\widehat{p} \right)}{n}}\right)$$ \\
    $\widehat{p}=\frac{240}{400}=\frac{3}{5}=0.6$ \\
    $P(Z<z_{\alpha/2}) = 0.975  \to z_{\alpha/2}\approx 1.96$ \\
    $$\left(0.6 - 1.96\cdot\sqrt{\dfrac{0.6 \cdot 0.4}{400}} ; 0.6 + 1.96\cdot\sqrt{\dfrac{0.6 \cdot 0.4}{400}} \right)\approx$$$$\approx(0.551990883236447; 0.648009116763553)$$
    
\end{solution}



% CUESTIÓN 9. Se ha estimado que el consumo medio de gasolina de los automóviles de un concesionario se distribuye según una distribución normal con una desviación típica de 0,5 litros. Se han probado 10 automóviles, elegidos aleatoriamente, de este concesionario por conductores con la misma forma de conducir y en carreteras similares, obteniendo un consumo medio de 6,5 litros por cada 100km. 
% a) Determine un intervalo de confianza, al 95\% de confianza, para la media del gasto de gasolina de estos vehículos. (1,25 puntos)
% b) Hallar el tamaño mínimo que debe tener la muestra para que, con un nivel de confianza del 95\%, el error cometido del consumo de gasolina sea inferior a 0,2. (0,75
% puntos)

% (Murcia 2020)

% CUESTIÓN 8. En una empresa multinacional el 60\% de las reuniones se realizan a través de videoconferencia. El 40\% de los empleados que asisten a estas videoconferencias son de países de la Unión Europea, mientras que en las reuniones presenciales solo el 20\% son trabajadores que no pertenecen a la Unión Europea. Si elegimos un trabajador al azar:
% a) Calcule la probabilidad de que pertenezca a la Unión Europea. (1 punto). 
% b) Sabiendo que el trabajador es de la Unión Europea, ¿Cuál es la probabilidad de que haya asistido a la reunión por videoconferencia? (1 punto).

% Un jugador de baloncesto tiene una probabilidad de 0.8 de encestar un iro libre. Si en un partido lanza 6 tiros libres, halle la probabilidad e que enceste: 
% i) Exactamente cuatro tiros libres. 
% ii) Al menos cuatro iros. 
% iii) Ninguno de ellos. 
% iv) Alguno de ellos.

% SOCIALES II. 2024 MODELO. EJERCICIO 3B Andalucía

% Problema 4 (2 puntos) La duración de las baterías de un determinado modelo de teléfono móvil tiene una distribución normal de media 34.5 horas y una desviación típica de 6.9 horas. Se toma una muestra aleatoria simple de 36 teléfonos móviles.

% a) ¿Cuál es la probabilidad de que la duración media de las baterías de la muestra este comprendida entre 32 y 33.5 horas?.
% b) ¿Y de que sea mayor de 38 horas?.

% Junio 2001. En cierta población cercana a una estación de esquí se quiere estimar con un nivel de confianza del 95\% la proporción de habitantes que practican el esquí. Se toma una muestra de 400 habitantes de la población de la que 240 afirman que practican este deporte. Determinar el correspondiente intervalo de confianza. Explicar los pasos seguidos para obtener la respuesta. (10 puntos)

\end{questions}

    \newgeometry{left=1 cm,bottom=2cm}
% \begin{landscape}
\begin{table}
% \Large
\centering

% \caption{Extracto de tabla de probabilidades de la \textbf{normal estándar $Z(0,1)$}}
\caption{Tabla de probabilidades de la \textbf{normal estándar $Z(0,1)$}}
\label{my-label}

\begin{tabular}{l|llllllllll}
z   & 0       & 0,01    & 0,02    & 0,03    & 0,04    & 0,05    & 0,06    & 0,07    & 0,08    & 0,09    \\
\hline
0   & 0,5     & 0,50399 & 0,50798 & 0,51197 & 0,51595 & 0,51994 & 0,52392 & 0,5279  & 0,53188 & 0,53586 \\
0,1 & 0,53983 & 0,5438  & 0,54776 & 0,55172 & 0,55567 & 0,55962 & 0,56356 & 0,56749 & 0,57142 & 0,57535 \\
0,2 & 0,57926 & 0,58317 & 0,58706 & 0,59095 & 0,59483 & 0,59871 & 0,60257 & 0,60642 & 0,61026 & 0,61409 \\
0,3 & 0,61791 & 0,62172 & 0,62552 & 0,6293  & 0,63307 & 0,63683 & 0,64058 & 0,64431 & 0,64803 & 0,65173 \\
0,4 & 0,65542 & 0,6591  & 0,66276 & 0,6664  & 0,67003 & 0,67364 & 0,67724 & 0,68082 & 0,68439 & 0,68793 \\
0,5 & 0,69146 & 0,69497 & 0,69847 & 0,70194 & 0,7054  & 0,70884 & 0,71226 & 0,71566 & 0,71904 & 0,7224  \\
0,6 & 0,72575 & 0,72907 & 0,73237 & 0,73565 & 0,73891 & 0,74215 & 0,74537 & 0,74857 & 0,75175 & 0,7549  \\
0,7 & 0,75804 & 0,76115 & 0,76424 & 0,7673  & 0,77035 & 0,77337 & 0,77637 & 0,77935 & 0,7823  & 0,78524 \\
0,8 & 0,78814 & 0,79103 & 0,79389 & 0,79673 & 0,79955 & 0,80234 & 0,80511 & 0,80785 & 0,81057 & 0,81327 \\
0,9 & 0,81594 & 0,81859 & 0,82121 & 0,82381 & 0,82639 & 0,82894 & 0,83147 & 0,83398 & 0,83646 & 0,83891 \\
1   & 0,84134 & 0,84375 & 0,84614 & 0,84849 & 0,85083 & 0,85314 & 0,85543 & 0,85769 & 0,85993 & 0,86214 \\
1,1 & 0,86433 & 0,8665  & 0,86864 & 0,87076 & 0,87286 & 0,87493 & 0,87698 & 0,879   & 0,881   & 0,88298 \\
1,2 & 0,88493 & 0,88686 & 0,88877 & 0,89065 & 0,89251 & 0,89435 & 0,89617 & 0,89796 & 0,89973 & 0,90147 \\
1,3 & 0,9032  & 0,9049  & 0,90658 & 0,90824 & 0,90988 & 0,91149 & 0,91309 & 0,91466 & 0,91621 & 0,91774 \\
1,4 & 0,91924 & 0,92073 & 0,9222  & 0,92364 & 0,92507 & 0,92647 & 0,92785 & 0,92922 & 0,93056 & 0,93189 \\
1,5 & 0,93319 & 0,93448 & 0,93574 & 0,93699 & 0,93822 & 0,93943 & 0,94062 & 0,94179 & 0,94295 & 0,94408 \\
1,6 & 0,9452  & 0,9463  & 0,94738 & 0,94845 & 0,9495  & 0,95053 & 0,95154 & 0,95254 & 0,95352 & 0,95449 \\
1,7 & 0,95543 & 0,95637 & 0,95728 & 0,95818 & 0,95907 & 0,95994 & 0,9608  & 0,96164 & 0,96246 & 0,96327 \\
1,8 & 0,96407 & 0,96485 & 0,96562 & 0,96638 & 0,96712 & 0,96784 & 0,96856 & 0,96926 & 0,96995 & 0,97062 \\
1,9 & 0,97128 & 0,97193 & 0,97257 & 0,9732  & 0,97381 & 0,97441 & 0,975   & 0,97558 & 0,97615 & 0,9767  \\
2   & 0,97725 & 0,97778 & 0,97831 & 0,97882 & 0,97932 & 0,97982 & 0,9803  & 0,98077 & 0,98124 & 0,98169 \\
2,1 & 0,98214 & 0,98257 & 0,983   & 0,98341 & 0,98382 & 0,98422 & 0,98461 & 0,985   & 0,98537 & 0,98574 \\
2,2 & 0,9861  & 0,98645 & 0,98679 & 0,98713 & 0,98745 & 0,98778 & 0,98809 & 0,9884  & 0,9887  & 0,98899 \\
2,3 & 0,98928 & 0,98956 & 0,98983 & 0,9901  & 0,99036 & 0,99061 & 0,99086 & 0,99111 & 0,99134 & 0,99158 \\
2,4 & 0,9918  & 0,99202 & 0,99224 & 0,99245 & 0,99266 & 0,99286 & 0,99305 & 0,99324 & 0,99343 & 0,99361 \\
2,5 & 0,99379 & 0,99396 & 0,99413 & 0,9943  & 0,99446 & 0,99461 & 0,99477 & 0,99492 & 0,99506 & 0,9952  \\
2,6 & 0,99534 & 0,99547 & 0,9956  & 0,99573 & 0,99585 & 0,99598 & 0,99609 & 0,99621 & 0,99632 & 0,99643 \\
2,7 & 0,99653 & 0,99664 & 0,99674 & 0,99683 & 0,99693 & 0,99702 & 0,99711 & 0,9972  & 0,99728 & 0,99736 \\
2,8 & 0,99744 & 0,99752 & 0,9976  & 0,99767 & 0,99774 & 0,99781 & 0,99788 & 0,99795 & 0,99801 & 0,99807 \\
2,9 & 0,99813 & 0,99819 & 0,99825 & 0,99831 & 0,99836 & 0,99841 & 0,99846 & 0,99851 & 0,99856 & 0,99861 \\
3   & 0,99865 & 0,99869 & 0,99874 & 0,99878 & 0,99882 & 0,99886 & 0,99889 & 0,99893 & 0,99896 & 0,999   \\
3,1 & 0,99903 & 0,99906 & 0,9991  & 0,99913 & 0,99916 & 0,99918 & 0,99921 & 0,99924 & 0,99926 & 0,99929 \\
3,2 & 0,99931 & 0,99934 & 0,99936 & 0,99938 & 0,9994  & 0,99942 & 0,99944 & 0,99946 & 0,99948 & 0,9995  \\
3,3 & 0,99952 & 0,99953 & 0,99955 & 0,99957 & 0,99958 & 0,9996  & 0,99961 & 0,99962 & 0,99964 & 0,99965 \\
3,4 & 0,99966 & 0,99968 & 0,99969 & 0,9997  & 0,99971 & 0,99972 & 0,99973 & 0,99974 & 0,99975 & 0,99976 \\
3,5 & 0,99977 & 0,99978 & 0,99978 & 0,99979 & 0,9998  & 0,99981 & 0,99981 & 0,99982 & 0,99983 & 0,99983 \\
3,6 & 0,99984 & 0,99985 & 0,99985 & 0,99986 & 0,99986 & 0,99987 & 0,99987 & 0,99988 & 0,99988 & 0,99989 \\
3,7 & 0,99989 & 0,9999  & 0,9999  & 0,9999  & 0,99991 & 0,99991 & 0,99992 & 0,99992 & 0,99992 & 0,99992 \\
3,8 & 0,99993 & 0,99993 & 0,99993 & 0,99994 & 0,99994 & 0,99994 & 0,99994 & 0,99995 & 0,99995 & 0,99995 \\
3,9 & 0,99995 & 0,99995 & 0,99996 & 0,99996 & 0,99996 & 0,99996 & 0,99996 & 0,99996 & 0,99997 & 0,99997 \\
4   & 0,99997 & 0,99997 & 0,99997 & 0,99997 & 0,99997 & 0,99997 & 0,99998 & 0,99998 & 0,99998 & 0,99998
\end{tabular}
\end{table}
% \end{landscape}
\restoregeometry

\end{document}

