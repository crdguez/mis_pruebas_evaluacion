\documentclass[addpoints,spanish, 12pt,a4paper]{exam}
%\documentclass[answers, spanish, 12pt,a4paper]{exam}
\printanswers
\renewcommand*\half{.5}
\pointpoints{punto}{puntos}
\hpword{Puntos:}
\vpword{Puntos:}
\htword{Total}
\vtword{Total}
\hsword{Resultado:}
\hqword{Ejercicio:}
\vqword{Ejercicio:}

\usepackage[utf8]{inputenc}
\usepackage[spanish]{babel}
\usepackage{eurosym}
%\usepackage[spanish,es-lcroman, es-tabla, es-noshorthands]{babel}


\usepackage[margin=1in]{geometry}
\usepackage{amsmath,amssymb}
\usepackage{multicol}
\usepackage{yhmath}

\pointsinrightmargin % Para poner las puntuaciones a la derecha. Se puede cambiar. Si se comenta, sale a la izquierda.
\extrawidth{-2.4cm} %Un poquito más de margen por si ponemos textos largos.
\marginpointname{ \emph{\points}}

\usepackage{graphicx}

\graphicspath{{../../img/}} 

\newcommand{\class}{2º Bachillerato CCSS}
\newcommand{\examdate}{\today}
\newcommand{\examnum}{Recuperación de Álgebra}
\newcommand{\tipo}{A}


\newcommand{\timelimit}{45 minutos}

\renewcommand{\solutiontitle}{\noindent\textbf{Solución:}\enspace}


\pagestyle{head}
\firstpageheader{\includegraphics[width=0.2\columnwidth]{header_left}}{\textbf{Departamento de Matemáticas\linebreak \class}\linebreak \examnum}{\includegraphics[width=0.1\columnwidth]{header_right}}
\runningheader{\class}{\examnum}{Página \thepage\ de \numpages}
\runningheadrule


\usepackage{pgf,tikz,pgfplots}
\pgfplotsset{compat=1.15}
\usepackage{mathrsfs}
\usetikzlibrary{arrows}

% % ##################### si queremos decimales distintos de 0.5 pero ojo, nos cargamos la tabla. Solo funciona  \gradetable 
% \usepackage{fp}
% \usepackage{numprint}
% \npdecimalsign{.}
% \nprounddigits{2}
% \usepackage{etoolbox}
% \makeatletter
% % points printed at each question
% \patchcmd{\point@block}{\@points}{\FPdiv\pointdiv{\@points}{100}\numprint{\pointdiv}}{}{}
% % points printed for each question in grade table
% \patchcmd{\do@oneline@v}{\pointsof@index{pq@index}}{\FPdiv\pointsdiv{\pointsof@index{pq@index}}{100}\numprint{\pointsdiv}}{}{}
% % total number of points in grade table
% \patchcmd{\prt@tablepoints}{\prt@hlfcntr{tbl@points}}{\FPdiv\pointsdiv{\prt@hlfcntr{tbl@points}}{100}\numprint{\pointsdiv}}{}{}
% % patching needed in many other places
% \makeatother
% % ##################### fin de si queremos decimales

\begin{document}

\noindent
\begin{tabular*}{\textwidth}{l @{\extracolsep{\fill}} r @{\extracolsep{6pt}} }
\textbf{Nombre:} \makebox[3.5in]{\hrulefill} & \textbf{Fecha:}\makebox[1in]{\hrulefill} \\
 & \\
\textbf{Tiempo: \timelimit} & Tipo: \tipo 
\end{tabular*}
\rule[2ex]{\textwidth}{2pt}
Esta prueba tiene \numquestions\ ejercicios. La puntuación máxima es de \numpoints. 
La nota final de la prueba será la parte proporcional de la puntuación obtenida sobre la puntuación máxima. 

\begin{center}


\addpoints
 % \gradetable[h][questions]
 % \gradetable[h]
	% \pointtable[h][questions]
    \pointtable[h]
    
\end{center}

\noindent
\rule[2ex]{\textwidth}{2pt}

\begin{questions}


\question Se considera el siguiente sistema lineal de ecuaciones, dependiente del par\'ametro real $k$:
\begin{equation*}
\begin{cases}
    x + y + kz = 4 \\
    2x - y + 2z = 5 \\
    -x + 3y - z = 0
\end{cases}
\end{equation*}

\begin{parts}
    \part[2] Disc\'utase el sistema para los distintos valores del par\'ametro $k$. 
    % \part[1] Resu\'elvase el sistema para el caso en que tenga infinitas soluciones. 
    \part[1] Resu\'elvase el sistema para $k = 0$. \hfill
\end{parts}
\begin{solution}
Considere el siguiente sistema de ecuaciones, dónde k  es un parámetro real: $$\left\{ \begin{matrix}k z + x + y = 4 \\ 2 x - y + 2 z = 5 \\ - x + 3 y - z = 0 \\ \end{matrix}\right.$$ Determine los valores del parámetro real k, para los que este sistema es compatible determinado,  compatible indeterminado o incompatible. \\ Resuelva el sistema cuando\begin{itemize}\item k=1\item k=0\end{itemize}\textbf{Discusión y resolución por Gauss:} Escalonando la matriz ampliada tenemos\\$A^*= \left(\begin{matrix}1 & 1 & k & 4\\2 & -1 & 2 & 5\\-1 & 3 & -1 & 0\end{matrix}\right) \thicksim \left(\begin{matrix}1 & 1 & k & 4\\0 & -3 & 2 - 2 k & -3\\0 & 0 & \frac{5}{3} - \frac{5 k}{3} & 0\end{matrix}\right)$. \\  De los valores de la última fila podemos concluir:\begin{itemize}\item Si $k = 1 \to$ $$\left(\begin{matrix}1 & 1 & 1 & 4\\0 & -3 & 0 & -3\\0 & 0 & 0 & 0\end{matrix}\right)$$ La última fila es $0z=0 \to $ S.C.I\item si $k\neq [1]  \to $ S.C.D.\end{itemize}  \textbf{Por rangos y determinantes:} \\$\left|A\right|=\left|\begin{matrix}1 & 1 & k\\2 & -1 & 2\\-1 & 3 & -1\end{matrix}\right|=5 k - 5 \to \left|A\right|=0 \quad si \quad k = \left[ 1\right]$\begin{itemize}\item Si $k=1 \to rg(A)=2 \land rg(A^*)=2 \to $ S.C.I. $\to$ solo se puede resolver por Gauss, (ver más arriba)\item Si $k \neq[1] \to rg(A)=3 \land rg(A^*)=3 \to $ S.C.D.  \\ \end{itemize}\textbf{Para k = 1 :}  \\ \textbf{Discusión y resolución por Gauss:} Escalonando la matriz ampliada tenemos\\$A^*= \left(\begin{matrix}1 & 1 & 1 & 4\\2 & -1 & 2 & 5\\-1 & 3 & -1 & 0\end{matrix}\right) \thicksim \left(\begin{matrix}1 & 1 & 1 & 4\\0 & -3 & 0 & -3\\0 & 0 & 0 & 0\end{matrix}\right)$. \\  De los valores de la última fila podemos concluir:\begin{itemize}\item S.C.D.\begin{itemize}\item $\left(\begin{matrix}0 & 0 & 0 & 0\end{matrix}\right) \to z = z$\end{itemize}\begin{itemize}\item $\left(\begin{matrix}0 & -3 & 0 & -3\end{matrix}\right) \to y = 1$\end{itemize}\begin{itemize}\item $\left(\begin{matrix}1 & 1 & 1 & 4\end{matrix}\right) \to x = 3 - z$\end{itemize}\end{itemize}  \textbf{Por rangos y determinantes:} \\$\left|A\right|=\left|\begin{matrix}1 & 1 & 1\\2 & -1 & 2\\-1 & 3 & -1\end{matrix}\right|=0  \neq 0 $\begin{itemize} \item $rg(A)=2 \land rg(A^*)=2 \to $ S.C.D.   \\ \\ Por Cramer: \begin{itemize}\item $x=\frac{\left|\begin{matrix}4 & 1 & 1\\5 & -1 & 2\\0 & 3 & -1\end{matrix}\right|}{0}=\frac{0}{0}=\text{NaN}$\item $y=\frac{\left|\begin{matrix}1 & 4 & 1\\2 & 5 & 2\\-1 & 0 & -1\end{matrix}\right|}{0}=\frac{0}{0}=\text{NaN}$\item $z=\frac{\left|\begin{matrix}1 & 1 & 4\\2 & -1 & 5\\-1 & 3 & 0\end{matrix}\right|}{0}=\frac{0}{0}=\text{NaN}$\end{itemize}\end{itemize}\textbf{Para k = 0 :}  \\ \textbf{Discusión y resolución por Gauss:} Escalonando la matriz ampliada tenemos\\$A^*= \left(\begin{matrix}1 & 1 & 0 & 4\\2 & -1 & 2 & 5\\-1 & 3 & -1 & 0\end{matrix}\right) \thicksim \left(\begin{matrix}1 & 1 & 0 & 4\\0 & -3 & 2 & -3\\0 & 0 & \frac{5}{3} & 0\end{matrix}\right)$. \\  De los valores de la última fila podemos concluir:\begin{itemize}\item S.C.D.\begin{itemize}\item $\left(\begin{matrix}0 & 0 & \frac{5}{3} & 0\end{matrix}\right) \to z = 0$\end{itemize}\begin{itemize}\item $\left(\begin{matrix}0 & -3 & 2 & -3\end{matrix}\right) \to y = 1$\end{itemize}\begin{itemize}\item $\left(\begin{matrix}1 & 1 & 0 & 4\end{matrix}\right) \to x = 3$\end{itemize}\end{itemize}  \textbf{Por rangos y determinantes:} \\$\left|A\right|=\left|\begin{matrix}1 & 1 & 0\\2 & -1 & 2\\-1 & 3 & -1\end{matrix}\right|=-5  \neq 0 $\begin{itemize} \item $rg(A)=3 \land rg(A^*)=3 \to $ S.C.D.   \\ \\ Por Cramer: \begin{itemize}\item $x=\frac{\left|\begin{matrix}4 & 1 & 0\\5 & -1 & 2\\0 & 3 & -1\end{matrix}\right|}{-5}=\frac{-15}{-5}=3$\item $y=\frac{\left|\begin{matrix}1 & 4 & 0\\2 & 5 & 2\\-1 & 0 & -1\end{matrix}\right|}{-5}=\frac{-5}{-5}=1$\item $z=\frac{\left|\begin{matrix}1 & 1 & 4\\2 & -1 & 5\\-1 & 3 & 0\end{matrix}\right|}{-5}=\frac{0}{-5}=0$\end{itemize}\end{itemize}
\end{solution}

\question[3] Un centro dedicado a la ense\~nanza personalizada de idiomas tiene dos cursos, uno b\'asico y otro avanzado, para los que dedica distintos recursos. Esta planificaci\'on hace que pueda atender entre 20 y 65 estudiantes del curso b\'asico y entre 20 y 40 estudiantes del curso avanzado. El n\'umero m\'aximo de estudiantes que en total puede atender es 100. 
Los beneficios que obtiene por cada estudiante en el curso b\'asico se estiman en 145 euros y en 150 euros por cada estudiante del curso avanzado. Hallar qu\'e n\'umero de estudiantes de cada curso proporcionan el m\'aximo beneficio. \hfill 
\begin{solution} 
    max $ z=145 x + 150 y$ s.a:
    $\left\{ \begin{matrix}x+y \leq 100  \\ x \leq 65 \\ x \geq 20 \\ y \leq 40 \\ y \geq 20 \\ \end{matrix}\right.$
    \\
    Vértices y valor de la función:
$$\left( 65, \  35\right)\to150 \cdot 35 + 145 \cdot 65=14675$$
$$\left( 60, \  40\right)\to150 \cdot 40 + 145 \cdot 60=14700$$
$$\left( 65, \  20\right)\to150 \cdot 20 + 145 \cdot 65=12425$$
$$\left( 20, \  40\right)\to145 \cdot 20 + 150 \cdot 40=8900$$
$$\left( 20, \  20\right)\to145 \cdot 20 + 150 \cdot 20=5900$$
Solución: 60 del básico y 40 del avanzado
\end{solution}


\question Dadas las matrices:

\[
A = \begin{pmatrix}
1 & 2 & 0 \\
1 & 0 & 3
\end{pmatrix} 
\quad
B = \begin{pmatrix}
2 & -1 \\
1 & 0 \\
0 & -2
\end{pmatrix} 
\quad
C = \begin{pmatrix}
5 & 3 \\
3 & 2
\end{pmatrix}
\quad
D = \begin{pmatrix}
5 & 2 \\
3 & 1
\end{pmatrix}
\]

Realiza las siguientes operaciones:

\begin{parts}
    % \part El producto \( A \cdot B \)
    % \begin{solution}
    % Para calcular el producto de las matrices \( A \cdot B \), debemos multiplicar las filas de \( A \) por las columnas de \( B \):
    
    % \[
    % A \cdot B = \begin{pmatrix}
    % 1 & 2 & 0 \\
    % 1 & 0 & 3
    % \end{pmatrix}
    % \begin{pmatrix}
    % 2 & -1 \\
    % 1 & 0 \\
    % 0 & -2
    % \end{pmatrix}
    % = \begin{pmatrix}
    % (1 \cdot 2 + 2 \cdot 1 + 0 \cdot 0) & (1 \cdot -1 + 2 \cdot 0 + 0 \cdot -2) \\
    % (1 \cdot 2 + 0 \cdot 1 + 3 \cdot 0) & (1 \cdot -1 + 0 \cdot 0 + 3 \cdot -2)
    % \end{pmatrix}
    % \]
    
    % \[
    % = \begin{pmatrix}
    % 4 & -1 \\
    % 2 & -7
    % \end{pmatrix}
    % \]
    % \end{solution}

    % \part[1] La inversa \( C^{-1} \)
    % \begin{solution}
    % La matriz inversa de una matriz \( 2 \times 2 \) se calcula como:
    % \[
    % C^{-1} = \frac{1}{\det(C)} \begin{pmatrix} d & -b \\ -c & a \end{pmatrix}
    % \]
    % donde \( C = \begin{pmatrix} a & b \\ c & d \end{pmatrix} = \begin{pmatrix} 5 & 3 \\ 3 & 2 \end{pmatrix} \).

    % El determinante de \( C \) es:
    % \[
    % \det(C) = (5 \cdot 2) - (3 \cdot 3) = 10 - 9 = 1
    % \]
    
    % Entonces, la matriz inversa es:
    % \[
    % C^{-1} = \frac{1}{1} \begin{pmatrix} 2 & -3 \\ -3 & 5 \end{pmatrix} = \begin{pmatrix} 2 & -3 \\ -3 & 5 \end{pmatrix}
    % \]
    % \end{solution}

    % \part La diferencia \( D - A \cdot B \)
    % \begin{solution}
    % Primero, calculamos el producto \( A \cdot B \) que ya hemos encontrado:

    % \[
    % A \cdot B = \begin{pmatrix} 4 & -1 \\ 2 & -7 \end{pmatrix}
    % \]

    % Ahora, restamos \( A \cdot B \) de \( D \):

    % \[
    % D - A \cdot B = \begin{pmatrix} 5 & 2 \\ 3 & 1 \end{pmatrix} - \begin{pmatrix} 4 & -1 \\ 2 & -7 \end{pmatrix}
    % = \begin{pmatrix} 5 - 4 & 2 - (-1) \\ 3 - 2 & 1 - (-7) \end{pmatrix}
    % \]

    % \[
    % = \begin{pmatrix} 1 & 3 \\ 1 & 8 \end{pmatrix}
    % \]
    % \end{solution}

    \part[2] Resuelve la ecuación matricial: \( A \cdot B + C \cdot X = D \); es decir, calcula la matriz \( X \)
    \begin{solution}
    Primero, despejamos \( C \cdot X \) de la ecuación:

    \[
    C \cdot X = D - A \cdot B
    \]

    Ya sabemos que \( D - A \cdot B = \begin{pmatrix} 1 & 3 \\ 1 & 8 \end{pmatrix} \), por lo que:

    \[
    C \cdot X = \begin{pmatrix} 1 & 3 \\ 1 & 8 \end{pmatrix}
    \]

    Ahora, multiplicamos ambos lados por \( C^{-1} \) para despejar \( X \):

    \[
    X = C^{-1} \cdot \left( D - A \cdot B \right)
    \]

    Sustituimos \( C^{-1} = \begin{pmatrix} 2 & -3 \\ -3 & 5 \end{pmatrix} \) y \( D - A \cdot B = \begin{pmatrix} 1 & 3 \\ 1 & 8 \end{pmatrix} \):

    \[
    X = \begin{pmatrix} 2 & -3 \\ -3 & 5 \end{pmatrix} \cdot \begin{pmatrix} 1 & 3 \\ 1 & 8 \end{pmatrix}
    \]

    Realizamos la multiplicación de matrices:

    \[
    X = \begin{pmatrix}
    (2 \cdot 1 + -3 \cdot 1) & (2 \cdot 3 + -3 \cdot 8) \\
    (-3 \cdot 1 + 5 \cdot 1) & (-3 \cdot 3 + 5 \cdot 8)
    \end{pmatrix}
    \]

    \[
    X = \begin{pmatrix}
    -1 & -18 \\
    2 & 29
    \end{pmatrix}
    \]

    Por lo tanto, la matriz \( X \) es:

    \[
    X = \begin{pmatrix} -1 & -18 \\ 2 & 29 \end{pmatrix}
    \]
    \end{solution}

\end{parts}


\question[2] En un grupo hay 288 personas de entre 18 y 25 años clasificadas como estudiantes, empleados y sin ocupación. Por cada cinco estudiantes hay tres empleados y los que se encuentran sin ocupación representan el 80\% del resto.

Plantea el correspondiente sistema de ecuaciones.

\begin{solution}

solve([x+y+z-288,x/5-y/3,z-0.8*(x+y)])
{x: 100.0, y: 60.0, z: 128.0}

Definimos las siguientes variables:
\[
x = \text{número de estudiantes}, \quad y = \text{número de empleados}, \quad z = \text{número de personas sin ocupación}.
\]

A partir del enunciado, tenemos las siguientes ecuaciones:

1. La suma de las personas debe ser igual a 288:
\[
x + y + z = 288
\]

2. La relación entre estudiantes y empleados es de 5 estudiantes por cada 3 empleados:
\[
y = \frac{3}{5} x
\]

3. Las personas sin ocupación representan el 80\% del resto, es decir, el 80\% de \( 288 - x - y \):
\[
z = 0.80 \times ( x +y)
\]

El sistema de ecuaciones es:

\[
\begin{aligned}
x + y + z &= 288 \\
y &= \frac{3}{5} x \\
z &= 0.80 \times ( x + y)
\end{aligned}
\]
\end{solution}

% \question Se consideran las siguientes matrices:
% \begin{equation*}
% A = \begin{pmatrix} 2 & 2 & 0 \\ 0 & 2 & 0 \\ 2 & 0 & 4 \end{pmatrix}, \quad
% B = \begin{pmatrix} -3 & 4 & -6 \\ -2 & 1 & -2 \\ -11 & 3 & -8 \end{pmatrix}
% \end{equation*}

% \begin{parts}
%     \part[1\half] Calc\'ulese $A^{-1} A^T$.
%     \begin{solution}$A^{-1}=\left(\begin{matrix}\frac{1}{2} & - \frac{1}{2} & 0\\0 & \frac{1}{2} & 0\\- \frac{1}{4} & \frac{1}{4} & \frac{1}{4}\end{matrix}\right) \to A^{-1}\cdot A^T=\left(\begin{matrix}0 & -1 & 1\\1 & 1 & 0\\0 & \frac{1}{2} & \frac{1}{2}\end{matrix}\right)$
%     \end{solution}
    
%     \textit{Nota: La notaci\'on $A^T$ representa a la matriz transpuesta de $A$.}
    
%     % \part[1\half] Resu\'elvase la ecuaci\'on matricial: \begin{equation*}
% %     \frac{1}{4} A^2 - AX = B
% %     \end{equation*}
% %     \begin{solution}
% %         $A^2=\left(\begin{matrix}4 & 8 & 0\\0 & 4 & 0\\12 & 4 & 16\end{matrix}\right)$
% %         $\frac{1}{4} A^2-B=\left(\begin{matrix}4 & -2 & 6\\2 & 0 & 2\\14 & -2 & 12\end{matrix}\right)$
% %         $X=\left(\begin{matrix}1 & -1 & 2\\1 & 0 & 1\\3 & 0 & 2\end{matrix}\right)
% % $
% %     \end{solution}
% \end{parts}


\end{questions}
% \gradetable \\
% \gradetable[h] \\
% \pointtable
\end{document}
\grid
