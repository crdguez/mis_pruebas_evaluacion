\documentclass[addpoints,spanish, 12pt,a4paper]{exam}
%\documentclass[answers, spanish, 12pt,a4paper]{exam}
\printanswers
\renewcommand*\half{.5}
\pointpoints{punto}{puntos}
\hpword{Puntos:}
\vpword{Puntos:}
\htword{Total}
\vtword{Total}
\hsword{Resultado:}
\hqword{Ejercicio:}
\vqword{Ejercicio:}


\usepackage[utf8]{inputenc}
\usepackage[spanish]{babel}
\usepackage{eurosym}
%\usepackage[spanish,es-lcroman, es-tabla, es-noshorthands]{babel}


\usepackage[margin=1in]{geometry}
\usepackage{amsmath,amssymb}
\usepackage{multicol}
\usepackage{yhmath}

\pointsinrightmargin % Para poner las puntuaciones a la derecha. Se puede cambiar. Si se comenta, sale a la izquierda.
\extrawidth{-2.4cm} %Un poquito más de margen por si ponemos textos largos.
\marginpointname{ \emph{\points}}

\usepackage{graphicx}

\graphicspath{{../../img/}} 

\newcommand{\class}{2º Bachillerato CCSS}
\newcommand{\examdate}{\today}
\newcommand{\examnum}{Examen Final 1ªEv.}
\newcommand{\tipo}{A}


\newcommand{\timelimit}{90 minutos}

\renewcommand{\solutiontitle}{\noindent\textbf{Solución:}\enspace}


\pagestyle{head}
\firstpageheader{\includegraphics[width=0.2\columnwidth]{header_left}}{\textbf{Departamento de Matemáticas\linebreak \class}\linebreak \examnum}{\includegraphics[width=0.1\columnwidth]{header_right}}
\runningheader{\class}{\examnum}{Página \thepage\ de \numpages}
\runningheadrule


\usepackage{pgf,tikz,pgfplots}
\pgfplotsset{compat=1.15}
\usepackage{mathrsfs}
\usetikzlibrary{arrows}

% ##################### si queremos decimales distintos de 0.5 pero ojo, nos cargamos la tabla. Solo funciona  \gradetable 
\usepackage{fp}
\usepackage{numprint}
\npdecimalsign{.}
\nprounddigits{2}
\usepackage{etoolbox}
\makeatletter
% points printed at each question
\patchcmd{\point@block}{\@points}{\FPdiv\pointdiv{\@points}{100}\numprint{\pointdiv}}{}{}
% points printed for each question in grade table
\patchcmd{\do@oneline@v}{\pointsof@index{pq@index}}{\FPdiv\pointsdiv{\pointsof@index{pq@index}}{100}\numprint{\pointsdiv}}{}{}
% total number of points in grade table
\patchcmd{\prt@tablepoints}{\prt@hlfcntr{tbl@points}}{\FPdiv\pointsdiv{\prt@hlfcntr{tbl@points}}{100}\numprint{\pointsdiv}}{}{}
% patching needed in many other places
\makeatother
% ##################### fin de si queremos decimales

\begin{document}

\noindent
\begin{tabular*}{\textwidth}{l @{\extracolsep{\fill}} r @{\extracolsep{6pt}} }
\textbf{Nombre:} \makebox[3.5in]{\hrulefill} & \textbf{Fecha:}\makebox[1in]{\hrulefill} \\
 & \\
\textbf{Tiempo: \timelimit} & Tipo: \tipo 
\end{tabular*}
% \rule[2ex]{\textwidth}{2pt}
% Esta prueba tiene \numquestions\ ejercicios. La puntuación máxima es de \numpoints. 
% La nota final de la prueba será la parte proporcional de la puntuación obtenida sobre la puntuación máxima. 

% \begin{center}


% \addpoints
%  % \gradetable[h][questions]
%  % \gradetable[h]
% 	% \pointtable[h][questions]
%     \pointtable[h]
    
% \end{center}

\noindent
\rule[2ex]{\textwidth}{2pt}

\begin{questions}


\question Considera el sistema de ecuaciones lineales
    \[
    \begin{pmatrix}
    1 & 1 & 3 \\ 
    2 & 3 & 4 \\ 
    3 & 4 & m
    \end{pmatrix}
    \begin{pmatrix} x \\ y \\ z \end{pmatrix}
    =
    \begin{pmatrix} 0 \\ 0 \\ 0 \end{pmatrix}
    \]

\begin{parts}

    
    \part[100]
    Discutir si existe solución según los valores del parámetro \( m \). 
    % En caso afirmativo, resolver el sistema.
    \begin{solution}
         $|A|=m-7$. Si \( m = 7 \), el sistema es compatible e indeterminado. Soluciones: \( x = -5\lambda \), \( y = 2\lambda \), \( z = \lambda \)
 Si \( m \neq 7 \), el sistema es compatible determinado. Soluciones: \( x = y = z = 0 \)


    \end{solution}
    \part[100] Resolver el sistema anterior para \( m = 7 \)

\end{parts}


\question Dadas las matrices  A  B  y  C
\[
A = \begin{pmatrix} 2 & -1 \\ 1 & 0 \end{pmatrix}
\quad
% B = \begin{pmatrix} 1 & 0 \\ -1 & 1 \end{pmatrix}
B = \begin{pmatrix} 2 & 0 \\ 5 & 2 \end{pmatrix}
\quad
C = \begin{pmatrix} 1 & 2 \\ 0 & -1 \end{pmatrix}
\]


\begin{parts}
    % \part Calcula  $C^2$  
    \part[50] Halla  $A + B + C^2$  
    % \part Encuentra  $(A - B)^{-1}$  
    \part[200] Resuelve la ecuación matricial  $AX - BX = A + B + C^2$
    \end{parts}

\begin{solution}
    $C^2=I$ \\
    $A+B+C^2=\left(\begin{matrix}5 & -1\\6 & 3\end{matrix}\right)$ \\
    $(A-B)^{-1} =\left(\begin{matrix}\frac{1}{2} & - \frac{1}{4}\\-1 & 0\end{matrix}\right)$\\
    $X=(A-B)^1\cdot(A+B+C^2)=\left(\begin{matrix}1 & - \frac{5}{4}\\-5 & 1\end{matrix}\right)$

\end{solution}

\question[150] En un museo las entradas cuestan 1 euro para los niños, 2 euros para los jóvenes y 5
euros para los adultos. Ayer se recaudaron un total de 600 euros y se sabe que el número de adultos
que visitó el museo fue igual al doble de la suma del número de niños más el número de jóvenes;
además, si hubiesen visitado el museo 100 jóvenes más, el número de jóvenes habría sido igual a la
suma del número de niños más el número de adultos. Plantear \textbf{(sin resolver)} un sistema de ecuaciones
lineales para determinar el número de niños, jóvenes y adultos que visitaron el museo.
\begin{solution}
    \[
\begin{cases}
x + 2y + 5z = 600 \\
z = 2(x + y) \\
y + 100 = x + z
\end{cases}
\]

\end{solution}

\question Un corredor aficionado tiene dos tipos de entrenamiento, el corto y el largo. En cada
entrenamiento corto, al que dedica 1 hora, corre 15 km y consume 1200 kilocalorías. En cada
entrenamiento largo, al que dedica 3 horas, corre 30 km y consume 2500 kilocalorías. Quiere
planificar los entrenamientos del verano de forma que haga al menos 24 entrenamientos, pero no
corra más de 660 km ni dedique más de 48 horas, en total. 
\begin{parts}
    \part[300] Si su objetivo es maximizar el número
total de kilocalorías consumidas, plantear y resolver un problema de programación lineal para
determinar cuántos entrenamientos de cada tipo tiene que hacer. ¿Cuántas kilocalorías consumirá en
ese caso?

\begin{solution}
    \textbf{Planteamiento del problema:}

Sean:

- \( x \) = número de entrenamientos cortos.
- \( y \) = número de entrenamientos largos.

La función objetivo es maximizar las kilocalorías consumidas:

\[
\text{Maximizar } Z = 1200x + 2500y
\]

Sujeta a las siguientes restricciones:

\[
\begin{cases}
x + y \geq 24 \quad \text{(Número total de entrenamientos)} \\

15x+30y \leq  660 \to x+2y=44  \\
x+3y \leq 48 \\
x \geq 0, \quad y \geq 0 
\end{cases}
\]

\includegraphics[width=1\textwidth]{grafica_corredor.png}

\textbf{Vértices de la región factible:}

 \( (x, y) = (24, 0) \to f(24,0)=28800\) \\
 \( (x, y) = (12, 12) \to f(12,12)=44400 \) \\
 \( (x, y) = (36, 4) \to f(36,4)=53200\) \\
 \( (x, y) = (44, 0) \to f(44,0)=52800\)

 Solución: 36 y 4. 53200 kc
\end{solution}
    \part[100] ¿Qué pasaría si en el entrenamiento largo consumiera 3600 kilocalorías?
    \begin{solution}
        (infinitas soluciones) Segmento que une $(12,12)$ y $(36,4)$ ya que $f(z)=57600$ en ambos vértices.
    \end{solution}
\end{parts}


% \question En un almacén se guarda aceite de oliva y aceite de girasol  Para atender a los clientes se ha de tener almacenado un mínimo de 40 bidones de aceite de oliva y 20 de girasol  Ademas el numero de bidones de aceite de oliva no debe ser inferior a la mitad del numero de bidones de aceite de girasol  La capacidad total del almacén es de 150 bidones  Se sabe que el gasto de almacenaje de un bidón de aceite de oliva es de 2 euros y el de girasol es de 1 euro  

% Se desea saber cuantos bidones de aceite de oliva y de girasol se deben almacenar para que el gasto de almacenaje sea mínimo  

% \begin{parts}
% \part Plantea el problema de programación lineal que minimiza el gasto de almacenaje
% \part Representa la región factible S
% \part Calcula las coordenadas de los vértices de dicha región S
% \part Cuantos bidones de cada tipo habrá que almacenar para que el gasto sea mínimo
% \end{parts}

%\question 
%
%\begin{parts}
%\part[2] 
%\begin{solution}
%\end{solution}
%
%
%\end{parts}
%\addpoints


% \question
% Tres amigas, Sara, Cristina y Jimena tienen un total de 15000 seguidores en una red social. 
% Si Jimena perdiera el 25\% de sus seguidores, todavía tendría el triple de seguidores que Sara. 
% Además, la mitad de los seguidores de Sara más la quinta parte de los de Cristina suponen la cuarta parte de los seguidores de Jimena. 
% Calcule cuántos seguidores tiene cada una de las tres amigas.

% \question[3] Dada la matriz 
% \[
% A = \begin{pmatrix} 0 & 3 & 4 \\ 1 & -4 & -5 \\ -1 & 3 & 4 \end{pmatrix}
% \]
% se pide:

% \begin{parts}
%     \part[1] Comprobar que verifica la igualdad \( A^3 + I = O \), siendo \( I \) la matriz identidad y \( O \) la matriz nula.
%     \part[1]  Justificar que \( A \) tiene inversa y obtener \( A^{-1} \).
%     \part  Calcular \( A^{100} \).
% \end{parts}

% https://tex.stackexchange.com/questions/434157/exam-package-decimal-points



% \question Sean las matrices $A=\left(\begin{matrix}2 & 0\\0 & 1\end{matrix}\right)$, $B= \  \left(\begin{matrix}1 & 0\\1 & 2\end{matrix}\right)$, $C=\left(\begin{matrix}10 & 11\\4 & 7\end{matrix}\right)$
% \begin{parts}
%     \part[75] Determina la matriz inversa de la matriz $I + B$ , siendo $I$ la matriz identidad de orden 2.
%         \begin{solution}
%             $I+B=\left(\begin{matrix}2 & 0\\1 & 3\end{matrix}\right) \to (I+B)^{-1}=\left(\begin{matrix}\frac{1}{2} & 0\\- \frac{1}{6} & \frac{1}{3}\end{matrix}\right)$
%         \end{solution} 
%     \part[175] Calcular las matrices $X$ e $Y$ que verifican que: $\left\{\begin{matrix} 
%         AX+BY=C \\ 
%         AX=Y   
%     \end{matrix}\right.$
%         \begin{solution}
%             $BY=C-Y \to BY+Y=C \to (B+I)Y=C \to Y= (B+Y)^{-1}C$ \\
%             $Y=\left(\begin{matrix}\frac{1}{2} & 0\\- \frac{1}{6} & \frac{1}{3}\end{matrix}\right)\cdot \left(\begin{matrix}10 & 11\\4 & 7\end{matrix}\right) = \left(\begin{matrix}5 & \frac{11}{2}\\- \frac{1}{3} & \frac{1}{2}\end{matrix}\right)$ \\
%             $X=A^{-1}\cdot Y=\left(\begin{matrix}\frac{1}{2} & 0\\0 & 1\end{matrix}\right) \cdot \left(\begin{matrix}5 & \frac{11}{2}\\- \frac{1}{3} & \frac{1}{2}\end{matrix}\right) = \left(\begin{matrix}\frac{5}{2} & \frac{11}{4}\\- \frac{1}{3} & \frac{1}{2}\end{matrix}\right)$
%         \end{solution}
% \end{parts}

% \question Dadas las matrices  $A=\left(\begin{matrix}2 & 1 & 3\\1 & -1 & 0\end{matrix}\right)$,  $B=\left(\begin{matrix}1 & -1\\2 & 0\\0 & 1\end{matrix}\right)$, $C=\left(\begin{matrix}2 & 3\\1 & 4\end{matrix}\right)$ y $D=\left(\begin{matrix}1 & 8\\0 & 5\end{matrix}\right)$ 
% %\noaddpoints % to omit double points count

% \begin{parts}
% \part[1] Calcule $2C$ y $AB$ 

% \part[2] Encontrar, si existe, una matriz $X$ tal que: $AB+2CX=D$
% \begin{solution}
% $AB=\left(\begin{matrix}4 & 1\\-1 & -1\end{matrix}\right)$, $2C=\left(\begin{matrix}4 & 6\\2 & 8\end{matrix}\right)$, $D-AB = \left(\begin{matrix}-3 & 7\\1 & 6\end{matrix}\right)$ y $(2C)^{-1}:\left(\begin{matrix}4 & 6\\2 & 8\end{matrix}\right)\xrightarrow{traspuesta}\left(\begin{matrix}4 & 2\\6 & 8\end{matrix}\right)\xrightarrow{adjunta}\left(\begin{matrix}8 & -6\\-2 & 4\end{matrix}\right)\xrightarrow{inversa}\left(\begin{matrix}\frac{2}{5} & - \frac{3}{10}\\- \frac{1}{10} & \frac{1}{5}\end{matrix}\right)$. \\ Por tanto, $X=(2C)^{-1}\cdot (D-AB)= \left(\begin{matrix}\frac{2}{5} & - \frac{3}{10}\\- \frac{1}{10} & \frac{1}{5}\end{matrix}\right)\cdot \left(\begin{matrix}-3 & 7\\1 & 6\end{matrix}\right)=\left(\begin{matrix}- \frac{3}{2} & 1\\\frac{1}{2} & \frac{1}{2}\end{matrix}\right)$

% \end{solution}

% \part[1] Calcule, justificadamente, el rango de la matriz:$\left(\begin{matrix}1 & 0 & 2\\-1 & -1 & 1\\1 & -1 & 5\end{matrix}\right)$
% \begin{solution}
% $det(A)=0 \land $ Matriz de adjuntos $\rightarrow\left(\begin{matrix}-4 & 6 & 2\\-2 & 3 & 1\\2 & -3 & -1\end{matrix}\right)$ \\
% Por tanto, $ran(A)=2$
% \end{solution}

% \end{parts}
% \addpoints

% \question[1] Sean las matrices $A=\displaystyle \left(\begin{matrix}2 & 1\\3 & 2\end{matrix}\right)$ y $B=\left(\begin{matrix}2 & -1\\-3 & 2\end{matrix}\right)$
% \begin{parts}
%     \part Comprueba que $B$ es la inversa de $A$
%     \begin{solution}
%         $A\cdot B=I$
%     \end{solution}
%     \part Calcula $(A-2I)^2$
%         $A-2I=\left(\begin{matrix}0 & 1\\3 & 0\end{matrix}\right) \to  (A-2I)^2= \left(\begin{matrix}3 & 0\\0 & 3\end{matrix}\right)$
%     \part Calcula la matriz $X$ tal que $AX=B$
%         \begin{solution}
%             $X=B^2=\left(\begin{matrix}7 & -4\\-12 & 7\end{matrix}\right)$
%         \end{solution}
% \end{parts}


% \question Resuelve el sistema: $\left(\begin{matrix}1 & 3 & -1\\1 & 2 & 0\\0 & 3 & -1\end{matrix}\right)\cdot \left(\begin{matrix}x\\y\\z\end{matrix}\right)=\left(\begin{matrix}4\\5\\3\end{matrix}\right)$

% \begin{parts}
% \part[1] Por el método de Gauss
% \part[1] Por el método de la matriz inversa
% \part[1] Por la regla de Cramer
% \end{parts}
% \begin{solution}
% $\left\{\begin{matrix}x + 3 y - z = 4\\x + 2 y = 5\\3 y - z = 3\\\end{matrix}\right.$ \\  Por Gauss: \\ $\left(\begin{matrix}1 & 3 & -1 & 4\\1 & 2 & 0 & 5\\0 & 3 & -1 & 3\end{matrix}\right)\rightarrow\left(\begin{matrix}1 & 3 & -1 & 4\\0 & -1 & 1 & 1\\0 & 0 & 2 & 6\end{matrix}\right)\to$ Sol:$\left\{\left( 1, \  2, \  3\right)\right\}$ \\ Por Matriz inversa: \\ $X=A^{-1}\cdot b=\left(\begin{matrix}1 & 0 & -1\\- \frac{1}{2} & \frac{1}{2} & \frac{1}{2}\\- \frac{3}{2} & \frac{3}{2} & \frac{1}{2}\end{matrix}\right)\cdot\left(\begin{matrix}4\\5\\3\end{matrix}\right)=\left(\begin{matrix}1\\2\\3\end{matrix}\right)$. \ Ya que $\left(\begin{matrix}1 & 3 & -1\\1 & 2 & 0\\0 & 3 & -1\end{matrix}\right)\xrightarrow{traspuesta}\left(\begin{matrix}1 & 1 & 0\\3 & 2 & 3\\-1 & 0 & -1\end{matrix}\right)\xrightarrow{adjunta}\left(\begin{matrix}-2 & 0 & 2\\1 & -1 & -1\\3 & -3 & -1\end{matrix}\right)\xrightarrow{inversa}\left(\begin{matrix}1 & 0 & -1\\- \frac{1}{2} & \frac{1}{2} & \frac{1}{2}\\- \frac{3}{2} & \frac{3}{2} & \frac{1}{2}\end{matrix}\right)$. \\ Por Cramer: \\ $det(A)=-2$ \ $\Delta_0$, $s_0$: $\left( -2, \  1\right)$ \ $\Delta_1$, $s_1$: $\left( -4, \  2\right)$ \ $\Delta_2$, $s_2$: $\left( -6, \  3\right)$ \
% \end{solution}

% \question En una librería hubo la semana pasada una promoción de tres libros: una novela, un libro de
% poesía y un cuento. Se vendieron 200 ejemplares de la novela, 100 de poesía y 150 cuentos. La librería ingresó por dicha promoción 8600 euros, que el precio de un ejemplar de novela es el doble que el de un cuento y que el triple de la diferencia entre el precio del ejemplar de poesía y del cuento es igual al
% precio de una novela.
% \begin{parts}
% \part[2] Plantea un sistema de ecuaciones que refleje el enunciado

% \part[1] Resuelva el problema por cualquiera de los métodos vistos en clase
% \begin{solution} \\
% \includegraphics[scale = 0.8]{sol1}
% \end{solution}
% \end{parts}

% \question Un padre decide repartir su fortuna de 480 monedas de oro entre sus tres hijas: Ana,
% Carla y Pilar. La cantidad que recibe Ana
% es el doble de la suma de las cantidades que reciben Carla y Pilar. Además, la suma de
% las cantidades que reciben Ana y Pilar es igual al triple de la cantidad que recibe Carla.
% \begin{parts}
% \part[125] Plantea un sistema de ecuaciones que refleje el enunciado
% \part[125] Resuelva el problema por cualquiera de los métodos vistos en clase
% \end{parts}
% \begin{solution}
% $\left\{\begin{matrix}x + y + z = 480\\x = 2 y + 2 z\\x + z = 3 y\\\end{matrix}\right.$ \\  Por Gauss: \\ $\left(\begin{matrix}1 & 1 & 1 & 480\\1 & -2 & -2 & 0\\1 & -3 & 1 & 0\end{matrix}\right)\rightarrow\left(\begin{matrix}1 & 1 & 1 & 480\\0 & -3 & -3 & -480\\0 & 0 & 4 & 160\end{matrix}\right)\to$ Sol:$\left\{\left( 320, \  120, \  40\right)\right\}$ \\ Por Matriz inversa: \\ $X=A^{-1}\cdot b=\left(\begin{matrix}\frac{2}{3} & \frac{1}{3} & 0\\\frac{1}{4} & 0 & - \frac{1}{4}\\\frac{1}{12} & - \frac{1}{3} & \frac{1}{4}\end{matrix}\right)\cdot\left(\begin{matrix}480\\0\\0\end{matrix}\right)=\left(\begin{matrix}320\\120\\40\end{matrix}\right)$. \ Ya que $\left(\begin{matrix}1 & 1 & 1\\1 & -2 & -2\\1 & -3 & 1\end{matrix}\right)\xrightarrow{traspuesta}\left(\begin{matrix}1 & 1 & 1\\1 & -2 & -3\\1 & -2 & 1\end{matrix}\right)\xrightarrow{adjunta}\left(\begin{matrix}-8 & -4 & 0\\-3 & 0 & 3\\-1 & 4 & -3\end{matrix}\right)\xrightarrow{inversa}\left(\begin{matrix}\frac{2}{3} & \frac{1}{3} & 0\\\frac{1}{4} & 0 & - \frac{1}{4}\\\frac{1}{12} & - \frac{1}{3} & \frac{1}{4}\end{matrix}\right)$. \\ Por Cramer: \\ $det(A)=-12$ \ $\Delta_0$, $s_0$: $\left( -3840, \  320\right)$ \ $\Delta_1$, $s_1$: $\left( -1440, \  120\right)$ \ $\Delta_2$, $s_2$: $\left( -480, \  40\right)$ \ \end{solution}

% \question Dado el sistema: $$\left\{\begin{matrix}x + y + z = a - 1\\a z + 2 x + y = a\\a y + x + z = 1\\\end{matrix}\right.$$
% %\noaddpoints % to omit double points count
% \begin{parts}
% \part[150] Discutir la solución del mismo según el valor de $a$ 

% \part[100] Resolver el sistema para $a=2$

% % \part[3] Resolver el sistema según el valor de $a$


% \end{parts}
% \begin{solution}
% $\left(\begin{matrix}1 & 1 & 1 & a - 1\\2 & 1 & a & a\\1 & a & 1 & 1\end{matrix}\right) \to \left(\begin{matrix}1 & 1 & 1 & a - 1\\0 & -1 & a - 2 & 2 - a\\0 & 0 & \left(a - 2\right) \left(a - 1\right) & a \left(2 - a\right)\end{matrix}\right)\to det(A)=- a^{2} + 3 a - 2\to- \left(a - 2\right) \left(a - 1\right)$. \\ \\ Discusión: \\Si a $\neq\left( 1, \  2\right)\to det(A) \neq 0 \to ran(A)=ran(A^*)=3 \to $ S.C.D. $\to$ Sol:$\left(\begin{matrix}a + 1\\\frac{2 - a}{a - 1}\\- \frac{a}{a - 1}\end{matrix}\right)$ \\Si a=1: \\ $\left(\begin{matrix}1 & 1 & 1 & 0\\0 & -1 & -1 & 1\\0 & 0 & 0 & 1\end{matrix}\right) \to ran(A^*)=3 \land ran(A)=2 \to$  S.I. \\Si a=2: \\ $\left(\begin{matrix}1 & 1 & 1 & 1\\0 & -1 & 0 & 0\\0 & 0 & 0 & 0\end{matrix}\right) \to ran(A^*)=2 \land ran(A)=2 \to$  S.C.I.  $\to$ Sol:$\left\{ x : 1 - z, \  y : 0\right\}$
% \end{solution}
% \addpoints



% \question Dadas las siguientes restricciones: $$\left\{\begin{matrix}
% 2x  \leqslant  8 - y\\
% x  \leqslant  3\\
% y  \leqslant  4\\
% x  \geqslant  0 \\
% y  \geqslant  0 \\
% \end{matrix}\right.$$ 
% %\noaddpoints % to omit double points count

% \begin{parts}
% \part[1] Razonar si $z=5x+2y$ alcanza un valor máximo y uno mínimo con las restricciones anteriores. En caso afirmativo, calcular dichos valores y los puntos en los que se alcanzan.   
% \begin{solution}\\
% \includegraphics[scale=0.6]{fi1_1} 
% \\Vértices:\\
% $A(0 , 0) \to f(0,0)=0$\\
% $B(3 , 0) \to f(3,0)=15$\\
% $C(3 , 2) \to f(3,2)=19$\\
% $D(2 , 4) \to f(2,4)=18$\\
% $E(0 , 4) \to f(0,4)=8$\\
% \\
% Mínimo en $D$ y $f(D)=0$ \\
% Máximo en $C$ y $f(C)=19$
% \end{solution}
% \part[1] Igual que el apartado anterior pero para $z=6x+3y$ 
% \begin{solution}\\
% \includegraphics[scale=0.6]{fi1_2} 
% \\ Vértices:\\
% $A(0 , 0) \to f(0,0)=0$\\
% $B(3 , 0) \to f(3,0)=18$\\
% $C(3 , 2) \to f(3,2)=24$\\
% $D(2 , 4) \to f(2,4)=24$\\
% $E(0 , 4) \to f(0,4)=12$\\
% \\
% Mínimo en $D$ y $f(D)=0$\\
% Máximo en $\overline{CD}$ y $f(C)=24 \land f(D)=24$
% $$\overline{CD}\equiv \left\{ \begin{matrix}
% x = 3 + (2-3)\lambda \\
% y = 2 + (4-2)\lambda \\
% \end{matrix} , \lambda \in \left( 0, 1 \right)
% \right.
% $$
% \end{solution}


% \end{parts}

% \addpoints


% \question[3] Los 400 alumnos de un colegio van a ir de excursión. Para ello se contrata el viaje a una
% empresa que dispone de 8 autobuses de 40 plazas y 10 con 50 plazas, pero sólo de 9 conductores para ese
% día. Dada la diferente capacidad y calidad, el alquiler de cada autobús de los grandes cuesta 80 \euro. y
% el de cada uno de los pequeños 60 \euro . ¿Cuántos autobuses de cada clase se tiene
% que alquilar para que el coste del viaje sea mínimo?

% \begin{solution}
% min $z=6000 x + 8000 y$ \\
% s.a: \\
% $\left\{ \begin{matrix}x \leq 8  \\ y \leq 10 \\ x+y \leq 9 \\ 40x+50y \geq 400 \\ x \geq 0 \\ y \geq 0 \\ \end{matrix}\right.$ \\
% Región factible:\\
% \includegraphics[scale=0.30]{region1.jpg} \\

% Vértices y valor de la función: \\

% (5, 4)→6000x5+8000x4=62000\\
% (0, 9)→6000x0+8000x9=72000\\
% (0, 8)→6000x0+8000x8=64000\\

% La función se optimiza para x=5, y=4.0 y el valor óptimo es 62000
% \end{solution}



% \question Una empresa de transportes valora la apertura de sucursales rurales y/o urbanas. Las
% sucursales rurales emplean a tres personas, requieren de una inversión de 100.000 euros para su apertura y
% generan unos ingresos de 15000 euros al mes. Las sucursales urbanas emplean a 6 personas, requieren de
% 150.000 euros de inversión y generan un ingreso de 18.000 euros al mes. La empresa de transportes tiene
% hasta tres millones de euros disponibles para abrir nuevas sucursales, han decidido limitar el número de
% nuevas sucursales a 25 y se han comprometido a crear como mínimo 60 empleos.
% \begin{parts}
%     \part[200] Calcule el ingreso mensual máximo que se obtendría
%     \part[50] En la solución óptima, ¿cuántos empleos generará?, ¿se gasta todo el dinero disponible?
% \end{parts}
% \begin{solution}
%     max $z=15000 x + 18000 y$ \\
%     s.a: \\
%     $\left\{ \begin{matrix}x+y \leq 25  \\ 3x+6y \geq 60 \\ 100000x+150000y \leq 3000000 \\ x \geq 0 \\ y \geq 0 \\ \end{matrix}\right.$ \\
%     Región factible: \\
%     \includegraphics[scale=0.30]{pl2022.png} \\
%     Vértices y valor de la función: \\

%     (15,10)→18000x10+15000x15=405000 \\
%     (25,0)→18000x0+15000x25=375000 \\
%     (0,10)→15000x0+18000x10=180000 \\
%     (20,0)→18000x0+15000x20=300000 \\
%     (0,20)→15000x0+18000x20=360000 \\
    
%     Solución: \\
%     La función se optimiza para $x=15$, $y=10$ y el valor óptimo es 405000. Se gasta todo el dinero ya que $100000*15+150000*10 = 3000000$

    

% \end{solution}


% \question[3] Un camionero transporta dos tipos de mercancías, X e Y, ganando 60 y 50 euros por
% tonelada respectivamente. Al menos debe transportar 8 toneladas de X y como mucho el doble de cantidad
% que de Y. ¿A cuánto asciende su ganancia total máxima si dispone de un camión que puede transportar
% hasta 30 toneladas? 


  
% \begin{solution}
% Maximizar $f(x,y)=60x+50y$ s.a: $\left\{ \begin{matrix}
% x \geqslant 8 \\
% x \leqslant 2y \\
% x+y \leqslant 30 \\
%  \end{matrix} \right.$ 
 
% \includegraphics[scale=0.2]{sep2007}


% Vértices:\\
% $A(8 , 4) \to f(8,4)=680$\\
% $B(8 , 22) \to f(8,22)=1580$\\
% $C(20 , 10) \to f(20,10)=1700$\\

% 1700 \euro (debe transportar 20 toneladas de X y 10 toneladas de Y). \end{solution}


\addpoints

\end{questions}
% \gradetable \\
% \gradetable[h] \\
% \pointtable
\end{document}
\grid
