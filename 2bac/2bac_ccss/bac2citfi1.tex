\documentclass[addpoints,spanish, 12pt,a4paper]{exam}
%\documentclass[answers, spanish, 12pt,a4paper]{exam}
\printanswers
\pointpoints{punto}{puntos}
\hpword{Puntos:}
\vpword{Puntos:}
\htword{Total}
\vtword{Total}
\hsword{Resultado:}
\hqword{Ejercicio:}
\vqword{Ejercicio:}

\usepackage[utf8]{inputenc}
\usepackage[spanish]{babel}
\usepackage{eurosym}
%\usepackage[spanish,es-lcroman, es-tabla, es-noshorthands]{babel}


\usepackage[margin=1in]{geometry}
\usepackage{amsmath,amssymb}
\usepackage{multicol}
\usepackage{yhmath}

\pointsinrightmargin % Para poner las puntuaciones a la derecha. Se puede cambiar. Si se comenta, sale a la izquierda.
\extrawidth{-2.4cm} %Un poquito más de margen por si ponemos textos largos.
\marginpointname{ \emph{\points}}

\usepackage{graphicx}

\graphicspath{{../img/}} 

\newcommand{\class}{2º Bachillerato CIT}
\newcommand{\examdate}{\today}
\newcommand{\examnum}{Final 1ªEv.}
\newcommand{\tipo}{A}


\newcommand{\timelimit}{105 minutos}

\renewcommand{\solutiontitle}{\noindent\textbf{Solución:}\enspace}


\pagestyle{head}
\firstpageheader{\includegraphics[width=0.2\columnwidth]{header_left}}{\textbf{Departamento de Matemáticas\linebreak \class}\linebreak \examnum}{\includegraphics[width=0.1\columnwidth]{header_right}}
\runningheader{\class}{\examnum}{Página \thepage\ of \numpages}
\runningheadrule


\usepackage{pgf,tikz,pgfplots}
\pgfplotsset{compat=1.15}
\usepackage{mathrsfs}
\usetikzlibrary{arrows}


\begin{document}

\noindent
\begin{tabular*}{\textwidth}{l @{\extracolsep{\fill}} r @{\extracolsep{6pt}} }
\textbf{Nombre:} \makebox[3.5in]{\hrulefill} & \textbf{Fecha:}\makebox[1in]{\hrulefill} \\
 & \\
\textbf{Tiempo: \timelimit} & Tipo: \tipo 
\end{tabular*}
\rule[2ex]{\textwidth}{2pt}
Esta prueba tiene \numquestions\ ejercicios. La puntuación máxima es de \numpoints. 
La nota final de la prueba será la parte proporcional de la puntuación obtenida sobre la puntuación máxima. 

\begin{center}


\addpoints
 %\gradetable[h][questions]
	\pointtable[h][questions]
\end{center}

\noindent
\rule[2ex]{\textwidth}{2pt}

\begin{questions}

%\question 
%
%\begin{parts}
%\part[2] 
%\begin{solution}
%\end{solution}
%
%
%\end{parts}
%\addpoints

\question[6] Calcula el siguiente límite: $$\lim_{x \to 1} \left(\frac{x^{3} - 2 x + 2}{3 x^{2} - 2}\right)^{\frac{x^{2} - 3 x}{x^{2} + x - 2}}$$
\begin{solution}
$\lim_{x \to 1} \left(\frac{x^{3} - 2 x + 2}{3 x^{2} - 2}\right)^{\frac{x^{2} - 3 x}{x^{2} + x - 2}}=1^\infty$. Indeterminación \\ $g(x)\cdot\left[f(x)-1\right]=\frac{x \left(x - 3\right)}{\left(x - 1\right) \left(x + 2\right)}\cdot\frac{\left(x - 1\right) \left(x^{2} - 2 x - 4\right)}{3 x^{2} - 2}=\frac{x \left(x - 3\right) \left(x^{2} - 2 x - 4\right)}{\left(x + 2\right) \left(3 x^{2} - 2\right)}$. \\ $\lim_{x \to 1}\frac{x \left(x - 3\right) \left(x^{2} - 2 x - 4\right)}{\left(x + 2\right) \left(3 x^{2} - 2\right)}=\frac{10}{3}$. Por tanto: \\ $\lim_{x \to 1} \left(\frac{x^{3} - 2 x + 2}{3 x^{2} - 2}\right)^{\frac{x^{2} - 3 x}{x^{2} + x - 2}}=e^{\frac{10}{3}}$

\end{solution}
\addpoints

\question[4] Hallar los puntos de la gráfica $y=2 x^{3} + 3 x^{2} - 30 x - 6$ en el cual la recta tangente es paralela a la recta $y=6 x - 5$
\begin{solution}$f'(x)=6 x^{2} + 6 x - 30\land m=6$. Luego: \\ $6 x^{2} + 6 x - 30 = 6 \to x =\left[ -3, \  2\right] $. \ Los puntos de la gráfica son: \\   $\left(-3,57 \right)$ . \\  $\left(2,-38 \right)$ . \\ 
\end{solution}
\addpoints

\question Halla la derivadas de la siguientes funciones:

\begin{parts}
\part[7] $\log{\left(\sqrt{\frac{\cos{\left(x \right)} + 1}{1 - \cos{\left(x \right)}}} \right)}$
\begin{solution}
$f'(x)=\left(- \frac{\log{\left(1 - \cos{\left(x \right)} \right)}}{2} + \frac{\log{\left(\cos{\left(x \right)} + 1 \right)}}{2}\right)^{'}=- \frac{\sin{\left(x \right)}}{2 \left(\cos{\left(x \right)} + 1\right)} - \frac{\sin{\left(x \right)}}{2 \left(1 - \cos{\left(x \right)}\right)}=\frac{\sin{\left(x \right)}}{\left(\cos{\left(x \right)} - 1\right) \left(\cos{\left(x \right)} + 1\right)}=- \frac{1}{\sin{\left(x \right)}}$
\end{solution}

\part[7] $\frac{x \sqrt{x^{2} - 1}}{2} - \tan{\left(x + \sqrt{x^{2} - 1} \right)}$
\begin{solution}


$f'(x)=\left(\frac{x \sqrt{x^{2} - 1}}{2}\right)^{'}+\left(- \tan{\left(x + \sqrt{x^{2} - 1} \right)}\right)^{'}=\frac{2x^{2} - 1}{2\sqrt{x^{2} - 1}}- \frac{\left(x + \sqrt{x^{2} - 1}\right) \left(\sec^{2}{\left(x + \sqrt{x^{2} - 1} \right)} \right)}{\sqrt{x^{2} - 1}}=\frac{2 x^{2} - 2 \left(x + \sqrt{x^{2} - 1}\right) \sec^{2}{\left(x + \sqrt{x^{2} - 1} \right)}-1}{2 \sqrt{x^{2} - 1}}
$

\end{solution}

\end{parts}
\addpoints


\question[5] Determina a y b para que sea derivable en $\mathbb{R}$ la función: $$f(x)=\left\{\begin{matrix}
\log{\left(\sin{\left(x \right)} + e \right)} & si & x<0 \\
a x + b + x^{3} & si & x\geqslant 0 \\
\end{matrix}\right.$$ 


\begin{solution}
Continuidad en 0:\\ $\lim_{x \to 0^-} \log{\left(\sin{\left(x \right)} + e \right)} = \lim_{x \to 0^+}\left(a x + b + x^{3}\right) \to 1 = b $ \\ Derivabilidad en 0: \\ $\lim_{x \to 0^-}\left(\frac{\cos{\left(x \right)}}{\sin{\left(x \right)} + e}\right) = \lim_{x \to 0^+}\left(a + 3 x^{2}\right) \to e^{-1} = a $ \\ Por tanto: $\left\{ a : e^{-1}, \  b : 1\right\} $ \\ 
\end{solution}

\addpoints

\question[6] Aprovechando como hipotenusa una pared de $10 \sqrt{2}$ m. se desea acotar una superficie triangular de 
área máxima.¿Qué medidas deberán tener los otros dos lados (catetos)?

\begin{solution}
Función a optimizar: $\frac{x \sqrt{200 - x^{2}}}{2}$ \\ $f'(x)=\frac{100 - x^{2}}{\sqrt{200 - x^{2}}}\land f''(x)=\frac{x \left(x^{2} - 300\right)}{\left(200 - x^{2}\right)^{\frac{3}{2}}}$ \\ Extremos relativos: \\ $-10\land f''(-10)=2$ \\ $10\land f''(10)=-2$ \\
\end{solution}

\addpoints


\question Dada la función: $x + e^{- x}$ 

\begin{parts}
\part[5] Determina los intervalos de crecimiento y decrecimiento de f, así como los extremos relativos
\begin{solution}
$dom(f(x))=\mathbb{R}$ \\
$f'(x)=1 - e^{- x} \to f'(x)=0 \to 1 - e^{- x} = 0 \to x=0 $  \\

Intervalos de crecimiento: $\left[ \left[ \left(-\infty, 0\right), \  \text{False}\right], \  \left[ \left(0, \infty\right), \  \text{True}\right]\right]$ \\
 
\end{solution}
\part[5] Determina los intervalos de concavidad y convexidad de f
\begin{solution}$f''(x)=e^{- x} \to f''(x)>0$ \\ Intervalos de convexidad:$\left[ \mathbb{R}, \  \text{True}\right]$ \\ 
\end{solution}
\part[5] Determina las asíntotas de la gráfica de f
\begin{solution}
$dom(f(x))=\mathbb{R} \to \nexists A.V.$ \\
$lim_{x \to +\infty} f(x)= \infty + 0 = \infty \land lim_{x \to -\infty} f(x)= -\infty \to \nexists A.H.$ \\
Asíntota oblícua: \\ 

pendiente: $lim_{x \to \infty}\frac{f(x)}{x}=lim_{x \to \infty}1+ \frac{e^{-x}}{x} = 1$ \\
ordenada: $lim_{x \to \infty}f-m\cdot x=lim_{x \to \infty}e^{-x}=0 \to A.O: y=x$

\end{solution}


\end{parts}
\addpoints

\end{questions}

\end{document}
\grid
