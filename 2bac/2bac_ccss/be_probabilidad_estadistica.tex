
% \documentclass[11pt]{beamer}
\documentclass[11pt,handout]{beamer}
\usetheme{Boadilla}
%\usetheme{metropolis}

\usecolortheme{crane}


\usepackage[utf8]{inputenc}
\usepackage{amsmath}
\usepackage{amsfonts}
\usepackage{amssymb}

\usepackage{hyperref}
\usepackage{graphicx}
\usepackage[spanish]{babel}
\graphicspath{{../../img/}}

\usepackage{pgf,tikz}

\usetikzlibrary{shapes, calc, shapes, arrows, math, babel, positioning}
\newcommand{\degre}{\ensuremath{^\circ}}
\usepackage{pgf,tikz,pgfplots}
\pgfplotsset{compat=1.15}
\usepackage{mathrsfs}
\usetikzlibrary{math}
\usetikzlibrary{arrows}



%\author{}
\title{Estadística y Probabilidad}

\setbeamercovered{transparent} 
\setbeamertemplate{navigation symbols}{} 
\logo{} 
%\institute{} 
\date{} 
\author{Dep. de Matemáticas}
%\subject{} 
\titlegraphic{\includegraphics[width=0.2\columnwidth]{header_right}}
\begin{document}




%\AtBeginSection[]{
%    \begin{frame}
%        \frametitle{Tabla de Contenidos}
%        \tableofcontents[currentsection]
%    \end{frame}
%}


\AtBeginSubsection[]{
    \begin{frame}
%        \frametitle{Tabla de Contenidos}
        \tableofcontents[currentsubsection]
    \end{frame}
}

%\pgfmathdeclarefunction{gauss}{2}{
%  \pgfmathparse{1/(#2*sqrt(2*pi))*exp(-((x-#1)^2)/(2*#2^2))}	
%}

\begin{frame}
\titlepage
\end{frame}

%\begin{frame}
%\tableofcontents
%\end{frame}

\section{Probabilidad}
\subsection{Probabilidad Simple}

\begin{frame}{Experimentos aleatorios}
\begin{block}{}
Un experimento es \textbf{aleatorio} cuando depende de muchos factores y cualquier pequeña modificación de alguno implica obtener un resultado diferente.
\end{block}
\pause
\textbf{Ejemplos:}
\begin{itemize}[<+->]
 \item \textbf{Aleatorio}: Lanzar un dado y ver el resultado
 \item \textbf{Determinista}: Calcular el tiempo que tarda en caer un objeto al suelo desde una distancia determinada
 \end{itemize} 

\end{frame}



\begin{frame}{Espacio muestral y sucesos}
\begin{itemize}[<+->]
\item \textbf{Espacio muestral}: Conjunto de los posibles resultados del experimento. Se denota: $E$
\item \textbf{Sucesos simples o elementales}: Cualquiera de los elementos del espacio muestral
\item \textbf{Sucesos compuestos}: Sucesos formados por varios simples. 
\item \textbf{Suceso seguro}: Suceso compuesto por los elementos del Espacio muestral. Se cumple siempre
\item \textbf{Suceso imposible}: Cualquier suceso que no se cumpla nunca. Se denota con el símbolo: $\varnothing$
\item \textbf{Suceso contrario}: Si $A$ es un suceso, $\overline{A}$ es el suceso contrario. Es aquel que se cumple cuando no se cumple $A$
\end{itemize}

\end{frame}

\begin{frame}{Ejemplo}
Lanzamos \textbf{un dado} y comprobamos la cara que sale:
\begin{itemize}[<+->]
\item \textbf{Espacio muestral}: $E=\lbrace 1,2,3,4,5,6 \rbrace $
\item \textbf{Sucesos simples o elementales}: $1$, $2$, $3$, $4$, $5$ ó $6$
\item \textbf{Sucesos compuestos}: $A=\lbrace que\ salga\ par\rbrace=\lbrace2,4,6\rbrace$
\item \textbf{Suceso seguro}: $E=\lbrace 1,2,3,4,5,6 \rbrace $
\item \textbf{Suceso imposible}: $\varnothing=\lbrace que\ salga \ mayor \ que \ 6\rbrace$
\item \textbf{Suceso contrario}: Si $A=\lbrace que\ salga\ par\rbrace=\lbrace2,4,6\rbrace$, $\overline{A}=\lbrace que\ salga\ impar\rbrace=\lbrace1,3,5\rbrace$ 
\end{itemize}
\end{frame}
\begin{frame}
{Operaciones con sucesos y relaciones}
\begin{columns}
\begin{column}{0.7\textwidth}
\begin{block}{}
Suceso \textbf{contrario o complementario} de $A$ es otro suceso, $\overline{A}$ que contiene todos los sucesos elementales que no están en $A$.
\end{block}

\end{column}
\begin{column}{0.3\textwidth}
%\scalebox{.7}{%\input{img_tikz/tikz_complementario}}
\begin{tikzpicture}[scale=0.6]
	\draw[draw=blue!50, fill=blue!20,thick] (-2,-2) rectangle (4,2) node[above] {$E$};
    \begin{scope}
        \clip (0,0) circle (1.5cm);
        %\fill[fill=blue!20, draw=blue!50, thick] (0:2cm) circle (1.5cm);
    \end{scope}
    \draw[draw=blue!50,fill=white,  thick] (0,0) circle (1.5cm) node {$A$};
%    \draw[draw=blue!50, thick] (0:2cm) circle (1.5cm) node {$B$};
    %\node[anchor=south] at (current bounding box.north) {$A \cap B$};
    \node[anchor=south] at (1,1.3) {$\overline{A}$};
\end{tikzpicture}


\end{column}
\end{columns}





\end{frame}

\begin{frame}
{Operaciones con sucesos y relaciones}
\begin{columns}
\begin{column}{0.6\textwidth}
\begin{block}{}
la \textbf{unión} de los sucesos $A$ y $B$ es aquel suceso que contiene a todos los elementos de $A$ y a  los de $B$. Por tanto los sucesos que están en $A$ \textbf{o} en $B$ o en los dos a la vez. Se denota: $A\cup B$ 
\end{block}
\begin{block}{}
la \textbf{intersección} de los sucesos $A$ y $B$ es aquel suceso que contiene a todos los elementos que están tanto en $A$ como en $B$. Están en $A$ \textbf{y} en $B$. Se denota: $A\cap B$ 
\end{block}


\end{column}
\begin{column}{0.4\textwidth}
%%\input{img_tikz/tikz_union}
\begin{tikzpicture}[scale=0.8]
\draw[draw=blue!50, thick] (-2,-2) rectangle (4,2) node[above] {$E$};
    \draw[fill=blue!20, draw=blue!50, thick] 
    	(0,0) circle (1.5cm) node {$A$}
        (0:2cm) circle (1.5cm) node {$B$};
    \node[anchor=south] at (1,1.3) {$A \cup B$};
\end{tikzpicture}


%%\input{img_tikz/tikz_interseccion}
\begin{tikzpicture}[scale=0.8]
	\draw[draw=blue!50, thick] (-2,-2) rectangle (4,2) node[above] {$E$};
    \begin{scope}
        \clip (0,0) circle (1.5cm);
        \fill[fill=blue!20, draw=blue!50, thick] (0:2cm) circle (1.5cm);
    \end{scope}
    \draw[draw=blue!50, thick] (0,0) circle (1.5cm) node {$A$};
    \draw[draw=blue!50, thick] (0:2cm) circle (1.5cm) node {$B$};
    %\node[anchor=south] at (current bounding box.north) {$A \cap B$};
    \node[anchor=south] at (1,1.3) {$A \cap B$};
\end{tikzpicture}

\end{column}
\end{columns}



\end{frame}

\begin{frame}
{Ejemplo}

 Tomamos como experimento el resultado de \textbf{lanzar un dado}, y los sucesos: \\
\begin{tabular}{l}
$A=\lbrace que\ salga\ par\rbrace=\lbrace2,4,6\rbrace$ \\
$B=\lbrace que\ sea\ mayor\ que\ 3\rbrace=\lbrace4,5,6\rbrace$ \\
$C=\lbrace que\ salga\ impar\rbrace=\lbrace1,3,5\rbrace$
\end{tabular}

\begin{itemize}	[<+->]
	\item $A\cup B=\lbrace2,4,5,6\rbrace$ \\
	%%\input{img_tikz/tikz_ej_union}
	\item $A\cap B=\lbrace4,6\rbrace$\\
	%%\input{img_tikz/tikz_ej_interseccion}

	\item $A\cup C=\lbrace1,2,3,4,5,6\rbrace=E$
	\item $A\cap C=\varnothing$
\end{itemize}

\end{frame}


\begin{frame}{Compatibilidad de sucesos}
\begin{block}{}
Se dice que dos sucesos son \textbf{incompatibles} cuando su intersección es el conjunto vacío. En caso contrario se dice que son \textbf{compatibles}.
\end{block}
\pause
\textbf{Ejemplo:} Tomamos como experimento el resultado de \textbf{lanzar un dado}, y los sucesos: \\
\begin{tabular}{l}
$A=\lbrace que\ salga\ par\rbrace=\lbrace2,4,6\rbrace$ \\
$B=\lbrace que\ sea\ mayor\ que\ 3\rbrace=\lbrace4,5,6\rbrace$ \\
$C=\lbrace que\ salga\ impar\rbrace=\lbrace1,3,5\rbrace$
\end{tabular}
\newline
$A$ y $B$ son \textbf{compatibles} y $A$ y $C$ \textbf{incompatibles}.

\end{frame}

\begin{frame}{Regla de Laplace}
\begin{block}{}
 La probabilidad de un suceso de un experimento regular viene determinada por la \textbf{Regla de Laplace}:
$$P(A)=\dfrac{Casos\ favorables}{Casos\ posibles} $$
\end{block}

\pause

\textbf{Ejemplo:} Al \textbf{lanzar un dado}, los casos posibles son 6 ($\lbrace1,2,3,4,5,6\rbrace$):\\
\begin{tabular}{l}
La probabilidad de sacar un 3: $\lbrace3\rbrace\to \dfrac{1}{6}$\\
La probabilidad de sacar par: $\lbrace2,4,6\rbrace\to\dfrac{3}{6}$ \\
La probabilidad de sacar más de 4: $\lbrace5,6\rbrace\to\dfrac{2}{6}$
\end{tabular}

\end{frame}

\begin{frame}{Propiedades de la probabilidad} La probabilidad de un experimento regular cumple las siguientes propiedades:
\begin{block}{}
    \begin{itemize}[<+->]
        \item $0 \leq P(A) \leq 1$ 
        \item $P(E) = 1$ y $P(\varnothing) = 0$
        \item $P(A) + P(\overline A) = 1$
        \item $P(A \cup B) = P(A) + P(B) - P(A \cap B)$. Si $A$ y $B$ son incompatibles: $P(A \cup B) = P(A) + P(B)$
    \end{itemize}

\end{block}

Para los sucesos contrarios se cumplen las siguientes propiedades:

\begin{itemize}
\item $P(\overline{A})=1-P(A)$
\item $P(\overline{A \cap B})=P(\overline{A})\cup P(\overline{B})$
\item $P(\overline{A \cup B})=P(\overline{A})\cap P(\overline{B})$
\end{itemize}

\end{frame}


\subsection{Probabilidad Compuesta}
\begin{frame}{Probabilidad condicionada}
\begin{block}{}
$$P(A|B)=\dfrac{P(A\cap B)}{P(B)}$$
\end{block}

\pause
Despejando:
$$P(A\cap B) = P(A|B)\cdot P(B)$$

\pause

Se dice que dos sucesos son \textbf{independientes} cuando la probabilidad de cada uno no depende del resultado del otro. 




\begin{itemize}[<+->]

\item $A\ y \ B\ son \ independientes \Longleftrightarrow P(A|B)=P(A)$

\item $A\ y \ B\ son \ independientes \Rightarrow P(A \cap B)=P(A) \cdot P(B)$

\end{itemize}


\end{frame}

\begin{frame}{Ejemplo sin remplazamiento}
En una urna hay tres bolas blancas y dos negras. Se extraen dos bolas \textbf{sin} reemplazamiento:
%%\input{img_tikz/tikz_arbol_1}

\tikzstyle{bag} = [text width=4em, text centered]
\tikzstyle{end} = [circle, minimum width=3pt,fill, inner sep=0pt]
\tikzstyle{level 1} = [level distance=3.5cm, sibling distance=3.5cm]
\tikzstyle{level 2} = [level distance=3.5cm, sibling distance=2cm]

\begin{tikzpicture}[grow=right, sloped, scale=0.7]
\node[bag] {$3_B, 2_N$}
    child {
        node[bag] {$3_B, 1_N$}        
            child {
                node[end, label=right:
                    {$P(N_1\cap N_2)=\frac{2}{5}\cdot\frac{1}{4}$}] {}
                edge from parent
                node[above] {$N$}
                node[below]  {$1/4$}
            }
            child {
                node[end, label=right:
                    {$P(N_1\cap B_2)=\frac{2}{5}\cdot\frac{3}{4}$}] {}
                edge from parent
                node[above] {$B$}
                node[below] {$3/4$}
            }
            edge from parent 
            node[above] {$N$}
            node[below]  {$2/5$}
    }
    child {
        node[bag] {$2_B, 2_N$}        
        child {
                node[end, label=right:
                    {$P(B_1\cap N_2)=\frac{3}{5}\cdot\frac{2}{4}$}] {}
                edge from parent
                node[above] {$N$}
                node[below]  {$2/4$}
            }
            child {
                node[end, label=right:
                    {$P(B_1\cap B_2)=\frac{3}{5}\cdot\frac{2}{4}$}] {}
                edge from parent
                node[above] {$B$}
                node[below]  {$2/4$}
            }
        edge from parent         
            node[above] {$B$}
            node[below]  {$3/5$}
    };
\end{tikzpicture}

\pause

\begin{itemize}[<+->]
\item Probabilidad de que sean del mismo color: \\
\pause
$P((B_1\cap B_2)\cup (N_1\cap N_2))=\frac{3}{10}+\frac{1}{10}=\frac{2}{5}$
\end{itemize}

\end{frame}

\begin{frame}{Ejemplo con remplazamiento}
En una urna hay tres bolas blancas y dos negras. Se extraen dos bolas \textbf{con} reemplazamiento:
%%\input{img_tikz/tikz_arbol_2}

\tikzstyle{bag} = [text width=4em, text centered]
\tikzstyle{end} = [circle, minimum width=3pt,fill, inner sep=0pt]
\tikzstyle{level 1} = [level distance=3.5cm, sibling distance=3.5cm]
\tikzstyle{level 2} = [level distance=3.5cm, sibling distance=2cm]

\begin{tikzpicture}[grow=right, sloped, scale=0.7]
\node[bag] {$3_B, 2_N$}
    child {
        node[bag] {$3_B, 2_N$}        
            child {
                node[end, label=right:
                    {$P(N_1\cap N_2)=\frac{2}{5}\cdot\frac{2}{5}$}] {}
                edge from parent
                node[above] {$N$}
                node[below]  {$2/5$}
            }
            child {
                node[end, label=right:
                    {$P(N_1\cap B_2)=\frac{2}{5}\cdot\frac{3}{5}$}] {}
                edge from parent
                node[above] {$B$}
                node[below] {$3/5$}
            }
            edge from parent 
            node[above] {$N$}
            node[below]  {$2/5$}
    }
    child {
        node[bag] {$3_B, 2_N$}        
        child {
                node[end, label=right:
                    {$P(B_1\cap N_2)=\frac{3}{5}\cdot\frac{2}{5}$}] {}
                edge from parent
                node[above] {$N$}
                node[below]  {$2/5$}
            }
            child {
                node[end, label=right:
                    {$P(B_1\cap B_2)=\frac{3}{5}\cdot\frac{3}{5}$}] {}
                edge from parent
                node[above] {$B$}
                node[below]  {$3/5$}
            }
        edge from parent         
            node[above] {$B$}
            node[below]  {$3/5$}
    };
\end{tikzpicture}

\pause
\begin{itemize}
\item Probabilidad de que sean del mismo color: \\
\pause
$P((B_1\cap B_2)\cup (N_1\cap N_2))=\frac{9}{25}+\frac{4}{25}=\frac{13}{25}$
\end{itemize}

\end{frame}

\begin{frame}{Tablas de Contingencia}
\begin{block}{}
Si tenemos \textbf{dos particiones del espacio muestral}, generalmente por dos características diferentes de la población, resulta útil el uso de \textbf{tablas de contingencia} para el cálculo de probabilidades.
\end{block}
\end{frame}

\begin{frame}{Ejemplo de tabla de contingencia}

\begin{block}{}
En una cadena de televisión se hizo una encuesta a 2500 personas para saber la audiencia de un \textbf{debate} y de una \textbf{película} que se emitieron en horas distintas: 2100 vieron la película, 1500 vieron el debate y 350 no
vieron ninguno de los dos programas.
\end{block}

Si llamamos $D$ al suceso 'haber visto el debate'  y $P$ a 'haber visto la película', podemos organizar la información en una tabla y completarla: 
\newline

\pause

\begin{columns}
\begin{column}{0.41\textwidth}  
     \begin{tabular}{|c | c | c | c |} 
     \hline
       &  $D$ & $\overline{D}$ & Total \\ [0.5ex] 
     \hline
     $P$ &  &  & 2100 \\ 
     \hline
     $\overline{P}$ &  & 350 &  \\
     \hline
     Total & 1500 &  &  2500 \\
     \hline
    \end{tabular}
\end{column}
\begin{column}{0.05\textwidth}
  $$\to$$  
\end{column}
\begin{column}{0.41\textwidth}
\pause
     \begin{tabular}{|c | c | c | c |} 
     \hline
      &  $D$ & $\overline{D}$ & Total \\ [0.5ex] 
     \hline
     $P$ & \color{red}1450 & \color{red}650 & 2100 \\ 
     \hline
     $\overline{P}$ & \color{red}50  & 350 & \color{red}400 \\
     \hline
     Total & 1500 & \color{red}1000 &  2500 \\
     \hline
    \end{tabular}
\end{column}
\end{columns}
\end{frame}

\begin{frame}{Ejemplo de tabla de contingencia}
Ejemplos de cálculo de probabilidades a través de Laplace y condicionales:
\begin{columns}
\begin{column}{0.41\textwidth}
     \begin{tabular}{|c | c | c | c |} 
     \hline
      &  $D$ & $\overline{D}$ & Total \\ [0.5ex] 
     \hline
     $P$ & \color{red}1450 & \color{red}650 & 2100 \\ 
     \hline
     $\overline{P}$ & \color{red}50  & 350 & \color{red}400 \\
     \hline
     Total & 1500 & \color{red}1000 &  2500 \\
     \hline
    \end{tabular}
\end{column}
\begin{column}{0.65\textwidth}  
\begin{itemize}[<+->]
    \item $P(P \cap D)= \frac{1450}{2500}=\frac{29}{50}$, o bien $P(P \cap D)= P(P)\cdot P(D|P)=\frac{2100}{2500}\cdot \frac{1450}{2100}=\frac{29}{50}$
    \item $P(P|D)=\frac{1450}{1500}=\frac{29}{30}$, o bien $P(P|D)=\frac{P(P \cap D)}{P(D)} =\frac{\frac{1450}{2500}}{\frac{1500}{2500}}=\frac{29}{30}$
    \item $P(D|P)=\frac{1450}{2100}=\frac{29}{42}$, o bien $P(D|P)= \frac{P(D \cap P)}{P(P)} =\frac{\frac{1450}{2500}}{\frac{2100}{2500}}=\frac{29}{42}$
\end{itemize}
\end{column}
\end{columns}
\end{frame}

\begin{frame}{Teorema de la probabilidad total}
Si $A_1$, $A_2$, ..., $A_n$   son sucesos incompatibles dos a dos y cuya unión es todo el espacio muestral, entonces la probabilidad de cualquier otro suceso $B$ es:
\begin{block}{}
$$P(B)=\sum_{i=1}^n P(A_i)\cdot  P(B|A_i) $$
\end{block}

\end{frame}

\begin{frame}{Ejemplo de probabilidad total}
En una urna en la que hay tres bolas blancas y dos negras. Si se extraen dos bolas \textbf{sin} reemplazamiento:
%%\input{img_tikz/tikz_arbol_3}
\tikzstyle{bag} = [text width=4em, text centered]
\tikzstyle{end} = [circle, minimum width=3pt,fill, inner sep=0pt]
\tikzstyle{level 1} = [level distance=3.5cm, sibling distance=3.5cm]
\tikzstyle{level 2} = [level distance=3.5cm, sibling distance=2cm]
\begin{tikzpicture}[grow=right, sloped, scale=0.7]
\node[bag] {$3_B, 2_N$}
    child {
        node[bag] {$3_B, 1_N$}        
            child {
                node[end, label=right:
                    {$P(N_1\cap N_2)=\frac{2}{5}\cdot\frac{1}{4}$}] {}
                edge from parent
                node[above] {$N$}
                node[below]  {$1/4$}
            }
            child {
                node[end, label=right:
                    {$P(N_1\cap B_2)=\frac{2}{5}\cdot\frac{3}{4}$}] {}
                edge from parent
                node[above] {$B$}
                node[below] {$3/4$}
            }
            edge from parent 
            node[above] {$N$}
            node[below]  {$2/5$}
    }
    child {
        node[bag] {$2_B, 2_N$}        
        child {
                node[end, label=right:
                    {$P(B_1\cap N_2)=\frac{3}{5}\cdot\frac{2}{4}$}] {}
                edge from parent
                node[above] {$N$}
                node[below]  {$2/4$}
            }
            child {
                node[end, label=right:
                    {$P(B_1\cap B_2)=\frac{3}{5}\cdot\frac{2}{4}$}] {}
                edge from parent
                node[above] {$B$}
                node[below]  {$2/4$}
            }
        edge from parent         
            node[above] {$B$}
            node[below]  {$3/5$}
    };
\end{tikzpicture}
\pause
$$P(B_2)=P(B_1)\cdot P(B_2|B_1) + P(N_1)\cdot P(B_2|N_1)
= \frac{3}{5}\cdot\frac{2}{4} + \frac{2}{5}\cdot\frac{3}{4}$$
\end{frame}


\begin{frame}{Teorema de Bayes}
Si $A_1$, $A_2$, ..., $A_n$   son sucesos incompatibles dos a dos y cuya unión es todo el espacio muestral, y $B$ otro suceso cualquiera:
\begin{block}{}
$$P(A_i|B)=\dfrac{P(A_i \cap B)}{\sum_{i=1}^n P(A_i)\cdot  P(B|A_i)} $$
\end{block}
\end{frame}

\begin{frame}{Ejemplo de Bayes}
En una urna en la que hay tres bolas blancas y dos negras. Si se extraen dos bolas \textbf{sin} reemplazamiento:
%%\input{img_tikz/tikz_arbol_3}
\tikzstyle{bag} = [text width=4em, text centered]
\tikzstyle{end} = [circle, minimum width=3pt,fill, inner sep=0pt]
\tikzstyle{level 1} = [level distance=3.5cm, sibling distance=3.5cm]
\tikzstyle{level 2} = [level distance=3.5cm, sibling distance=2cm]
\begin{tikzpicture}[grow=right, sloped, scale=0.7]
\node[bag] {$3_B, 2_N$}
    child {
        node[bag] {$3_B, 1_N$}        
            child {
                node[end, label=right:
                    {$P(N_1\cap N_2)=\frac{2}{5}\cdot\frac{1}{4}$}] {}
                edge from parent
                node[above] {$N$}
                node[below]  {$1/4$}
            }
            child {
                node[end, label=right:
                    {$P(N_1\cap B_2)=\frac{2}{5}\cdot\frac{3}{4}$}] {}
                edge from parent
                node[above] {$B$}
                node[below] {$3/4$}
            }
            edge from parent 
            node[above] {$N$}
            node[below]  {$2/5$}
    }
    child {
        node[bag] {$2_B, 2_N$}        
        child {
                node[end, label=right:
                    {$P(B_1\cap N_2)=\frac{3}{5}\cdot\frac{2}{4}$}] {}
                edge from parent
                node[above] {$N$}
                node[below]  {$2/4$}
            }
            child {
                node[end, label=right:
                    {$P(B_1\cap B_2)=\frac{3}{5}\cdot\frac{2}{4}$}] {}
                edge from parent
                node[above] {$B$}
                node[below]  {$2/4$}
            }
        edge from parent         
            node[above] {$B$}
            node[below]  {$3/5$}
    };
\end{tikzpicture}
\pause 
$P(B_1|B_2)=\dfrac{P(B_1 \cap B_2)}{P(B_1)\cdot  P(B_2|B_1)+P(N_1)\cdot  P(B_2|N_1)}=\dfrac{\dfrac{3}{5}\cdot\dfrac{2}{4}}{\dfrac{3}{5}\cdot\dfrac{2}{4} + \dfrac{2}{5}\cdot\dfrac{3}{4}}$
\end{frame}





\section{Distribuciones de Probabilidad}
\subsection{Variables y distribuciones}

\begin{frame}{Variables aleatorias y distribuciones de probabilidad}
\begin{block}{Finalidad:} \textbf{Abstraer matemáticamente} un tipo de \textbf{experimento aleatorio}. Y, con ello, poder estimar de manera teórica lo que sucedería de manera experimental mediante una estadística.
\end{block}

\pause

\begin{block}{¿Cómo?:} Mediante \textbf{variables aleatorias} y \textbf{distribuciones de probabilidad} asociadas a esas variables

\end{block}


\end{frame}

\begin{frame}{Variables aleatorias}
\begin{block}{}
Una \textbf{variable aleatoria} es una función que a cada suceso
elemental de un espacio muestral le asigna un número. \\ Para hacer referencia a las variables se usan las letras: $X$, $Y$, ...
\end{block}
\end{frame}

\begin{frame}{Ejemplo}
\begin{block}{}
Sea el \textbf{experimento aleatorio “lanzar un dado”} 
\end{block}
\pause
El espacio muestral lo componen las 6 caras del dado.\\
Podemos asignar la variable $X$ que a cada cara le asocia el número que represente su cara.
\begin{center}
\begin{tabular}{ccc}
 & $X$ &  \\
Suceso &  &  $x_i$\\ \hline 
Cara 1 & $\rightarrow$ & 1 \\ 
Cara 2 & $\rightarrow$ & 2 \\ 
Cara 3 & $\rightarrow$ & 3 \\ 
Cara 4 & $\rightarrow$ & 4 \\ 
Cara 5 & $\rightarrow$ & 5 \\ 
Cara 6 & $\rightarrow$ & 6 \\ 
\end{tabular} 
\end{center}
\end{frame}


\begin{frame}{Ejemplo}
\begin{block}{}
Sea el \textbf{experimento compuesto "lanzar dos monedas"} 
\end{block}
\pause
Podemos asignar la variable aleatoria:
\begin{center}
\emph{Y = \{Número de caras\}} \\
\begin{tabular}{ccc}
 & $Y$ &  \\
Suceso &  &  $y_i$\\ \hline 
C,C & $\rightarrow$ & 2 \\ 
C,X & $\rightarrow$ & 1 \\ 
X,C & $\rightarrow$ & 1 \\ 
X,X & $\rightarrow$ & 0 \\ 
\end{tabular} 
\end{center}
\end{frame}




\begin{frame}{Tipos de variables aleatorias}
\begin{itemize}[<+->] 

    \item{Discretas:} Toman un \textbf{número finito o numerable} de valores. \textbf{Ejemplo:} Sea la \emph{X =“El número de caras al lanzar dos dados”}. Los valores posibles son 0, 1 o 2 (que es un conjunto finito de datos, en concreto 3 datos)
\item{Continuas:} Toman valores en un \textbf{rango continuo}. \textbf{Ejemplo:} \emph{X = “Distancia al centro de la diana medida desde la posición en que cae un dardo lanzado por un tirador experto” }. En este caso la variable puede tomar cualquier valor en el rango entre 0 y el radio de la diana 
\end{itemize}

\end{frame}

\begin{frame}{Distribuciones de probabilidad}
\begin{block}{}
Llamaremos \textbf{distribución de probabilidad} a la relación entre los valores de la variable y sus probabilidades.
\end{block}

\begin{itemize}[<+->]
    \item Estas relaciones se pueden indicar mediante el uso de funciones
    \item El tipo de función y su tratamiento es diferente según las variables sean discretas o continuas
    
\end{itemize}


Veamos algunas distribuciones ...
\end{frame}


\begin{frame}
{Distribución uniforme discreta}
\begin{block}{Ejemplo}
Sea la variable \emph{X = “Número obtenido al lanzar una dado”}
\end{block}
\pause
\begin{columns}
\begin{column}{0.5 \textwidth}
A cada valor de la variable podemos asignarle su probabilidad:
\begin{center}
\begin{tabular}{ccc}
 & $P(X)$ &  \\
$x_i$ &  &  $P(x_i)$\\ \hline 
1 & $\rightarrow$ & $\tfrac{1}{6}$ \\ 
2 & $\rightarrow$ & $\tfrac{1}{6}$ \\ 
3 & $\rightarrow$ & $\tfrac{1}{6}$ \\ 
4 & $\rightarrow$ & $\tfrac{1}{6}$ \\ 
5 & $\rightarrow$ & $\tfrac{1}{6}$ \\ 
6 & $\rightarrow$ & $\tfrac{1}{6}$ \\ 
\end{tabular} 
\end{center}
\end{column}
\pause
\begin{column}{0.5 \textwidth}
 Podemos representar la relación anterior mediante una función:

$$P\colon \begin{array}{ll} 
          X \rightarrow \mathbb{R} \\ 
          x_i\mapsto P(X=x_i)=\frac{1}{n} 
         \end{array}$$
\end{column}
\end{columns}

\pause
A este tipo de distribución se le llama \textbf{uniforme discreta}.
\end{frame}

\subsection{Binomial y Normal}

\begin{frame}
{Distribución Binomial. Introducción}
\begin{block}{Ejemplo} Queremos calcular las probabilidades de que al lanzar 5 monedas, obtengamos tres caras.
\end{block}
Un suceso que cumple el enunciado es:
$$S_1=\left\lbrace C,C,C,X,X \right\rbrace$$
Teniendo en cuenta que lanzar cada moneda son experimentos independientes, la probabilidad de ese suceso será:
$$P\left(S_1\right)=P\left(C_1\right)\cdot P\left(C_2 C_1 \right)$$
\begin{eqnarray*}
P\left(S_1\right) & = &P\left(C_1\right)\cdot P\left(C_2 | C_1 \right)\cdot ... \cdot P\left(X_5 | C_1 \cap C_2  \cap C_3  \cap X_4   \right)= \\ &  = & P\left(C_1\right)\cdot  P\left(C_2\right) \cdot P\left(C_3\right) \cdot P\left(X_4\right) \cdot P\left(X_5\right)= \\
& = & P\left(C\right)^3\cdot  P\left(X\right)^2
\end{eqnarray*}


\end{frame}

\begin{frame}{}
Como la probabilidad de que una moneda sea cara es $\frac{1}{2}$ y la de que sea cruz también:

$$P\left(S_1\right)=\left(\frac{1}{2}\right)^3\cdot  \left(\frac{1}{2}\right)^2$$
Ahora bien, habrá tantos sucesos que cumplan el enunciado como combinaciones de 5 elementos tomados de 3 en 3. Por tanto la probabilidad de que salgan 3 caras será:

$$P\left(Salgan \ 3 \ caras\right)=\binom{5}{3}\left(\frac{1}{2}\right)^3\cdot  \left(\frac{1}{2}\right)^2$$

\end{frame}

\begin{frame}{}

Podemos determinar la probabilidad mediante la siguiente función.

$$P\colon \begin{array}{l} 
          X \rightarrow \mathbb{R} \\ 
          k\mapsto P(X=k)=\binom{5}{k}\left(\frac{1}{2}\right)^k\cdot  \left(\frac{1}{2}\right)^{5-k} 
         \end{array}$$

\end{frame}


\begin{frame}{}
Realizando los cálculos para $k = 1,...,5 $, la distribución de probabilidad de $X$ que resulta es:
\begin{columns}
\begin{column}{0.3\textwidth}
\begin{center}
\begin{tabular}{ccc}
 & $P(X)$ &  \\
$x_i$ &  &  $P(x_i)$\\ \hline 
0 & $\rightarrow$ & $0.03125$ \\ 
1 & $\rightarrow$ & $0.15625$ \\ 
2 & $\rightarrow$ & $0.3125$ \\ 
3 & $\rightarrow$ & $0.3125$ \\ 
4 & $\rightarrow$ & $0.15625$ \\ 
5 & $\rightarrow$ & $0.03125$ \\ 
\end{tabular} 
\end{center}
\end{column}
\begin{column}{0.7\textwidth}
\begin{center}
% %\input{img_tikz/tikz_binomial_0}
\begin{tikzpicture}[
    declare function={binom(\k,\n,\p)=\n!/(\k!*(\n-\k)!)*\p^\k*(1-\p)^(\n-\k);}
]
\begin{axis}[
    samples at={0,...,5},
    yticklabel style={
        /pgf/number format/fixed,
        /pgf/number format/fixed zerofill,
        /pgf/number format/precision=1
    },
    ybar=0pt, bar width=1
]
\addplot [fill=cyan, fill opacity=0.5] {binom(x,5,0.5)}; \addlegendentry{$p=0.5$}

\end{axis}
\end{tikzpicture}
\end{center}
\end{column}

\end{columns}
\pause
Esto es un ejemplo de \textbf{distribución binomial} de tamaño $5$ y probabilidad $0.5$ .
\end{frame}



\begin{frame}{Distribución Binomial}
Hablaremos de una \textbf{distribución binomial} cuando:
\begin{itemize}[<+->]
\item Se parte de un \textbf{experimento compuesto} de varios simples independientes
\item Los experimentos simples son \textbf{dicotómicos}. Es decir, solo puede haber dos sucesos elementales: uno al que llamaremos acierto y otro al que llamaremos fracaso
\item Asociado al experimento compuesto tenemos la variable \textbf{número de aciertos} cuando realizamos el experimento simple un número determinado de veces
\end{itemize}
\end{frame}

\begin{frame}{Distribución Binomial}
En la situación anterior, la distribución binomial vendrá determinada por dos parámetros:
\begin{itemize}[<+->]
\item Parámetro $n$: Número de veces que se realiza el experimento simple 
\item Parámetro $p$: La probabilidad de que ocurra el suceso acierto
\end{itemize}
\pause

A este tipo de variable y su distribución de probabilidades se le llama \textbf{binomial} y se denota $$X \sim \mathcal{B}(n,\,p)$$.
\end{frame}

\begin{frame}{Ejemplo}
\begin{block}{Lanzamineto de 5 monedas} El experimento se compone de \textbf{5 lanzamientos de moneda}. Si sale cara es acierto y si no fracaso. La variable aleatoria asociada al experimento será el número de caras que salen al lanzar 5 monedas. Esta variable sigue una distribución binomial $X \sim \mathcal{B}(5,\,0.5)$.
\end{block}



\end{frame}

\begin{frame}{Probabilidad de la binomial}
\textbf{En general tendremos una binomial de tamaño $n$ y probabilidad $p$}, cuando el experimento simple se haga $n$ veces y la probabilidad de acierto sea $p$.

La función de probabilidad en este caso nos queda:

\begin{block}{}
 $$P\colon \begin{array}{l} 
          X \rightarrow \mathbb{R} \\ 
          k\mapsto P(X=k)=\binom{n}{k}\left(p\right)^k\cdot  \left(1-p\right)^{n-k} 
         \end{array}$$

\end{block}

\end{frame}

\begin{frame}{Parámetros estadísticos de la binomial}
\textbf{La media y la desviación típica} de la distribución binomial $X \sim \mathcal{B}(n,\,p)$:

\begin{block}{}
\begin{itemize}
    \item La media de la Binomial es:  $$\overline{x}=n\cdot p$$
    \item La desviación típica de la Binomial es:  $$\sigma=\sqrt{n\cdot p\cdot q}$$
\end{itemize}
\end{block}

\textbf{Ejemplo:} Si $X \sim \mathcal{B}(10000,\,0.002)$
\begin{itemize}
    \item La media:  $$\overline{x}=10000\cdot 0.002= 200$$
    \item La desviación típica:  $$\sigma=\sqrt{10000\cdot 0.002\cdot (1-0.002)}=\sqrt{196}=14$$
\end{itemize}

    
\end{frame}


\begin{frame}{Distribución Binomial}
Ejemplos de representación gráfica de la distribución de probabilidad de algunas binomiales:    
         
%%\input{img_tikz/tikz_binomial}
\begin{tikzpicture}[
    declare function={binom(\k,\n,\p)=\n!/(\k!*(\n-\k)!)*\p^\k*(1-\p)^(\n-\k);}
]
\begin{axis}[
    samples at={0,...,40},
    yticklabel style={
        /pgf/number format/fixed,
        /pgf/number format/fixed zerofill,
        /pgf/number format/precision=1
    }
]
\addplot [only marks, cyan] {binom(x,40,0.2)}; \addlegendentry{$p=0.2 \, n=40$}
\addplot [only marks, orange] {binom(x,40,0.5)}; \addlegendentry{$p=0.5 \, n=40 $}
\end{axis}
\end{tikzpicture}
    
\end{frame}


\begin{frame}{Distribución de probabilidades en una variable continua}

Observa que la probabilidad total tiene que ser 1 y la variable $X$, al ser continua, puede tomar $\infty$ valores.
\begin{block}{}
Si queremos repartir la probabilidad entre infinitos valores, \textbf{la probabilidad de un valor concreto será 0} ($\lim_{x \to \infty}\frac{1}{x}=0$). 
\end{block}
   

Para calcular las probabilidades no se utilizarán funciones de probabilidad como hemos visto hasta ahora sino \textbf{funciones de densidad}:
\end{frame}

\pgfmathdeclarefunction{gauss}{2}{%
  \pgfmathparse{1/(#2*sqrt(2*pi))*exp(-((x-#1)^2)/(2*#2^2))}%
}

\begin{frame}{Función de densidad}
En variables continuas solo tiene sentido calcular la probabilidad en intervalos.
Se llama \textbf{función de densidad} ($f(x)$) a aquella que: 
\begin{block}{}
$$P\left(a\leq X \leq b \right) = \int_{a}^{b} f(x) dx$$
\end{block}    


La interpretación gráfica de lo anterior nos dice que la probabilidad de un intervalo corresponde con el área de la función de densidad en ese intervalo.
\begin{center}
% %\input{img_tikz/tikz_densidad_1}

\begin{tikzpicture}[scale=0.7]

\tikzmath{
			\conf = 0.96; \crit= 2.05; \a=(1-\conf)/2;
          }

%\begin{axis}[
%   no markers, domain=0:10, samples=100,
%   axis lines*=left, xlabel=$x$, ylabel=$y$,
%   every axis y label/.style={at=(current axis.above origin),anchor=south},
%   every axis x label/.style={at=(current axis.right of origin),anchor=west},
%   height=5cm, width=12cm,
%   xtick={4,6.5}, ytick=\empty,
%   enlargelimits=false, clip=false, axis on top,
%   %grid = major
%   ]
%   \addplot [fill=cyan!20, draw=none, domain=0:5.96] {gauss(6.5,1)} \closedcycle;
%   \addplot [very thick,cyan!50!black] {gauss(4,1)};
%   \addplot [very thick,cyan!50!black] {gauss(6.5,1)};


% %\draw [yshift=-0.6cm, latex-latex](axis cs:4,0) -- node [fill=white] {$1.96\sigma$} (axis cs:5.96,0);
% \end{axis}

\begin{axis}[
  no markers, domain=-5:5, samples=100,
  axis lines=left, 
  xlabel=$ $, ylabel=$ $,
  every axis y label/.style={at=(current axis.above origin),anchor=south},
  every axis x label/.style={at=(current axis.right of origin),anchor=west},
  height=5cm, width=12cm,
  xtick={-1,\crit}, ytick=\empty,
  xticklabels = {$a$, $b$},
  enlargelimits=false, clip=false, axis on top,
  %grid = major
  ]
  \addplot [fill=cyan!20, draw=none, domain=-1:\crit] {gauss(0,1)} \closedcycle;
  \addplot [very thick,cyan!50!black] {gauss(0,1)};
  %\addplot [very thick,cyan!50!black] {gauss(6.5,1)};
  
%\draw [yshift=-0.6cm, latex-latex](axis cs:4,0) -- node [fill=white] {$1.96\sigma$} (axis cs:5.96,0);
\end{axis}
\node[] at (5.5,1.5) {$P\left(a\leq X \leq b \right) $};	

\end{tikzpicture}

\end{center}
\end{frame}


\begin{frame}
{Distribución Normal}

Se trata de una distribución asociada a a un variable continua . Resulta de aproximar una binomial para valores muy grandes de $n$. 
\begin{block}{} Y se denota: $
X \sim \mathcal{N}(\mu,\,\sigma)\,.
    $ 
\end{block}
\pause
Tiene dos parámetros:
\begin{itemize} [<+->]
    \item $\mu :$ Media de la distribución
    \item $\sigma :$ Desviación típica
\end{itemize}

\pause    
Esta distribución aparece asociada a muchos fenómenos naturales. 
\end{frame}


\begin{frame}{Función de densidad de una distribución normal}  
\begin{columns}
\begin{column}{0.4\textwidth}
\begin{block}{}
Si $
X \sim \mathcal{N}(\mu,\,\sigma)
    $, la función de densidad es:
    $$f(x)=\frac{1}{{\sigma \sqrt {2\pi } }}e^{{{ - \left( {x - \mu } \right)^2 } \mathord{\left/ {\vphantom {{ - \left( {x - \mu } \right)^2 } {2\sigma ^2 }}} \right. \kern-\nulldelimiterspace} {2\sigma ^2 }}}$$
\end{block}
\end{column}
\begin{column}{0.6\textwidth} Ejemplos de gráficas:
% %\input{img_tikz/tikz_normal}
\begin{tikzpicture}
\begin{axis}[every axis plot post/.append style={
  mark=none,domain=-2:3,samples=50,smooth}, % All plots: from -2:2, 50 samples, smooth, no marks
  axis x line*=bottom, % no box around the plot, only x and y axis
  axis y line*=left, % the * suppresses the arrow tips
  enlargelimits=upper] % extend the axes a bit to the right and top
  \addplot {gauss(0,0.5)};
  \addlegendentry{$\mu=0 \land \sigma^{2}=0.5$}
  \addplot {gauss(1,0.75)};
  \addlegendentry{$\mu=1 \land \sigma^{2}=0.75$}
  \addplot {gauss(0,1)};
  \addlegendentry{$\mu=0 \land \sigma^{2}=1$}
\end{axis}
\end{tikzpicture}
\end{column}
\end{columns}{}

\end{frame}

\begin{frame}{Cálculo práctico de la probabilidad de la Normal:}
\begin{columns}
\begin{column}{0.4\textwidth}
\begin{block}{}
En realidad, para calcular la probabilidad no se hace la integral, sino que \textbf{se utiliza una tabla} que ya tiene calculadas probabilidades de la  $Z \sim \mathcal{N}(0,\,1)
    $ ...
\end{block}

\end{column}
\begin{column}{0.6\textwidth}
% \includegraphics[page=1,width=1\textwidth]{probabilidad/distribucion_normal}

\resizebox{0.99\textwidth}{!}{
\begin{tabular}{l|llllllllll}

z   & 0       & 0,01    & 0,02    & 0,03    & 0,04    & 0,05    & 0,06    & 0,07    & 0,08    & 0,09    \\
\hline
0   & 0,5     & 0,50399 & 0,50798 & 0,51197 & 0,51595 & 0,51994 & 0,52392 & 0,5279  & 0,53188 & 0,53586 \\
0,1 & 0,53983 & 0,5438  & 0,54776 & 0,55172 & 0,55567 & 0,55962 & 0,56356 & 0,56749 & 0,57142 & 0,57535 \\
0,2 & 0,57926 & 0,58317 & 0,58706 & 0,59095 & 0,59483 & 0,59871 & 0,60257 & 0,60642 & 0,61026 & 0,61409 \\
0,3 & 0,61791 & 0,62172 & 0,62552 & 0,6293  & 0,63307 & 0,63683 & 0,64058 & 0,64431 & 0,64803 & 0,65173 \\
0,4 & 0,65542 & 0,6591  & 0,66276 & 0,6664  & 0,67003 & 0,67364 & 0,67724 & 0,68082 & 0,68439 & 0,68793 \\
0,5 & 0,69146 & 0,69497 & 0,69847 & 0,70194 & 0,7054  & 0,70884 & 0,71226 & 0,71566 & 0,71904 & 0,7224  \\
0,6 & 0,72575 & 0,72907 & 0,73237 & 0,73565 & 0,73891 & 0,74215 & 0,74537 & 0,74857 & 0,75175 & 0,7549  \\
0,7 & 0,75804 & 0,76115 & 0,76424 & 0,7673  & 0,77035 & 0,77337 & 0,77637 & 0,77935 & 0,7823  & 0,78524 \\
0,8 & 0,78814 & 0,79103 & 0,79389 & 0,79673 & 0,79955 & 0,80234 & 0,80511 & 0,80785 & 0,81057 & 0,81327 \\
0,9 & 0,81594 & 0,81859 & 0,82121 & 0,82381 & 0,82639 & 0,82894 & 0,83147 & 0,83398 & 0,83646 & 0,83891 \\
1   & 0,84134 & 0,84375 & 0,84614 & 0,84849 & 0,85083 & 0,85314 & 0,85543 & 0,85769 & 0,85993 & 0,86214 \\
1,1 & 0,86433 & 0,8665  & 0,86864 & 0,87076 & 0,87286 & 0,87493 & 0,87698 & 0,879   & 0,881   & 0,88298 \\
1,2 & 0,88493 & 0,88686 & 0,88877 & 0,89065 & 0,89251 & 0,89435 & 0,89617 & 0,89796 & 0,89973 & 0,90147 \\
1,3 & 0,9032  & 0,9049  & 0,90658 & 0,90824 & 0,90988 & 0,91149 & 0,91309 & 0,91466 & 0,91621 & 0,91774 \\
1,4 & 0,91924 & 0,92073 & 0,9222  & 0,92364 & 0,92507 & 0,92647 & 0,92785 & 0,92922 & 0,93056 & 0,93189 \\
1,5 & 0,93319 & 0,93448 & 0,93574 & 0,93699 & 0,93822 & 0,93943 & 0,94062 & 0,94179 & 0,94295 & 0,94408 \\
1,6 & 0,9452  & 0,9463  & 0,94738 & 0,94845 & 0,9495  & 0,95053 & 0,95154 & 0,95254 & 0,95352 & 0,95449 \\
1,7 & 0,95543 & 0,95637 & 0,95728 & 0,95818 & 0,95907 & 0,95994 & 0,9608  & 0,96164 & 0,96246 & 0,96327 \\
1,8 & 0,96407 & 0,96485 & 0,96562 & 0,96638 & 0,96712 & 0,96784 & 0,96856 & 0,96926 & 0,96995 & 0,97062 \\
1,9 & 0,97128 & 0,97193 & 0,97257 & 0,9732  & 0,97381 & 0,97441 & 0,975   & 0,97558 & 0,97615 & 0,9767  \\
2   & 0,97725 & 0,97778 & 0,97831 & 0,97882 & 0,97932 & 0,97982 & 0,9803  & 0,98077 & 0,98124 & 0,98169 \\
2,1 & 0,98214 & 0,98257 & 0,983   & 0,98341 & 0,98382 & 0,98422 & 0,98461 & 0,985   & 0,98537 & 0,98574 \\
2,2 & 0,9861  & 0,98645 & 0,98679 & 0,98713 & 0,98745 & 0,98778 & 0,98809 & 0,9884  & 0,9887  & 0,98899 \\
2,3 & 0,98928 & 0,98956 & 0,98983 & 0,9901  & 0,99036 & 0,99061 & 0,99086 & 0,99111 & 0,99134 & 0,99158 \\
2,4 & 0,9918  & 0,99202 & 0,99224 & 0,99245 & 0,99266 & 0,99286 & 0,99305 & 0,99324 & 0,99343 & 0,99361 \\
2,5 & 0,99379 & 0,99396 & 0,99413 & 0,9943  & 0,99446 & 0,99461 & 0,99477 & 0,99492 & 0,99506 & 0,9952  \\
2,6 & 0,99534 & 0,99547 & 0,9956  & 0,99573 & 0,99585 & 0,99598 & 0,99609 & 0,99621 & 0,99632 & 0,99643 \\
2,7 & 0,99653 & 0,99664 & 0,99674 & 0,99683 & 0,99693 & 0,99702 & 0,99711 & 0,9972  & 0,99728 & 0,99736 \\
2,8 & 0,99744 & 0,99752 & 0,9976  & 0,99767 & 0,99774 & 0,99781 & 0,99788 & 0,99795 & 0,99801 & 0,99807 \\
2,9 & 0,99813 & 0,99819 & 0,99825 & 0,99831 & 0,99836 & 0,99841 & 0,99846 & 0,99851 & 0,99856 & 0,99861 \\
3   & 0,99865 & 0,99869 & 0,99874 & 0,99878 & 0,99882 & 0,99886 & 0,99889 & 0,99893 & 0,99896 & 0,999   \\
3,1 & 0,99903 & 0,99906 & 0,9991  & 0,99913 & 0,99916 & 0,99918 & 0,99921 & 0,99924 & 0,99926 & 0,99929 \\
3,2 & 0,99931 & 0,99934 & 0,99936 & 0,99938 & 0,9994  & 0,99942 & 0,99944 & 0,99946 & 0,99948 & 0,9995  \\
3,3 & 0,99952 & 0,99953 & 0,99955 & 0,99957 & 0,99958 & 0,9996  & 0,99961 & 0,99962 & 0,99964 & 0,99965 \\
3,4 & 0,99966 & 0,99968 & 0,99969 & 0,9997  & 0,99971 & 0,99972 & 0,99973 & 0,99974 & 0,99975 & 0,99976 \\
3,5 & 0,99977 & 0,99978 & 0,99978 & 0,99979 & 0,9998  & 0,99981 & 0,99981 & 0,99982 & 0,99983 & 0,99983 \\
3,6 & 0,99984 & 0,99985 & 0,99985 & 0,99986 & 0,99986 & 0,99987 & 0,99987 & 0,99988 & 0,99988 & 0,99989 \\
3,7 & 0,99989 & 0,9999  & 0,9999  & 0,9999  & 0,99991 & 0,99991 & 0,99992 & 0,99992 & 0,99992 & 0,99992 \\
3,8 & 0,99993 & 0,99993 & 0,99993 & 0,99994 & 0,99994 & 0,99994 & 0,99994 & 0,99995 & 0,99995 & 0,99995 \\
3,9 & 0,99995 & 0,99995 & 0,99996 & 0,99996 & 0,99996 & 0,99996 & 0,99996 & 0,99996 & 0,99997 & 0,99997 \\
4   & 0,99997 & 0,99997 & 0,99997 & 0,99997 & 0,99997 & 0,99997 & 0,99998 & 0,99998 & 0,99998 & 0,99998
\end{tabular}

}
\end{column}
\end{columns}


\end{frame}

\begin{frame}{Cálculo práctico de la probabilidad de la Normal:}
\begin{columns}
\begin{column}{0.6\textwidth}
\begin{block}{}
... y que refleja:

$$P\left(Z\leq k \right), \ \  k \in \left[ 0 , 4.09 \right]$$
\begin{center}
    % %\input{img_tikz/tikz_normal_z}

\begin{tikzpicture}[scale=0.6]

\tikzmath{
			\conf = 0; \crit= 1.05; \a=round(1-\conf)/2,2);
          }

\begin{axis}[
  no markers, domain=-5:5, samples=100,
  axis lines=left, 
  every axis y label/.style={at=(current axis.above origin),anchor=south},
  every axis x label/.style={at=(current axis.right of origin),anchor=west},
  height=5cm, width=12cm,
  xtick={\crit}, ytick=\empty,
  xticklabels = { $k$},
  enlargelimits=false, clip=false, axis on top,
  %grid = major
  ]
  \addplot [fill=cyan!20, draw=none, domain=-5:\crit] {gauss(0,1)} \closedcycle;
  \addplot [very thick,cyan!50!black] {gauss(0,1)};

\end{axis}
\node[] at (5,1.5) {$P\left(Z \leq k \right) $};	
\end{tikzpicture}
    
\end{center}
\end{block}
\end{column}
\begin{column}{0.4\textwidth}
% \includegraphics[page=1,width=1\textwidth]{probabilidad/distribucion_normal}
\resizebox{0.9\textwidth}{!}{
\begin{tabular}{l|llllllllll}

z   & 0       & 0,01    & 0,02    & 0,03    & 0,04    & 0,05    & 0,06    & 0,07    & 0,08    & 0,09    \\
\hline
0   & 0,5     & 0,50399 & 0,50798 & 0,51197 & 0,51595 & 0,51994 & 0,52392 & 0,5279  & 0,53188 & 0,53586 \\
0,1 & 0,53983 & 0,5438  & 0,54776 & 0,55172 & 0,55567 & 0,55962 & 0,56356 & 0,56749 & 0,57142 & 0,57535 \\
0,2 & 0,57926 & 0,58317 & 0,58706 & 0,59095 & 0,59483 & 0,59871 & 0,60257 & 0,60642 & 0,61026 & 0,61409 \\
0,3 & 0,61791 & 0,62172 & 0,62552 & 0,6293  & 0,63307 & 0,63683 & 0,64058 & 0,64431 & 0,64803 & 0,65173 \\
0,4 & 0,65542 & 0,6591  & 0,66276 & 0,6664  & 0,67003 & 0,67364 & 0,67724 & 0,68082 & 0,68439 & 0,68793 \\
0,5 & 0,69146 & 0,69497 & 0,69847 & 0,70194 & 0,7054  & 0,70884 & 0,71226 & 0,71566 & 0,71904 & 0,7224  \\
0,6 & 0,72575 & 0,72907 & 0,73237 & 0,73565 & 0,73891 & 0,74215 & 0,74537 & 0,74857 & 0,75175 & 0,7549  \\
0,7 & 0,75804 & 0,76115 & 0,76424 & 0,7673  & 0,77035 & 0,77337 & 0,77637 & 0,77935 & 0,7823  & 0,78524 \\
0,8 & 0,78814 & 0,79103 & 0,79389 & 0,79673 & 0,79955 & 0,80234 & 0,80511 & 0,80785 & 0,81057 & 0,81327 \\
0,9 & 0,81594 & 0,81859 & 0,82121 & 0,82381 & 0,82639 & 0,82894 & 0,83147 & 0,83398 & 0,83646 & 0,83891 \\
1   & 0,84134 & 0,84375 & 0,84614 & 0,84849 & 0,85083 & 0,85314 & 0,85543 & 0,85769 & 0,85993 & 0,86214 \\
1,1 & 0,86433 & 0,8665  & 0,86864 & 0,87076 & 0,87286 & 0,87493 & 0,87698 & 0,879   & 0,881   & 0,88298 \\
1,2 & 0,88493 & 0,88686 & 0,88877 & 0,89065 & 0,89251 & 0,89435 & 0,89617 & 0,89796 & 0,89973 & 0,90147 \\
1,3 & 0,9032  & 0,9049  & 0,90658 & 0,90824 & 0,90988 & 0,91149 & 0,91309 & 0,91466 & 0,91621 & 0,91774 \\
1,4 & 0,91924 & 0,92073 & 0,9222  & 0,92364 & 0,92507 & 0,92647 & 0,92785 & 0,92922 & 0,93056 & 0,93189 \\
1,5 & 0,93319 & 0,93448 & 0,93574 & 0,93699 & 0,93822 & 0,93943 & 0,94062 & 0,94179 & 0,94295 & 0,94408 \\
1,6 & 0,9452  & 0,9463  & 0,94738 & 0,94845 & 0,9495  & 0,95053 & 0,95154 & 0,95254 & 0,95352 & 0,95449 \\
1,7 & 0,95543 & 0,95637 & 0,95728 & 0,95818 & 0,95907 & 0,95994 & 0,9608  & 0,96164 & 0,96246 & 0,96327 \\
1,8 & 0,96407 & 0,96485 & 0,96562 & 0,96638 & 0,96712 & 0,96784 & 0,96856 & 0,96926 & 0,96995 & 0,97062 \\
1,9 & 0,97128 & 0,97193 & 0,97257 & 0,9732  & 0,97381 & 0,97441 & 0,975   & 0,97558 & 0,97615 & 0,9767  \\
2   & 0,97725 & 0,97778 & 0,97831 & 0,97882 & 0,97932 & 0,97982 & 0,9803  & 0,98077 & 0,98124 & 0,98169 \\
2,1 & 0,98214 & 0,98257 & 0,983   & 0,98341 & 0,98382 & 0,98422 & 0,98461 & 0,985   & 0,98537 & 0,98574 \\
2,2 & 0,9861  & 0,98645 & 0,98679 & 0,98713 & 0,98745 & 0,98778 & 0,98809 & 0,9884  & 0,9887  & 0,98899 \\
2,3 & 0,98928 & 0,98956 & 0,98983 & 0,9901  & 0,99036 & 0,99061 & 0,99086 & 0,99111 & 0,99134 & 0,99158 \\
2,4 & 0,9918  & 0,99202 & 0,99224 & 0,99245 & 0,99266 & 0,99286 & 0,99305 & 0,99324 & 0,99343 & 0,99361 \\
2,5 & 0,99379 & 0,99396 & 0,99413 & 0,9943  & 0,99446 & 0,99461 & 0,99477 & 0,99492 & 0,99506 & 0,9952  \\
2,6 & 0,99534 & 0,99547 & 0,9956  & 0,99573 & 0,99585 & 0,99598 & 0,99609 & 0,99621 & 0,99632 & 0,99643 \\
2,7 & 0,99653 & 0,99664 & 0,99674 & 0,99683 & 0,99693 & 0,99702 & 0,99711 & 0,9972  & 0,99728 & 0,99736 \\
2,8 & 0,99744 & 0,99752 & 0,9976  & 0,99767 & 0,99774 & 0,99781 & 0,99788 & 0,99795 & 0,99801 & 0,99807 \\
2,9 & 0,99813 & 0,99819 & 0,99825 & 0,99831 & 0,99836 & 0,99841 & 0,99846 & 0,99851 & 0,99856 & 0,99861 \\
3   & 0,99865 & 0,99869 & 0,99874 & 0,99878 & 0,99882 & 0,99886 & 0,99889 & 0,99893 & 0,99896 & 0,999   \\
3,1 & 0,99903 & 0,99906 & 0,9991  & 0,99913 & 0,99916 & 0,99918 & 0,99921 & 0,99924 & 0,99926 & 0,99929 \\
3,2 & 0,99931 & 0,99934 & 0,99936 & 0,99938 & 0,9994  & 0,99942 & 0,99944 & 0,99946 & 0,99948 & 0,9995  \\
3,3 & 0,99952 & 0,99953 & 0,99955 & 0,99957 & 0,99958 & 0,9996  & 0,99961 & 0,99962 & 0,99964 & 0,99965 \\
3,4 & 0,99966 & 0,99968 & 0,99969 & 0,9997  & 0,99971 & 0,99972 & 0,99973 & 0,99974 & 0,99975 & 0,99976 \\
3,5 & 0,99977 & 0,99978 & 0,99978 & 0,99979 & 0,9998  & 0,99981 & 0,99981 & 0,99982 & 0,99983 & 0,99983 \\
3,6 & 0,99984 & 0,99985 & 0,99985 & 0,99986 & 0,99986 & 0,99987 & 0,99987 & 0,99988 & 0,99988 & 0,99989 \\
3,7 & 0,99989 & 0,9999  & 0,9999  & 0,9999  & 0,99991 & 0,99991 & 0,99992 & 0,99992 & 0,99992 & 0,99992 \\
3,8 & 0,99993 & 0,99993 & 0,99993 & 0,99994 & 0,99994 & 0,99994 & 0,99994 & 0,99995 & 0,99995 & 0,99995 \\
3,9 & 0,99995 & 0,99995 & 0,99996 & 0,99996 & 0,99996 & 0,99996 & 0,99996 & 0,99996 & 0,99997 & 0,99997 \\
4   & 0,99997 & 0,99997 & 0,99997 & 0,99997 & 0,99997 & 0,99997 & 0,99998 & 0,99998 & 0,99998 & 0,99998
\end{tabular}

}
\end{column}
\end{columns}
\end{frame}

\begin{frame}{Cálculo en la $Z \sim \mathcal{N}(0,\,1)$ o normal estándar}
% \includegraphics[page=1,width=0.8\textwidth]{probabilidad/calculonormal.png}
\resizebox{0.7\textwidth}{!}{
\begin{tabular}{l|llllllllll}

z   & 0       & 0,01    & 0,02    & 0,03    & 0,04    & 0,05    & 0,06    & 0,07    & 0,08    & 0,09    \\
\hline
0   & \textbf{0,5 }    & 0,50399 & 0,50798 & 0,51197 & 0,51595 & 0,51994 & 0,52392 & 0,5279  & 0,53188 & 0,53586 \\
0,1 & 0,53983 & 0,5438  & 0,54776 & 0,55172 & 0,55567 & 0,55962 & 0,56356 & 0,56749 & 0,57142 & 0,57535 \\
0,2 & 0,57926 & 0,58317 & 0,58706 & 0,59095 & 0,59483 & 0,59871 & 0,60257 & 0,60642 & 0,61026 & 0,61409 \\
0,3 & 0,61791 & 0,62172 & 0,62552 & 0,6293  & 0,63307 & 0,63683 & 0,64058 & 0,64431 & 0,64803 & 0,65173 \\
0,4 & 0,65542 & 0,6591  & 0,66276 & 0,6664  & 0,67003 & 0,67364 & 0,67724 & 0,68082 & 0,68439 & 0,68793 \\
0,5 & 0,69146 & 0,69497 & 0,69847 & 0,70194 & 0,7054  & 0,70884 & 0,71226 & 0,71566 & 0,71904 & 0,7224  \\
0,6 & 0,72575 & 0,72907 & 0,73237 & 0,73565 & 0,73891 & 0,74215 & 0,74537 & 0,74857 & 0,75175 & 0,7549  \\
0,7 & 0,75804 & 0,76115 & 0,76424 & 0,7673  & 0,77035 & 0,77337 & 0,77637 & 0,77935 & 0,7823  & 0,78524 \\
0,8 & 0,78814 & 0,79103 & 0,79389 & 0,79673 & 0,79955 & 0,80234 & 0,80511 & 0,80785 & 0,81057 & 0,81327 \\
0,9 & 0,81594 & 0,81859 & 0,82121 & 0,82381 & 0,82639 & 0,82894 & 0,83147 & 0,83398 & 0,83646 & 0,83891 \\
1   & 0,84134 & 0,84375 & 0,84614 & 0,84849 & 0,85083 & 0,85314 & 0,85543 & 0,85769 & 0,85993 & 0,86214 \\
1,1 & 0,86433 & 0,8665  & 0,86864 & 0,87076 & 0,87286 & 0,87493 & 0,87698 & 0,879   & 0,881   & 0,88298 \\
1,2 & 0,88493 & 0,88686 & 0,88877 & \textbf{0,89065} & 0,89251 & 0,89435 & 0,89617 & 0,89796 & 0,89973 & 0,90147 \\
1,3 & 0,9032  & 0,9049  & 0,90658 & 0,90824 & 0,90988 & 0,91149 & 0,91309 & 0,91466 & 0,91621 & 0,91774 \\

\end{tabular}

}

\begin{itemize} [<+->]
    \item $P\left(Z\leq 0 \right)=0.5 $. Ya que en este caso $k=0=0.00$ la suma del valor de la fila 0 con el valor de la columna 0 me da el valor de $k$, y la probabilidad asociada es $0.5$
    \item $P\left(Z\leq 1.23 \right)= 0.89065$. En este caso la probabilidad asociada a 1.23 se busca en la fila 1.2 y la columna 0.03 
\end{itemize}
\end{frame}



\begin{frame}{Cálculo en la $Z \sim \mathcal{N}(0,\,1)$ o normal estándar} 
\begin{block}{}
    Las probabilidades en conjuntos de valores de la distribución que no se puedan obtener directamente de la tabla se transformarán en operaciones con probabilidades que sí estén en la tabla:
\end{block}
Veamos algunos ejemplos:
\end{frame}


\begin{frame}{Cálculo en la $Z \sim \mathcal{N}(0,\,1)$ o normal estándar}
\begin{block}{Ejemplo}
$P\left(Z\geq 1.23 \right)= 1 - P\left(Z\leq 1.23 \right) = 1 - 0.89065= 0.10935$.
\end{block}
Basta fijarse en la relación que hay entre las áreas:
\begin{center}
   %%\input{img_tikz/tikz_normal_z21} 
\begin{tikzpicture}[scale=0.5]

\tikzmath{
			\conf = 0; \crit= 1.05; \a=round(1-\conf)/2,2);
          }


\begin{axis}[
  no markers, domain=-5:5, samples=100,
  axis lines=left, 
  every axis y label/.style={at=(current axis.above origin),anchor=south},
  every axis x label/.style={at=(current axis.right of origin),anchor=west},
  height=5cm, width=12cm,
  xtick={\crit}, ytick=\empty,
  xticklabels = { $k$},
  enlargelimits=false, clip=false, axis on top,
  %grid = major
  ]
  \addplot [fill=cyan!20, draw=none, domain=\crit:5] {gauss(0,1)} \closedcycle;
  \addplot [very thick,cyan!50!black] {gauss(0,1)};

\end{axis}
\node[] at (7,0.5) {$P\left(Z \geq k \right) $};	
\end{tikzpicture}
\end{center}

\begin{columns}
\begin{column}{0.5\textwidth}
 %\input{img_tikz/tikz_normal_z22}
\begin{tikzpicture}[scale=0.5]

\tikzmath{
			\conf = 0; \crit= 1.05; \a=round(1-\conf)/2,2);
          }

\begin{axis}[
  no markers, domain=-5:5, samples=100,
  axis lines=left, 
  every axis y label/.style={at=(current axis.above origin),anchor=south},
  every axis x label/.style={at=(current axis.right of origin),anchor=west},
  height=5cm, width=12cm,
  xtick={\crit}, ytick=\empty,
  xticklabels = { $k$},
  enlargelimits=false, clip=false, axis on top,
  %grid = major
  ]
  \addplot [fill=cyan!20, draw=none, domain=-5:5] {gauss(0,1)} \closedcycle;
  \addplot [very thick,cyan!50!black] {gauss(0,1)};

\end{axis}
\node[] at (5.25,1.5) {$1$};	
\end{tikzpicture}

\end{column}
\begin{column}{0.5\textwidth}
 %\input{img_tikz/tikz_normal_z23}
\begin{tikzpicture}[scale=0.5]

\tikzmath{
			\conf = 0; \crit= 1.05; \a=round(1-\conf)/2,2);
          }

\begin{axis}[
  no markers, domain=-5:5, samples=100,
  axis lines=left, 
  every axis y label/.style={at=(current axis.above origin),anchor=south},
  every axis x label/.style={at=(current axis.right of origin),anchor=west},
  height=5cm, width=12cm,
  xtick={\crit}, ytick=\empty,
  xticklabels = { $k$},
  enlargelimits=false, clip=false, axis on top,
  %grid = major
  ]
  \addplot [fill=cyan!20, draw=none, domain=-5:\crit] {gauss(0,1)} \closedcycle;
  \addplot [very thick,cyan!50!black] {gauss(0,1)};

\end{axis}
\node[] at (5,1.5) {$P\left(Z \leq k \right) $};	
\end{tikzpicture}
\end{column}
\end{columns}
\end{frame}

\begin{frame}{Cálculo en la $Z \sim \mathcal{N}(0,\,1)$ o normal estándar}
\begin{block}{Ejemplo}
$P\left(Z\leq -2.15 \right)=P\left(Z\geq 2.15 \right)=1-P\left(Z\leq 2.15 \right)=1-0.98422=0.01578$.
\end{block}
Basta fijarse en la relación que hay entre las áreas:
\begin{center}
   %\input{img_tikz/tikz_normal_z31}
\begin{tikzpicture}[scale=0.5]
\tikzmath{
			\conf = 0; \crit= 1.05; \a=round(1-\conf)/2,2);
          }

\begin{axis}[
  no markers, domain=-5:5, samples=100,
  axis lines=left, 
  every axis y label/.style={at=(current axis.above origin),anchor=south},
  every axis x label/.style={at=(current axis.right of origin),anchor=west},
  height=5cm, width=12cm,
  xtick={-\crit, 0,\crit}, ytick=\empty,
  xticklabels = {$-k$,$0$, $k$},
  enlargelimits=false, clip=false, axis on top,
  %grid = major
  ]
  \addplot [fill=cyan!20, draw=none, domain=-5:-\crit] {gauss(0,1)} \closedcycle;
  \addplot [fill=cyan!20, draw=none, domain=\crit:5] {gauss(0,1)} \closedcycle;
  \addplot [very thick,cyan!50!black] {gauss(0,1)};

\end{axis}
\node[] at (8,0.5) {$P\left(Z \geq k \right) $};
\node[] at (2.5,0.5) {$P\left(Z \leq -k \right) $};	
\end{tikzpicture}
 
\end{center}
\end{frame}

\begin{frame}{Cálculo en la $Z \sim \mathcal{N}(0,\,1)$ o normal estándar}
\begin{block}{Ejemplo}
$P\left( -1.3 < Z < 3.1\right)=P\left( Z < 3.1\right)-P\left(  Z < -1.3\right)=
    P\left( Z < 3.1\right) - \left[ 1 - P\left(  Z < 1.3\right) \right]= 0.99903 - 1 + 0.9032 = 0.90223$.
\end{block}
Basta fijarse en la relación que hay entre las áreas:
\begin{center}
   %\input{img_tikz/tikz_normal_z41} 
\begin{tikzpicture}[scale=0.5]

\tikzmath{
			\conf = 0.96; \crit= 2.05; \a=round(1-\conf)/2,2);
          }


\begin{axis}[
  no markers, domain=-5:5, samples=100,
  axis lines=left, 
  every axis y label/.style={at=(current axis.above origin),anchor=south},
  every axis x label/.style={at=(current axis.right of origin),anchor=west},
  height=5cm, width=12cm,
  xtick={-1,0,\crit}, ytick=\empty,
  xticklabels = {$a$, $0$, $b$},
  enlargelimits=false, clip=false, axis on top,
  %grid = major
  ]
  \addplot [fill=cyan!20, draw=none, domain=-1:\crit] {gauss(0,1)} \closedcycle;
  \addplot [very thick,cyan!50!black] {gauss(0,1)};
\end{axis}
\node[] at (5.5,1.5) {$P\left(a < Z < b \right) $};	
\end{tikzpicture}
\end{center}

\begin{columns}
\begin{column}{0.5\textwidth}
 %\input{img_tikz/tikz_normal_z42}
 \begin{tikzpicture}[scale=0.5]

\tikzmath{
			\conf = 0.96; \crit= 2.05; \a=round(1-\conf)/2,2);
          }

\begin{axis}[
  no markers, domain=-5:5, samples=100,
  axis lines=left, 
  every axis y label/.style={at=(current axis.above origin),anchor=south},
  every axis x label/.style={at=(current axis.right of origin),anchor=west},
  height=5cm, width=12cm,
  xtick={-1,0,\crit}, ytick=\empty,
  xticklabels = {$a$, $0$, $b$},
  enlargelimits=false, clip=false, axis on top,
  %grid = major
  ]
  \addplot [fill=cyan!20, draw=none, domain=-5:\crit] {gauss(0,1)} \closedcycle;
  \addplot [very thick,cyan!50!black] {gauss(0,1)};

\end{axis}
\node[] at (5.5,1.5) {$P\left(Z < b \right) $};	
\end{tikzpicture}

\end{column}
\begin{column}{0.5\textwidth}
 %\input{img_tikz/tikz_normal_z43}
\begin{tikzpicture}[scale=0.5]

\tikzmath{
			\conf = 0.96; \crit= 2.05; \a=round(1-\conf)/2,2);
          }

\begin{axis}[
  no markers, domain=-5:5, samples=100,
  axis lines=left, 
  every axis y label/.style={at=(current axis.above origin),anchor=south},
  every axis x label/.style={at=(current axis.right of origin),anchor=west},
  height=5cm, width=12cm,
  xtick={-1,0,\crit}, ytick=\empty,
  xticklabels = {$a$, $0$, $b$},
  enlargelimits=false, clip=false, axis on top,
  %grid = major
  ]
  \addplot [fill=cyan!20, draw=none, domain=-5:-1] {gauss(0,1)} \closedcycle;
  \addplot [very thick,cyan!50!black] {gauss(0,1)};
\end{axis}
\node[] at (3,1.5) {$P\left(Z < a \right) $};	
\end{tikzpicture}
\end{column}
\end{columns}
\end{frame}

\begin{frame}{Cálculo en la $Z \sim \mathcal{N}(0,\,1)$ o normal estándar}
\begin{block}{Cálculo del valor de la variable a partir de la probabilidad}
El uso de la tabla normal nos permite realizar el proceso inverso. Es decir, fijada una probabilidad $Pr$, encontrar el valor de la variable $k$ que cumpla:
$$P(Z<=k)=Pr$$
    \begin{center}
        %\input{img_tikz/tikz_normal_z5}
 \begin{tikzpicture}[scale=0.7]
\tikzmath{
			\conf = 0; \crit= 1.05; \a=round(1-\conf)/2,2);
          }

\begin{axis}[
  no markers, domain=-5:5, samples=100,
  axis lines=left, 
  every axis y label/.style={at=(current axis.above origin),anchor=south},
  every axis x label/.style={at=(current axis.right of origin),anchor=west},
  height=5cm, width=12cm,
  xtick={\crit}, ytick=\empty,
  xticklabels = { $k$},
  enlargelimits=false, clip=false, axis on top,
  %grid = major
  ]
  \addplot [fill=cyan!20, draw=none, domain=-5:\crit] {gauss(0,1)} \closedcycle;
  \addplot [very thick,cyan!50!black] {gauss(0,1)};

\end{axis}
\node[] at (5,1.5) {$Pr$};	
\end{tikzpicture}
    \end{center}
\end{block}

\end{frame}

\begin{frame}{Ejemplo}

Dada $Z \sim \mathcal{N}(0,\,1)$, calcula el valor de la variable sabiendo que la probabilidad de que tome un valor menor que ese es de un 85\%. 

$$P(Z<=k)=0.85$$
    \begin{center}
        %\input{img_tikz/tikz_normal_z6}
        \begin{tikzpicture}[scale=0.5]
\tikzmath{
			\conf = 0; \crit= 1.05; \a=round(1-\conf)/2,2);
          }

\begin{axis}[
  no markers, domain=-5:5, samples=100,
  axis lines=left, 
  xlabel=$ $, ylabel=$ $,
  every axis y label/.style={at=(current axis.above origin),anchor=south},
  every axis x label/.style={at=(current axis.right of origin),anchor=west},
  height=5cm, width=12cm,
  xtick={\crit}, ytick=\empty,
  xticklabels = { $k$},
  enlargelimits=false, clip=false, axis on top,
  %grid = major
  ]
  \addplot [fill=cyan!20, draw=none, domain=-5:\crit] {gauss(0,1)} \closedcycle;
  \addplot [very thick,cyan!50!black] {gauss(0,1)};
\end{axis}
\node[] at (5,1.5) {$0.85$};	
\end{tikzpicture}
    \end{center}

Vamos a la tabla y buscamos los dos valores seguidos de la tabla entre los que se quede el $0.85$ y encontramos:
$$0.84849 < 0.85 < 0.85083 $$
Como queda más cerca del $0.85083$, me quedo con la celda correspondiente a la fila $1$ y columna $0.04$  $\Rightarrow k=1.04$.
\end{frame}

\begin{frame}{Cálculo en $X \sim \mathcal{N}(\mu,\,\sigma)$ - Tipificación}
Para manejar $X \sim \mathcal{N}(\mu,\,\sigma)$ reduciremos los cálculos a cálculos en la $Z \sim \mathcal{N}(0,\,1)$ a partir de la siguiente propiedad:

\begin{block}{}
$$ Si \  X \sim \mathcal{N}(\mu,\,\sigma) \Rightarrow X^{'}=\frac{X - \mu}{\sigma} \sim \mathcal{N}(0,\,1) $$
\end{block}

\pause

Al proceso de transformar la variable anterior a una $Z\sim \mathcal{N}(0,\,1)$ se denomina \textbf{tipificarla}.


\end{frame}



\begin{frame}{Ejemplos}

\begin{itemize}[<+->]
    \item $X \sim \mathcal{N}(1,\,2) \to P(X<=2.32)=P(X'<=\frac{2.32-1}{2})=P(Z<=0.66)=0.74537$
    \item $X \sim \mathcal{N}(5,\,3) \to P(X<=3.59)=P(X'<=\frac{3.59-5}{3})=P(Z<=-0.47)=1-P(Z<=0.47)=1-0.68082=0.31918$
\end{itemize}
\end{frame}


\begin{frame}{Aproximación de la Binomial a partir de la Normal}

\begin{block}

\begin{columns}
\begin{column}{0.3\textwidth}
Observa que si: 
\begin{itemize}
\item $X \sim \mathcal{B}(n,\,p)$ \textbf{(Binomial)}
\item $n\cdot p \geq 5 \land n\cdot q \geq 5$
\end{itemize} 
\end{column}
\begin{column}{0.6\textwidth}
\begin{tikzpicture}[scale=0.6]


  \begin{axis}[ 
  	axis lines=left,
    width=12cm, height=8cm, 
    xlabel={$ $}, ylabel={$ $}, 
    %title={Distribución Normal $N(np, \sqrt{npq})$ y Binomial $B(n, p)$},
    ymin=0, ymax=0.3,
    xtick=\empty, % Quitar números del eje x
    ytick=\empty, % Quitar números del eje y
    axis line style= - , % Eliminar líneas de los ejes
    extra x ticks={6}, % Agregar el valor k en x=6
    extra x tick labels={k}, % Etiqueta k en x=6
    extra tick style={grid=major}, % Establecer el estilo de las marcas extra
  ]
  \addplot[red, thick, domain=0:0] coordinates {(0,0)};
      \addlegendentry{$X \sim \mathcal{B}(n,\,p)$}

  
     Graficar la distribución normal N(5, sqrt(2.5))
    \addplot[
      domain=-1:11, 
      samples=100, 
      thick, 
      blue, 
      unbounded coords=jump,
      opacity=0.7
    ]
    {1/(sqrt(2*pi*2.5))*exp(-((x-5)^2)/(2*2.5))};
    \addlegendentry{$X' \sim \mathcal{N}(np,\,\sqrt{npq})$}  

     %Sombrear la normal entre 5.5 y 6.5
    \addplot [fill=cyan!20, draw=none, domain=5.5:6.6] {gauss(5,sqrt(2.5))}\closedcycle;    
    
    % Graficar la distribución binomial Binomial(10, 0.5)
    \addplot[
      ycomb,
      ybar,
      red,
      fill opacity=0.5,
      %nodes near coords,
    ]
    plot coordinates {
      (0, 0.000977)
      (1, 0.009766)
      (2, 0.043945)
      (3, 0.117188)
      (4, 0.205078)
      (5, 0.246094)
      (6, 0.205078)
      (7, 0.117188)
      (8, 0.043945)
      (9, 0.009766)
      (10, 0.000977)
    };
   
  \end{axis}
\end{tikzpicture}
\end{column}
\end{columns}
La gráfica de $X$ \emph{se parece a} $X' \sim \mathcal{N}(np,\,\sqrt{npq})$ \textbf{(Normal)}
\end{block}

Por tanto podemos aproximar la probabilidad de la \textbf{Binomial ($X$)} a partir de la \textbf{Normal ($X'$)} siguiendo el siguiente criterio:


$$\boxed{P(X=k)\approx P(k-0.5<X'<k+0.5)}$$

\end{frame}

\begin{frame}{Ejemplo de aproximación de la binomial}
\begin{block}{Problema}
Se lanza una moneda 200 veces. Calcula:
\begin{itemize}
\item la probabilidad de que salgan exactamente 110 caras
\item la probabilidad de que salgan a lo sumo 110 caras
\end{itemize} 
\end{block}
El experimento sigue una distribución $X \sim B(200;\  0,5)$.

Además, $n\cdot p=200\cdot 0.5 =100 > 5$ y $n\cdot q=200\cdot 0.5 =100 > 5$. Por tanto $X \approx X'\sim N(100, \sqrt{200\cdot 0,5 \cdot 0,5}=N(100, 5t\sqrt{2})$.

\begin{itemize}
\item $P(X=110)=\begin{pmatrix} 200 \\ 110 \end{pmatrix}\cdot 0.5^{110} \cdot 0,5^{90}\approx 0.0207986943323103$
\item $P(109.5<X'<110.5)\approx 0.0207726494084092$
\end{itemize} 

Por lo que prácticamente es la misma cantidad, y por tanto una buena aproximación la que se realiza a través de la normal correspondiente.

\end{frame}

\begin{frame}{Ejemplo de aproximación de la binomial}
\begin{block}{Problema}
Se lanza una moneda 200 veces. Calcula:
\begin{itemize}
\item la probabilidad de que salgan exactamente 110 caras
\item la probabilidad de que salgan a lo sumo 110 caras
\end{itemize} 

\end{block}

La distribución es la misma que en el ejercicio anterior: $X \sim B(200;\  0,5)$

Si quisieramos calcular $P(X \leq 110)$ tendríamos que calcular 111 probabilidades y sumarlas, lo que nos llevaría un tiempo. En este caso podemos hacer la siguiente aproximación:

$$\boxed{P(X\leq 110) \approx P(X' < 110.5)\approx 0.931218053045048}$$


\textbf{Nota:} El añadir y/o quitar 0.5 al pasar a la normal, se conoce como corrección por continuidad o de Yates.

\end{frame}

\section{Inferencia Estadística}
\subsection{Estimación puntual}
\begin{frame}{Inferencia Estadística}
\begin{block}{Finalidad:} Obtener conclusiones válidas para toda la población a partir del estudio de una muestra.
\end{block}

\textbf{Ejemplo:} He preguntado la nota de matemáticas a 3 alumnos y la media de las notas es $6.4$. ¿Podemos extraer alguna conclusión sobre la nota media de la clase?¿Con qué grado de confianza?

\pause

\begin{block}{¿Cómo?:} Mediante los m\textbf{étodos de estimación puntual} y de \textbf{intervalos de confianza}
\end{block}

\end{frame}



\begin{frame}{Estimación Puntual de la media y la varianza}
Dada una población de media $\mu$ y desviación típica $\sigma$
\begin{block}{Estimación de la media}
 Un buen estimador de $\mu$ es la \textbf{media muestral} $\overline{x}$:
$$\overline{x}= \frac{x_1 + x_2 + ....+x_n} {n}=\frac{{\sum_{i=1}^n x_i }}{n}$$
\end{block}
\pause
\begin{block}{Estimación de la varianza}
Un buen estimador de $\sigma^2$ es la cuasivarianza muestral:
$$\widehat{\sigma}^2=\frac{n}{n-1}s^2$$
siendo $s^2$ la varianza muestral:
$$s^2=\frac{\sum_{i=1}^n x_i^2 }{n} - \bar x^2$$
\end{block}
\end{frame}

\begin{frame}{Ejemplo}
   
    \begin{block}{Parámetros de una muestra} Se ha extraído una muestra de 10 alumnos de los alumnos de 2º Bachillerato del instituto. Se les ha preguntado por las notas de matemáticas y estas han sido:  5,7,6,5,10,10,6,8,7,3.
    
    Determina la media y la varianza de los datos muestrales
    
    \end{block}
\begin{itemize}
\item $\overline{x}=\dfrac{\sum x_i}{n}=\dfrac{5+7+6+5+10+10+6+8+7+3}{10}=\dfrac{67}{10}=6.7$
\item $var=\dfrac{\sum (x_i - \overline{x})^2}{n}=\dfrac{\sum x_i^2}{n}-\overline{x}^2=\dfrac{493}{10}-6.7^2\approx4.41$
\end{itemize}    
    
\end{frame}

\begin{frame}{Ejemplo}
\begin{block}{}
Una muestra aleatoria de 36 personas, empleadas en una gran industria, da el número medio de días al año que faltan al trabajo es $\overline{x} = 12$ con $s^2 = 4$
\end{block}

\begin{itemize}[<+->]
    \item Dar una estimación puntual de $\mu$ (media poblacional)
    \pause 
    \\ Un estimador es la media muestral que en este caso vale 12
    \item Dar una estimación puntual de $\sigma^2$ (varianza poblacional)
    \pause 
    \\ Un estimador es la cuasivarianza muestral:
    $$\widehat{\sigma}^2=\frac{n}{n-1}s^2=\frac{36}{35}\cdot 4\approx 4.1$$
\end{itemize}

\end{frame}


\begin{frame}{Ejemplo}
\begin{itemize}[<+->]
    \item  Se ha seleccionado una muestra al azar de 50 mujeres de una población de mayores
de 18 años. En la muestra se ha observado que la media de las 50 tallas es 1,60 m. Si se sabe que
la desviación típica en la población es de 3,3 cm, determina la probabilidad de que la
media de la población no difiere en más de 1 cm de la de la muestra. 
    % \item Como $\overline{X} \approx N\left(\mu,\frac{\sigma}{\sqrt{n}}\right)$, tipificando $\frac{\overline{X} -\mu}{\frac{\sigma}{\sqrt{n}}} \rightarrow Z(0,1)$ y por tanto: $P(\mu-1 < \overline{X} < \mu +1)=P(\frac{-1}{\frac{\sigma}{\sqrt{n}}} < Z < \frac{1}{\frac{\sigma}{\sqrt{n}}})\approx 0.967866646858422$
    \item $\left( \overline{x} - z_{\alpha / 2}\cdot \frac{\sigma}{\sqrt{n}} ,  \overline{x} + z_{\alpha / 2}\cdot \frac{\sigma}{\sqrt{n}} \right)$ es un intervalo para $\mu$ con un grado de confianza de $1-\alpha \to E=z_{\alpha / 2}\cdot \frac{\sigma}{\sqrt{n}}$ es el error máximo cometido $\to 1=z_{\alpha / 2}\cdot \frac{\sigma}{\sqrt{n}} \to z_{\alpha / 2}=\frac{\sqrt{n}}{\sigma}=\frac{\sqrt{50}}{3,3} \to  \alpha / 2=P(Z>\frac{\sqrt{50}}{3,3}) \to$ el grado de confianza es  $1-2\cdot P(Z>\frac{\sqrt{50}}{3,3})\approx 0.967866646858422$
    
\end{itemize}
\end{frame}



\subsection{Estimación por intervalos de confianza}

\begin{frame}
{Estimación de la media por intervalo de confianza}
A partir de una muestra de tamaño n y un grado de confianza $1-\alpha$: 
\begin{block}{Intervalo de confianza para la media}
$$ \left( \overline{x} - z_{\alpha / 2}\cdot \frac{\sigma}{\sqrt{n}} ,  \overline{x} + z_{\alpha / 2}\cdot \frac{\sigma}{\sqrt{n}}
\right)$$
siendo: \begin{itemize}
\item $\overline{x}$: La media de los datos de la muestra
\item $z_{\alpha / 2}$ o valor crítico: El valor de la distribución normal  $Z\leadsto N\left((0,1\right)$ tal que $P(Z>z_{\alpha / 2})=\frac{\alpha}{2}$ ($\alpha$ es el nivel de significación y $1-\alpha$ el grado de confianza)
\item $\sigma$: La desviación típica de la distribución de la población (o si no se conoce de un estimador sesgado de la misma)
\item $n$: El tamaño de la muestra
\end{itemize}
\end{block}
\pause

\end{frame}

\begin{frame}
{Ejemplo}
\begin{block}{Problema} La duración de las bombillas de un fabricante es una variable aleatoria con distribución
normal de desviación típica 75 horas. Decidimos tomar un tamaño de la muestra igual a 150, comprobamos la duración de cada
bombilla y calculamos su promedio, que resulta ser 1053 horas. Calcular el intervalo de confianza al 98\%
para la media de la duración de las bombillas del fabricante.
\end{block}
\end{frame}

\begin{frame}{Ejemplo: Solución}
\begin{columns}
\begin{column}{0.5\textwidth}
    \textbf{Datos:} $\overline{x}=1053$, Confianza=$98$\%, $\sigma=75$ y n=150 \\
    $\alpha=1-0.98=0.02 \to \frac{\alpha}{2}=\frac{0.02}{2}=0.01$
    . \\ Por tanto, el valor crítico será: \\
    $P\left(Z \leqslant z_{\alpha / 2} \right)= 0.98 + 0.01 = 0.99 \to z_{\alpha / 2} = 2.33$ \\
    A partir de la definición de intervalo de confianza de la media:
    $$ \left( \overline{x} - z_{\alpha / 2}\cdot \frac{\sigma}{\sqrt{n}} ,  \overline{x} + z_{\alpha / 2}\cdot \frac{\sigma}{\sqrt{n}}
    \right)$$

\end{column}
\begin{column}{0.5\textwidth}
    % \input{img_tikz/tikz_intervalo_2}
    \begin{tikzpicture}[scale=0.5]

\tikzmath{
			\conf = 0.98; \crit= 2.33; \a=(1-\conf)/2);
          }

\begin{axis}[
  no markers, domain=-5:5, samples=100,
  axis lines=left, 
  height=5cm, width=12cm,
  xtick={0,\crit}, ytick=\empty,
  xticklabels = {$0$, $z_{\frac{\alpha}{2}}=\crit$},
  enlargelimits=false, clip=false, axis on top,
  %grid = major
  ]
  \addplot [fill=cyan!20, draw=none, domain=-\crit:\crit] {gauss(0,1)} \closedcycle;
  \addplot [very thick,cyan!50!black] {gauss(0,1)};
\end{axis}
\node[] at (5.2,1.5) {$\conf$};	
\draw[->]   (\crit+6.5,1)node[right]{$\a$}  --  (\crit+5.6,0.1) ;
\end{tikzpicture}
\end{column}
\end{columns}
    
    Operando nos queda:
    $$ \left( 1053 - 2.33 \cdot \frac{75}{\sqrt{150}} ,  1053 + 2.33 \cdot \frac{75}{\sqrt{150}}
    \right)=\left(1038.73 ,  1067.27\right)
    $$
\end{frame}


\begin{frame}{Error máximo del intervalo de la media}
\begin{center}
        \begin{tikzpicture}[scale=0.4]
        
        \tikzmath{
        			\a = -10; \b = 10; \aa = \a -1; \bb = \b + 1 ;
        			\dist = \b - \a; \med = (\a + \b)/2;
                  }
        
        \draw[very thick] (\a,0) -- (\b,0);
        \path [draw=black, fill=white] (\b,0) circle (2pt);
        \path [draw=black, fill=white] (\a,0.0) circle (2pt);
        \draw[latex-latex] (\a - 1.5,0) -- (\b + 1.5,0) ;
        
        % \foreach \x in  {\a,...,\b}
        % \draw[shift={(\x,0)},color=black] (0pt,3pt) -- (0pt,-3pt);
        % \foreach \x in  {\aa,...,\bb}
        % \draw[shift={(\x,0)},color=black] (0pt,0pt) -- (0pt,-3pt) node[below] % {$\pgfmathprintnumber{\x}$};
        \draw[shift={(\a,0)},color=black] (0pt,3pt) -- (0pt,-3pt);
        \draw[shift={(\a,0)},color=black] (0pt,0pt) -- (0pt,-3pt) node[below] {$\overline{x} - z_{\alpha / 2}\cdot \frac{\sigma}{\sqrt{n}}$};
        \draw[shift={(\med,0)},color=black] (0pt,3pt) -- (0pt,-3pt);
        \draw[shift={(\med,0)},color=black] (0pt,0pt) -- (0pt,-3pt) node[below] {$\overline{x}$};
        \draw[shift={(\b,0)},color=black] (0pt,3pt) -- (0pt,-3pt);
        \draw[shift={(\b,0)},color=black] (0pt,0pt) -- (0pt,-3pt) node[below] {$\overline{x} + z_{\alpha / 2}\cdot \frac{\sigma}{\sqrt{n}}$};
        
          \draw[decorate,decoration={brace}, thick]
            (\med,0.2)--(\b,0.2) node[above, midway] 
        {$E=z_{\alpha / 2}\cdot \frac{\sigma}{\sqrt{n}}$}; 
        \end{tikzpicture}
\end{center}

\begin{block}{} $$E=z_{\alpha / 2}\cdot \frac{\sigma}{\sqrt{n}}$$
Es el radio del entorno dado por el intervalo de confianza.
\end{block}

\textbf{NOTA:} Disminuye al aumentar el tamaño de la muestra y por tanto si se quiere garantizar un error determinado para un nivel de confianza habrá que tomar muestras de al menos un \textbf{tamaño determinado de la muestra}:.

\begin{block}{} $$E=z_{\alpha / 2}\cdot \frac{\sigma}{\sqrt{n}} \Rightarrow n =\frac{\sigma^2 \cdot z_{\alpha / 2}^2}{E^2}$$

\end{block}


\end{frame}

\begin{frame}{Ejemplo}


\begin{block}{}
Se desea realizar una investigación para estimar el peso medio de los recién nacidos de madres fumadoras. Se admite un error máximo de 50 gramos, con una confianza del 95\%. Si por estudios anteriores se sabe que la desviación típica del peso medio de tales recién nacidos es de 400 gramos, ¿qué tamaño mínimo de muestra se necesita en la investigación?
\end{block}
    
\end{frame}{}


\begin{frame}{Ejemplo: Solución}

    \textbf{Datos:} $E=50$, Confianza=$95$ y $\sigma=400$ \\
$\alpha=1-0.95=0.05 \to \frac{\alpha}{2}=\frac{0.05}{2}=0.025$
. \\ Por tanto, el valor crítico será: \\
$P\left(Z \leqslant z_{\alpha / 2} \right)= 0.95 + 0.025 = 0.975 \to z_{\alpha / 2} = 1.96$\\ A partir de la definición de error máximo admitido:
$$E=z_{\alpha / 2}\cdot \frac{\sigma}{\sqrt{n}} \to 
n = \left( \frac{z_{\alpha / 2} \cdot \sigma}{E} \right) ^ 2$$
Luego: \\
$$n = \left( \frac{1.96 \cdot 400}{50} \right) ^ 2\approx 245.8534 \to n=246
$$
\end{frame}

\begin{frame}{Estimación de la proporción por intervalo de confianza}
A partir de una muestra de tamaño n y un grado de confianza $1-\alpha$: 
\begin{block}{Intervalo de confianza para la proporción}
$$ \left( \widehat{p} - z_{\alpha / 2}\cdot \sqrt{\frac{\widehat{p}\cdot\left(1-\overline{p} \right)}{n}} ,  \widehat{p} + z_{\alpha / 2}\cdot \sqrt{\frac{\widehat{p}\cdot\left(1-\widehat{p} \right)}{n}}\right)$$
siendo: \begin{itemize}
\item $\widehat{p}$: La proporción muestral
\item $z_{\alpha / 2}$ o valor crítico:  $P(Z>z_{\alpha / 2})=\frac{\alpha}{2}$ 
\item $\sigma$: La desviación típica de la distribución de la población \item $n$: El tamaño de la muestra
\end{itemize}
\end{block}
Llamaremos error máximo a:
\begin{block}{}
$$E=z_{\alpha / 2}\cdot \sqrt{\frac{\widehat{p}\cdot\left(1-\widehat{p} \right)}{n}}$$
\end{block}

\end{frame}

\begin{frame}
{Ejemplo}
\begin{block}{Problema} Para estimar la proporción de personas con sobrepeso en una población se ha tomado una
muestra aleatoria simple de tamaño 100 personas, de las cuales 21 tienen sobrepeso. Calcular el intervalo
de confianza al 96\% para la proporción de personas con sobrepeso en la población.
\end{block}
\end{frame}

\begin{frame}{Ejemplo: Solución}
\begin{columns}
\begin{column}{0.55\textwidth}
    \textbf{Datos:} $\widehat{p}=0.21$, Confianza=$96$ \\
$\alpha=1-0.96=0.04 \to \frac{\alpha}{2}=\frac{0.04}{2}=0.02$
. \\ Por tanto, el valor crítico será: \\

$P\left(Z \leqslant z_{\alpha / 2} \right)= 0.96 + 0.02 = 0.98 \to z_{\alpha / 2} \approx 2.05$ \\ 
A partir de la definición de error máximo admitido:
$$E=z_{\alpha / 2}\cdot \sqrt{\frac{\widehat{p}\cdot\left(1-\widehat{p} \right)}{n}}\approx 2.05 \cdot \sqrt{\frac{0.21\cdot 0.79}{100}}\approx 0.08$$

\end{column}
\begin{column}{0.45\textwidth}
    % \input{img_tikz/tikz_intervalo_1}

    \begin{tikzpicture}[scale=0.45]

\tikzmath{
			\conf = 0.96; \crit= 2.05; \a=round((1-\conf)/2),2);
          }

\begin{axis}[
  no markers, domain=-5:5, samples=100,
  axis lines=left, 
  %xlabel=$xa$, ylabel=$ya$,
  %every axis y label/.style={at=(current axis.above origin),anchor=south},
  %every axis x label/.style={at=(current axis.right of origin),anchor=west},
  height=5cm, width=12cm,
  xtick={0,\crit}, ytick=\empty,
  xticklabels = {$0$, $z_{\frac{\alpha}{2}}=\crit$},
  enlargelimits=false, clip=false, axis on top,
  %grid = major
  ]
  \addplot [fill=cyan!20, draw=none, domain=-\crit:\crit] {gauss(0,1)} \closedcycle;
  \addplot [very thick,cyan!50!black] {gauss(0,1)};

\end{axis}
\node[] at (5.2,1.5) {$\conf$};	
\draw[->]   (\crit+6.5,1)node[right]{$\a$}  --  (\crit+5.6,0.1) ;



\end{tikzpicture}
\end{column}
\end{columns}
Luego el intervalo es: $$\left( 0.21 - 0.08 , 0.21 + 0,08 \right) = \left(0.13, 0.29 \right)$$ 
\end{frame}



\end{document}
