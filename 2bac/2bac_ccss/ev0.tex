\documentclass{exam}
\usepackage{amsmath, amsthm, amssymb, yhmath} 
\printanswers

\usepackage{tikz}
\usetikzlibrary{angles,quotes} % Para los ángulos

\begin{document}

\begin{center}
\bfseries Prueba inicial
\end{center}
\textbf{Nombre:} \\
\textbf{Fecha:} \\
\textbf{Curso:} \\
\hline

\begin{questions}

\question Dadas las matrices:
\[
A = \begin{pmatrix} 
1 & 2 & 3 \\ 
0 & 1 & 4 \\ 
5 & 6 & 0 
\end{pmatrix}, \quad
B = \begin{pmatrix} 
7 & 8 & 9 \\ 
0 & 1 & 2 \\ 
3 & 4 & 5 
\end{pmatrix}
\]
Calcula el producto \( A \times B \) y verifica si el producto es conmutativo, es decir, ¿es cierto que \( A \times B = B \times A \)?
\begin{solution}
    $A\cdot B = \left[\begin{matrix}16 & 22 & 28\\12 & 17 & 22\\35 & 46 & 57\end{matrix}\right]$ y $B \cdot A= \left[\begin{matrix}52 & 76 & 53\\10 & 13 & 4\\28 & 40 & 25\end{matrix}\right]$
\end{solution}

\vspace{250pt}

\question Sea la matriz
$
A = \begin{pmatrix} 0 & 1 \\ 1 & 0 \end{pmatrix}
$. Encuentra una fórmula general para \( A^n \), donde \( n \) es un número entero positivo. Calcula \( A^2 \), \( A^3 \) y \( A^4 \) como ejemplos.

\begin{solution}

Sea la matriz \( A \):

\[
A = \begin{pmatrix} 0 & 1 \\ 1 & 0 \end{pmatrix}
\]

Calculamos las primeras potencias de \( A \) para detectar patrones:

\[
A^2 = A \times A = \begin{pmatrix} 0 & 1 \\ 1 & 0 \end{pmatrix} \times \begin{pmatrix} 0 & 1 \\ 1 & 0 \end{pmatrix} = \begin{pmatrix} 1 & 0 \\ 0 & 1 \end{pmatrix} = I
\]

\[
A^3 = A^2 \times A = I \times A = A = \begin{pmatrix} 0 & 1 \\ 1 & 0 \end{pmatrix}
\]

\[
A^4 = A^3 \times A = A \times A = I = \begin{pmatrix} 1 & 0 \\ 0 & 1 \end{pmatrix}
\]

Notamos que las potencias de \( A \) alternan entre \( A \) y la matriz identidad \( I \). Esto nos lleva a la siguiente fórmula general:

\[
A^n = \begin{cases} 
I, & \text{si } n \text{ es par} \\
A, & \text{si } n \text{ es impar} 
\end{cases}
\]

Por lo tanto, la potencia de la matriz \( A \) sigue un ciclo de longitud 2:
- Si \( n \) es par, \( A^n = I \).
- Si \( n \) es impar, \( A^n = A \).

\end{solution}

\newpage


\question Calcula la inversa de la siguiente matriz \(D\):

\[
D = \begin{pmatrix}
1 & 2 & 1 \\
2 & 3 & 1 \\
1 & 1 & 1
\end{pmatrix}
\]

\begin{solution}
    \[
D^{-1} = \begin{pmatrix}
-2 & 1 & 1 \\
1 & 0 & -1 \\
1 & -1 & 1
\end{pmatrix}
\]
\end{solution}


\end{questions}

\end{document}
