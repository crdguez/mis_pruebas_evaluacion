\documentclass[addpoints,spanish, 12pt,a4paper]{exam}
%\documentclass[answers, spanish, 12pt,a4paper]{exam}
\printanswers
\pointpoints{punto}{puntos}
\hpword{Puntos:}
\vpword{Puntos:}
\htword{Total}
\vtword{Total}
\hsword{Resultado:}
\hqword{Ejercicio:}
\vqword{Ejercicio:}

\usepackage[utf8]{inputenc}
\usepackage[spanish]{babel}
\usepackage{eurosym}
%\usepackage[spanish,es-lcroman, es-tabla, es-noshorthands]{babel}


\usepackage[margin=1in]{geometry}
\usepackage{amsmath,amssymb}
\usepackage{multicol}
\usepackage{yhmath}

\pointsinrightmargin % Para poner las puntuaciones a la derecha. Se puede cambiar. Si se comenta, sale a la izquierda.
\extrawidth{-2.4cm} %Un poquito más de margen por si ponemos textos largos.
\marginpointname{ \emph{\points}}

\usepackage{graphicx}

\graphicspath{{../img/}} 

\newcommand{\class}{2º Bachillerato CCSS}
\newcommand{\examdate}{\today}
\newcommand{\examnum}{Extraordinario}
\newcommand{\tipo}{A}


\newcommand{\timelimit}{90 minutos}

\renewcommand{\solutiontitle}{\noindent\textbf{Solución:}\enspace}


\pagestyle{head}
\firstpageheader{\includegraphics[width=0.2\columnwidth]{header_left}}{\textbf{Departamento de Matemáticas\linebreak \class}\linebreak \examnum}{\includegraphics[width=0.1\columnwidth]{header_right}}
\runningheader{\class}{\examnum}{Página \thepage\ of \numpages}
\runningheadrule


\usepackage{pgf,tikz,pgfplots}
\pgfplotsset{compat=1.15}
\usepackage{mathrsfs}
\usetikzlibrary{arrows}


\begin{document}

\noindent
\begin{tabular*}{\textwidth}{l @{\extracolsep{\fill}} r @{\extracolsep{6pt}} }
\textbf{Nombre:} \makebox[3.5in]{\hrulefill} & \textbf{Fecha:}\makebox[1in]{\hrulefill} \\
 & \\
\textbf{Tiempo: \timelimit} & Tipo: \tipo 
\end{tabular*}
\rule[2ex]{\textwidth}{2pt}
Esta prueba tiene \numquestions\ ejercicios. La puntuación máxima es de \numpoints. 
La nota final de la prueba será la parte proporcional de la puntuación obtenida sobre la puntuación máxima. 

\begin{center}


\addpoints
 %\gradetable[h][questions]
	\pointtable[h][questions]
\end{center}

\noindent
\rule[2ex]{\textwidth}{2pt}

\begin{questions}

\question Dado  el sistema lineal de ecuaciones:

$$\left\{\begin{matrix}
    x + ay + z = 1 \\
    2y + az = 2 \\
    x+y+z=1
\end{matrix}\right.$$
\begin{parts}
    \part[2] Discute el tipo de sistema en función de los valores de $a$
    \begin{solution}
        $\left(\begin{matrix}1 & a & 1 & 1\\0 & 2 & a & 2\\0 & 0 & - a \left(\frac{1}{2} - \frac{a}{2}\right) & a - 1\end{matrix}\right)$\\
        Si $a=0 \to S.I$ \\
        Si $a=1 \to S.C.I$ \\
        Para el resto de casos $S.C.D$
        
    \end{solution}
    \part[2] Resuelve el sistema para $a=3$
\end{parts}


\question[4] La repoblación forestal de un bosque quemado en un gran incendio se va
a llevar a cabo por dos empresas diferentes de jardinería. Hay que repoblar con pinos, eucaliptos
y chopos. La primera empresa es capaz de plantar, en una semana, 30 pinos, 20 eucaliptos y 20
chopos. La segunda empresa planta 20 pinos, 30 eucaliptos y 20 chopos. El coste semanal se
estima en 33.000€ para la primera empresa de jardinería y de 35.000€ para la segunda. Se necesita
plantar un mínimo de 60 pinos, 120 eucaliptos y 100 chopos. ¿Cuántas semanas deberá trabajar
cada grupo para finalizar el proyecto con el mínimo coste?
\begin{solution}
    
\end{solution}

\question Sea $C(x)=100+140x+x^2$ el coste total, en euros, de producir $x$ unidades de un
producto y cada unidad del producto se vende a $(400-2x)$ euros.
\begin{parts}
    \part[2] ¿Cuántas unidades deben venderse para que el beneficio (ingresos menos costes) sea
máximo?, ¿a cuánto asciende dicho beneficio máximo?
    \begin{solution}
        $B=- 3 x^{2} + 260 x - 100$ \\
        $B'(x)=260-2x \to x=130/3 \to x \approx 43 \to b=5533$
    \end{solution}
    \part[2] ¿Cuántas unidades hay que producir para minimizar el coste medio $CM(x)=\dfrac{C(x)}{x}$?
Obtén dicho valor mínimo.
\begin{solution}
    $\frac{x^{2} + 140 x + 100}{x} \to \frac{2 x + 140}{x} - \frac{x^{2} + 140 x + 100}{x^{2}}=0 \to 1 - \\frac{100}{x^{2}}=0 \to x= 10 \to c = 160$
\end{solution}
\end{parts}

\question Sea la función $f(x)=x^3+ax^2x+bx+1$:
\begin{parts}
    \part[2] Determinar los valores de a y b de forma que la función tenga un extremo relativo en $x=1$
 y la recta tangente a la gráfica de la función
$f(x)$ en el punto de abscisa $x=0$ tenga de
pendiente $m=-1$
    \part[2] Si en la función anterior $a=2$ y $b=4$, determinar los intervalos de crecimiento y decrecimiento de la función, así como sus extremos relativos
\end{parts}

\question[4] Dado dos sucesos independientes
A y B se conoce que $P(A)=0,3$ y $P(B)=0,4$ calcula:
\begin{parts}
    \part $P(A \cup B)$
    \part $P(\overline{A} \cap \overline{B})$
    \part $P(A|\overline{B})$
\end{parts}

% septiembre de 2014
\question Se desea estimar la proporción de individuos con sobrepeso en una
población. Para ello se va a tomar una muestra aleatoria simple y se va a determinar, de cada individuo, si
tiene sobrepeso o no, y a partir de los resultados se construirá un intervalo de confianza para la proporción
de individuos con sobrepeso en la población. El intervalo se hará a un nivel de confianza del 96\%.
\begin{parts}
    \part[2] Si queremos que el intervalo no tenga una amplitud mayor que 0,1 ¿qué tamaño de la
muestra debemos escoger?
    \begin{solution}
        421 individuos
    \end{solution}
    \part[2] Decidimos tomar una muestra de tamaño 200 individuos, de los cuales 40 tienen sobrepeso.
Calcular el intervalo de confianza al 96\% para la proporción de individuos con sobrepeso en la población.
    \begin{solution}
        (0,142 ; 0,25)
    \end{solution}
\end{parts}
% \question[4] Una empresa ha invertido 12 millones de euros en tres fondos
% A, B y C. Entre los fondos A y B ha invertido el triple que en el fondo C. Si
% hubiese invertido un millón menos en el fondo A y otro millón menos en el
% fondo B (invirtiendo en ese caso dos millones más en el fondo C), entonces
% habría invertido entre el fondo B y el C lo mismo que en el fondo A. Plantear y
% resolver un sistema de ecuaciones para determinar cuánto dinero ha invertido
% la empresa en cada fondo.
% \begin{solution}
%     $\left\{
%     \begin{matrix}
%         x+y+z=12 \\
%         x+y=3z \\
%         x-1=y-1+z+2
%     \end{matrix}
%     \right. \to x=7, y=2, z=3$
% \end{solution}

% \question[4] Dadas las matrices $A=\left(\begin{matrix}x & y\\2 & x\end{matrix}\right)$ y 
% % A, B = Matrix(2,2,[x,y,2,x]), Matrix(2,2,[-1, 5/2, 2, -1])
% $B=\left(\begin{matrix}-1 & \frac{5}{2}\\2 & -1\end{matrix}\right)
% $:
% \begin{parts}
%     \part Encontrar, si existen, $x$ e $y$ de forma que se cumpla $$AB = A + B$$
%     \part Encontrar, si existe, la matriz inversa de B
% \end{parts}
% \begin{solution}
%     $$\left[\begin{matrix}- 2 x + 2 y + 1 & 2.5 x - 2 y - 2.5\\2 x - 6 & 6.0 - 2 x\end{matrix}\right]\to x=3, y=\frac{5}{2}
% $$
% $$\left[\begin{matrix}0.25 & \frac{5}{8}\\0.5 & 0.25\end{matrix}\right]
% $$
% \end{solution}

% % \question[4] Dada la función $$f(x)=\dfrac{2x^2}{9-x^2}$$
% % \begin{parts}
% %     \part El dominio de la función y los puntos de corte con los ejes
% %     \part Asíntotas verticales y horizontales
% %     \part Intervalos de crecimiento y decrecimiento
% %     \part Máximos y mínimos locales
% % \end{parts}

% \question[4] Dada la función $$f(x)=\dfrac{1}{x+1}-\dfrac{8}{2x+7}$$
% \begin{parts}
%     \part Calcular $$\lim_{x\to + \infty} f(x)$$
%     \part ¿Cuál es el mínimo valor que toma la función en el intervalo $\left[0,4\right]$
%     \part Calcular:
%     $$\int_{0}^{4}f(x)\ dx$$
% \end{parts}
% \begin{solution}
%     $0$, $\frac{- 6 x - 1}{\left(x + 1\right) \left(2 x + 7\right)}$, $f(4)=-\frac{1}{3}$, $\log{\left(x + 1 \right)} - 4 \log{\left(x + \frac{7}{2} \right)} \to - 4 \log{\left(\frac{15}{2} \right)} + \log{\left(5 \right)} + 4 \log{\left(\frac{7}{2} \right)}= \log{\left(\frac{2401}{10125} \right)}$
% \end{solution}

% % \question[4] Sea $C(x) = 100 + 140 x + x^2$
% % el coste total, en euros, de producir $x$ unidades de un
% % producto y cada unidad del producto se vende a $(400 - 12x)$ euros.
% % \begin{parts}
% %     \part ¿Cuántas unidades deben venderse para que el beneficio sea
% % máximo?, ¿a cuánto asciende dicho beneficio máximo?
% % \begin{solution}
% % % C= 100+140*x+x**2
% % % B=x*(400-12*x)
% %     $- 13 x^{2} + 260 x - 100 \to x=10$, $260 - 26 x$
% % \end{solution}
% %     \part  ¿Cuántas unidades hay que producir para minimizar el coste medio $CM(x)=\dfrac{C(x)}{x}$? Obtén dicho valor mínimo.
% %     \begin{solution}
% %         $ \frac{x^{2} + 140 x + 100}{x} \to  \frac{x^{2} + 140 x + 100}{x} \to \frac{200}{x^3} \to x=10$
% %     \end{solution}

% % \end{parts}

% \question Las existencias en el almacén de una empresa (en miles de kilogramos), durante un turno de 8 horas de cierto día, vienen dadas por la función:
% $E(x)=\left\{\begin{matrix}
%      -2x^2+16x+10 & si \ 0\leq x \leq 3 \\
%     -4x+52 & si \ 3<x\leq 8   
% \end{matrix} \right.$
% \begin{parts}
%     \part[1] Justifique si la función $E(x)$ es continua a lo largo de la jornada laboral.
%     \part[1] Justifique si a partir de la tercera hora de trabajo, cada hora que pasa, las existencias
% aumentan o disminuyen y en qué cantidad (por hora).
%     \part[1] Calcule los kilogramos de mercancía en el almacén en el momento de máxima existencia y
% en qué momento de la jornada se alcanzan.
%     \part[1] ¿En qué intervalo de la jornada laboral las existencias son mayores a 24.000 kilogramos?
% \end{parts}

% % \question[4] El ayuntamiento de una ciudad realiza una encuesta para estimar
% % la proporción de ciudadanos que están a favor de la apertura de un nuevo
% % centro comercial. Para ello, pregunta a 400 ciudadanos de los cuales 168
% % afirman estar a favor de la apertura del centro. Calcular el intervalo de
% % confianza al 98\% para la proporción de ciudadanos a favor de la apertura del
% % centro comercial.

% % \question[4] Una familia gastó 60 euros en la compra de un pack de mascarillas quirúrgicas, un pack
% % de mascarillas ffp2 y un pack de botes de gel hidroalcohólico. El pack de botes de gel costó la mitad
% % que las mascarillas quirúrgicas y ffp2 juntas. Ante el anuncio de una vacuna, los mayoristas
% % procedieron a realizar descuentos para agilizar las ventas y las mascarillas quirúrgicas, ffp2 y gel
% % tuvieron descuentos del 50\%, 25\% y 40\%, respectivamente. Con estos descuentos y con la misma
% % compra, la familia pagó 35 euros. Calcula el precio de cada artículo.



% % \question[4] Responde a las siguientes cuestiones:
% % \begin{parts}
% %     \part Lanzamos dos dados. ¿Cuál es la probabilidad de que la suma de sus puntuaciones sea mayor que 9?
% % \begin{solution}  \end{solution}
    
% %     \part Una empresa de telefonıía móvil quiere conocer la proporción de
% % clientes que están satisfechos con su servicio de atención al cliente. Para
% % ello toma una muestra de 200 clientes y les pregunta su opinión: 120 están
% % satisfechos y los 80 restantes no. Calcular un intervalo de confianza del 98\%
% % para la proporción de clientes de la empresa satisfechos con el servicio de
% % atención.
% % \begin{solution}
    
% % \end{solution}
% % \end{parts}

% \question Cierta asociación quiere trasladar a las autoridades información sobre el peso de las chicas
% menores de 18 años de su ciudad. Saben que el peso de las chicas menores de 18 años se distribuye
% según una normal de desviación típica 8,96 kg. Se pide:
% \begin{parts}
%     \part[2] Calcule el tamaño mínimo que debe tener la muestra para que, con un nivel de confianza del
% 97\%, el error de estimación del peso medio no supere los 2 kg.
%     \part[2] El peso, en kilogramos, correspondiente a una muestra de 9 chicas menores de 18 años es: 52, 53, 50, 45, 66, 42, 69, 55, 52. Calcule un intervalo de confianza al 97\% para la media poblacional.
% \end{parts}




% % \question[4] Se lanzan tres dados a la vez. Determinar:
% % \begin{parts}
% %     \part La probabilidad de que en los tres dados salga un 6
% %     \part La probabilidad de que en al menos uno de los tres dados salga un 6
% %     \part La probabilidad de que los tres dados muestren el mismo resultado
% % \end{parts}

% % \question[4] El precio por noche de alojamiento en un hotel de 4 estrellas en España se distribuye según
% % una distribución normal de desviación típica igual a 15 euros.
% % \begin{parts}
% %     \part Para calcular el intervalo de confianza al 96\% para el precio medio del alojamiento de
% % forma que su amplitud no sea mayor que 6 \euro ¿qué tamaño de la muestra debemos tomar?
% %     \part Decidimos tomar una muestra de tamaño 12. Se eligen 12 hoteles de cuatro estrellas
% % españoles y consultamos sus precios, obteniéndose los siguientes resultados:
% % $$41, 57, 55, 57, 60, 65, 44, 63, 50, 54, 79, 97$$
% %  Calcula el intervalo de confianza al 96\% para el precio medio por noche del alojamiento en España.
% % \end{parts}


% \question[4] Tenemos tres cajas, una verde, una roja y una amarilla, y en cada caja hay una moneda. La de
% la caja verde está trucada y la probabilidad de que salga cara es el doble de la probabilidad de que salga
% cruz, la moneda de la caja roja tiene dos caras y la de la caja amarilla no está trucada. Se toma una caja al
% azar y se lanza la moneda que está en esa caja. Calcular razonadamente:
% \begin{parts}
%     \part La probabilidad de que salga cara
%     \part La probabilidad de que sabiendo que ha salido cara, se haya lanzado la moneda de la caja roja
% \end{parts}
% \begin{solution}
    
% \end{solution}

\end{questions}

    \newgeometry{left=1 cm,bottom=2cm}
% \begin{landscape}
\begin{table}
% \Large
\centering

% \caption{Extracto de tabla de probabilidades de la \textbf{normal estándar $Z(0,1)$}}
\caption{Tabla de probabilidades de la \textbf{normal estándar $Z(0,1)$}}
\label{my-label}

\begin{tabular}{l|llllllllll}
z   & 0       & 0,01    & 0,02    & 0,03    & 0,04    & 0,05    & 0,06    & 0,07    & 0,08    & 0,09    \\
\hline
0   & 0,5     & 0,50399 & 0,50798 & 0,51197 & 0,51595 & 0,51994 & 0,52392 & 0,5279  & 0,53188 & 0,53586 \\
0,1 & 0,53983 & 0,5438  & 0,54776 & 0,55172 & 0,55567 & 0,55962 & 0,56356 & 0,56749 & 0,57142 & 0,57535 \\
0,2 & 0,57926 & 0,58317 & 0,58706 & 0,59095 & 0,59483 & 0,59871 & 0,60257 & 0,60642 & 0,61026 & 0,61409 \\
0,3 & 0,61791 & 0,62172 & 0,62552 & 0,6293  & 0,63307 & 0,63683 & 0,64058 & 0,64431 & 0,64803 & 0,65173 \\
0,4 & 0,65542 & 0,6591  & 0,66276 & 0,6664  & 0,67003 & 0,67364 & 0,67724 & 0,68082 & 0,68439 & 0,68793 \\
0,5 & 0,69146 & 0,69497 & 0,69847 & 0,70194 & 0,7054  & 0,70884 & 0,71226 & 0,71566 & 0,71904 & 0,7224  \\
0,6 & 0,72575 & 0,72907 & 0,73237 & 0,73565 & 0,73891 & 0,74215 & 0,74537 & 0,74857 & 0,75175 & 0,7549  \\
0,7 & 0,75804 & 0,76115 & 0,76424 & 0,7673  & 0,77035 & 0,77337 & 0,77637 & 0,77935 & 0,7823  & 0,78524 \\
0,8 & 0,78814 & 0,79103 & 0,79389 & 0,79673 & 0,79955 & 0,80234 & 0,80511 & 0,80785 & 0,81057 & 0,81327 \\
0,9 & 0,81594 & 0,81859 & 0,82121 & 0,82381 & 0,82639 & 0,82894 & 0,83147 & 0,83398 & 0,83646 & 0,83891 \\
1   & 0,84134 & 0,84375 & 0,84614 & 0,84849 & 0,85083 & 0,85314 & 0,85543 & 0,85769 & 0,85993 & 0,86214 \\
1,1 & 0,86433 & 0,8665  & 0,86864 & 0,87076 & 0,87286 & 0,87493 & 0,87698 & 0,879   & 0,881   & 0,88298 \\
1,2 & 0,88493 & 0,88686 & 0,88877 & 0,89065 & 0,89251 & 0,89435 & 0,89617 & 0,89796 & 0,89973 & 0,90147 \\
1,3 & 0,9032  & 0,9049  & 0,90658 & 0,90824 & 0,90988 & 0,91149 & 0,91309 & 0,91466 & 0,91621 & 0,91774 \\
1,4 & 0,91924 & 0,92073 & 0,9222  & 0,92364 & 0,92507 & 0,92647 & 0,92785 & 0,92922 & 0,93056 & 0,93189 \\
1,5 & 0,93319 & 0,93448 & 0,93574 & 0,93699 & 0,93822 & 0,93943 & 0,94062 & 0,94179 & 0,94295 & 0,94408 \\
1,6 & 0,9452  & 0,9463  & 0,94738 & 0,94845 & 0,9495  & 0,95053 & 0,95154 & 0,95254 & 0,95352 & 0,95449 \\
1,7 & 0,95543 & 0,95637 & 0,95728 & 0,95818 & 0,95907 & 0,95994 & 0,9608  & 0,96164 & 0,96246 & 0,96327 \\
1,8 & 0,96407 & 0,96485 & 0,96562 & 0,96638 & 0,96712 & 0,96784 & 0,96856 & 0,96926 & 0,96995 & 0,97062 \\
1,9 & 0,97128 & 0,97193 & 0,97257 & 0,9732  & 0,97381 & 0,97441 & 0,975   & 0,97558 & 0,97615 & 0,9767  \\
2   & 0,97725 & 0,97778 & 0,97831 & 0,97882 & 0,97932 & 0,97982 & 0,9803  & 0,98077 & 0,98124 & 0,98169 \\
2,1 & 0,98214 & 0,98257 & 0,983   & 0,98341 & 0,98382 & 0,98422 & 0,98461 & 0,985   & 0,98537 & 0,98574 \\
2,2 & 0,9861  & 0,98645 & 0,98679 & 0,98713 & 0,98745 & 0,98778 & 0,98809 & 0,9884  & 0,9887  & 0,98899 \\
2,3 & 0,98928 & 0,98956 & 0,98983 & 0,9901  & 0,99036 & 0,99061 & 0,99086 & 0,99111 & 0,99134 & 0,99158 \\
2,4 & 0,9918  & 0,99202 & 0,99224 & 0,99245 & 0,99266 & 0,99286 & 0,99305 & 0,99324 & 0,99343 & 0,99361 \\
2,5 & 0,99379 & 0,99396 & 0,99413 & 0,9943  & 0,99446 & 0,99461 & 0,99477 & 0,99492 & 0,99506 & 0,9952  \\
2,6 & 0,99534 & 0,99547 & 0,9956  & 0,99573 & 0,99585 & 0,99598 & 0,99609 & 0,99621 & 0,99632 & 0,99643 \\
2,7 & 0,99653 & 0,99664 & 0,99674 & 0,99683 & 0,99693 & 0,99702 & 0,99711 & 0,9972  & 0,99728 & 0,99736 \\
2,8 & 0,99744 & 0,99752 & 0,9976  & 0,99767 & 0,99774 & 0,99781 & 0,99788 & 0,99795 & 0,99801 & 0,99807 \\
2,9 & 0,99813 & 0,99819 & 0,99825 & 0,99831 & 0,99836 & 0,99841 & 0,99846 & 0,99851 & 0,99856 & 0,99861 \\
3   & 0,99865 & 0,99869 & 0,99874 & 0,99878 & 0,99882 & 0,99886 & 0,99889 & 0,99893 & 0,99896 & 0,999   \\
3,1 & 0,99903 & 0,99906 & 0,9991  & 0,99913 & 0,99916 & 0,99918 & 0,99921 & 0,99924 & 0,99926 & 0,99929 \\
3,2 & 0,99931 & 0,99934 & 0,99936 & 0,99938 & 0,9994  & 0,99942 & 0,99944 & 0,99946 & 0,99948 & 0,9995  \\
3,3 & 0,99952 & 0,99953 & 0,99955 & 0,99957 & 0,99958 & 0,9996  & 0,99961 & 0,99962 & 0,99964 & 0,99965 \\
3,4 & 0,99966 & 0,99968 & 0,99969 & 0,9997  & 0,99971 & 0,99972 & 0,99973 & 0,99974 & 0,99975 & 0,99976 \\
3,5 & 0,99977 & 0,99978 & 0,99978 & 0,99979 & 0,9998  & 0,99981 & 0,99981 & 0,99982 & 0,99983 & 0,99983 \\
3,6 & 0,99984 & 0,99985 & 0,99985 & 0,99986 & 0,99986 & 0,99987 & 0,99987 & 0,99988 & 0,99988 & 0,99989 \\
3,7 & 0,99989 & 0,9999  & 0,9999  & 0,9999  & 0,99991 & 0,99991 & 0,99992 & 0,99992 & 0,99992 & 0,99992 \\
3,8 & 0,99993 & 0,99993 & 0,99993 & 0,99994 & 0,99994 & 0,99994 & 0,99994 & 0,99995 & 0,99995 & 0,99995 \\
3,9 & 0,99995 & 0,99995 & 0,99996 & 0,99996 & 0,99996 & 0,99996 & 0,99996 & 0,99996 & 0,99997 & 0,99997 \\
4   & 0,99997 & 0,99997 & 0,99997 & 0,99997 & 0,99997 & 0,99997 & 0,99998 & 0,99998 & 0,99998 & 0,99998
\end{tabular}
\end{table}
% \end{landscape}
\restoregeometry

\end{document}
\grid
