\documentclass[addpoints,spanish, 12pt,a4paper]{exam}
%\documentclass[answers, spanish, 12pt,a4paper]{exam}
\printanswers
\pointpoints{punto}{puntos}
\hpword{Puntos:}
\vpword{Puntos:}
\htword{Total}
\vtword{Total}
\hsword{Resultado:}
\hqword{Ejercicio:}
\vqword{Ejercicio:}

\usepackage[utf8]{inputenc}
\usepackage[spanish]{babel}
\usepackage{eurosym}
%\usepackage[spanish,es-lcroman, es-tabla, es-noshorthands]{babel}


\usepackage[margin=1in]{geometry}
\usepackage{amsmath,amssymb}
\usepackage{multicol}
\usepackage{yhmath}

\pointsinrightmargin % Para poner las puntuaciones a la derecha. Se puede cambiar. Si se comenta, sale a la izquierda.
\extrawidth{-2.4cm} %Un poquito más de margen por si ponemos textos largos.
\marginpointname{ \emph{\points}}

\usepackage{graphicx}

\graphicspath{{../../img/}} 

\newcommand{\class}{2º Bachillerato CCSS}
\newcommand{\examdate}{\today}
\newcommand{\examnum}{Final 2ªEv.}
\newcommand{\tipo}{A}


\newcommand{\timelimit}{90 minutos}

\renewcommand{\solutiontitle}{\noindent\textbf{Solución:}\enspace}


\pagestyle{head}
\firstpageheader{\includegraphics[width=0.2\columnwidth]{header_left}}{\textbf{Departamento de Matemáticas\linebreak \class}\linebreak \examnum}{\includegraphics[width=0.1\columnwidth]{header_right}}
\runningheader{\class}{\examnum}{Página \thepage\ of \numpages}
\runningheadrule


\usepackage{pgf,tikz,pgfplots}
\pgfplotsset{compat=1.15}
\usepackage{mathrsfs}
\usetikzlibrary{arrows}


\begin{document}

\noindent
\begin{tabular*}{\textwidth}{l @{\extracolsep{\fill}} r @{\extracolsep{6pt}} }
\textbf{Nombre:} \makebox[3.5in]{\hrulefill} & \textbf{Fecha:}\makebox[1in]{\hrulefill} \\
 & \\
\textbf{Tiempo: \timelimit} & Tipo: \tipo 
\end{tabular*}
\rule[2ex]{\textwidth}{2pt}
Esta prueba tiene \numquestions\ ejercicios. La puntuación máxima es de \numpoints. 
La nota final de la prueba será la parte proporcional de la puntuación obtenida sobre la puntuación máxima. 

\begin{center}


\addpoints
 %\gradetable[h][questions]
	\pointtable[h][questions]
\end{center}

\noindent
\rule[2ex]{\textwidth}{2pt}

\begin{questions}

%\question 
%
%\begin{parts}
%\part[2] 
%\begin{solution}
%\end{solution}
%
%
%\end{parts}
%\addpoints

\question[2] Halla el límite de la función $f(x)=\dfrac{x^2-2x}{2x^2-8}$ cuando $x\to 2$, $x\to -2$, $x\to \infty$, $x\to -\infty$.

\question Dada la función $f(x)=\left\{ \begin{matrix}
x^2+2x-1 &  si & x\leq 1 \\
\dfrac{4}{x+1} &  si & x > 1
\end{matrix}\right.$
\begin{parts}
    \part[1] Estudia la continuidad de la función
    \part[1] ¿Existe algún punto dónde $f'(x)=0$ ?
\end{parts}

\question Halla las derivadas de las siguientes funciones:
\begin{parts}
    \part[1] $f(x)=\dfrac{x}{\left(x+2\right)^2}$
    \part[1] $f(x)=\left(\dfrac{1-x}{1+x}\right)^2$
    \part[1] $f(x)=\ln\left(\dfrac{x}{3}+1\right)$
    \part[1] $f(x)=e^{2x+1}$
\end{parts} 
 
% \question[2] Halla la ecuación de la recta tangente a $f(x)=x^2\sqrt{1-3x}$ en el punto de abscisa $x=-1$

\question[4] Estudia las asíntotas y los intervalos de crecimiento y decrecimiento de la función $f(x)=\dfrac{2x}{1-x^2}$

\question[3] Halla el dominio de definición y los extremos relativos de $f(x)=x+\sqrt{1-x}$

% \question[1] Un heladero ha comprobado que, a un precio de 50 céntimos de euro la unidad, vende una media de 200
% helados diarios. Por cada céntimo que aumenta el precio, vende dos helados menos al día. Si el coste por
% unidad es de 40 céntimos
% \begin{parts}
%    \part[2] ¿A qué precio de venta es máximo el beneficio diario que obtiene el heladero?
%    \begin{solution}
%        $B(x)=(50+x-10)(200-2x)=- 2 x^{2} + 180 x + 2000$ \\
%        $B'(x)=180-4x \to B'(x)=0 \to x=45$ \\
%        Solución: 95 helados 
       
%    \end{solution}
%    \part[1] ¿Cuál será ese beneficio
%    \begin{solution}
%         $B(45)=6050$ \euro
%    \end{solution}
       
% \end{parts}

\question[4] Un granjero desea vallar un terreno rectangular de pasto adyacente a un río. El pastizal debe tener 180.000
$m^2$
para producir suficiente forraje para su ganado. ¿Qué dimensiones tendrá el terreno rectangular de
forma que utilice la mínima cantidad de valla, si el lado que da al río no necesita ser vallado
\begin{solution}
    $300 x 600 m$
\end{solution}

\question Sea $C(q) = 100 + 140 q + q^2$ el coste total, en euros, de producir $q$ unidades de un
producto y cada unidad del producto se vende a $(400 - 12 q)$ euros.
\begin{parts}
    \part[2] ¿Cuántas unidades deben venderse para que el beneficio (ingresos menos costes) sea máximo?, ¿a cuánto asciende dicho beneficio máximo
    \begin{solution}
        $B(x)=x^{2} - x \left(400 - 12 x\right) + 140 x + 100 = -13 x^{2} + 260 x - 100 \to B'(x)= - 26 x + 260 \to q=10 (B''(x)=-26)$
    \end{solution}
    \part[2] ¿Cuántas unidades hay que producir para minimizar el coste medio $CM(q)=\dfrac{C(q)}{q}$ Obtén dicho valor mínimo.
\end{parts}


% \question Calcula:


% \begin{parts}
% % \part[2] 
% % $$\lim_{x \to -1}\left(\dfrac{x^{3} + x^{2} - x - 1}{2 x^{3} + 5 x^{2} + 4 x + 1}\right)$$
% % \begin{solution}
% % $2$
% % \end{solution}
% % \part[2] 
% % $$\lim_{x \to 3}\left(\frac{\sqrt{x + 1} - 2}{x^{2} - 3 x}\right)$$
% % \begin{solution}
% % $\frac{1}{12}$
% % \end{solution}
% % \part[1] 
% % $$\lim_{x \to -\infty} \left(\dfrac{x+3}{2x}\right)^{\frac{x^{2}}{x - 3}}$$
% % \begin{solution}
% % $\frac{1}{2^{-\infty}}=\infty$
% % \end{solution}

% \part[2] 
% $$\lim_{x \to -1}\left(\frac{x^{3} - 3 x - 2}{2 x^{3} + 5 x^{2} + 4 x + 1}\right)$$
% \begin{solution}
% $3$
% \end{solution}
% \part[2] 
% $$\lim_{x \to 8}\left(\frac{\sqrt{x + 1} - 3}{x^{2} - 8 x}\right)$$
% \begin{solution}
% $\frac{1}{48}$
% \end{solution}
% \part[1] 
% $$\lim_{x \to \infty} \left(\dfrac{x+3}{2x}\right)^{\frac{x^{2}}{x - 3}}$$
% \begin{solution}
% $\frac{1}{2^{\infty}}=0$
% \end{solution}
% \end{parts}
% \addpoints

% % \question Se considera la función $$f(x)=\left\{ \begin{matrix}
% % ax^2-1 &  si & x<1 \\
% % (x-a)^2 &  si & x \geq 1
% % \end{matrix}\right.$$ 

% % \begin{parts}
% % \part[2] Determine los valores de 
% % $a \in \mathbb{R}$ que hacen que $f$ es continua en su dominio
% % \begin{solution}
% % $f=\begin{cases} a x^{2} - 1 & \text{for}\: x < 1 \\\left(- a + x\right)^{2} & \text{otherwise} \end{cases}
% % $ \\
% % $\lim_{x \to 1^-} f=a - 1
% %  \land \lim_{x \to 1^-} f=\left(1 - a\right)^{2} \to - a^{2} + 3 a - 2
% % =0 \to a=1, a=2$
% % \end{solution}
% % \part[2] Para 
% % $a=\dfrac{1}{2}$, determine, si existen, los puntos de corte de la gráfica con el eje de las $x$
% % \begin{solution}
% % $\dfrac{{x}^{2}}{2} - 1=0 \to x=-\sqrt{2}$

% % $\left(x - \frac{1}{2}\right)^{2}
% % \to x=\frac{1}{2}=0 \notin x\geq1$
% % \end{solution}
% % \end{parts}
% % \addpoints


% % \question Dada la función $f(x)=ax+\frac{b}{x}$:
% % \begin{parts}
% % \part[2] Determine los valores de los parámetros 
% % $a$, $b \in \mathbb{R}$  para que pase por el punto $(2,4)$ y tenga un extremo relativo en ese punto.
% % \begin{solution}
% % $ 
% % \left\{
% % \begin{matrix}
% % f(2)=4 \\
% % f'(2)=0
% % \end{matrix}
% % \right.
% % \to f'(x)=a-\frac{b}{x^2}
% % \to
% % \left\{
% % \begin{matrix}
% % 2 a + \frac{b}{2}
% % =4 \\
% % a - \frac{b}{4}
% % =0
% % \end{matrix}
% % \right.
% % \to a=1 \land b=4$
% % \end{solution}
% % \part[2] Justifica qué tipo de extremo relativo es (máximo relativo o mínimo relativo)
% % \begin{solution}
% % $f''(x)=\frac{2b}{x^3} \to f''(2)=\frac{b}{4} \to f''(2)=\frac{4}{4}=1>0 \to MIN REL$ 
% % \end{solution}
% % \end{parts}




% % \question Sea la función $f(x)=\dfrac{x^2-x+1}{x-1}$:
% % \begin{parts}
% % \part[1] Determine sus asíntotas
% % \begin{solution}
% % $\lim_{x \to 1^-}\left(\frac{x^{2} - x + 1}{x - 1}\right)= -\infty
% % \land
% % \lim_{x \to 1^+}\left(\frac{x^{2} - x + 1}{x - 1}\right)=\infty \to x=1 A.V.
% % $\\
% % $\lim_{x \to \infty}\left(\frac{x^{2} - x + 1}{x - 1}\right)=\infty
% % \land
% % \lim_{x \to -\infty}\left(\frac{x^{2} - x + 1}{x - 1}\right)=-\infty \to \nexists A.H.
% % $ \\
% % $\lim_{x \to \infty}\frac{f(x)}{x}=\left(\frac{x^{2} - x + 1}{x^2 - x}\right)=1 \to m = 1 \land \lim_{x \to \infty} \left(f(x)-mx\right)=0 \to y=x A.O.$
% % \end{solution}
% % \part[1] Calcule $f'(2)$
% % \begin{solution}
% % $f'(x)=\frac{x \left(x - 2\right)}{x^{2} - 2 x + 1} \to f'(2)=0$
% % \end{solution}
% % \end{parts}
% % \addpoints

% % \question Se considera la función $f(x)=\frac{10}{x^2+2x-3}$:
% % \begin{parts}
% % \part[1] Determine el dominio de $f$ y sus asíntotas
% % \begin{solution}$f(x)=0 \to x= -3, x=1 \to \lim_{x \to -3}=\infty \land \lim_{x \to -3}=\infty \to x=-3 A.V. x=-1 A.V.$\\$\lim_{x\to\infty}f=\dfrac{10}{\infty}=0 \to y =0  A.H.$
% % \end{solution}
% % \part[1] Obtenga los intervalos de crecimiento y decrecimiento de $f(x)$ y determine los extremos relativos indicando si corresponden a máximos o mínimos 	
% % \begin{solution}$f'(x)=\frac{10 \left(- 2 x - 2\right)}{\left(x^{2} + 2 x - 3\right)^{2}}$\\$f'(x)=0 \to x =-1$\\$f''(-1)=-\frac{5}{4}\to \left(-1,-\frac{5}{2}\right) MAX \ REL$
% % \end{solution}
% % \end{parts}



% % \question Dada la función $f(x)=\dfrac{2x}{1-x^2}$:
% % \begin{parts}
% % \part[2] Determine el dominio de $f$ y sus asíntotas
% % \part[2] Obtenga los intervalos de crecimiento y decrecimiento de $f(x)$
% % \part[2] Determine los extremos relativos indicando si corresponden a máximos o mínimos 	
% % \end{parts}


% \question Dada la función $f(x)=\dfrac{-3x}{1-x^2}$:
% \begin{parts}
% \part[2] Determine el dominio de $f$ y sus asíntotas
% \part[2] Obtenga los intervalos de crecimiento y decrecimiento de $f(x)$
% \part[2] Determine los extremos relativos indicando si corresponden a máximos o mínimos 	
% \end{parts}



% % \question Dada la función $f(x)=x+\sqrt{1-x}$:
% % \begin{parts}
% % \part[1] Determine el dominio de $f$ 
% % \part[2] Obtenga los intervalos de crecimiento y decrecimiento de $f(x)$ y determine los extremos relativos indicando si corresponden a máximos o mínimos 	
% % \end{parts}

% % \question Dada la función: $$f(x)=\left\{ \begin{matrix}
% % \dfrac{x+2}{x-1} &  si & x\leq 2 \\
% % \dfrac{3x^2-2x}{x+2} &  si & x > 2
% % \end{matrix}\right.$$
% % \begin{parts}
% % \part[1] Estudie la continuidad de $f$ 
% % \part[2] Determine la recta tangente a $f$ en el punto donde $x=3$
% % \part[2] Calcule las asíntotas oblicuas 	
% % \end{parts}


% \question Dada la función: $$f(x)=\left\{ \begin{matrix}
% \dfrac{x+2}{x-1} &  si & x\leq 2 \\
% \dfrac{3x^2-2x}{x+2} &  si & x > 2
% \end{matrix}\right.$$
% \begin{parts}
% \part[1] Estudie la continuidad de $f$ 
% \part[2] Determine la recta normal a $f$ en el punto donde $x=-2$
% \part[2] Calcule las asíntotas horizontales 	
% \end{parts}


\end{questions}

\end{document}
\grid
