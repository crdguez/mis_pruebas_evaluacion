\documentclass[addpoints,spanish, 12pt,a4paper]{exam}
%\documentclass[answers, spanish, 12pt,a4paper]{exam}
\printanswers
\pointpoints{punto}{puntos}
\hpword{Puntos:}
\vpword{Puntos:}
\htword{Total}
\vtword{Total}
\hsword{Resultado:}
\hqword{Ejercicio:}
\vqword{Ejercicio:}

\usepackage[utf8]{inputenc}
\usepackage[spanish]{babel}
\usepackage{eurosym}
%\usepackage[spanish,es-lcroman, es-tabla, es-noshorthands]{babel}


\usepackage[margin=1in]{geometry}
\usepackage{amsmath,amssymb}
\usepackage{multicol}
\usepackage{yhmath}

\pointsinrightmargin % Para poner las puntuaciones a la derecha. Se puede cambiar. Si se comenta, sale a la izquierda.
\extrawidth{-2.4cm} %Un poquito más de margen por si ponemos textos largos.
\marginpointname{ \emph{\points}}

\usepackage{graphicx}

\graphicspath{{../../img/}} 

\newcommand{\class}{2º Bachillerato CCSS}
\newcommand{\examdate}{\today}
\newcommand{\examnum}{Final 2ªEv.}
\newcommand{\tipo}{A}


\newcommand{\timelimit}{90 minutos}

\renewcommand{\solutiontitle}{\noindent\textbf{Solución:}\enspace}


\pagestyle{head}
\firstpageheader{\includegraphics[width=0.2\columnwidth]{header_left}}{\textbf{Departamento de Matemáticas\linebreak \class}\linebreak \examnum}{\includegraphics[width=0.1\columnwidth]{header_right}}
\runningheader{\class}{\examnum}{Página \thepage\ of \numpages}
\runningheadrule


\usepackage{pgf,tikz,pgfplots}
\pgfplotsset{compat=1.15}
\usepackage{mathrsfs}
\usetikzlibrary{arrows}


\begin{document}

\noindent
\begin{tabular*}{\textwidth}{l @{\extracolsep{\fill}} r @{\extracolsep{6pt}} }
\textbf{Nombre:} \makebox[3.5in]{\hrulefill} & \textbf{Fecha:}\makebox[1in]{\hrulefill} \\
 & \\
\textbf{Tiempo: \timelimit} & Tipo: \tipo 
\end{tabular*}
\rule[2ex]{\textwidth}{2pt}
Esta prueba tiene \numquestions\ ejercicios. La puntuación máxima es de \numpoints. 
La nota final de la prueba será la parte proporcional de la puntuación obtenida sobre la puntuación máxima. 

\begin{center}


\addpoints
 %\gradetable[h][questions]
	\pointtable[h][questions]
\end{center}

\noindent
\rule[2ex]{\textwidth}{2pt}

\begin{questions}

%\question 
%
%\begin{parts}
%\part[2] 
%\begin{solution}
%\end{solution}
%
%
%\end{parts}
%\addpoints





\question[2] Halla el límite de la función $f(x)=\dfrac{2x^2-4x}{4x^2-16}$ cuando $x\to 2$, $x\to -2$, $x\to \infty$, $x\to -\infty$.
\begin{solution}
    $\left[ \frac{1}{4}, \  -\infty, \  \frac{1}{2}, \  \frac{1}{2}\right]$
\end{solution}

\question Dada la función $f(x)=\left\{ \begin{matrix}
x^{2} - 4 x + 5 &  si & x\leq 1 \\
\dfrac{4}{x+1} &  si & x > 1
\end{matrix}\right.$
% f=Piecewise((x**2-4*x+5,x<=1),(4/(x+1),True))
\begin{parts}
    \part[1] Estudia la continuidad de la función
    \begin{solution}
        Fuera de x=1 f es continua por serlo sus trozos en 
 sus dominios \\
        $\lim_{x\to1^-}f=\lim_{x\to1^+}f=2$ \\ Por tanto $f$ es continua en $\mathbf{R}$    
    \end{solution}
    \part[1] ¿Existe algún punto dónde $f'(x)=0$ ?
    \begin{solution}
        $\begin{cases} 2 x - 4 & \text{for}\: x \leq 1 \\- \frac{4}{\left(x + 1\right)^{2}} & \text{otherwise} \end{cases}$ \\ El segundo trozo no se anula para ningún valor. El primer trozo se anularía en x=2 pero no pertenece al dominio. Por tanto, no hay ninguno
    \end{solution}
\end{parts}

% \question Halla las derivadas de las siguientes funciones:
% \begin{solution}
%     $\displaystyle \left[ \frac{2 - x}{\left(x + 2\right)^{3}}, \  \frac{4 \left(x - 1\right)}{\left(x + 1\right)^{3}}, \  \frac{1}{x + 3}, \  2 e^{2 x + 1}\right]$
% \end{solution}
% \begin{parts}
%     \part[1] $f(x)=\dfrac{x}{\left(x+2\right)^2}$
%     \part[1] $f(x)=\left(\dfrac{1-x}{1+x}\right)^2$
%     \part[1] $f(x)=\ln\left(\dfrac{x}{3}+1\right)$
%     \part[1] $f(x)=e^{2x+1}$
% \end{parts} 

\question Calcula las derivadas de las siguientes funciones: 
\begin{parts}
\part[1] $f(x)=\frac{1}{x}+2\ln x - \frac{\ln x}{x}$
\begin{solution}
$\frac{2}{x} + \frac{1}{x^{2}} \ln{\left (x \right )} - \frac{2}{x^{2}}=\frac{1}{x^{2}} \left(2 x + \ln{\left (x \right )} - 2\right)$
\end{solution}
\part[1] $g(x)=\dfrac{e^x}{(x-1)^2}$
\begin{solution}
$\frac{\left(- 2 x + 2\right) e^{x}}{\left(x - 1\right)^{4}} + \frac{e^{x}}{\left(x - 1\right)^{2}}=\frac{e^{x}}{\left(x - 1\right)^{4}} \left(- 2 x + \left(x - 1\right)^{2} + 2\right)$

\end{solution}
\end{parts}
 
% \question[2] Halla la ecuación de la recta tangente a $f(x)=x^2\sqrt{1-3x}$ en el punto de abscisa $x=-1$

\question[4] Calcula las ecuaciones de las asíntotas y los intervalos de crecimiento y decrecimiento de la función $f(x)=\dfrac{2x}{1-x^2}$
\begin{solution}
    A.V: $x=-1$ y $x=1$ \\
    A.H: $y=0$ \\
    $f'(x)=\frac{2 \left(x^{2} + 1\right)}{\left(x^{2} - 1\right)^{2}} \to \nexists$ puntos críticos. \\
    La función es creciente en $\left(-\infty, -1\right) \cup \left(-1, 1\right) \cup \left(1, \infty \right)$ 
\end{solution}

% \question[2] Calcula os intervalos de crecimiento y decrecimiento de  $f(x)=\dfrac{2x}{1-x^2}$
% \begin{solution}
%     A.V: $x=-1$ y $x=1$ \\
%     A.H: $y=0$ \\
%     $f'(x)=\frac{2 \left(x^{2} + 1\right)}{\left(x^{2} - 1\right)^{2}} \to \nexists$ puntos críticos. \\
%     La función es creciente en $\left(-\infty, -1\right) \cup \left(-1, 1\right) \cup \left(1, \infty \right)$ 
% \end{solution}

\question[3] Halla el dominio de definición y los extremos relativos (puntos de la gráfica) de $f(x)=x+\sqrt{1-x}$
\begin{solution}
    Dominio: $\left\{x| 1-x >0\right\}\to \left(-\infty, 1\right)$ \\
    $f'(x)=1 - \frac{1}{2 \sqrt{1 - x}} \to  f'(x)=0 \to x=\frac{3}{4} \to \left(\frac{3}{4}, \frac{5}{4}\right)$
    
\end{solution}

% \question[1] Un heladero ha comprobado que, a un precio de 50 céntimos de euro la unidad, vende una media de 200
% helados diarios. Por cada céntimo que aumenta el precio, vende dos helados menos al día. Si el coste por
% unidad es de 40 céntimos
% \begin{parts}
%    \part[2] ¿A qué precio de venta es máximo el beneficio diario que obtiene el heladero?
%    \begin{solution}
%        $B(x)=(50+x-10)(200-2x)=- 2 x^{2} + 180 x + 2000$ \\
%        $B'(x)=180-4x \to B'(x)=0 \to x=45$ \\
%        Solución: 95 helados 
       
%    \end{solution}
%    \part[1] ¿Cuál será ese beneficio
%    \begin{solution}
%         $B(45)=6050$ \euro
%    \end{solution}
       
% \end{parts}

\question[4] Un granjero desea vallar un terreno rectangular de pasto adyacente a un río. El terreno debe tener 180.000
$m^2$
para producir suficiente pasto. ¿Qué dimensiones tendrá el terreno de
forma que utilice la mínima cantidad de valla, si el lado que da al río no necesita ser vallado?
\begin{solution}
    Minimizar $f(x)=x +2 \frac{180000}{x}$ \\
    $f'(x)= 1 - \frac{360000}{x^{2}} \land f''(x)=\frac{720000}{x^{3}} \to $
    $x=600 m$ e $y=300 m$
\end{solution}

% \question Sea $C(q) = 100 + 140 q + q^2$ el coste total, en euros, de producir $q$ unidades de un
% producto y cada unidad del producto se vende a $(400 - 12 q)$ euros.
% \begin{parts}
%     \part[2] ¿Cuántas unidades deben venderse para que el beneficio (ingresos menos costes) sea máximo?, ¿Cuánto será el beneficio?
%     \begin{solution}
%         $B(x)=x^{2} - x \left(400 - 12 x\right) + 140 x + 100 = -13 x^{2} + 260 x - 100 \to B'(x)= - 26 x + 260 \to q=10 \  (B''(x)=-26)$
%     \end{solution}
%     \part[2] ¿Cuántas unidades hay que producir para minimizar el coste medio $CM(q)=\dfrac{C(q)}{q}$ Obtén dicho valor mínimo.
%     \begin{solution}
%         $cm(x)=x + 140 + \frac{100}{x}$ \\
%         $cm'(x)=1 - \frac{100}{x^{2}}$ \\
%         $cm''(x)=\frac{200}{x^{3}}$\\
%         $cm'(x)=0 \to x=10$
        
        
        
%     \end{solution}
% \end{parts}

\question La función de costes de una empresa es 
$C(q)=q^3+3q+10$, donde $q$ representa las unidades producidas.  Sabiendo que el precio de venta, en euros, de cada unidad
producida es $p = 30$, se desea conocer:
\begin{parts}
    \part[1] La función de beneficio de esta empresa.
    \begin{solution}
        $f(x)=- x^{3} + 27 x - 10$
    \end{solution}
    \part[2] El número de unidades producidas que maximiza el beneficio de la empresa. 
    \begin{solution}
        $f'(x)=27 - 3 x^{2} \land f''(x)=-6x \to f'(3)=0 \land f''(3) = -18 \to x=3$
    \end{solution}
    \part[1] El beneficio máximo que puede lograr la empresa.
    \begin{solution}
    $f(3)=44$    
    \end{solution}
    
\end{parts}

% \question Calcula:


% \begin{parts}
% % \part[2] 
% % $$\lim_{x \to -1}\left(\dfrac{x^{3} + x^{2} - x - 1}{2 x^{3} + 5 x^{2} + 4 x + 1}\right)$$
% % \begin{solution}
% % $2$
% % \end{solution}
% % \part[2] 
% % $$\lim_{x \to 3}\left(\frac{\sqrt{x + 1} - 2}{x^{2} - 3 x}\right)$$
% % \begin{solution}
% % $\frac{1}{12}$
% % \end{solution}
% % \part[1] 
% % $$\lim_{x \to -\infty} \left(\dfrac{x+3}{2x}\right)^{\frac{x^{2}}{x - 3}}$$
% % \begin{solution}
% % $\frac{1}{2^{-\infty}}=\infty$
% % \end{solution}

% \part[2] 
% $$\lim_{x \to -1}\left(\frac{x^{3} - 3 x - 2}{2 x^{3} + 5 x^{2} + 4 x + 1}\right)$$
% \begin{solution}
% $3$
% \end{solution}
% \part[2] 
% $$\lim_{x \to 8}\left(\frac{\sqrt{x + 1} - 3}{x^{2} - 8 x}\right)$$
% \begin{solution}
% $\frac{1}{48}$
% \end{solution}
% \part[1] 
% $$\lim_{x \to \infty} \left(\dfrac{x+3}{2x}\right)^{\frac{x^{2}}{x - 3}}$$
% \begin{solution}
% $\frac{1}{2^{\infty}}=0$
% \end{solution}
% \end{parts}
% \addpoints

% % \question Se considera la función $$f(x)=\left\{ \begin{matrix}
% % ax^2-1 &  si & x<1 \\
% % (x-a)^2 &  si & x \geq 1
% % \end{matrix}\right.$$ 

% % \begin{parts}
% % \part[2] Determine los valores de 
% % $a \in \mathbb{R}$ que hacen que $f$ es continua en su dominio
% % \begin{solution}
% % $f=\begin{cases} a x^{2} - 1 & \text{for}\: x < 1 \\\left(- a + x\right)^{2} & \text{otherwise} \end{cases}
% % $ \\
% % $\lim_{x \to 1^-} f=a - 1
% %  \land \lim_{x \to 1^-} f=\left(1 - a\right)^{2} \to - a^{2} + 3 a - 2
% % =0 \to a=1, a=2$
% % \end{solution}
% % \part[2] Para 
% % $a=\dfrac{1}{2}$, determine, si existen, los puntos de corte de la gráfica con el eje de las $x$
% % \begin{solution}
% % $\dfrac{{x}^{2}}{2} - 1=0 \to x=-\sqrt{2}$

% % $\left(x - \frac{1}{2}\right)^{2}
% % \to x=\frac{1}{2}=0 \notin x\geq1$
% % \end{solution}
% % \end{parts}
% % \addpoints


% \question Dada la función $f(x)=ax+\frac{b}{x}$:
% \begin{parts}
% \part[2] Determine los valores de los parámetros 
% $a$, $b \in \mathbb{R}$  para que pase por el punto $(2,4)$ y tenga un extremo relativo en ese punto.
% \begin{solution}
% $ 
% \left\{
% \begin{matrix}
% f(2)=4 \\
% f'(2)=0
% \end{matrix}
% \right.
% \to f'(x)=a-\frac{b}{x^2}
% \to
% \left\{
% \begin{matrix}
% 2 a + \frac{b}{2}
% =4 \\
% a - \frac{b}{4}
% =0
% \end{matrix}
% \right.
% \to a=1 \land b=4$
% \end{solution}
% \part[2] Justifica qué tipo de extremo relativo es (máximo relativo o mínimo relativo)
% \begin{solution}
% $f''(x)=\frac{2b}{x^3} \to f''(2)=\frac{b}{4} \to f''(2)=\frac{4}{4}=1>0 \to MIN REL$ 
% \end{solution}
% \end{parts}




% % \question Sea la función $f(x)=\dfrac{x^2-x+1}{x-1}$:
% % \begin{parts}
% % \part[1] Determine sus asíntotas
% % \begin{solution}
% % $\lim_{x \to 1^-}\left(\frac{x^{2} - x + 1}{x - 1}\right)= -\infty
% % \land
% % \lim_{x \to 1^+}\left(\frac{x^{2} - x + 1}{x - 1}\right)=\infty \to x=1 A.V.
% % $\\
% % $\lim_{x \to \infty}\left(\frac{x^{2} - x + 1}{x - 1}\right)=\infty
% % \land
% % \lim_{x \to -\infty}\left(\frac{x^{2} - x + 1}{x - 1}\right)=-\infty \to \nexists A.H.
% % $ \\
% % $\lim_{x \to \infty}\frac{f(x)}{x}=\left(\frac{x^{2} - x + 1}{x^2 - x}\right)=1 \to m = 1 \land \lim_{x \to \infty} \left(f(x)-mx\right)=0 \to y=x A.O.$
% % \end{solution}
% % \part[1] Calcule $f'(2)$
% % \begin{solution}
% % $f'(x)=\frac{x \left(x - 2\right)}{x^{2} - 2 x + 1} \to f'(2)=0$
% % \end{solution}
% % \end{parts}
% % \addpoints

% % \question Se considera la función $f(x)=\frac{10}{x^2+2x-3}$:
% % \begin{parts}
% % \part[1] Determine el dominio de $f$ y sus asíntotas
% % \begin{solution}$f(x)=0 \to x= -3, x=1 \to \lim_{x \to -3}=\infty \land \lim_{x \to -3}=\infty \to x=-3 A.V. x=-1 A.V.$\\$\lim_{x\to\infty}f=\dfrac{10}{\infty}=0 \to y =0  A.H.$
% % \end{solution}
% % \part[1] Obtenga los intervalos de crecimiento y decrecimiento de $f(x)$ y determine los extremos relativos indicando si corresponden a máximos o mínimos 	
% % \begin{solution}$f'(x)=\frac{10 \left(- 2 x - 2\right)}{\left(x^{2} + 2 x - 3\right)^{2}}$\\$f'(x)=0 \to x =-1$\\$f''(-1)=-\frac{5}{4}\to \left(-1,-\frac{5}{2}\right) MAX \ REL$
% % \end{solution}
% % \end{parts}



% % \question Dada la función $f(x)=\dfrac{2x}{1-x^2}$:
% % \begin{parts}
% % \part[2] Determine el dominio de $f$ y sus asíntotas
% % \part[2] Obtenga los intervalos de crecimiento y decrecimiento de $f(x)$
% % \part[2] Determine los extremos relativos indicando si corresponden a máximos o mínimos 	
% % \end{parts}


% \question Dada la función $f(x)=\dfrac{-3x}{1-x^2}$:
% \begin{parts}
% \part[2] Determine el dominio de $f$ y sus asíntotas
% \part[2] Obtenga los intervalos de crecimiento y decrecimiento de $f(x)$
% \part[2] Determine los extremos relativos indicando si corresponden a máximos o mínimos 	
% \end{parts}



% % \question Dada la función $f(x)=x+\sqrt{1-x}$:
% % \begin{parts}
% % \part[1] Determine el dominio de $f$ 
% % \part[2] Obtenga los intervalos de crecimiento y decrecimiento de $f(x)$ y determine los extremos relativos indicando si corresponden a máximos o mínimos 	
% % \end{parts}

% % \question Dada la función: $$f(x)=\left\{ \begin{matrix}
% % \dfrac{x+2}{x-1} &  si & x\leq 2 \\
% % \dfrac{3x^2-2x}{x+2} &  si & x > 2
% % \end{matrix}\right.$$
% % \begin{parts}
% % \part[1] Estudie la continuidad de $f$ 
% % \part[2] Determine la recta tangente a $f$ en el punto donde $x=3$
% % \part[2] Calcule las asíntotas oblicuas 	
% % \end{parts}


% \question Dada la función: $$f(x)=\left\{ \begin{matrix}
% \dfrac{x+2}{x-1} &  si & x\leq 2 \\
% \dfrac{3x^2-2x}{x+2} &  si & x > 2
% \end{matrix}\right.$$
% \begin{parts}
% \part[1] Estudie la continuidad de $f$ 
% \part[2] Determine la recta normal a $f$ en el punto donde $x=-2$
% \part[2] Calcule las asíntotas horizontales 	
% \end{parts}


\end{questions}

\end{document}
\grid
