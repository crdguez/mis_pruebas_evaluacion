\documentclass[addpoints,spanish, 12pt,a4paper]{exam}
%\documentclass[answers, spanish, 12pt,a4paper]{exam}
% \printanswers
\pointpoints{punto}{puntos}
\hpword{Puntos:}
\vpword{Puntos:}
\htword{Total}
\vtword{Total}
\hsword{Resultado:}
\hqword{Ejercicio:}
\vqword{Ejercicio:}

\usepackage[utf8]{inputenc}
\usepackage[spanish]{babel}
\usepackage{eurosym}
%\usepackage[spanish,es-lcroman, es-tabla, es-noshorthands]{babel}


\usepackage[margin=1in]{geometry}
\usepackage{amsmath,amssymb}
\usepackage{multicol}
\usepackage{yhmath}

\pointsinrightmargin % Para poner las puntuaciones a la derecha. Se puede cambiar. Si se comenta, sale a la izquierda.
\extrawidth{-2.4cm} %Un poquito más de margen por si ponemos textos largos.
\marginpointname{ \emph{\points}}

\usepackage{graphicx}

\graphicspath{{../img/}} 

\newcommand{\class}{2º Bachillerato CCSS}
\newcommand{\examdate}{\today}
\newcommand{\examnum}{Final 3ªEv.}
\newcommand{\tipo}{A}


\newcommand{\timelimit}{90 minutos}

\renewcommand{\solutiontitle}{\noindent\textbf{Solución:}\enspace}


\pagestyle{head}
\firstpageheader{\includegraphics[width=0.2\columnwidth]{header_left}}{\textbf{Departamento de Matemáticas\linebreak \class}\linebreak \examnum}{\includegraphics[width=0.1\columnwidth]{header_right}}
\runningheader{\class}{\examnum}{Página \thepage\ of \numpages}
\runningheadrule


\usepackage{pgf,tikz,pgfplots}
\pgfplotsset{compat=1.15}
\usepackage{mathrsfs}
\usetikzlibrary{arrows}


\begin{document}

\noindent
\begin{tabular*}{\textwidth}{l @{\extracolsep{\fill}} r @{\extracolsep{6pt}} }
\textbf{Nombre:} \makebox[3.5in]{\hrulefill} & \textbf{Fecha:}\makebox[1in]{\hrulefill} \\
 & \\
\textbf{Tiempo: \timelimit} & Tipo: \tipo 
\end{tabular*}
\rule[2ex]{\textwidth}{2pt}
Esta prueba tiene \numquestions\ ejercicios. La puntuación máxima es de \numpoints. 
La nota final de la prueba será la parte proporcional de la puntuación obtenida sobre la puntuación máxima. 

\begin{center}


\addpoints
 %\gradetable[h][questions]
	\pointtable[h][questions]
\end{center}

\noindent
\rule[2ex]{\textwidth}{2pt}

\begin{questions}

% \question[1] Se valorará hasta con un punto el uso de la notación adecuada en los ejercicios propuestos


% 22.9.4
\question Sean A y B dos sucesos asociados a un mismo experimento aleatorio. Suponga que $P(A) = 0. 7$, $P(\overline{B}) = 0.7$ y $P(A\cap B) = 0.2$. 
\begin{parts}
    \part[1] ¿Son A y B independientes? Justifique su respuesta.
    \part[1] Calcule $P(\overline{A} \cap \overline{B})$
\end{parts}

\question[1] Un examen de EVAU de Historia consiste en desarrollar 2 de los 4 temas propuestos. Si el temario se compone de 50 temas y un alumno se ha estudiado solo 10. ¿Cuál es la probabilidad de que se sepa al menos uno de los temas propuestos? 

\begin{solution}
Solución: Por costumbre utilizaremos B = Be. 
a) P(A n B) = O, 2 y P(A)P(B) = O, 7(1 — O, 7) = O, 21 P(A n B) # P(A)P(B) A y B no son independientes. 
b) P(A n B) = P(A U B) =1 — P(A U B) = 1 — (P(A) ± P(B) — P(A n B)) =1 — (O, 7 + O, 3 — O, 2) = O, 2  
\end{solution}

% Problema 21.1.4 (2 puntos) 
\question En un mercado agropecuario el 70\% de las verduras que se comercializan son de proximidad y el resto no. El 30\% de las verduras de proximidad son ecológicas, mientras que de las que no son de proximidad, solo son ecológicas el 10\%. Si un cliente elegido al azar ha realizado una compra de una verdura, calcule las siguientes probabilidades: 
\begin{parts}
    \part[1] Probabilidad de que la verdura comprada no sea ecológica
    \part[1] Probabilidad de que la verdura sea de proximidad o ecológica
\end{parts}
\begin{solution}
$0.76$ y $0.73$  
\end{solution}

% Problema 21.1.5 (2 puntos) 
\question El número de kilómetros que un corredor entrena a la semana mientras prepara una carrera popular se puede aproximar por una variable aleatoria de distribución normal de media $\mu$ horas y desviación típica $\sigma = 10$ horas
\begin{parts}
    \part[1] Se toma una muestra aleatoria simple de 20 atletas, obteniéndose una media muestral de 30 kilómetros. Determine un intervalo de confianza al 95\% para $\mu$
    \part[2] Suponga que $\mu=28$ kilómetros. Calcule la probabilidad de que al tomar una muestra aleatoria simple de 10 atletas, la media muestral, $\overline{X}$, esté entre 28 y 30 kilómetros
\end{parts}
\begin{solution}
    Solución: 
N(p; 10) 
a) n = 20, .TC = 30 y NC = 95% z012 = 1, 96 
10 , E= za./2=_ = 1,m = 4, 004 V20 
IC = (rC — E, Tc + E) = (30 — 4, 383; 30 + 4, 383) = (25, 617; 34, 383) 
b) p = 28 y rt = 10: 
28 — 28 30 — 28 P(28 <7(<30)=P( << a10/ )— 10/ alT  
P(0 < Z < 0,63) = P(Z < 0,63)— "Z < 0) = 0,7357 — 0.5 = 0,2357 
\end{solution}

% Problema 20.8.4 (2 puntos) 
\question En un instituto se decide que los alumnos y alumnas solo pueden utilizar un único color (azul o negro) al realizar los exámenes. Dos de cada tres exámenes están escritas en azul. La probabilidad de que un examen escrito en azul sea de una alumna es de 0,7. La probabilidad de que un examen esté escrito en negro y sea de un alumno es 0,2. Se elige un examen al azar. Calcule las siguientes probabilidades:
\begin{parts}
    \part[1] Sea el examen de un alumno
    \part[1] Sabiendo que está escrito en negro, sea de un alumno
\end{parts}
\begin{solution}
    $0.4$ y $0.6$
\end{solution}

% Problema 21.6.5 (2 puntos) 
\question Para que una determinada marca de chocolate estudie entre sus clientes la demanda de sus cajas de bombones, se desea estimar la proporción de cajas grandes en relación al número de cajas de bombones vendidas, P. 
\begin{parts}
    \part[1] Sabiendo que la proporción poblacional de la demanda es P = 0,2, determine el tamaño mínimo necesario de una muestra de ventas de cajas de bombones para garantizar que, con una confianza del 99\%, el margen de error en la estimación no sea mayor que $0.08$
    \part[2] Tomada al azar una muestra de 200 cajas de bombones vendidas, se encontró que 25 habían sido cajas grandes. Determine un intervalo de confianza al 95\% para la proporción de cajas grandes en relación a la venta total de cajas de bombones
\end{parts}


\begin{solution}
    $166$ y $(0,07916;0,17084)$
\end{solution}

\end{questions}

    \newgeometry{left=1 cm,bottom=2cm}
% \begin{landscape}
\begin{table}
% \Large
\centering

% \caption{Extracto de tabla de probabilidades de la \textbf{normal estándar $Z(0,1)$}}
\caption{Tabla de probabilidades de la \textbf{normal estándar $Z(0,1)$}}
\label{my-label}

\begin{tabular}{l|llllllllll}
z   & 0       & 0,01    & 0,02    & 0,03    & 0,04    & 0,05    & 0,06    & 0,07    & 0,08    & 0,09    \\
\hline
0   & 0,5     & 0,50399 & 0,50798 & 0,51197 & 0,51595 & 0,51994 & 0,52392 & 0,5279  & 0,53188 & 0,53586 \\
0,1 & 0,53983 & 0,5438  & 0,54776 & 0,55172 & 0,55567 & 0,55962 & 0,56356 & 0,56749 & 0,57142 & 0,57535 \\
0,2 & 0,57926 & 0,58317 & 0,58706 & 0,59095 & 0,59483 & 0,59871 & 0,60257 & 0,60642 & 0,61026 & 0,61409 \\
0,3 & 0,61791 & 0,62172 & 0,62552 & 0,6293  & 0,63307 & 0,63683 & 0,64058 & 0,64431 & 0,64803 & 0,65173 \\
0,4 & 0,65542 & 0,6591  & 0,66276 & 0,6664  & 0,67003 & 0,67364 & 0,67724 & 0,68082 & 0,68439 & 0,68793 \\
0,5 & 0,69146 & 0,69497 & 0,69847 & 0,70194 & 0,7054  & 0,70884 & 0,71226 & 0,71566 & 0,71904 & 0,7224  \\
0,6 & 0,72575 & 0,72907 & 0,73237 & 0,73565 & 0,73891 & 0,74215 & 0,74537 & 0,74857 & 0,75175 & 0,7549  \\
0,7 & 0,75804 & 0,76115 & 0,76424 & 0,7673  & 0,77035 & 0,77337 & 0,77637 & 0,77935 & 0,7823  & 0,78524 \\
0,8 & 0,78814 & 0,79103 & 0,79389 & 0,79673 & 0,79955 & 0,80234 & 0,80511 & 0,80785 & 0,81057 & 0,81327 \\
0,9 & 0,81594 & 0,81859 & 0,82121 & 0,82381 & 0,82639 & 0,82894 & 0,83147 & 0,83398 & 0,83646 & 0,83891 \\
1   & 0,84134 & 0,84375 & 0,84614 & 0,84849 & 0,85083 & 0,85314 & 0,85543 & 0,85769 & 0,85993 & 0,86214 \\
1,1 & 0,86433 & 0,8665  & 0,86864 & 0,87076 & 0,87286 & 0,87493 & 0,87698 & 0,879   & 0,881   & 0,88298 \\
1,2 & 0,88493 & 0,88686 & 0,88877 & 0,89065 & 0,89251 & 0,89435 & 0,89617 & 0,89796 & 0,89973 & 0,90147 \\
1,3 & 0,9032  & 0,9049  & 0,90658 & 0,90824 & 0,90988 & 0,91149 & 0,91309 & 0,91466 & 0,91621 & 0,91774 \\
1,4 & 0,91924 & 0,92073 & 0,9222  & 0,92364 & 0,92507 & 0,92647 & 0,92785 & 0,92922 & 0,93056 & 0,93189 \\
1,5 & 0,93319 & 0,93448 & 0,93574 & 0,93699 & 0,93822 & 0,93943 & 0,94062 & 0,94179 & 0,94295 & 0,94408 \\
1,6 & 0,9452  & 0,9463  & 0,94738 & 0,94845 & 0,9495  & 0,95053 & 0,95154 & 0,95254 & 0,95352 & 0,95449 \\
1,7 & 0,95543 & 0,95637 & 0,95728 & 0,95818 & 0,95907 & 0,95994 & 0,9608  & 0,96164 & 0,96246 & 0,96327 \\
1,8 & 0,96407 & 0,96485 & 0,96562 & 0,96638 & 0,96712 & 0,96784 & 0,96856 & 0,96926 & 0,96995 & 0,97062 \\
1,9 & 0,97128 & 0,97193 & 0,97257 & 0,9732  & 0,97381 & 0,97441 & 0,975   & 0,97558 & 0,97615 & 0,9767  \\
2   & 0,97725 & 0,97778 & 0,97831 & 0,97882 & 0,97932 & 0,97982 & 0,9803  & 0,98077 & 0,98124 & 0,98169 \\
2,1 & 0,98214 & 0,98257 & 0,983   & 0,98341 & 0,98382 & 0,98422 & 0,98461 & 0,985   & 0,98537 & 0,98574 \\
2,2 & 0,9861  & 0,98645 & 0,98679 & 0,98713 & 0,98745 & 0,98778 & 0,98809 & 0,9884  & 0,9887  & 0,98899 \\
2,3 & 0,98928 & 0,98956 & 0,98983 & 0,9901  & 0,99036 & 0,99061 & 0,99086 & 0,99111 & 0,99134 & 0,99158 \\
2,4 & 0,9918  & 0,99202 & 0,99224 & 0,99245 & 0,99266 & 0,99286 & 0,99305 & 0,99324 & 0,99343 & 0,99361 \\
2,5 & 0,99379 & 0,99396 & 0,99413 & 0,9943  & 0,99446 & 0,99461 & 0,99477 & 0,99492 & 0,99506 & 0,9952  \\
2,6 & 0,99534 & 0,99547 & 0,9956  & 0,99573 & 0,99585 & 0,99598 & 0,99609 & 0,99621 & 0,99632 & 0,99643 \\
2,7 & 0,99653 & 0,99664 & 0,99674 & 0,99683 & 0,99693 & 0,99702 & 0,99711 & 0,9972  & 0,99728 & 0,99736 \\
2,8 & 0,99744 & 0,99752 & 0,9976  & 0,99767 & 0,99774 & 0,99781 & 0,99788 & 0,99795 & 0,99801 & 0,99807 \\
2,9 & 0,99813 & 0,99819 & 0,99825 & 0,99831 & 0,99836 & 0,99841 & 0,99846 & 0,99851 & 0,99856 & 0,99861 \\
3   & 0,99865 & 0,99869 & 0,99874 & 0,99878 & 0,99882 & 0,99886 & 0,99889 & 0,99893 & 0,99896 & 0,999   \\
3,1 & 0,99903 & 0,99906 & 0,9991  & 0,99913 & 0,99916 & 0,99918 & 0,99921 & 0,99924 & 0,99926 & 0,99929 \\
3,2 & 0,99931 & 0,99934 & 0,99936 & 0,99938 & 0,9994  & 0,99942 & 0,99944 & 0,99946 & 0,99948 & 0,9995  \\
3,3 & 0,99952 & 0,99953 & 0,99955 & 0,99957 & 0,99958 & 0,9996  & 0,99961 & 0,99962 & 0,99964 & 0,99965 \\
3,4 & 0,99966 & 0,99968 & 0,99969 & 0,9997  & 0,99971 & 0,99972 & 0,99973 & 0,99974 & 0,99975 & 0,99976 \\
3,5 & 0,99977 & 0,99978 & 0,99978 & 0,99979 & 0,9998  & 0,99981 & 0,99981 & 0,99982 & 0,99983 & 0,99983 \\
3,6 & 0,99984 & 0,99985 & 0,99985 & 0,99986 & 0,99986 & 0,99987 & 0,99987 & 0,99988 & 0,99988 & 0,99989 \\
3,7 & 0,99989 & 0,9999  & 0,9999  & 0,9999  & 0,99991 & 0,99991 & 0,99992 & 0,99992 & 0,99992 & 0,99992 \\
3,8 & 0,99993 & 0,99993 & 0,99993 & 0,99994 & 0,99994 & 0,99994 & 0,99994 & 0,99995 & 0,99995 & 0,99995 \\
3,9 & 0,99995 & 0,99995 & 0,99996 & 0,99996 & 0,99996 & 0,99996 & 0,99996 & 0,99996 & 0,99997 & 0,99997 \\
4   & 0,99997 & 0,99997 & 0,99997 & 0,99997 & 0,99997 & 0,99997 & 0,99998 & 0,99998 & 0,99998 & 0,99998
\end{tabular}
\end{table}
% \end{landscape}
\restoregeometry

\end{document}
\grid
