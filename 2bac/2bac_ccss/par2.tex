\documentclass[addpoints,spanish, 12pt,a4paper]{exam}
%\documentclass[answers, spanish, 12pt,a4paper]{exam}
\printanswers
\pointpoints{punto}{puntos}
\hpword{Puntos:}
\vpword{Puntos:}
\htword{Total}
\vtword{Total}
\hsword{Resultado:}
\hqword{Ejercicio:}
\vqword{Ejercicio:}

\usepackage[utf8]{inputenc}
\usepackage[spanish]{babel}
\usepackage{eurosym}
%\usepackage[spanish,es-lcroman, es-tabla, es-noshorthands]{babel}


\usepackage[margin=1in]{geometry}
\usepackage{amsmath,amssymb}
\usepackage{multicol}
\usepackage{yhmath}

\pointsinrightmargin % Para poner las puntuaciones a la derecha. Se puede cambiar. Si se comenta, sale a la izquierda.
\extrawidth{-2.4cm} %Un poquito más de margen por si ponemos textos largos.
\marginpointname{ \emph{\points}}

\usepackage{graphicx}

\graphicspath{{../../img/}} 

\newcommand{\class}{2º Bachillerato CCSS}
\newcommand{\examdate}{\today}
\newcommand{\examnum}{Parcial 2ªEv.}
\newcommand{\tipo}{B}


\newcommand{\timelimit}{45 minutos}

\renewcommand{\solutiontitle}{\noindent\textbf{Solución:}\enspace}


\pagestyle{head}
\firstpageheader{\includegraphics[width=0.2\columnwidth]{header_left}}{\textbf{Departamento de Matemáticas\linebreak \class}\linebreak \examnum}{\includegraphics[width=0.1\columnwidth]{header_right}}
\runningheader{\class}{\examnum}{Página \thepage\ of \numpages}
\runningheadrule


\usepackage{pgf,tikz,pgfplots}
\pgfplotsset{compat=1.15}
\usepackage{mathrsfs}
\usetikzlibrary{arrows}


\begin{document}

\noindent
\begin{tabular*}{\textwidth}{l @{\extracolsep{\fill}} r @{\extracolsep{6pt}} }
\textbf{Nombre:} \makebox[3.5in]{\hrulefill} & \textbf{Fecha:}\makebox[1in]{\hrulefill} \\
 & \\
\textbf{Tiempo: \timelimit} & Tipo: \tipo 
\end{tabular*}
\rule[2ex]{\textwidth}{2pt}
Esta prueba tiene \numquestions\ ejercicios. La puntuación máxima es de \numpoints. 
La nota final de la prueba será la parte proporcional de la puntuación obtenida sobre la puntuación máxima. 

\begin{center}


\addpoints
 %\gradetable[h][questions]
	\pointtable[h][questions]
\end{center}

\noindent
\rule[2ex]{\textwidth}{2pt}

\begin{questions}

%\question 
%
%\begin{parts}
%\part[2] 
%\begin{solution}
%\end{solution}
%
%
%\end{parts}
%\addpoints

\question Calcula:


\begin{parts}
% \part[2] 
% $$\lim_{x \to -1}\left(\dfrac{x^{3} + x^{2} - x - 1}{2 x^{3} + 5 x^{2} + 4 x + 1}\right)$$
% \begin{solution}
% $2$
% \end{solution}
% \part[2] 
% $$\lim_{x \to 3}\left(\frac{\sqrt{x + 1} - 2}{x^{2} - 3 x}\right)$$
% \begin{solution}
% $\frac{1}{12}$
% \end{solution}
% \part[1] 
% $$\lim_{x \to -\infty} \left(\dfrac{x+3}{2x}\right)^{\frac{x^{2}}{x - 3}}$$
% \begin{solution}
% $\frac{1}{2^{-\infty}}=\infty$
% \end{solution}

\part[2] 
$$\lim_{x \to -1}\left(\frac{x^{3} - 3 x - 2}{2 x^{3} + 5 x^{2} + 4 x + 1}\right)$$
\begin{solution}
$\lim_{x \to -1}\left(\frac{x - 2}{2 x + 1}\right)=3$
\end{solution}
\part[2] 
$$\lim_{x \to 8}\left(\frac{\sqrt{x + 1} - 3}{x^{2} - 8 x}\right)$$
\begin{solution}
$\frac{1}{48}$
\end{solution}
\part[1] 
$$\lim_{x \to \infty} \left(\dfrac{x+3}{2x}\right)^{\frac{x^{2}}{x - 3}}$$
\begin{solution}
$\frac{1}{2^{\infty}}=0$
\end{solution}
\end{parts}
\addpoints

% \question Se considera la función $$f(x)=\left\{ \begin{matrix}
% ax^2-1 &  si & x<1 \\
% (x-a)^2 &  si & x \geq 1
% \end{matrix}\right.$$ 

% \begin{parts}
% \part[2] Determine los valores de 
% $a \in \mathbb{R}$ que hacen que $f$ es continua en su dominio
% \begin{solution}
% $f=\begin{cases} a x^{2} - 1 & \text{for}\: x < 1 \\\left(- a + x\right)^{2} & \text{otherwise} \end{cases}
% $ \\
% $\lim_{x \to 1^-} f=a - 1
%  \land \lim_{x \to 1^-} f=\left(1 - a\right)^{2} \to - a^{2} + 3 a - 2
% =0 \to a=1, a=2$
% \end{solution}
% \part[2] Para 
% $a=\dfrac{1}{2}$, determine, si existen, los puntos de corte de la gráfica con el eje de las $x$
% \begin{solution}
% $\dfrac{{x}^{2}}{2} - 1=0 \to x=-\sqrt{2}$

% $\left(x - \frac{1}{2}\right)^{2}
% \to x=\frac{1}{2}=0 \notin x\geq1$
% \end{solution}
% \end{parts}
% \addpoints


% \question Dada la función $f(x)=ax+\frac{b}{x}$:
% \begin{parts}
% \part[2] Determine los valores de los parámetros 
% $a$, $b \in \mathbb{R}$  para que pase por el punto $(2,4)$ y tenga un extremo relativo en ese punto.
% \begin{solution}
% $ 
% \left\{
% \begin{matrix}
% f(2)=4 \\
% f'(2)=0
% \end{matrix}
% \right.
% \to f'(x)=a-\frac{b}{x^2}
% \to
% \left\{
% \begin{matrix}
% 2 a + \frac{b}{2}
% =4 \\
% a - \frac{b}{4}
% =0
% \end{matrix}
% \right.
% \to a=1 \land b=4$
% \end{solution}
% \part[2] Justifica qué tipo de extremo relativo es (máximo relativo o mínimo relativo)
% \begin{solution}
% $f''(x)=\frac{2b}{x^3} \to f''(2)=\frac{b}{4} \to f''(2)=\frac{4}{4}=1>0 \to MIN REL$ 
% \end{solution}
% \end{parts}




% \question Sea la función $f(x)=\dfrac{x^2-x+1}{x-1}$:
% \begin{parts}
% \part[1] Determine sus asíntotas
% \begin{solution}
% $\lim_{x \to 1^-}\left(\frac{x^{2} - x + 1}{x - 1}\right)= -\infty
% \land
% \lim_{x \to 1^+}\left(\frac{x^{2} - x + 1}{x - 1}\right)=\infty \to x=1 A.V.
% $\\
% $\lim_{x \to \infty}\left(\frac{x^{2} - x + 1}{x - 1}\right)=\infty
% \land
% \lim_{x \to -\infty}\left(\frac{x^{2} - x + 1}{x - 1}\right)=-\infty \to \nexists A.H.
% $ \\
% $\lim_{x \to \infty}\frac{f(x)}{x}=\left(\frac{x^{2} - x + 1}{x^2 - x}\right)=1 \to m = 1 \land \lim_{x \to \infty} \left(f(x)-mx\right)=0 \to y=x A.O.$
% \end{solution}
% \part[1] Calcule $f'(2)$
% \begin{solution}
% $f'(x)=\frac{x \left(x - 2\right)}{x^{2} - 2 x + 1} \to f'(2)=0$
% \end{solution}
% \end{parts}
% \addpoints

% \question Se considera la función $f(x)=\frac{10}{x^2+2x-3}$:
% \begin{parts}
% \part[1] Determine el dominio de $f$ y sus asíntotas
% \begin{solution}$f(x)=0 \to x= -3, x=1 \to \lim_{x \to -3}=\infty \land \lim_{x \to -3}=\infty \to x=-3 A.V. x=-1 A.V.$\\$\lim_{x\to\infty}f=\dfrac{10}{\infty}=0 \to y =0  A.H.$
% \end{solution}
% \part[1] Obtenga los intervalos de crecimiento y decrecimiento de $f(x)$ y determine los extremos relativos indicando si corresponden a máximos o mínimos 	
% \begin{solution}$f'(x)=\frac{10 \left(- 2 x - 2\right)}{\left(x^{2} + 2 x - 3\right)^{2}}$\\$f'(x)=0 \to x =-1$\\$f''(-1)=-\frac{5}{4}\to \left(-1,-\frac{5}{2}\right) MAX \ REL$
% \end{solution}
% \end{parts}



% \question Dada la función $f(x)=\dfrac{2x}{1-x^2}$:
% \begin{parts}
% \part[2] Determine el dominio de $f$ y sus asíntotas
% \part[2] Obtenga los intervalos de crecimiento y decrecimiento de $f(x)$
% \part[2] Determine los extremos relativos indicando si corresponden a máximos o mínimos 	
% \end{parts}


\question Dada la función $f(x)=\dfrac{-3x}{1-x^2}$:
\begin{parts}
\part[2] Determine el dominio de $f$ y sus asíntotas
% \part[2] Obtenga los intervalos de crecimiento y decrecimiento de $f(x)$
% \part[2] Determine los extremos relativos indicando si corresponden a máximos o mínimos 	
\end{parts}



% \question Dada la función $f(x)=x+\sqrt{1-x}$:
% \begin{parts}
% \part[1] Determine el dominio de $f$ 
% \part[2] Obtenga los intervalos de crecimiento y decrecimiento de $f(x)$ y determine los extremos relativos indicando si corresponden a máximos o mínimos 	
% \end{parts}

\question Calcula y simplifica las derivadas de las siguientes funciones:
\begin{parts}
    \part[2] $f(x)=\left(x^{4} - 2\right) \left(x^{4} + 2\right)$ \begin{solution} $4 x^{3} \left(x^{4} - 2\right) + 4 x^{3} \left(x^{4} + 2\right)= 8 x^{7}$\end{solution}
    % \part[2] $f(x)=\ln{\left(\dfrac{x - 1}{x + 1} \right)}$ \begin{solution}$\frac{\left(x + 1\right) \left(- \frac{x - 1}{\left(x + 1\right)^{2}} + \frac{1}{x + 1}\right)}{x - 1}=\frac{2}{x^{2} - 1}$\end{solution}
    \part[2] $f(x)=x^{5} \ln{\left(x \right)}$ \begin{solution}$ x^{4} \ln{\left(x \right)} + x^{4}$\end{solution}
    % \part[2] $f(x)=x^{3} e^{2 x^{2}}$ \begin{solution}$4 x^{4} e^{2 x^{2}} + 3 x^{2} e^{2 x^{2}}$ \end{solution}
\end{parts}


% \question Dada la función: $$f(x)=\left\{ \begin{matrix}
% \dfrac{x+2}{x-1} &  si & x\leq 2 \\
% \dfrac{3x^2-2x}{x+2} &  si & x > 2
% \end{matrix}\right.$$
% \begin{parts}
% \part[1] Estudie la continuidad de $f$ 
% \part[2] Determine la recta tangente a $f$ en el punto donde $x=3$
% \part[2] Calcule las asíntotas oblicuas 	
% \end{parts}




\question Dada la función: $$f(x)=\left\{ \begin{matrix}
\dfrac{x+2}{x-1} &  si & x\leq 2 \\
\dfrac{3x^2-2x}{x+2} &  si & x > 2
\end{matrix}\right.$$
\begin{parts}
\part[2] Estudie la continuidad de $f$ \begin{solution}$\mathbf{R}-\left\{ 1, 2 \right\}$\end{solution} 
\part[2] Determine la recta normal a $f$ en el punto donde $x=-2$\begin{solution}$y=3x+6$\end{solution}
% \part[2] Calcule las asíntotas horizontales 	
\end{parts}


\end{questions}

\end{document}
\grid
