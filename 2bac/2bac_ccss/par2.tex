\documentclass[addpoints,spanish, 12pt,a4paper]{exam}
%\documentclass[answers, spanish, 12pt,a4paper]{exam}
\printanswers
\pointpoints{punto}{puntos}
\hpword{Puntos:}
\vpword{Puntos:}
\htword{Total}
\vtword{Total}
\hsword{Resultado:}
\hqword{Ejercicio:}
\vqword{Ejercicio:}

\usepackage[utf8]{inputenc}
\usepackage[spanish]{babel}
\usepackage{eurosym}
%\usepackage[spanish,es-lcroman, es-tabla, es-noshorthands]{babel}


\usepackage[margin=1in]{geometry}
\usepackage{amsmath,amssymb}
\usepackage{multicol}
\usepackage{yhmath}

\pointsinrightmargin % Para poner las puntuaciones a la derecha. Se puede cambiar. Si se comenta, sale a la izquierda.
\extrawidth{-2.4cm} %Un poquito más de margen por si ponemos textos largos.
\marginpointname{ \emph{\points}}

\usepackage{graphicx}

\graphicspath{{../../img/}} 

\newcommand{\class}{2º Bachillerato CCSS}
\newcommand{\examdate}{\today}
\newcommand{\examnum}{Parcial 2ªEv.}
\newcommand{\tipo}{A}


\newcommand{\timelimit}{45 minutos}

\renewcommand{\solutiontitle}{\noindent\textbf{Solución:}\enspace}


\pagestyle{head}
\firstpageheader{\includegraphics[width=0.2\columnwidth]{header_left}}{\textbf{Departamento de Matemáticas\linebreak \class}\linebreak \examnum}{\includegraphics[width=0.1\columnwidth]{header_right}}
\runningheader{\class}{\examnum}{Página \thepage\ of \numpages}
\runningheadrule


\usepackage{pgf,tikz,pgfplots}
\pgfplotsset{compat=1.15}
\usepackage{mathrsfs}
\usetikzlibrary{arrows}


\begin{document}

\noindent
\begin{tabular*}{\textwidth}{l @{\extracolsep{\fill}} r @{\extracolsep{6pt}} }
\textbf{Nombre:} \makebox[3.5in]{\hrulefill} & \textbf{Fecha:}\makebox[1in]{\hrulefill} \\
 & \\
\textbf{Tiempo: \timelimit} & Tipo: \tipo 
\end{tabular*}
\rule[2ex]{\textwidth}{2pt}
Esta prueba tiene \numquestions\ ejercicios. La puntuación máxima es de \numpoints. 
La nota final de la prueba será la parte proporcional de la puntuación obtenida sobre la puntuación máxima. 

\begin{center}


\addpoints
 %\gradetable[h][questions]
	\pointtable[h][questions]
\end{center}

\noindent
\rule[2ex]{\textwidth}{2pt}

\begin{questions}

%\question 
%
%\begin{parts}
%\part[2] 
%\begin{solution}
%\end{solution}
%
%
%\end{parts}
%\addpoints

\question Se considera la función $$f(x)=\left\{ \begin{matrix}
ax^2-1 &  si & x<1 \\
(x-a)^2 &  si & x \geq 1
\end{matrix}\right.$$ 

\begin{parts}
\part[2] Determine los valores de 
$a \in \mathbb{R}$ que hacen que $f$ es continua en su dominio
\begin{solution}
$f=\begin{cases} a x^{2} - 1 & \text{for}\: x < 1 \\\left(- a + x\right)^{2} & \text{otherwise} \end{cases}
$ \\
$\lim_{x \to 1^-} f=a - 1
 \land \lim_{x \to 1^-} f=\left(1 - a\right)^{2} \to - a^{2} + 3 a - 2
=0 \to a=1, a=2$
\end{solution}
\part[2] Para 
$a=\dfrac{1}{2}$, determine, si existen, los puntos de corte de la gráfica con el eje de las $x$
\begin{solution}
$\dfrac{{x}^{2}}{2} - 1=0 \to x=-\sqrt{2}$

$\left(x - \frac{1}{2}\right)^{2}
\to x=\frac{1}{2}=0 \notin x\geq1$
\end{solution}


\end{parts}
\addpoints


\question[1] Dada la función $f(x)=ax+\frac{b}{x}$:


\begin{parts}
\part[2] Determine los valores de los parámetros 
$a$, $b \in \mathbb{R}$  para que pase por el punto $(2,4)$ y tenga un extremo relativo en ese punto.
\begin{solution}
$ 
\left\{
\begin{matrix}
f(2)=4 \\
f'(2)=0
\end{matrix}
\right.
\to f'(x)=a-\frac{b}{x^2}
\to
\left\{
\begin{matrix}
2 a + \frac{b}{2}
=4 \\
a - \frac{b}{4}
=0
\end{matrix}
\right.
\to a=1 \land b=4$

\end{solution}
\part[1] Justifica qué tipo de extremo relativo es (máximo relativo o mínimo relativo)
\begin{solution}
$f''(x)=\frac{2b}{x^3} \to f''(2)=\frac{b}{4} \to f''(2)=\frac{4}{4}=1>0 \to MIN REL$ 
\end{solution}
\end{parts}

\question Sea la función $f(x)=\frac{x^2-x+1}{x-1}$:
\begin{parts}
\part[1] Determine sus asíntotas
\begin{solution}
$\lim_{x \to 1^-}\left(\frac{x^{2} - x + 1}{x - 1}\right)= -\infty
\land
\lim_{x \to 1^+}\left(\frac{x^{2} - x + 1}{x - 1}\right)=\infty \to x=1 A.V.
$\\
$\lim_{x \to \infty}\left(\frac{x^{2} - x + 1}{x - 1}\right)=\infty
\land
\lim_{x \to -\infty}\left(\frac{x^{2} - x + 1}{x - 1}\right)=-\infty \to \nexists A.H.
$ \\
$\lim_{x \to \infty}\frac{f(x)}{x}=\left(\frac{x^{2} - x + 1}{x^2 - x}\right)=1 \to m = 1 \land \lim_{x \to \infty} \left(f(x)-mx\right)=0 \to y=x A.O.$


\end{solution}
\part[1] Calcule $f'(2)$
\begin{solution}
$f'(x)=\frac{x \left(x - 2\right)}{x^{2} - 2 x + 1} \to f'(2)=0$
\end{solution}
\end{parts}
\addpoints

\question 22.2.2 Se considera la función $f(x)=\frac{10}{x^2+2x-3}$:
\begin{parts}
\part[1] Determine el dominio de $f$ y sus asíntotas

\part[1] Obtenga los intervalos de crecimiento y decrecimiento de $f(x)$ y determine los extremos relativos indicando si corresponden a máximos o mínimos 	
\end{parts}

\end{questions}

\end{document}
\grid
