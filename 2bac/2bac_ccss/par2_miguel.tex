\documentclass[addpoints,spanish, 12pt,a4paper]{exam}
%\documentclass[answers, spanish, 12pt,a4paper]{exam}
% \printanswers
\pointpoints{punto}{puntos}
\hpword{Puntos:}
\vpword{Puntos:}
\htword{Total}
\vtword{Total}
\hsword{Resultado:}
\hqword{Ejercicio:}
\vqword{Ejercicio:}

\usepackage[utf8]{inputenc}
\usepackage[spanish]{babel}
\usepackage{eurosym}
%\usepackage[spanish,es-lcroman, es-tabla, es-noshorthands]{babel}


\usepackage[margin=1in]{geometry}
\usepackage{amsmath,amssymb}
\usepackage{multicol}
\usepackage{yhmath}

\pointsinrightmargin % Para poner las puntuaciones a la derecha. Se puede cambiar. Si se comenta, sale a la izquierda.
\extrawidth{-2.4cm} %Un poquito más de margen por si ponemos textos largos.
\marginpointname{ \emph{\points}}

\usepackage{graphicx}
\graphicspath{{../img/}} 

\newcommand{\class}{2º Bachillerato CCSS}
\newcommand{\examdate}{\today}
\newcommand{\examnum}{Parcial 2ªEv.}
\newcommand{\tipo}{A}


\newcommand{\timelimit}{50 minutos}

\renewcommand{\solutiontitle}{\noindent\textbf{Solución:}\enspace}


\pagestyle{head}
\firstpageheader{\includegraphics[width=0.2\columnwidth]{header_left}}{\textbf{Departamento de Matemáticas\linebreak \class}\linebreak \examnum}{\includegraphics[width=0.1\columnwidth]{header_right}}
\runningheader{\class}{\examnum}{Página \thepage\ of \numpages}
\runningheadrule


\begin{document}

\noindent
\begin{tabular*}{\textwidth}{l @{\extracolsep{\fill}} r @{\extracolsep{6pt}} }
\textbf{Nombre:} \makebox[3.5in]{\hrulefill} & \textbf{Fecha:}\makebox[1in]{\hrulefill} \\
 & \\
\textbf{Tiempo: \timelimit} & Tipo: \tipo 
\end{tabular*}
\rule[2ex]{\textwidth}{2pt}
Esta prueba tiene \numquestions\ ejercicios. La puntuación máxima es de \numpoints. 
La nota final de la prueba será la parte proporcional de la puntuación obtenida sobre la puntuación máxima. 

\begin{center}


\addpoints
 %\gradetable[h][questions]
	\pointtable[h][questions]
\end{center}

\noindent
\rule[2ex]{\textwidth}{2pt}

\begin{questions}

\question[2] Halla el límite de la función $f(x)=\dfrac{x^{2} - 2 x}{2 x^{2} - 8}$ en $x=2$, $x=-2$, $x=\infty$ y $x=-\infty$  

%\noaddpoints % to omit double points count

\begin{solution}
$f(x)=\frac{x^{2} - 2 x}{2 x^{2} - 8}=\frac{x}{2 \left(x + 2\right)}$ . \\  \\ $\lim_{x \to 2^+}\left(\frac{x^{2} - 2 x}{2 x^{2} - 8}\right)=\frac{1}{4}$ \\ $\lim_{x \to 2^-}\left(\frac{x^{2} - 2 x}{2 x^{2} - 8}\right)=\frac{1}{4}$ \\ $\lim_{x \to -2^+}\left(\frac{x^{2} - 2 x}{2 x^{2} - 8}\right)=-\infty$ \\ $\lim_{x \to -2^-}\left(\frac{x^{2} - 2 x}{2 x^{2} - 8}\right)=\infty$ \\ $\lim_{x \to \infty}\left(\frac{x^{2} - 2 x}{2 x^{2} - 8}\right)=\frac{1}{2}$ \\ $\lim_{x \to \infty}\left(\frac{x^{2} - 2 x}{2 x^{2} - 8}\right)=\frac{1}{2}$ \\ $\lim_{x \to -\infty}\left(\frac{x^{2} - 2 x}{2 x^{2} - 8}\right)=\frac{1}{2}$ \\ $\lim_{x \to -\infty}\left(\frac{x^{2} - 2 x}{2 x^{2} - 8}\right)=\frac{1}{2}$

\end{solution}


\addpoints

\question Calcula y simplifica las siguientes derivadas: 

\begin{parts}


\part[1] $y=\ln(\frac{x}{3}+1)$ 
\begin{solution} $\frac{1}{3 \left(\frac{x}{3} + 1\right)}$ \end{solution}
\part[1]  $y=e^{2x+1}$ 
\begin{solution} $ 2 e^{2 x + 1}$\end{solution}
\part[2]   $y=\left(\dfrac{1-x}{1+x}\right)^2$
\begin{solution} $- \frac{2 \left(1 - x\right)^{2}}{\left(x + 1\right)^{3}} + \frac{2 x - 2}{\left(x + 1\right)^{2}}=\frac{4 \left(x - 1\right)}{\left(x + 1\right)^{3}}$ \end{solution}
\end{parts}


\question Dada la función $$f(x)=x+\sqrt{1-x}$$:
\begin{parts}
\part[1] Determine el dominio de $f$
\part[2] Determine las asíntotas de $f$
% \part[2] Obtenga los intervalos de crecimiento y decrecimiento de $f(x)$ y determine los extremos relativos indicando si corresponden a máximos o mínimos 	
\end{parts}


\question Dada la función:
$$f(x)=\begin{cases} x^{2} + 2 x - 1 & si \ x \leq 1 \\\dfrac{4}{x + 1} & si \ x > 1\end{cases}$$ 

\begin{parts}

\part[1] Estudia la continuidad

% \part[1] Estudia la derivabilidad

\part[1]   ¿Existe algún punto donde $f'(x)=0$?

\end{parts}\begin{solution} Las funciones parciales son continuas y derivables en todo su dominio por ser polinómicas y de proporcionalidad inversa. \\
$dom(x^{2} + 2 x - 1)=\mathbb{R} \to$ continua y derivable en $\left(- \infty , 1 \right]$ \\
$ dom(\dfrac{4}{x + 1})= \mathbb{R}-\{ -1 \} \to $ continua y derivable en $\left(1 , \infty \right)$\\
Continuidad en $x=1$: \\
$\lim_{x \to 1^-} f =1^2+2\cdot 1-1=2 \land \lim_{x \to 1^+} f =\frac{4}{1+1}=2 \to$ es continua
$f'(x)=\begin{cases} 2x + 2 & si \ x \leq 1 \\\dfrac{-4}{(x + 1)^2} & si \ x > 1\end{cases}$ \\
Derivabilidad en $x=1$: \\
$\lim_{x \to 1^-}f'=2\cdot 1+2=3 \land \lim_{x \to 1^+}f'=\frac{-4}{(1+1)^2}=-1 \to$ no es derivable


 

 \end{solution}


\addpoints


% \question Sea la función  $$\frac{x^{2} - 3 x + 3}{x - 1}$$.
% %\noaddpoints % to omit double points count
% \begin{parts}
% \part[2] Determinar los intervalos de crecimiento y decrecimiento

% \part[1] Determinar los extremos relativos  

% \part[2] Determinar los intervalos de concavidad y convexidad

% \part[1] Determinar los puntos de inflexión  

% \part[2] Determina sus asíntotas  

% \end{parts}

% \begin{solution}
% \includegraphics[scale=0.3]{pa2_1}

% $f(x)=\frac{x^{2} - 3 x + 3}{x - 1}=\frac{x^{2} - 3 x + 3}{x - 1}$ . \\ Dominio de continuidad: $\left(-\infty, 1\right) \cup \left(1, \infty\right)$ . \\ $f'(x)=\frac{2 x - 3}{x - 1} - \frac{x^{2} - 3 x + 3}{\left(x - 1\right)^{2}}=\frac{x \left(x - 2\right)}{x^{2} - 2 x + 1}$ . \\ $f''(x)=\frac{2}{x - 1} - \frac{2 \left(2 x - 3\right)}{\left(x - 1\right)^{2}} + \frac{2 \left(x^{2} - 3 x + 3\right)}{\left(x - 1\right)^{3}}=\frac{2}{x^{3} - 3 x^{2} + 3 x - 1}$ . \\ $f'''(x)=- \frac{6}{\left(x - 1\right)^{2}} + \frac{6 \left(2 x - 3\right)}{\left(x - 1\right)^{3}} - \frac{6 \left(x^{2} - 3 x + 3\right)}{\left(x - 1\right)^{4}}=- \frac{6}{x^{4} - 4 x^{3} + 6 x^{2} - 4 x + 1}$ . \\ Asíntotas:\\A.V. $x=1$\\A.O. $y=x - 2$ \\A.O. $y=x - 2$ \\ \\ Crecimiento: $\left(-\infty, 0\right) \cup \left(2, \infty\right)$ \\ Decrecimiento: $\left(0, 1\right) \cup \left(1, 2\right) $\\ Extremos relativos: $\left[ \left[ 0, \  max\right], \  \left[ 2, \  min\right]\right]$ \\ Concavidad: $\left(1, \infty\right)$ \\ Convexidad: $\left(-\infty, 1\right) $\\ Puntos de inflexión: $\left[ \right]$ \\



% \end{solution}

\addpoints



% \question[4] Tenemos que hacer dos chapas cuadradas de dos  materiales distintos de 2 y 3 \euro /$cm^2$ respectivamente. ¿Cómo hemos de elegir los lados de los cuadrados si queremos que el coste total sea mínimo y que la suma de los perímetros de los dos cuadrados sea de un metro?

% \begin{solution}
% $x \to$ lado del cuadrado de 2\euro /$cm^2$ \\
% $y \to$ lado del cuadrado de 3\euro /$cm^2$ \\
% De la condición del perímetro$4x+4y=100 \to y=25-x$ \\
% La función a optimizar es: $$f(x)=2x^2+3(25-x)^2$$
% Extremos relativos: \\
% $f'(x)=4x+6(x-25)=10x-150 \to f'(x)=0 \iff x=15$
% , como $f''(x)=10>0 \to f''(15)=10>0$ y por tanto en $x=15$ hay un mínimo relativo.\\


% Luego son cuadrados de lado 10 y 15 cm
% \end{solution}

\addpoints


\addpoints




\end{questions}

\end{document}
\grid