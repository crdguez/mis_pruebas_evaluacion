\documentclass[addpoints,spanish, 12pt,a4paper]{exam}
%\documentclass[answers, spanish, 12pt,a4paper]{exam}
\printanswers
\pointpoints{punto}{puntos}
\hpword{Puntos:}
\vpword{Puntos:}
\htword{Total}
\vtword{Total}
\hsword{Resultado:}
\hqword{Ejercicio:}
\vqword{Ejercicio:}

\usepackage[utf8]{inputenc}
\usepackage[spanish]{babel}
\usepackage{eurosym}
%\usepackage[spanish,es-lcroman, es-tabla, es-noshorthands]{babel}


\usepackage[margin=1in]{geometry}
\usepackage{amsmath,amssymb}
\usepackage{multicol}
\usepackage{yhmath}

\pointsinrightmargin % Para poner las puntuaciones a la derecha. Se puede cambiar. Si se comenta, sale a la izquierda.
\extrawidth{-2.4cm} %Un poquito más de margen por si ponemos textos largos.
\marginpointname{ \emph{\points}}

\usepackage{graphicx}

\graphicspath{{../../img/}} 

\newcommand{\class}{2º Bachillerato CCSS}
\newcommand{\examdate}{\today}
\newcommand{\examnum}{Final 1ªEv.}
\newcommand{\tipo}{A}


\newcommand{\timelimit}{105 minutos}

\renewcommand{\solutiontitle}{\noindent\textbf{Solución:}\enspace}


\pagestyle{head}
\firstpageheader{\includegraphics[width=0.2\columnwidth]{header_left}}{\textbf{Departamento de Matemáticas\linebreak \class}\linebreak \examnum}{\includegraphics[width=0.1\columnwidth]{header_right}}
\runningheader{\class}{\examnum}{Página \thepage\ of \numpages}
\runningheadrule


\usepackage{pgf,tikz,pgfplots}
\pgfplotsset{compat=1.15}
\usepackage{mathrsfs}
\usetikzlibrary{arrows}


\begin{document}

\noindent
\begin{tabular*}{\textwidth}{l @{\extracolsep{\fill}} r @{\extracolsep{6pt}} }
\textbf{Nombre:} \makebox[3.5in]{\hrulefill} & \textbf{Fecha:}\makebox[1in]{\hrulefill} \\
 & \\
\textbf{Tiempo: \timelimit} & Tipo: \tipo 
\end{tabular*}
\rule[2ex]{\textwidth}{2pt}
Esta prueba tiene \numquestions\ ejercicios. La puntuación máxima es de \numpoints. 
La nota final de la prueba será la parte proporcional de la puntuación obtenida sobre la puntuación máxima. 

\begin{center}


\addpoints
 %\gradetable[h][questions]
	\pointtable[h][questions]
\end{center}

\noindent
\rule[2ex]{\textwidth}{2pt}

\begin{questions}

\question Dado el siguiente sistema:
$$\left\{ \begin{matrix}x + y + z = -2 \\ - k x + 3 y + z = -7 \\ x + 2 y + z \left(k + 2\right) = -5 \\ \end{matrix}\right.
$$
\begin{parts}
    \part[2]  Estudiar las soluciones del sistema según los valores del parámetro $k$.
    \begin{solution}
    $|A|=\left|\begin{matrix}1 & 1 & 1\\- k & 3 & 1\\1 & 2 & k + 2\end{matrix}\right|=\left(k + 1\right) \left(k + 2\right)$
    \\ - Si $k \neq -2, -1\to |A| \neq 0 \to \exists A^{-1}$ 
    \\ Como $rg(A)=rg(A^*)=3$  --> S.C.D -- > Se puede resolver por Gauss, Matriz inversa o Cramer
    \\ - Si $k=-2\to |A| = 0 \to \nexists A^{-1}$
        \\ Como $rg(A)=rg(A^*)=2 \to$  S.C.I --> Solo se puede resolver por Gauss
        \\ - Si $k=-1\to |A| = 0 \to \nexists A^{-1}$
        \\ Como $rg(A)=2 \land rg(A^*)=3 \to$  S.I. 

    \end{solution}
    \part[2]  Resolver el sistema cuando sea compatible indeterminado.
    \begin{solution}$A^*=\left(\begin{matrix}1 & 1 & 1 & -2\\2 & 3 & 1 & -7\\1 & 2 & 0 & -5\end{matrix}\right)\sim\left(\begin{matrix}1 & 1 & 1 & -2\\0 & 1 & -1 & -3\\0 & 0 & 0 & 0\end{matrix}\right)\to x=1 - 2 \lambda, y=\lambda - 3, z=\lambda$\end{solution}
\end{parts}

\question Dadas las matrices  $A=\left(\begin{matrix}k & k & k^{2}\\1 & -1 & k\\2 k & -2 & 2\end{matrix}\right)$,  $B=\left(\begin{matrix}12\\6\\8\end{matrix}\right)$
\begin{parts}
\part[2] Hallar el rango de $A$ en función de los valores de $k$.
\begin{solution} 
- Si $k \neq -1, 0, 1\to |A| \neq 0 \to \exists A^{-1} \to rg(A)=3$
- Si no $rg(A)=2$
\end{solution}
\part[2] Para $k = 2$, hallar, si existe, la solución de la ecuación $AX = B$
\begin{solution}
- Gauss: \\
$A^*=\left(\begin{matrix}2 & 2 & 4 & 12\\1 & -1 & 2 & 6\\4 & -2 & 2 & 8\end{matrix}\right)\sim\left(\begin{matrix}2 & 2 & 4 & 12\\0 & -2 & 0 & 0\\0 & 0 & -6 & -16\end{matrix}\right)\to x=\frac{2}{3}, y=0, z=\frac{8}{3}$
- Matriz inversa: \\
$A^{-1}=\left(\begin{matrix}\frac{1}{12} & - \frac{1}{2} & \frac{1}{3}\\\frac{1}{4} & - \frac{1}{2} & 0\\\frac{1}{12} & \frac{1}{2} & - \frac{1}{6}\end{matrix}\right) \to  X=A^{-1}\cdot b =\left(\begin{matrix}\frac{1}{12} & - \frac{1}{2} & \frac{1}{3}\\\frac{1}{4} & - \frac{1}{2} & 0\\\frac{1}{12} & \frac{1}{2} & - \frac{1}{6}\end{matrix}\right)\cdot \left(\begin{matrix}12\\6\\8\end{matrix}\right) =\left(\begin{matrix}\frac{2}{3}\\0\\\frac{8}{3}\end{matrix}\right)$
- Cramer: \\
$x=\frac{\left|\begin{matrix}12 & 2 & 4\\6 & -1 & 2\\8 & -2 & 2\end{matrix}\right|}{24}=\frac{16}{24}=\frac{2}{3}$ \\
$y=\frac{\left|\begin{matrix}2 & 12 & 4\\1 & 6 & 2\\4 & 8 & 2\end{matrix}\right|}{24}=\frac{0}{24}=0$ \\
$z=\frac{\left|\begin{matrix}2 & 2 & 12\\1 & -1 & 6\\4 & -2 & 8\end{matrix}\right|}{24}=\frac{64}{24}=\frac{8}{3}$
\end{solution}
\end{parts}

\question Un almacén de legumbres al por mayor tiene sacos de dos tipos, con
capacidad para 5 kg de peso y con capacidad para 10 kg de peso. Sólo tiene 180 sacos de capacidad
10 kg. Debe poner a la venta como mucho 2000 kg de alubias en sacos de ambos tipos. Por cada
3 sacos de 10 kg puede vender como mucho 2 sacos de 5 kg, y como mínimo tiene que poner a la
venta 20 sacos de 5 kg y 60 de 10 kg. Por cada saco de 10 kg obtiene un beneficio de 5 € y por
cada saco de 5 kg obtiene un beneficio de 2 €. Determine cuántos sacos de cada tipo debe vender para obtener el máximo beneficio y calcule dicho beneficio. (Justifica todos los pasos realizados)
\begin{solution}\\
\includegraphics[scale=0.5]{prglin1}\\
\includegraphics[scale=0.8]{prglin2}
\end{solution}

\question Una voluntaria quiere preparar helado artesano y horchata de
auténtica chufa para un rastrillo solidario. La elaboración de cada litro de helado lleva 1 hora
de trabajo y la elaboración de un litro de horchata 2 horas. Como la horchata no necesita leche,
sabe que puede preparar hasta 15 litros de helado con los 5 bricks de 1,5 litros de leche que tiene. Para que haya suficiente
para todos los asistentes, tiene que preparar al menos 10 litros entre helado y horchata, en un
máximo de 20 horas. Si el beneficio por litro es de 25 euros para el helado y 12 euros para la horchata, obténgase
la cantidad de cada producto que se deberá preparar para maximizar el beneficio y calcúlese
el beneficio máximo que podría obtenerse (Justifica todos los pasos realizados)
\begin{solution}\\
\includegraphics[scale=0.5]{prglin3}\\
\includegraphics[scale=0.8]{prglin4}
\end{solution}

\question Dadas las matrices  $A=\left(\begin{matrix}2 & 1 & 3\\1 & -1 & 0\end{matrix}\right)$,  $B=\left(\begin{matrix}1 & -1\\2 & 0\\0 & 1\end{matrix}\right)$, $C=\left(\begin{matrix}2 & 3\\1 & 4\end{matrix}\right)$ y $D=\left(\begin{matrix}1 & 8\\0 & 5\end{matrix}\right)$ 
%\noaddpoints % to omit double points count

\begin{parts}
\part[1] Calcule $2C$ y $AB$ 

\part[2] Encontrar, si existe, una matriz $X$ tal que: $AB+2CX=D$
\begin{solution}
$AB=\left(\begin{matrix}4 & 1\\-1 & -1\end{matrix}\right)$, $2C=\left(\begin{matrix}4 & 6\\2 & 8\end{matrix}\right)$, $D-AB = \left(\begin{matrix}-3 & 7\\1 & 6\end{matrix}\right)$ y $(2C)^{-1}:\left(\begin{matrix}4 & 6\\2 & 8\end{matrix}\right)\xrightarrow{traspuesta}\left(\begin{matrix}4 & 2\\6 & 8\end{matrix}\right)\xrightarrow{adjunta}\left(\begin{matrix}8 & -6\\-2 & 4\end{matrix}\right)\xrightarrow{inversa}\left(\begin{matrix}\frac{2}{5} & - \frac{3}{10}\\- \frac{1}{10} & \frac{1}{5}\end{matrix}\right)$. \\ Por tanto, $X=(2C)^{-1}\cdot (D-AB)= \left(\begin{matrix}\frac{2}{5} & - \frac{3}{10}\\- \frac{1}{10} & \frac{1}{5}\end{matrix}\right)\cdot \left(\begin{matrix}-3 & 7\\1 & 6\end{matrix}\right)=\left(\begin{matrix}- \frac{3}{2} & 1\\\frac{1}{2} & \frac{1}{2}\end{matrix}\right)$

\end{solution}

% \part[1] Calcule, justificadamente, el rango de la matriz:$\left(\begin{matrix}1 & 0 & 2\\-1 & -1 & 1\\1 & -1 & 5\end{matrix}\right)$
% \begin{solution}
% $det(A)=0 \land $ Matriz de adjuntos $\rightarrow\left(\begin{matrix}-4 & 6 & 2\\-2 & 3 & 1\\2 & -3 & -1\end{matrix}\right)$ \\
% Por tanto, $ran(A)=2$
% \end{solution}

\end{parts}
\addpoints

% \question Resuelve el sistema: $\left(\begin{matrix}1 & 3 & -1\\1 & 2 & 0\\0 & 3 & -1\end{matrix}\right)\cdot \left(\begin{matrix}x\\y\\z\end{matrix}\right)=\left(\begin{matrix}4\\5\\3\end{matrix}\right)$

% \begin{parts}
% \part[1] Por el método de Gauss
% \part[1] Por el método de la matriz inversa
% \part[1] Por la regla de Cramer
% \end{parts}
% \begin{solution}
% $\left\{\begin{matrix}x + 3 y - z = 4\\x + 2 y = 5\\3 y - z = 3\\\end{matrix}\right.$ \\  Por Gauss: \\ $\left(\begin{matrix}1 & 3 & -1 & 4\\1 & 2 & 0 & 5\\0 & 3 & -1 & 3\end{matrix}\right)\rightarrow\left(\begin{matrix}1 & 3 & -1 & 4\\0 & -1 & 1 & 1\\0 & 0 & 2 & 6\end{matrix}\right)\to$ Sol:$\left\{\left( 1, \  2, \  3\right)\right\}$ \\ Por Matriz inversa: \\ $X=A^{-1}\cdot b=\left(\begin{matrix}1 & 0 & -1\\- \frac{1}{2} & \frac{1}{2} & \frac{1}{2}\\- \frac{3}{2} & \frac{3}{2} & \frac{1}{2}\end{matrix}\right)\cdot\left(\begin{matrix}4\\5\\3\end{matrix}\right)=\left(\begin{matrix}1\\2\\3\end{matrix}\right)$. \ Ya que $\left(\begin{matrix}1 & 3 & -1\\1 & 2 & 0\\0 & 3 & -1\end{matrix}\right)\xrightarrow{traspuesta}\left(\begin{matrix}1 & 1 & 0\\3 & 2 & 3\\-1 & 0 & -1\end{matrix}\right)\xrightarrow{adjunta}\left(\begin{matrix}-2 & 0 & 2\\1 & -1 & -1\\3 & -3 & -1\end{matrix}\right)\xrightarrow{inversa}\left(\begin{matrix}1 & 0 & -1\\- \frac{1}{2} & \frac{1}{2} & \frac{1}{2}\\- \frac{3}{2} & \frac{3}{2} & \frac{1}{2}\end{matrix}\right)$. \\ Por Cramer: \\ $det(A)=-2$ \ $\Delta_0$, $s_0$: $\left( -2, \  1\right)$ \ $\Delta_1$, $s_1$: $\left( -4, \  2\right)$ \ $\Delta_2$, $s_2$: $\left( -6, \  3\right)$ \
% \end{solution}

\question En una librería hubo la semana pasada una promoción de tres libros: una novela, un libro de
poesía y un cuento. Se vendieron 200 ejemplares de la novela, 100 de poesía y 150 cuentos. La librería ingresó por dicha promoción 8600 euros, que el precio de un ejemplar de novela es el doble que el de un cuento y que el triple de la diferencia entre el precio del ejemplar de poesía y del cuento es igual al
precio de una novela.
\begin{parts}
\part[2] Plantea un sistema de ecuaciones que refleje el enunciado

\part[1] Resuelva el problema por cualquiera de los métodos vistos en clase
\begin{solution} \\
\includegraphics[scale = 0.8]{sol1}
\end{solution}
\end{parts}

% \question Un padre decide repartir su fortuna de 480 monedas de oro entre sus tres hijas: Ana,
% Carla y Pilar. La cantidad que recibe Ana
% es el doble de la suma de las cantidades que reciben Carla y Pilar. Además, la suma de
% las cantidades que reciben Ana y Pilar es igual al triple de la cantidad que recibe Carla.
% \begin{parts}
% \part[2] Plantea un sistema de ecuaciones que refleje el enunciado
% \part[1] Resuelva el problema por cualquiera de los métodos vistos en clase
% \end{parts}
% \begin{solution}
% $\left\{\begin{matrix}x + y + z = 480\\x = 2 y + 2 z\\x + z = 3 y\\\end{matrix}\right.$ \\  Por Gauss: \\ $\left(\begin{matrix}1 & 1 & 1 & 480\\1 & -2 & -2 & 0\\1 & -3 & 1 & 0\end{matrix}\right)\rightarrow\left(\begin{matrix}1 & 1 & 1 & 480\\0 & -3 & -3 & -480\\0 & 0 & 4 & 160\end{matrix}\right)\to$ Sol:$\left\{\left( 320, \  120, \  40\right)\right\}$ \\ Por Matriz inversa: \\ $X=A^{-1}\cdot b=\left(\begin{matrix}\frac{2}{3} & \frac{1}{3} & 0\\\frac{1}{4} & 0 & - \frac{1}{4}\\\frac{1}{12} & - \frac{1}{3} & \frac{1}{4}\end{matrix}\right)\cdot\left(\begin{matrix}480\\0\\0\end{matrix}\right)=\left(\begin{matrix}320\\120\\40\end{matrix}\right)$. \ Ya que $\left(\begin{matrix}1 & 1 & 1\\1 & -2 & -2\\1 & -3 & 1\end{matrix}\right)\xrightarrow{traspuesta}\left(\begin{matrix}1 & 1 & 1\\1 & -2 & -3\\1 & -2 & 1\end{matrix}\right)\xrightarrow{adjunta}\left(\begin{matrix}-8 & -4 & 0\\-3 & 0 & 3\\-1 & 4 & -3\end{matrix}\right)\xrightarrow{inversa}\left(\begin{matrix}\frac{2}{3} & \frac{1}{3} & 0\\\frac{1}{4} & 0 & - \frac{1}{4}\\\frac{1}{12} & - \frac{1}{3} & \frac{1}{4}\end{matrix}\right)$. \\ Por Cramer: \\ $det(A)=-12$ \ $\Delta_0$, $s_0$: $\left( -3840, \  320\right)$ \ $\Delta_1$, $s_1$: $\left( -1440, \  120\right)$ \ $\Delta_2$, $s_2$: $\left( -480, \  40\right)$ \ \end{solution}

\question Dado el sistema: $$\left\{\begin{matrix}x + y + z = a - 1\\a z + 2 x + y = a\\a y + x + z = 1\\\end{matrix}\right.$$
%\noaddpoints % to omit double points count
\begin{parts}
\part[2] Discutir la solución del mismo según el valor de $a$ 

% \part[1] Resolver el sistema para $a=2$

\part[3] Resolver el sistema según el valor de $a$


\end{parts}
\begin{solution}
$\left(\begin{matrix}1 & 1 & 1 & a - 1\\2 & 1 & a & a\\1 & a & 1 & 1\end{matrix}\right) \to \left(\begin{matrix}1 & 1 & 1 & a - 1\\0 & -1 & a - 2 & 2 - a\\0 & 0 & \left(a - 2\right) \left(a - 1\right) & a \left(2 - a\right)\end{matrix}\right)\to det(A)=- a^{2} + 3 a - 2\to- \left(a - 2\right) \left(a - 1\right)$. \\ \\ Discusión: \\Si a $\neq\left( 1, \  2\right)\to det(A) \neq 0 \to ran(A)=ran(A^*)=3 \to $ S.C.D. $\to$ Sol:$\left(\begin{matrix}a + 1\\\frac{2 - a}{a - 1}\\- \frac{a}{a - 1}\end{matrix}\right)$ \\Si a=1: \\ $\left(\begin{matrix}1 & 1 & 1 & 0\\0 & -1 & -1 & 1\\0 & 0 & 0 & 1\end{matrix}\right) \to ran(A^*)=3 \land ran(A)=2 \to$  S.I. \\Si a=2: \\ $\left(\begin{matrix}1 & 1 & 1 & 1\\0 & -1 & 0 & 0\\0 & 0 & 0 & 0\end{matrix}\right) \to ran(A^*)=2 \land ran(A)=2 \to$  S.C.I.  $\to$ Sol:$\left\{ x : 1 - z, \  y : 0\right\}$
\end{solution}
\addpoints



\question Dadas las siguientes restricciones: $$\left\{\begin{matrix}
2x  \leqslant  8 - y\\
x  \leqslant  3\\
y  \leqslant  4\\
x  \geqslant  0 \\
y  \geqslant  0 \\
\end{matrix}\right.$$ 
%\noaddpoints % to omit double points count

\begin{parts}
\part[1] Razonar si $z=5x+2y$ alcanza un valor máximo y uno mínimo con las restricciones anteriores. En caso afirmativo, calcular dichos valores y los puntos en los que se alcanzan.   
\begin{solution}\\
\includegraphics[scale=0.6]{fi1_1} 
\\Vértices:\\
$A(0 , 0) \to f(0,0)=0$\\
$B(3 , 0) \to f(3,0)=15$\\
$C(3 , 2) \to f(3,2)=19$\\
$D(2 , 4) \to f(2,4)=18$\\
$E(0 , 4) \to f(0,4)=8$\\
\\
Mínimo en $D$ y $f(D)=0$ \\
Máximo en $C$ y $f(C)=19$
\end{solution}
\part[1] Igual que el apartado anterior pero para $z=6x+3y$ 
\begin{solution}\\
\includegraphics[scale=0.6]{fi1_2} 
\\ Vértices:\\
$A(0 , 0) \to f(0,0)=0$\\
$B(3 , 0) \to f(3,0)=18$\\
$C(3 , 2) \to f(3,2)=24$\\
$D(2 , 4) \to f(2,4)=24$\\
$E(0 , 4) \to f(0,4)=12$\\
\\
Mínimo en $D$ y $f(D)=0$\\
Máximo en $\overline{CD}$ y $f(C)=24 \land f(D)=24$
$$\overline{CD}\equiv \left\{ \begin{matrix}
x = 3 + (2-3)\lambda \\
y = 2 + (4-2)\lambda \\
\end{matrix} , \lambda \in \left( 0, 1 \right)
\right.
$$
\end{solution}


\end{parts}

\addpoints


\question[3] Los 400 alumnos de un colegio van a ir de excursión. Para ello se contrata el viaje a una
empresa que dispone de 8 autobuses de 40 plazas y 10 con 50 plazas, pero sólo de 9 conductores para ese
día. Dada la diferente capacidad y calidad, el alquiler de cada autobús de los grandes cuesta 80 \euro. y
el de cada uno de los pequeños 60 \euro . ¿Cuántos autobuses de cada clase se tiene
que alquilar para que el coste del viaje sea mínimo?

\begin{solution}
min $z=6000 x + 8000 y$ \\
s.a: \\
$\left\{ \begin{matrix}x \leq 8  \\ y \leq 10 \\ x+y \leq 9 \\ 40x+50y \geq 400 \\ x \geq 0 \\ y \geq 0 \\ \end{matrix}\right.$ \\
Región factible:\\
\includegraphics[scale=0.30]{region1.jpg} \\

Vértices y valor de la función: \\

(5, 4)→6000⋅5+8000⋅4=62000\\
(0, 9)→6000⋅0+8000⋅9=72000\\
(0, 8)→6000⋅0+8000⋅8=64000\\

La función se optimiza para x=5, y=4.0 y el valor óptimo es 62000
\end{solution}






% \question[3] Un camionero transporta dos tipos de mercancías, X e Y, ganando 60 y 50 euros por
% tonelada respectivamente. Al menos debe transportar 8 toneladas de X y como mucho el doble de cantidad
% que de Y. ¿A cuánto asciende su ganancia total máxima si dispone de un camión que puede transportar
% hasta 30 toneladas? 


  
% \begin{solution}
% Maximizar $f(x,y)=60x+50y$ s.a: $\left\{ \begin{matrix}
% x \geqslant 8 \\
% x \leqslant 2y \\
% x+y \leqslant 30 \\
%  \end{matrix} \right.$ 
 
% \includegraphics[scale=0.2]{sep2007}


% Vértices:\\
% $A(8 , 4) \to f(8,4)=680$\\
% $B(8 , 22) \to f(8,22)=1580$\\
% $C(20 , 10) \to f(20,10)=1700$\\

% 1700 \euro (debe transportar 20 toneladas de X y 10 toneladas de Y). \end{solution}




\addpoints



\end{questions}

\end{document}
\grid
