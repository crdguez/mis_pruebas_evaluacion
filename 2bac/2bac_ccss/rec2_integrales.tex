\documentclass[addpoints,spanish, 12pt,a4paper]{exam}
%\documentclass[answers, spanish, 12pt,a4paper]{exam}
%\printanswers
\pointpoints{punto}{puntos}
\hpword{Puntos:}
\vpword{Puntos:}
\htword{Total}
\vtword{Total}
\hsword{Resultado:}
\hqword{Ejercicio:}
\vqword{Ejercicio:}

\usepackage[utf8]{inputenc}
\usepackage[spanish]{babel}
\usepackage{eurosym}
%\usepackage[spanish,es-lcroman, es-tabla, es-noshorthands]{babel}


\usepackage[margin=1in]{geometry}
\usepackage{amsmath,amssymb}
\usepackage{multicol}
\usepackage{yhmath}

\pointsinrightmargin % Para poner las puntuaciones a la derecha. Se puede cambiar. Si se comenta, sale a la izquierda.
\extrawidth{-2.4cm} %Un poquito más de margen por si ponemos textos largos.
\marginpointname{ \emph{\points}}

\usepackage{graphicx}

\graphicspath{{../../img/}} 

\newcommand{\class}{2º Bachillerato CCSS}
\newcommand{\examdate}{\today}
\newcommand{\examnum}{Integrales}
\newcommand{\tipo}{C}


\newcommand{\timelimit}{50 minutos}

\renewcommand{\solutiontitle}{\noindent\textbf{Solución:}\enspace}


\pagestyle{head}
\firstpageheader{\includegraphics[width=0.2\columnwidth]{header_left}}{\textbf{Departamento de Matemáticas\linebreak \class}\linebreak \examnum}{\includegraphics[width=0.1\columnwidth]{header_right}}
\runningheader{\class}{\examnum}{Página \thepage\ of \numpages}
\runningheadrule


\usepackage{pgf,tikz,pgfplots}
\pgfplotsset{compat=1.15}
\usepackage{mathrsfs}
\usetikzlibrary{arrows}


\begin{document}

\noindent
\begin{tabular*}{\textwidth}{l @{\extracolsep{\fill}} r @{\extracolsep{6pt}} }
\textbf{Nombre:} \makebox[3.5in]{\hrulefill} & \textbf{Fecha:}\makebox[1in]{\hrulefill} \\
 & \\
\textbf{Tiempo: \timelimit} & Tipo: \tipo 
\end{tabular*}
\rule[2ex]{\textwidth}{2pt}
Esta prueba tiene \numquestions\ ejercicios. La puntuación máxima es de \numpoints. 
La nota final de la prueba será la parte proporcional de la puntuación obtenida sobre la puntuación máxima. 

\begin{center}


\addpoints
 %\gradetable[h][questions]
	\pointtable[h][questions]
\end{center}

\noindent
\rule[2ex]{\textwidth}{2pt}

\begin{questions}


% \question[3] Dada la función $f(x) = \dfrac{-2x^2 -6x - 4}{x^2 - 4} 
% $
% hallar su dominio, los puntos de corte con los ejes y
% la pendiente de la recta tangente a la gráfica de la función en x = 1.




% \question La función correspondiente a los beneficios de una empresa  donde $t$  representa los años y $B(t)$ el beneficio expresado en miles de euros es:
% $$B(t)=\frac{20t-80}{t+2}$$
% \begin{parts}
%     \part[1] Determinar si $B(t)$ tiene máximos, mínimos relativos y puntos de inflexión en su dominio de definición.
%     \part[1] Determinar cuándo la empresa tiene ganancias y cuándo pérdidas.
%     \part[1]  ¿Están los beneficios limitados?. Razonar la respuesta. Si lo están, ¿cuál es su límite?.
% \end{parts}

% \question[2] Calcular el área del recinto limitado por la parábola de ecuación $y = -x^2 - 4x + 5$, el eje $OX$, y las rectas $x=-2$ y $x=3$ 
% \begin{solution}$\frac{98}{3} ud^2$ \end{solution}

% \question[2] Calcular el área del recinto limitado por la parábola de ecuación $y = x^2 - 3x + 2$, el eje $OX$, y las rectas $x=0$ y $x=2$ 
% \begin{solution}$1 ud^2$ \end{solution}
    

% \question Dado  $f(x)=x^2+8x$:
% \begin{parts}
%     \part[1] Calcular las coordenadas de los puntos donde la recta tangente a $f$ es paralela al eje $OX$
%     \part[1] Determina las ecuaciones de dichas rectas tangentes
%     \part[1] Calcular el área del recinto acotado por $f$ y la recta $g(x)=x+8$
% \end{parts}


% \question Dado  $f(x)=x^2+6x-7$:
% \begin{parts}
%     \part[1] Calcular las coordenadas de los puntos donde la recta tangente a $f$ es paralela al eje $OX$
%     \part[1] Determina las ecuaciones de dichas rectas tangentes
%     \part[1] Calcular el área del recinto acotado por $f$ y la recta $g(x)=x+7$
% \end{parts}

% \question Halla (reduciendo la expresión todo lo que puedas) las derivada de las siguientes funciones:
% \begin{parts}
% \part[1] $f(x)=\dfrac{2x^3+x^2}{x-1}$
% \begin{solution}$ \frac{x \left(4 x^{2} - 5 x - 2\right)}{x^{2} - 2 x + 1}$ \end{solution}
% \part[1] $f(x)=\left(1-x\right)^2e^x$
% \begin{solution}$\left(x^{2} - 1\right) e^{x} $ \end{solution}
% \part[2] $f(x)=\ln\left[\left(\dfrac{1-x^2}{2+x}\right)^2\right]$
% \begin{solution}$ $ \end{solution}

% \part[1] $f(x)=\dfrac{x^3+2x^2}{1-x}$
% \part[1] $f(x)=\left(x-1\right)^2e^{-x}$
% \part[2] $f(x)=\ln\sqrt{\dfrac{1-x^2}{2+x}}$
% \begin{solution}$ $\end{solution}
% \end{parts}

% %  evau aragon septiembre 96
% \question  Los beneficios de una empresa vienen dados por la función \\ $B(t)=\dfrac{30t-60}{t+1}$ donde $t$  representa los años y $B(t)$ el beneficio expresado en miles de euros:
% \begin{parts}
%     \part[1] Determinar cuándo la empresa tiene ganancias y cuándo pérdidas.
%     \part[1] Determinar si $B(t)$ tiene máximos, mínimos relativos y puntos de inflexión en su dominio de definición.
%     \part[1]  ¿Están los beneficios limitados?. Razonar la respuesta. Si lo están, ¿cuál es su límite?.
% \end{parts}


% Parte de solo integrales
% --------------------------
\question Calcula las siguientes integrales:
\begin{parts}
    \part[1] $\int\left(x^5-7+\frac{2}{x}+\frac{4}{x^2}\right)dx$
    \part[1] $\int x^2\cdot\sqrt{x} \ dx$
    \part[1] $\int x\cdot  e^{x^2+1} \ dx$
    \part[1] $\int\left(\sqrt{x}-\sqrt{2}\right)^2dx$
    \part[1] $\int \dfrac{2x}{5x^2+9}\ dx$
\end{parts}

\question[2]   Calcular el área del recinto limitado por la función $f(x)=x^3-9x$ y el eje $OX$
\begin{solution}
    $\frac{81}{2}ud^2$
\end{solution}   

\question Dada  $f(x)=x^2+6x-7$:
\begin{parts}
        \part[2] Calcula el área del recinto acotado por $f$ y la recta $g(x)=x+7$
\end{parts}

\addpoints



\end{questions}

\end{document}
\grid
