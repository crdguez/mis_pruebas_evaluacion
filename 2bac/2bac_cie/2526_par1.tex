\documentclass[addpoints,spanish, 12pt,a4paper]{exam}
%\documentclass[answers, spanish, 12pt,a4paper]{exam}
\printanswers
\pointpoints{punto}{puntos}
\hpword{Puntos:}
\vpword{Puntos:}
\htword{Total}
\vtword{Total}
\hsword{Resultado:}
\hqword{Ejercicio:}
\vqword{Ejercicio:}

\usepackage[utf8]{inputenc}
\usepackage[spanish]{babel}
\usepackage{eurosym}
%\usepackage[spanish,es-lcroman, es-tabla, es-noshorthands]{babel}


\usepackage[margin=1in]{geometry}
\usepackage{amsmath,amssymb}
\usepackage{multicol}
\usepackage{yhmath}

\pointsinrightmargin % Para poner las puntuaciones a la derecha. Se puede cambiar. Si se comenta, sale a la izquierda.
\extrawidth{-2.4cm} %Un poquito más de margen por si ponemos textos largos.
\marginpointname{ \emph{\points}}

\usepackage{graphicx}

\graphicspath{{../../img/}} 

\newcommand{\class}{2º Bachillerato CIT}
\newcommand{\examdate}{\today}
\newcommand{\examnum}{Parcial 1ªEv.}
\newcommand{\tipo}{A}


\newcommand{\timelimit}{45 minutos}

\renewcommand{\solutiontitle}{\noindent\textbf{Solución:}\enspace}


\pagestyle{head}
\firstpageheader{\includegraphics[width=0.2\columnwidth]{header_left}}{\textbf{Departamento de Matemáticas\linebreak \class}\linebreak \examnum}{\includegraphics[width=0.1\columnwidth]{header_right}}
\runningheader{\class}{\examnum}{Página \thepage\ of \numpages}
\runningheadrule


\usepackage{pgf,tikz,pgfplots}
\pgfplotsset{compat=1.15}
\usepackage{mathrsfs}
\usetikzlibrary{arrows}


\begin{document}

\noindent
\begin{tabular*}{\textwidth}{l @{\extracolsep{\fill}} r @{\extracolsep{6pt}} }
\textbf{Nombre:} \makebox[3.5in]{\hrulefill} & \textbf{Fecha:}\makebox[1in]{\hrulefill} \\
 & \\
\textbf{Tiempo: \timelimit} & Tipo: \tipo 
\end{tabular*}
\rule[2ex]{\textwidth}{2pt}
Esta prueba tiene \numquestions\ ejercicios. La puntuación máxima es de \numpoints. 
La nota final de la prueba será la parte proporcional de la puntuación obtenida sobre la puntuación máxima. 

\begin{center}


\addpoints
 %\gradetable[h][questions]
	\pointtable[h][questions]
\end{center}

\noindent
\rule[2ex]{\textwidth}{2pt}

\begin{questions}

%\question 
%
%\begin{parts}
%\part[2] 
%\begin{solution}
%\end{solution}
%
%
%\end{parts}
%\addpoints

\question[1] Calcula el siguiente límite:

$$\lim_{x \to 1}\left(\frac{1 - \sqrt{2 - x}}{x^{2} - 3 x + 2}\right)$$

\begin{solution}$- \frac{1}{2}$\end{solution}

% \question[1] Calcula el siguiente límite:

% $$\lim_{x \to 3}\left(\frac{2 - \sqrt{1 + x}}{x -3}\right)$$

% \begin{solution}$- \frac{1}{4}$\end{solution}

% \question[1] Calcula el siguiente límite:

% $$\lim_{x \to \infty} \left(\frac{x + 5}{x - 1}\right)^{2 x + 1}$$

% \begin{solution}$e^{12}$\end{solution}


\question[2] Calcula el siguiente límite:

\[
\lim_{x \to 0} \cos(2x)^{\frac{4}{x^2}}
\]

\begin{solution}
Cuando \(x \to 0\), se tiene:

\[
\cos(2x) \to 1 \quad \text{y} \quad \frac{4}{x^2} \to \infty,
\]

lo que nos da una forma indeterminada del tipo \(1^\infty\).

Tomamos logaritmo natural:

\[
\ln L = \lim_{x \to 0} \frac{4}{x^2} \ln(\cos(2x))
\]

Usamos el desarrollo de Taylor de \(\cos(2x)\):

\[
\cos(2x) = 1 - \frac{(2x)^2}{2} + \cdots = 1 - 2x^2 + \cdots
\]

Entonces:

\[
\ln(\cos(2x)) \approx \ln(1 - 2x^2) \approx -2x^2
\]

Sustituyendo:

\[
\ln L = \lim_{x \to 0} \frac{4}{x^2} \cdot (-2x^2) = -8
\]

Por lo tanto:

\[
L = e^{-8}
\]

\[
\boxed{e^{-8}}
\]
\end{solution}

% \question[1] Calcula el siguiente límite:

% $$\lim_{x \to \infty} \left(\frac{x + 2}{x - 3}\right)^{2 x}$$

% \begin{solution}$e^{10}$\end{solution}

% \question Sea la función: $$g(x)=\dfrac{x^2-4}{x-2}$$
% \begin{parts}
% \part[1] Estudia la continuidad de la función
% \begin{solution}$\mathbb{R}-\left\{ 2\right\}=\left(-\infty,2\right)\cup \left(2,\infty\right)
% $\end{solution}

% \part[2] Redefine la función para que sea continua en $\mathbb{R}$
% \begin{solution} $f(x)=\left\{ \begin{array}{ccl} \dfrac{x^2-4}{x-2} & , & x \neq 2 \\ 4 & , & x=2 \end{array} \right.$ \end{solution}
% \end{parts}

\question Dada la función:
$$f(x)=\left\{ \begin{array}{ccl} \dfrac{e^{-x}}{2} & , & x < 0 \\ & \\ -x^2+k & , & x\geqslant 0 \end{array} \right.$$
\begin{parts}
\part[2] Calcula el valor de $k$ para que la función sea continua
\begin{solution}
$k=\frac{1}{2}$
\end{solution}
\part[1] Representa gráficamente la función (esbozo razonado)
\begin{solution}
\includegraphics[width=0.3\columnwidth]{funcion1}
\end{solution}
\end{parts}

% \question Dada la función:
% $$f(x)=\dfrac{x^2+5x+3}{x+2}$$
% \begin{parts}
% \part[2] Calcula las asíntotas
% \begin{solution}
% $x=-2$ y $y=x+3$
% \end{solution}
% \part[1] Determina si la gráfica corta a alguna de las asíntotas
% \begin{solution}
% No corta a ninguna
% \end{solution}
% \end{parts}		

\question Dada la función:
$$f(x)=\dfrac{x^3-2x+1}{x^2-1}$$
\begin{parts}
\part[2] Calcula las asíntotas
\begin{solution}
$f(x)=\frac{\left(x - 1\right) \left(x^{2} + x - 1\right)}{(x+1)(x-1)}$ \\
$x=-1$ y $y=x$
\end{solution}
\part[1] Determina si la gráfica corta a alguna de las asíntotas
\begin{solution}
No corta a ninguna
\end{solution}
\end{parts}		

\question[3] Calcula las siguientes derivadas
\begin{parts}
    \part $y=1+x^2\cdot e^{-x^2}$  
    \begin{solution}
        Para derivar \( y \) con respecto a \( x \), aplicamos la regla del producto y la regla de la cadena:
        
        \[
        y' = \frac{d}{dx} \left( 1 + x^2 e^{-x^2} \right)
        \]
        
        La derivada de \( 1 \) es \( 0 \), y derivamos \( x^2 e^{-x^2} \) utilizando la regla del producto:
        
        \[
        y' = 2x e^{-x^2} + x^2 \cdot \frac{d}{dx}(e^{-x^2})
        \]
        
        La derivada de \( e^{-x^2} \) con respecto a \( x \) es \( -2x e^{-x^2} \), entonces:
        
        \[
        y' = 2x e^{-x^2} + x^2 \cdot (-2x e^{-x^2})
        \]
        
        Simplificando:
        
        \[
        y' = 2x e^{-x^2} - 2x^3 e^{-x^2}
        \]
        
        Finalmente, factorizamos \( 2x e^{-x^2} \):
        
        \[
        y' = 2x e^{-x^2} (1 - x^2)
        \]
    \end{solution}
    \part $y = \ln \sqrt{\dfrac{x^2-1}{x^2+x}}$    \begin{solution}
        Primero, simplificamos la expresión utilizando la propiedad de los logaritmos \( \ln \sqrt{a} = \frac{1}{2} \ln a \):
        
        \[
        y = \frac{1}{2} \ln \left( \frac{x^2 - 1}{x^2 + x} \right)
        \]
        
        Ahora derivamos con respecto a \( x \). Usamos la regla de la cadena y la regla del cociente para derivar el logaritmo:
        
        \[
        y' = \frac{1}{2} \cdot \frac{1}{\frac{x^2 - 1}{x^2 + x}} \cdot \frac{d}{dx} \left( \frac{x^2 - 1}{x^2 + x} \right)
        \]
        
        La derivada de \( \frac{x^2 - 1}{x^2 + x} \) la obtenemos utilizando la regla del cociente:
        
        \[
        \frac{d}{dx} \left( \frac{x^2 - 1}{x^2 + x} \right) = \frac{(2x)(x^2 + x) - (x^2 - 1)(2x + 1)}{(x^2 + x)^2}
        \]
        
        Simplificando el numerador:
        
        \[
        (2x)(x^2 + x) = 2x^3 + 2x^2
        \]
        \[
        (x^2 - 1)(2x + 1) = (x^2 - 1)(2x) + (x^2 - 1)(1) = 2x^3 - 2x + x^2 - 1
        \]
        
        Ahora, restamos ambos términos:
        
        \[
        (2x^3 + 2x^2) - (2x^3 - 2x + x^2 - 1) = 2x^3 + 2x^2 - 2x^3 + 2x - x^2 + 1 = x^2 + 2x + 1
        \]
        
        Entonces, la derivada es:
        
        \[
        \frac{d}{dx} \left( \frac{x^2 - 1}{x^2 + x} \right) = \frac{x^2 + 2x + 1}{(x^2 + x)^2}
        \]
        
        Sustituyendo en la expresión original de la derivada:
        
        \[
        y' = \frac{1}{2} \cdot \frac{1}{\frac{x^2 - 1}{x^2 + x}} \cdot \frac{x^2 + 2x + 1}{(x^2 + x)^2}
        \]
        
        Simplificamos el cociente:
        
        \[
        y' = \frac{1}{2} \cdot \frac{x^2 + 2x + 1}{(x^2 - 1)(x^2 + x)}
        \]
        
        Finalmente, podemos factorizar el numerador \( x^2 + 2x + 1 \) como \( (x + 1)^2 \), obteniendo la derivada final:
        
        \[
        y' = \frac{(x + 1)^2}{2(x^2 - 1)(x^2 + x)}
        \]
    \end{solution}

    \part $y=\arcsen{\left(x^2-4\right)}$
        \begin{solution}
        Para derivar \( y = \arcsin(x^2 - 4) \), utilizamos la fórmula para la derivada de \( \arcsin(u) \), que es:
        
        \[
        \frac{d}{dx} \arcsin(u) = \frac{1}{\sqrt{1 - u^2}} \cdot \frac{du}{dx}
        \]
        
        En este caso, \( u = x^2 - 4 \). Derivamos \( u \) con respecto a \( x \):
        
        \[
        \frac{du}{dx} = 2x
        \]
        
        Ahora, sustituimos en la fórmula de la derivada de \( \arcsin \):
        
        \[
        y' = \frac{1}{\sqrt{1 - (x^2 - 4)^2}} \cdot 2x
        \]
        
        Simplificamos la expresión dentro de la raíz:
        
        \[
        y' = \frac{2x}{\sqrt{1 - (x^2 - 4)^2}}
        \]
        
        Expandimos \( (x^2 - 4)^2 \):
        
        \[
        (x^2 - 4)^2 = x^4 - 8x^2 + 16
        \]
        
        Entonces, la derivada queda:
        
        \[
        y' = \frac{2x}{\sqrt{1 - (x^4 - 8x^2 + 16)}}
        \]
        
        Finalmente, simplificamos el término en el denominador:
        
        \[
        y' = \frac{2x}{\sqrt{1 - x^4 + 8x^2 - 16}}
        \]
    \end{solution}

\end{parts}

\question[2] Deriva tomando logaritmos la siguiente función:
 $$y=\left(2x\right)^{2x}$$
        \begin{solution}
 Aplicamos logaritmos en ambos lados de la ecuación:



\[
\ln y = \ln\left((2x)^{2x}\right)
\]



Usamos la propiedad del logaritmo: \( \ln(a^b) = b \ln a \):



\[
\ln y = 2x \cdot \ln(2x)
\]



Derivamos ambos lados con respecto a \( x \). Usamos la regla de la cadena en el lado izquierdo:



\[
\frac{1}{y} \cdot \frac{dy}{dx} = \frac{d}{dx}\left(2x \cdot \ln(2x)\right)
\]



Aplicamos la regla del producto en el lado derecho:



\[
\frac{dy}{dx} = y \cdot \left[2 \cdot \ln(2x) + 2x \cdot \frac{1}{2x} \cdot 2\right]
\]



Simplificamos:



\[
\frac{dy}{dx} = y \cdot \left[2 \cdot \ln(2x) + 2\right]
\]



Recordando que \( y = (2x)^{2x} \), sustituimos:



\[
\frac{dy}{dx} = (2x)^{2x} \cdot \left[2 \cdot \ln(2x) + 2\right]
\]
    \end{solution}


\addpoints

\end{questions}

\end{document}
\grid
