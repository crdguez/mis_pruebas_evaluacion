\documentclass[addpoints,spanish, 12pt,a4paper]{exam}
%\documentclass[answers, spanish, 12pt,a4paper]{exam}
% \printanswers
\renewcommand*\half{.5}
\pointpoints{punto}{puntos}
\hpword{Puntos:}
\vpword{Puntos:}
\htword{Total}
\vtword{Total}
\hsword{Resultado:}
\hqword{Ejercicio:}
\vqword{Ejercicio:}


\usepackage[utf8]{inputenc}
\usepackage[spanish]{babel}
\usepackage{eurosym}
%\usepackage[spanish,es-lcroman, es-tabla, es-noshorthands]{babel}


\usepackage[margin=1in]{geometry}
\usepackage{amsmath,amssymb}
\usepackage{multicol}
\usepackage{yhmath}

\pointsinrightmargin % Para poner las puntuaciones a la derecha. Se puede cambiar. Si se comenta, sale a la izquierda.
\extrawidth{-2.4cm} %Un poquito más de margen por si ponemos textos largos.
\marginpointname{ \emph{\points}}

\usepackage{graphicx}

\graphicspath{{../../img/}} 

\newcommand{\class}{Matemáticas II}
\newcommand{\examdate}{\today}
\newcommand{\examnum}{Examen Final 1ªEv.}
\newcommand{\tipo}{B}


\newcommand{\timelimit}{90 minutos}

\renewcommand{\solutiontitle}{\noindent\textbf{Solución:}\enspace}


\pagestyle{head}
\firstpageheader{\includegraphics[width=0.2\columnwidth]{header_left}}{\textbf{Departamento de Matemáticas\linebreak \class}\linebreak \examnum}{\includegraphics[width=0.1\columnwidth]{header_right}}
\runningheader{\class}{\examnum}{Página \thepage\ de \numpages}
\runningheadrule


\usepackage{pgf,tikz,pgfplots}
\pgfplotsset{compat=1.15}
\usepackage{mathrsfs}
\usetikzlibrary{arrows}

% ##################### si queremos decimales distintos de 0.5 pero ojo, nos cargamos la tabla. Solo funciona  \gradetable 
\usepackage{fp}
\usepackage{numprint}
\npdecimalsign{.}
\nprounddigits{2}
\usepackage{etoolbox}
\makeatletter
% points printed at each question
\patchcmd{\point@block}{\@points}{\FPdiv\pointdiv{\@points}{100}\numprint{\pointdiv}}{}{}
% points printed for each question in grade table
\patchcmd{\do@oneline@v}{\pointsof@index{pq@index}}{\FPdiv\pointsdiv{\pointsof@index{pq@index}}{100}\numprint{\pointsdiv}}{}{}
% total number of points in grade table
\patchcmd{\prt@tablepoints}{\prt@hlfcntr{tbl@points}}{\FPdiv\pointsdiv{\prt@hlfcntr{tbl@points}}{100}\numprint{\pointsdiv}}{}{}
% patching needed in many other places
\makeatother
% ##################### fin de si queremos decimales

\begin{document}

\noindent
\begin{tabular*}{\textwidth}{l @{\extracolsep{\fill}} r @{\extracolsep{6pt}} }
\textbf{Nombre:} \makebox[3.5in]{\hrulefill} & \textbf{Fecha:}\makebox[1in]{\hrulefill} \\
 & \\
\textbf{Tiempo: \timelimit} & Tipo: \tipo 
\end{tabular*}
% \rule[2ex]{\textwidth}{2pt}
% Esta prueba tiene \numquestions\ ejercicios. La puntuación máxima es de \numpoints. 
% La nota final de la prueba será la parte proporcional de la puntuación obtenida sobre la puntuación máxima. 

% \begin{center}


% \addpoints
%  % \gradetable[h][questions]
%  % \gradetable[h]
% 	% \pointtable[h][questions]
%     \pointtable[h]
    
% \end{center}

\noindent
\rule[2ex]{\textwidth}{2pt}

\begin{questions}

\question Dada la función $$f(x) = (x+2)e^{-x} $$

\begin{parts}

\part[100] Encuentra los extremos relativos, absolutos y los intervalos de crecimiento y decrecimiento de \( f \).

\part[50] Determina la concavidad y convexidad y los puntos de inflexión de la función \( f \).

\part[100] Estudia las asíntotas de \( f \).

\part[100] Halla la recta tangente a la gráfica de la función en $x=-1$
\end{parts}

\question[150] La vela de un barco tiene forma de triángulo rectángulo. Si la hipotenusa mide \( 8\,\text{m} \), calcula las dimensiones para que la superficie de la vela sea máxima.

% \question[200] Sea la función \( f(x) = -x^{2} + \alpha x + 10 \), donde \( \alpha \) es un parámetro real. 
% \begin{parts}
%     \part Calcula el valor de \( \alpha \) para que \( f(x) \) tenga un máximo relativo en \( x = \dfrac{1}{2} \).
%     \part Calcula el valor de \( \alpha \) para que \( f(x) \) tenga un máximo relativo en \( x = \dfrac{1}{2} \).
%     % \part Para ese valor de \( \alpha \), calcula el área encerrada entre las gráficas \( f(x) \) y \( f'(x) \).
% \end{parts}

\question Dada la función $f(x) = 3x^5
+ 10x^3
+ ax + b $:
\begin{parts}
    \part[100] Comprueba que la función tiene un único punto de
inflexión
    \part[100] Hallar a y b para que la tangente a la gráfica de dicha función en el punto de
inflexión sea la recta $y = x + 2$
\end{parts}

% \question Calcula las siguientes integrales
% \begin{parts}
%     \part[100] $\int \dfrac{1}{9 x^{2} + 4}\, dx
% $
% \part[100] $\int \left(x^{2} + 2 x\right) \ln{\left(x \right)}\, dx$
% \end{parts}
% \begin{solution}
%     $\frac{\operatorname{atan}{\left(\frac{3 x}{2} \right)}}{6}
% , $ \\ $\frac{x^{2} \left(- 2 x + 6 \left(x + 3\right) \ln{\left(x \right)} - 9\right)}{18}$
% \end{solution}


\question[200] Hallar \( a, b, c \) y \( d \) para que la función
\[
f(x)=
\begin{cases}
ax+b, & \text{si } x<-1,\\[4pt]
x^{3}+x^{2}+cx+d, & \text{si } x\ge -1
\end{cases}
\]
sea derivable en \( \mathbb{R} \), corte al eje de ordenadas en el punto \( 3 \) y además alcance un extremo relativo en \( x=1 \).

\question[100] Calcula:
$$\lim_{x \to 0}\left(\frac{\ln{\left(x + e^{x} \right)}}{x}\right)$$
    \begin{solution}
        $2$
    \end{solution}

% \begin{parts}
%     \part $$\lim_{x \to 0}\left(\frac{\ln{\left(x + e^{x} \right)}}{x}\right)$$
%     \begin{solution}
%         $2$
%     \end{solution}
% \end{parts}

\addpoints

\end{questions}
% \gradetable \\
% \gradetable[h] \\
% \pointtable
\end{document}
\grid
