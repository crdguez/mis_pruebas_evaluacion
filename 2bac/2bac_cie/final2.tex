\documentclass[addpoints,spanish, 12pt,a4paper]{exam}
%\documentclass[answers, spanish, 12pt,a4paper]{exam}

\printanswers
\pointpoints{punto}{puntos}
\hpword{Puntos:}
\vpword{Puntos:}
\htword{Total}
\vtword{Total}
\hsword{Resultado:}
\hqword{Ejercicio:}
\vqword{Ejercicio:}

\usepackage[utf8]{inputenc}
\usepackage[spanish]{babel}
\usepackage{eurosym}
%\usepackage[spanish,es-lcroman, es-tabla, es-noshorthands]{babel}


\usepackage[margin=1in]{geometry}
\usepackage{amsmath,amssymb}
\usepackage{multicol, xparse}

\usepackage{yhmath}

\usepackage{verbatim}
%\usepackage{pstricks}


\usepackage{graphicx}
\graphicspath{{../../img/}}




\let\multicolmulticols\multicols
\let\endmulticolmulticols\endmulticols
\RenewDocumentEnvironment{multicols}{mO{}}
 {%
  \ifnum#1=1
    #2%
  \else % More than 1 column
    \multicolmulticols{#1}[#2]
  \fi
 }
 {%
  \ifnum#1=1
  \else % More than 1 column
    \endmulticolmulticols
  \fi
 }
\renewcommand{\solutiontitle}{\noindent\textbf{Sol:}\enspace}

\newcommand{\samedir}{\mathbin{\!/\mkern-5mu/\!}}

\newcommand{\class}{2º Bachillerato - Matemáticas II}
\newcommand{\examdate}{\today}

%\newcommand{\tipo}{A}


\newcommand{\timelimit}{90 minutos}

\renewcommand{\solutiontitle}{\noindent\textbf{Solución:}\enspace}


\pagestyle{head}
\firstpageheader{\includegraphics[width=0.2\columnwidth]{header_left}}{\textbf{Departamento de Matemáticas\linebreak \class}\linebreak \examnum}{\includegraphics[width=0.1\columnwidth]{header_right}}
\runningheader{\class}{\examnum}{Página \thepage\ of \numpages}
\runningheadrule

\pointsinrightmargin % Para poner las puntuaciones a la derecha. Se puede cambiar. Si se comenta, sale a la izquierda.
\extrawidth{-2.4cm} %Un poquito más de margen por si ponemos textos largos.
\marginpointname{ \emph{\points}}


    \newcommand{\tipo}{A}\newcommand{\examnum}{Final 2ª evaluación}
\begin{document}
\noindent
\begin{tabular*}{\textwidth}{l @{\extracolsep{\fill}} r @{\extracolsep{6pt}} }
\textbf{Nombre:} \makebox[3.5in]{\hrulefill} & \textbf{Fecha:}\makebox[1in]{\hrulefill} \\
 & \\
\textbf{Tiempo: \timelimit} & Tipo: \tipo 
\end{tabular*}
\rule[2ex]{\textwidth}{2pt}
Esta prueba tiene \numquestions\ ejercicios. La puntuación máxima es de \numpoints. 
La nota final de la prueba será la parte proporcional de la puntuación obtenida sobre la puntuación máxima. 

\begin{center}


\addpoints
     %\gradetable[h][questions]
    \pointtable[h][questions]
\end{center}

\noindent
\rule[2ex]{\textwidth}{2pt}

\begin{questions}


\question Dos urnas A y B contienen bolas de colores con la siguiente composición: La urna
A contiene 6 bolas verdes y 4 bolas negras, y la urna B contiene 2 bolas verdes, 4
bolas negras y 3 bolas rojas. Se saca al azar una bola de la urna A y se mete en la
urna B. A continuación, se saca al azar una bola de la urna B. Calcule:
\begin{parts}
    \part[1] La probabilidad de que la bola que se saca de la urna B sea roja.
    \part[1] La probabilidad de que la bola que se saca de la urna B sea verde,
sabiendo que la bola que se sacó de la urna A era verde.
    \part[1] La probabilidad de que la bola que se saca de la urna B sea negra
\end{parts}

\question PUNTO EXTRA: El cociente intelectual (CI) de los estudiantes universitarios sigue una distribución
normal de media $\mu$ y desviación típica $\sigma$ desconocidas. Se sabe que la media es
igual a 10 veces la desviación típica y que el 93,32 \% de los estudiantes tiene un CI
menor de 115.
\begin{parts}
    \part[1] Calcule la media y la desviación típica de esta distribución.
    \part[1] Si se eligen al azar 5 estudiantes universitarios, ¿cuál es la probabilidad
de que exactamente 3 de ellos tengan un CI mayor de 115?
\end{parts}

\question La altura de los individuos de una población sigue una distribución normal de media
175 cm y desviación típica 4 cm.
\begin{parts}
    \part[1] Calcule la probabilidad de que un individuo elegido al azar mida más
de 170 cm.
    \part[1] Calcule qué porcentaje de la población mide entre 170 y 185 cm.
    \part[1] Calcule la altura que es superada por el 33 /% de la población.

\end{parts}\

\question Luis es un estudiante bastante despistado y su tutora está cansada de que llegue
tarde a clase. Él se defiende diciendo que no es para tanto y que la tutora le tiene
manía. Ella le propone el siguiente trato: si en los próximos 9 días Luis llega tarde como mucho 2 días, la tutora le sube 1 punto en la nota final de la evaluación.
Sabiendo que la probabilidad de que Luis llegue tarde a clase cada día es 0,45,
determine:
\begin{parts}
    \part[1] El tipo de distribución que sigue la variable aleatoria que cuenta el número
de días que Luis llega tarde a clase en los próximos 9 días. ¿Cuáles son sus
parámetros?
    \part[1] ¿Cuál es la media y la desviación típica de esta distribución?
    \part[1] ¿Cuál es la probabilidad de que Luis consiga la ansiada subida de 1
punto en la nota final?
\end{parts}


\question[1] La probabilidad de que un determinado equipo de fútbol gane cuando juega en casa
es 2/3, y la probabilidad de que gane cuando juega fuera es 2/5.
\begin{parts}
    \part[1] Sin saber dónde jugará el próximo partido, calcule la probabilidad de que
gane.
    \part[1] Si ganó el último partido del campeonato, ¿cuál es la probabilidad de
que jugara en casa?
\end{parts}

\question En una clase hay 40 estudiantes, de los cuales 25 son chicas y el resto
son chicos. Además, 30 estudiantes han aprobado las matemáticas, de los cuales 10
son chicos.
\begin{parts}
    \part[1] Elegido un estudiante al azar, ¿Cuál es la probabilidad de que no haya aprobado las matemáticas?
    \part[1] ¿Y de que sea chica y haya aprobado las matemáticas?
    \part[1] Si se elige un estudiante que ha aprobado las matemáticas, ¿cuál es la
probabilidad de que sea una chica?
\end{parts}


%        \question Calcula los siguientes límites:
%        \begin{multicols}{1}
%        \begin{parts} \part[1] $$\lim_{x \to 3}\left(\frac{3 x^{2} - 11 x + 6}{x^{3} - 3 x^{2} + x - 3}\right)$$  \begin{solution}   $\frac{7}{10}$   \end{solution} \part[1] $$\lim_{x \to \infty} e^{1 - x}$$  \begin{solution}   $0$   \end{solution} \part[1] $$\lim_{x \to -2}\left(\frac{x^{3} + x^{2} - x + 2}{x^{2} + 4 x + 4}\right)$$  \begin{solution}   No existe el límite   \end{solution} \part[2] $$\lim_{x \to 2} \left(\frac{x^{3} - 4}{x^{2}}\right)^{\frac{1}{x - 2}}$$  \begin{solution}   $e^{2}$   \end{solution}
%        \end{parts}
%        \end{multicols}
        

% \question Luis es saltador de altura, y en el 70\% de sus saltos consigue superar los 2.10 m. Sabiendo que en una competición tiene que saltar tres veces, halla la probabilidad de que:

% \begin{parts} 
% \part[1]  En todas supere los 2.10 m.  \begin{solution}  $ 0.3430$  \end{solution} 
% \part[1]  No los supere en ninguna  \begin{solution}  $0.02700 $  \end{solution}
% \part[1]  Si su primer salto fue nulo, supere los 2.10 m en, al menos, una ocasión.  \begin{solution}  $0.9100$  \end{solution}
%         \end{parts}

\question En una urna hay 10 bolas blancas y 3 negras. Se extrae una bola al azar y, sin verla ni
reemplazarla, se extrae una segunda bola.
\begin{parts}
    \part[1] ¿Cuál es la probabilidad de que la segunda bola extraída sea negra?\begin{solution}$\frac{3}{13}$\end{solution}
    
    \part[1] Sabiendo que la segunda bola ha sido negra, calcule la probabilidad de que la primera bola extraída fuera negra también.\begin{solution}$\frac{1}{6}$\end{solution}
\end{parts}

\question Un examen de tipo test consta de 8 preguntas, cada una con cinco respuestas, de las cuales solo
una es correcta. Si un alumno contesta al azar:
\begin{parts}
    \part[1] ¿Cuál es la probabilidad de que conteste correctamente 4 preguntas?\begin{solution}$0.0458752000000000$\end{solution}
    \part[1] ¿Y la de que conteste bien 2 preguntas o más?\begin{solution}$0.496683520000000$\end{solution}

\end{parts}
        
\question Una máquina produce recipientes cuyas capacidades siguen una distribución normal de media 100 cl y de desviación típica 0,9.
\begin{parts}
    \part[1] Calcula la probabilidad de que la capacidad de un recipiente elegido al azar sea menor que 101 \begin{solution}$0.866739737097495$\end{solution}
    \part[1] Calcula la probabilidad de que la capacidad de un recipiente elegido al azar sea mayor que 99\begin{solution}$0.866739737097495$\end{solution}
    \part[1] Si el fabricante considera que un recipiente es defectuoso si su capacidad no está entre 99 y 101. ¿Qué
probabilidad tiene un recipiente de ser considerado defectuoso?\begin{solution}$1-0.733479474194989=0.266520525805011$\end{solution}
\end{parts}

        % \question Determina el dominio de definición de las siguientes funciones:
        % \begin{parts}
        % % reduce_inequalities([(x**2-x)/(x+2)>=0]).as_set()
        % \part[1] $f(x)=\sqrt{\dfrac{x^2 -x}{x+2}}$\begin{solution}  $\left(-2, 0\right] \cup \left[1, \infty\right)$\end{solution}
        % \part[1] $f(x)=\dfrac{3x+2}{x^4-5x^2-36}$\begin{solution}$\left(-\infty, -3\right] \cup \left[3, \infty\right)$\end{solution}
        % \end{parts}
        
\end{questions}
    
    \newgeometry{left=1 cm,bottom=2cm}
% \begin{landscape}
\begin{table}
% \Large
\centering

% \caption{Extracto de tabla de probabilidades de la \textbf{normal estándar $Z(0,1)$}}
\caption{Tabla de probabilidades de la \textbf{normal estándar $Z(0,1)$}}
\label{my-label}

\begin{tabular}{l|llllllllll}
z   & 0       & 0,01    & 0,02    & 0,03    & 0,04    & 0,05    & 0,06    & 0,07    & 0,08    & 0,09    \\
\hline
0   & 0,5     & 0,50399 & 0,50798 & 0,51197 & 0,51595 & 0,51994 & 0,52392 & 0,5279  & 0,53188 & 0,53586 \\
0,1 & 0,53983 & 0,5438  & 0,54776 & 0,55172 & 0,55567 & 0,55962 & 0,56356 & 0,56749 & 0,57142 & 0,57535 \\
0,2 & 0,57926 & 0,58317 & 0,58706 & 0,59095 & 0,59483 & 0,59871 & 0,60257 & 0,60642 & 0,61026 & 0,61409 \\
0,3 & 0,61791 & 0,62172 & 0,62552 & 0,6293  & 0,63307 & 0,63683 & 0,64058 & 0,64431 & 0,64803 & 0,65173 \\
0,4 & 0,65542 & 0,6591  & 0,66276 & 0,6664  & 0,67003 & 0,67364 & 0,67724 & 0,68082 & 0,68439 & 0,68793 \\
0,5 & 0,69146 & 0,69497 & 0,69847 & 0,70194 & 0,7054  & 0,70884 & 0,71226 & 0,71566 & 0,71904 & 0,7224  \\
0,6 & 0,72575 & 0,72907 & 0,73237 & 0,73565 & 0,73891 & 0,74215 & 0,74537 & 0,74857 & 0,75175 & 0,7549  \\
0,7 & 0,75804 & 0,76115 & 0,76424 & 0,7673  & 0,77035 & 0,77337 & 0,77637 & 0,77935 & 0,7823  & 0,78524 \\
0,8 & 0,78814 & 0,79103 & 0,79389 & 0,79673 & 0,79955 & 0,80234 & 0,80511 & 0,80785 & 0,81057 & 0,81327 \\
0,9 & 0,81594 & 0,81859 & 0,82121 & 0,82381 & 0,82639 & 0,82894 & 0,83147 & 0,83398 & 0,83646 & 0,83891 \\
1   & 0,84134 & 0,84375 & 0,84614 & 0,84849 & 0,85083 & 0,85314 & 0,85543 & 0,85769 & 0,85993 & 0,86214 \\
1,1 & 0,86433 & 0,8665  & 0,86864 & 0,87076 & 0,87286 & 0,87493 & 0,87698 & 0,879   & 0,881   & 0,88298 \\
1,2 & 0,88493 & 0,88686 & 0,88877 & 0,89065 & 0,89251 & 0,89435 & 0,89617 & 0,89796 & 0,89973 & 0,90147 \\
1,3 & 0,9032  & 0,9049  & 0,90658 & 0,90824 & 0,90988 & 0,91149 & 0,91309 & 0,91466 & 0,91621 & 0,91774 \\
1,4 & 0,91924 & 0,92073 & 0,9222  & 0,92364 & 0,92507 & 0,92647 & 0,92785 & 0,92922 & 0,93056 & 0,93189 \\
1,5 & 0,93319 & 0,93448 & 0,93574 & 0,93699 & 0,93822 & 0,93943 & 0,94062 & 0,94179 & 0,94295 & 0,94408 \\
1,6 & 0,9452  & 0,9463  & 0,94738 & 0,94845 & 0,9495  & 0,95053 & 0,95154 & 0,95254 & 0,95352 & 0,95449 \\
1,7 & 0,95543 & 0,95637 & 0,95728 & 0,95818 & 0,95907 & 0,95994 & 0,9608  & 0,96164 & 0,96246 & 0,96327 \\
1,8 & 0,96407 & 0,96485 & 0,96562 & 0,96638 & 0,96712 & 0,96784 & 0,96856 & 0,96926 & 0,96995 & 0,97062 \\
1,9 & 0,97128 & 0,97193 & 0,97257 & 0,9732  & 0,97381 & 0,97441 & 0,975   & 0,97558 & 0,97615 & 0,9767  \\
2   & 0,97725 & 0,97778 & 0,97831 & 0,97882 & 0,97932 & 0,97982 & 0,9803  & 0,98077 & 0,98124 & 0,98169 \\
2,1 & 0,98214 & 0,98257 & 0,983   & 0,98341 & 0,98382 & 0,98422 & 0,98461 & 0,985   & 0,98537 & 0,98574 \\
2,2 & 0,9861  & 0,98645 & 0,98679 & 0,98713 & 0,98745 & 0,98778 & 0,98809 & 0,9884  & 0,9887  & 0,98899 \\
2,3 & 0,98928 & 0,98956 & 0,98983 & 0,9901  & 0,99036 & 0,99061 & 0,99086 & 0,99111 & 0,99134 & 0,99158 \\
2,4 & 0,9918  & 0,99202 & 0,99224 & 0,99245 & 0,99266 & 0,99286 & 0,99305 & 0,99324 & 0,99343 & 0,99361 \\
2,5 & 0,99379 & 0,99396 & 0,99413 & 0,9943  & 0,99446 & 0,99461 & 0,99477 & 0,99492 & 0,99506 & 0,9952  \\
2,6 & 0,99534 & 0,99547 & 0,9956  & 0,99573 & 0,99585 & 0,99598 & 0,99609 & 0,99621 & 0,99632 & 0,99643 \\
2,7 & 0,99653 & 0,99664 & 0,99674 & 0,99683 & 0,99693 & 0,99702 & 0,99711 & 0,9972  & 0,99728 & 0,99736 \\
2,8 & 0,99744 & 0,99752 & 0,9976  & 0,99767 & 0,99774 & 0,99781 & 0,99788 & 0,99795 & 0,99801 & 0,99807 \\
2,9 & 0,99813 & 0,99819 & 0,99825 & 0,99831 & 0,99836 & 0,99841 & 0,99846 & 0,99851 & 0,99856 & 0,99861 \\
3   & 0,99865 & 0,99869 & 0,99874 & 0,99878 & 0,99882 & 0,99886 & 0,99889 & 0,99893 & 0,99896 & 0,999   \\
3,1 & 0,99903 & 0,99906 & 0,9991  & 0,99913 & 0,99916 & 0,99918 & 0,99921 & 0,99924 & 0,99926 & 0,99929 \\
3,2 & 0,99931 & 0,99934 & 0,99936 & 0,99938 & 0,9994  & 0,99942 & 0,99944 & 0,99946 & 0,99948 & 0,9995  \\
3,3 & 0,99952 & 0,99953 & 0,99955 & 0,99957 & 0,99958 & 0,9996  & 0,99961 & 0,99962 & 0,99964 & 0,99965 \\
3,4 & 0,99966 & 0,99968 & 0,99969 & 0,9997  & 0,99971 & 0,99972 & 0,99973 & 0,99974 & 0,99975 & 0,99976 \\
3,5 & 0,99977 & 0,99978 & 0,99978 & 0,99979 & 0,9998  & 0,99981 & 0,99981 & 0,99982 & 0,99983 & 0,99983 \\
3,6 & 0,99984 & 0,99985 & 0,99985 & 0,99986 & 0,99986 & 0,99987 & 0,99987 & 0,99988 & 0,99988 & 0,99989 \\
3,7 & 0,99989 & 0,9999  & 0,9999  & 0,9999  & 0,99991 & 0,99991 & 0,99992 & 0,99992 & 0,99992 & 0,99992 \\
3,8 & 0,99993 & 0,99993 & 0,99993 & 0,99994 & 0,99994 & 0,99994 & 0,99994 & 0,99995 & 0,99995 & 0,99995 \\
3,9 & 0,99995 & 0,99995 & 0,99996 & 0,99996 & 0,99996 & 0,99996 & 0,99996 & 0,99996 & 0,99997 & 0,99997 \\
4   & 0,99997 & 0,99997 & 0,99997 & 0,99997 & 0,99997 & 0,99997 & 0,99998 & 0,99998 & 0,99998 & 0,99998
\end{tabular}
\end{table}
% \end{landscape}
\restoregeometry
\end{document}
    