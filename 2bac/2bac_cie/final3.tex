\documentclass[addpoints,spanish, 12pt,a4paper]{exam}
%\documentclass[answers, spanish, 12pt,a4paper]{exam}
% \printanswers
\pointpoints{punto}{puntos}
\hpword{Puntos:}
\vpword{Puntos:}
\htword{Total}
\vtword{Total}
\hsword{Resultado:}
\hqword{Ejercicio:}
\vqword{Ejercicio:}

\usepackage[utf8]{inputenc}
\usepackage[spanish]{babel}
\usepackage{eurosym}
%\usepackage[spanish,es-lcroman, es-tabla, es-noshorthands]{babel}


\usepackage[margin=1in]{geometry}
\usepackage{amsmath,amssymb}
\usepackage{multicol}
\usepackage{yhmath}

\pointsinrightmargin % Para poner las puntuaciones a la derecha. Se puede cambiar. Si se comenta, sale a la izquierda.
\extrawidth{-2.4cm} %Un poquito más de margen por si ponemos textos largos.
\marginpointname{ \emph{\points}}

\usepackage{graphicx}

\graphicspath{{../../img/}} 

\newcommand{\class}{2º Bachillerato CIT}
\newcommand{\examdate}{\today}
\newcommand{\examnum}{Final 3ªEv.}
\newcommand{\tipo}{A}


\newcommand{\timelimit}{90 minutos}

\renewcommand{\solutiontitle}{\noindent\textbf{Solución:}\enspace}


\pagestyle{head}
\firstpageheader{\includegraphics[width=0.2\columnwidth]{header_left}}{\textbf{Departamento de Matemáticas\linebreak \class}\linebreak \examnum}{\includegraphics[width=0.1\columnwidth]{header_right}}
\runningheader{\class}{\examnum}{Página \thepage\ of \numpages}
\runningheadrule


\usepackage{pgf,tikz,pgfplots}
\pgfplotsset{compat=1.15}
\usepackage{mathrsfs}
\usetikzlibrary{arrows}


\begin{document}

\noindent
\begin{tabular*}{\textwidth}{l @{\extracolsep{\fill}} r @{\extracolsep{6pt}} }
\textbf{Nombre:} \makebox[3.5in]{\hrulefill} & \textbf{Fecha:}\makebox[1in]{\hrulefill} \\
 & \\
\textbf{Tiempo: \timelimit} & Tipo: \tipo 
\end{tabular*}
\rule[2ex]{\textwidth}{2pt}
Esta prueba tiene \numquestions\ ejercicios. La puntuación máxima es de \numpoints. 
La nota final de la prueba será la parte proporcional de la puntuación obtenida sobre la puntuación máxima. 

\begin{center}


\addpoints
 %\gradetable[h][questions]
	\pointtable[h][questions]
\end{center}

\noindent
\rule[2ex]{\textwidth}{2pt}

\begin{questions}

%\question 
%
%\begin{parts}
%\part[2] 
%\begin{solution}
%\end{solution}
%



% \question Los puntos $A= (0,-1,1)$, $B= (1,1,1)$ son dos vértices de un triángulo. El tercer vértice, $C$, está contenido en la recta $r$ que pasa por el punto $B$ y es perpendicular al plano $\pi \equiv 2x - y + z = 1$. 
% \begin{parts}
%     \part Calcule la ecuación de la recta $r$
%     \part Calcule las coordenadas del vértice C sabiendo que el área del triángulo es $3\sqrt{30}$
% \end{parts}

% \question Los puntos $A = (2, 0, 0)$, $B = (-1, 12, 4)$ son dos vértices de un triángulo. El tercer vértice se encuentra en la recta $r\equiv \left\{\begin{matrix}
%     4x+3z= 33 \\
%     y=0
% \end{matrix}\right.$
% \begin{parts}
%     \part Calcule las coordenadas del tercer vértice $C$, sabiendo que la recta $r$ es perpendicular a la recta que pasa por los puntos $A$ y $C$
%     \part Determine el ángulo que forman los vectores $\overrightarrow{AB}$ y $\overrightarrow{AC}$
%     \part Calcule el área del triángulo $ABC$
    
% \end{parts}

% Problema 19.1.3 (2,5 puntos) 

\question Dados los puntos $A(1, 2, -3)$; $B(1, 5, 0)$; $C(5, 6, -1)$ y $D(4, -1, 3)$,
se pide:
\begin{parts}
    \part[1] Calcular el plano $\pi$ que contiene a los puntos $A$,$B$ y $C$
    \part[1] Calcular el volumen del tetraedro determinado por los cuatro puntos
    \part[1] Calcular el área del triángulo determinado por $A$,$B$ y $C$ 
\end{parts}
\begin{solution}
    $\left( x - 2 y + 2 z = -9, \  21, \  9\right)$
\end{solution}

% Problema 0.1.1

\question  Dados los vectores $\overrightarrow{u}\left(a,1+a,2a\right)$ , $\overrightarrow{v}\left(a,1,a\right)$ y $\overrightarrow{w}\left(1,a,1\right)$ se pide:
\begin{parts}
	\part[1] Determinar los valores de $a$ para que los vectores $\overrightarrow{u}$, $\overrightarrow{v}$ y $\overrightarrow{w}$ sean linealmente dependientes
    \begin{solution}
    si $a=0,-1,1 \to$ linealmente dependientes    
    \end{solution}
    % \part[1] Estudiar si el vector $\overrightarrow{c}=(3,3,0)$ depende linealmente de $\overrightarrow{u}$, $\overrightarrow{v}$ y $\overrightarrow{w}$ para el caso $a=2$. Justifica la respuesta
\end{parts}  


% problema 16.1.3 
\question Dados el plano $\pi\equiv x+2y-z =5$ y la recta $r: \left\{\begin{matrix}
    x+y-2z= 1 \\
    2x+y-z=2
\end{matrix}\right.$
\begin{parts}
	\part[1] Determina la ecuación del plano que contiene a la recta $r$ y pasa por el punto $P(1,0,1)$
	\part[1] Hallar la ecuación de la recta, en forma implícita (intersección de dos planos), que es perpendicular al plano $\pi$ y pasa por el punto $Q(2,1,1)$ 
\end{parts}


\question Dado el plano $\pi \equiv m x + n z + y = 1 $ y la recta
$r: \left\{\begin{matrix}
    x + 2 y - z = 2 \\
    - x + y + 2 z = 4
\end{matrix}\right.$ se pide: 
\begin{parts}
    \part[1] Determina, en forma paramétrica, la recta paralela a $r$ que pasa por el origen de coordenadas 
    \part[1] Determina los valores de $m$ y $n$ para que el plano $\pi$ contenga a la recta $r$
\end{parts}
\begin{solution}\\
    $x + 2 y - z - 2, \  - x + y + 2 z - 4,  \  m x + n z + y - 1, \ $ \\ $  \operatorname{Point3D}\left(-2, 2, 0\right), \  \operatorname{Point3D}\left(5, -1, 3\right),$ \\ $ \  \left[ 5 m + 3 n - 1, \  - 2 m + n z + 1\right], \  \left\{ m : \frac{3}{ 6}, \  n : - \frac{3}{6}\right\}$
\end{solution}



\addpoints

\newpage

\hrulefill

\textbf{Recuperación de Álgebra:} Esta parte solo tienes que hacerla si suspendiste el examen de Álgebra 
\\
\noaddpoints

\question \textbf{(4 puntos)} Dadas las matrices  $A=\left(\begin{matrix}2 & 1 & 3\\1 & -1 & 0\end{matrix}\right)$,  $B=\left(\begin{matrix}1 & -1\\2 & 0\\0 & 1\end{matrix}\right)$, $C=\left(\begin{matrix}2 & 3\\1 & 4\end{matrix}\right)$ y $D=\left(\begin{matrix}1 & 8\\0 & 5\end{matrix}\right)$ 
%\noaddpoints % to omit double points count

\begin{parts}
\part Calcule $2C$ y $AB$ 

\part Encontrar, si existe, una matriz $X$ tal que: $AB+2CX=D$
\end{parts}
\begin{solution}
$AB=\left(\begin{matrix}4 & 1\\-1 & -1\end{matrix}\right)$, $2C=\left(\begin{matrix}4 & 6\\2 & 8\end{matrix}\right)$, $D-AB = \left(\begin{matrix}-3 & 7\\1 & 6\end{matrix}\right)$ y $(2C)^{-1}:\left(\begin{matrix}4 & 6\\2 & 8\end{matrix}\right)\xrightarrow{traspuesta}\left(\begin{matrix}4 & 2\\6 & 8\end{matrix}\right)\xrightarrow{adjunta}\left(\begin{matrix}8 & -6\\-2 & 4\end{matrix}\right)\xrightarrow{inversa}\left(\begin{matrix}\frac{2}{5} & - \frac{3}{10}\\- \frac{1}{10} & \frac{1}{5}\end{matrix}\right)$. \\ Por tanto, $X=(2C)^{-1}\cdot (D-AB)= \left(\begin{matrix}\frac{2}{5} & - \frac{3}{10}\\- \frac{1}{10} & \frac{1}{5}\end{matrix}\right)\cdot \left(\begin{matrix}-3 & 7\\1 & 6\end{matrix}\right)=\left(\begin{matrix}- \frac{3}{2} & 1\\\frac{1}{2} & \frac{1}{2}\end{matrix}\right)$

\end{solution}

% \question \textbf{(4 puntos)} Se consideran las matrices $A=\left( \begin{matrix}
%    1 & 0  \\
%    2 & k  \\
%    0 & 1  \\
% \end{matrix} \right)$. y $B=\left( \begin{matrix}
%    k & 0 & -1  \\
%    1 & 1 & 2  \\
% \end{matrix} \right)$.
% \begin{parts}
%  \part Discutir en función de los valores que pueda tomar k, si la matriz AB tiene inversa.
%  \part Discutir, en función de los valores de k, si la matriz BA tiene inversa.
% \end{parts}

\question \textbf{(4 puntos)} Considera la matriz $A=\left( \begin{matrix}
   a & b & c  \\
   2a & -b & 3c  \\
   3a & 0 & 4c  \\
\end{matrix} \right)$ donde a, b y c son no nulos.
\begin{parts}
    \part Determina el número de columnas de A que son linealmente independientes.
    \part Calcula el rango de A y razona si la matriz tiene inversa
\end{parts}



\question \textbf{(4 puntos)} Discutir y resolver el siguiente sistema de acuerdo con los valores del parámetro m.
	$$\left\{ \begin{matrix}
  & 5x+4y+2z=0 \\ 
 & 2x+3y+z=0 \\ 
 & 4x-y+{{m}^{2}}z=m-1 \\ 
\end{matrix} \right.$$	


\end{questions}

\end{document}
\grid
