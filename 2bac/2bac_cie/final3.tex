\documentclass[addpoints,spanish, 12pt,a4paper]{exam}
%\documentclass[answers, spanish, 12pt,a4paper]{exam}
% \printanswers
\pointpoints{punto}{puntos}
\hpword{Puntos:}
\vpword{Puntos:}
\htword{Total}
\vtword{Total}
\hsword{Resultado:}
\hqword{Ejercicio:}
\vqword{Ejercicio:}

\usepackage[utf8]{inputenc}
\usepackage[spanish]{babel}
\usepackage{eurosym}
%\usepackage[spanish,es-lcroman, es-tabla, es-noshorthands]{babel}


\usepackage[margin=1in]{geometry}
\usepackage{amsmath,amssymb}
\usepackage{multicol}
\usepackage{yhmath}

\pointsinrightmargin % Para poner las puntuaciones a la derecha. Se puede cambiar. Si se comenta, sale a la izquierda.
\extrawidth{-2.4cm} %Un poquito más de margen por si ponemos textos largos.
\marginpointname{ \emph{\points}}

\usepackage{graphicx}

\graphicspath{{../../img/}} 

\newcommand{\class}{2º Bachillerato CIT}
\newcommand{\examdate}{\today}
\newcommand{\examnum}{Final 3ªEv. - Geometría y recuperación de Álgebra}
\newcommand{\tipo}{A}


\newcommand{\timelimit}{45 minutos}

\renewcommand{\solutiontitle}{\noindent\textbf{Solución:}\enspace}


\pagestyle{head}
\firstpageheader{\includegraphics[width=0.2\columnwidth]{header_left}}{\textbf{Departamento de Matemáticas\linebreak \class}\linebreak \examnum}{\includegraphics[width=0.1\columnwidth]{header_right}}
\runningheader{\class}{\examnum}{Página \thepage\ of \numpages}
\runningheadrule


\usepackage{pgf,tikz,pgfplots}
\pgfplotsset{compat=1.15}
\usepackage{mathrsfs}
\usetikzlibrary{arrows}


\begin{document}

\noindent
\begin{tabular*}{\textwidth}{l @{\extracolsep{\fill}} r @{\extracolsep{6pt}} }
\textbf{Nombre:} \makebox[3.5in]{\hrulefill} & \textbf{Fecha:}\makebox[1in]{\hrulefill} \\
 & \\
\textbf{Tiempo: \timelimit} & Tipo: \tipo 
\end{tabular*}
\rule[2ex]{\textwidth}{2pt}
Esta prueba tiene \numquestions\ ejercicios. La puntuación máxima es de \numpoints. 
La nota final de la prueba será la parte proporcional de la puntuación obtenida sobre la puntuación máxima. 

\begin{center}


\addpoints
 %\gradetable[h][questions]
	\pointtable[h][questions]
\end{center}

\noindent
\rule[2ex]{\textwidth}{2pt}

\begin{questions}

%\question 
%
%\begin{parts}
%\part[2] 
%\begin{solution}
%\end{solution}
%
%
%\end{parts}
%\addpoints


% % Aragón Septiembre 14
% \question[3] Resuelve el siguiente sistema matricial: $$\left\{ \begin{matrix}A+3B=\left(\begin{matrix}-4 & -2\\3 & -4\end{matrix}\right) \\ 2A-B= \left(\begin{matrix}-1 & 3\\-1 & -1\end{matrix}\right) \\ \end{matrix}\right.$$
% \begin{solution}
%     Sea $M=\left(\begin{matrix}-4 & -2\\3 & -4\end{matrix}\right)$ y $N=\left(\begin{matrix}-1 & 3\\-1 & -1\end{matrix}\right)$ \\

%     $A=\frac{1}{7}(M+3N)=\left(\begin{matrix}-1.0 & 1.0\\0 & -1.0\end{matrix}\right)$\\
%         $B=\frac{1}{7}(2M-N)=\left(\begin{matrix}-1.0 & -1.0\\1.0 & -1.0\end{matrix}\right)$ \\
% \end{solution}

% % Murcia Junio 21
% \question Dado el sistema de ecuaciones en función del parámetro $a$:$$\left\{ \begin{matrix}a x + y + z = 4 \\  x- a y + z = 1 \\ x + y + z = a + 2 \\ \end{matrix}\right.$$
% \begin{parts}
% \part[1] Determine para qué valores de a el sistema tiene solución única
% \part[2] Determine para qué valor de a el sistema tiene infinitas soluciones y resuélvalo en ese caso
% \part[1] Determine para qué valor de a el sistema no tiene solución
% \end{parts}
% \begin{solution}
% $\left\{ \begin{matrix}a x + y + z = 4 \\ - a y + x + z = 1 \\ x + y + z = a + 2 \\ \end{matrix}\right.$\\$|A|=\left|\begin{matrix}a & 1 & 1\\1 & - a & 1\\1 & 1 & 1\end{matrix}\right|=- \left(a - 1\right) \left(a + 1\right)$\\* Si $a \neq -1, 1\to |A| \neq 0 \to \exists A^{-1}$.\\Como $rg(A)=rg(A^*)=3$ --> S.C. (El sistema tiene solución) \\ Como además $rg(A)=3$ coincide con el número de incógnitas --> S.C.D -- > Se puede resolver por Gauss, Matriz inversa o Cramer \\ \textbf{Resolución por Gauss} \\ $A^*=\left(\begin{matrix}a & 1 & 1 & 4\\1 & - a & 1 & 1\\1 & 1 & 1 & a + 2\end{matrix}\right)\sim\left(\begin{matrix}1 & - a & 1 & 1\\0 & a^{2} + 1 & 1 - a & 4 - a\\0 & 0 & \frac{a^{2} - 1}{a^{2} + 1} & \frac{\left(a + 1\right) \left(a^{2} + a - 3\right)}{a^{2} + 1}\end{matrix}\right)\to x=\frac{2 - a}{a - 1}, y=1, z=\frac{a^{2} + a - 3}{a - 1}$ \\ \textbf{Método de la matríz inversa} \\ $A^{-1}=\left(\begin{matrix}\frac{1}{a - 1} & 0 & - \frac{1}{a - 1}\\0 & - \frac{1}{a + 1} & \frac{1}{a + 1}\\- \frac{1}{a - 1} & \frac{1}{a + 1} & \frac{a^{2} + 1}{a^{2} - 1}\end{matrix}\right) \to  X=A^{-1}\cdot b =\left(\begin{matrix}\frac{1}{a - 1} & 0 & - \frac{1}{a - 1}\\0 & - \frac{1}{a + 1} & \frac{1}{a + 1}\\- \frac{1}{a - 1} & \frac{1}{a + 1} & \frac{a^{2} + 1}{a^{2} - 1}\end{matrix}\right)\cdot \left(\begin{matrix}4\\1\\a + 2\end{matrix}\right) =\left(\begin{matrix}\frac{2 - a}{a - 1}\\1\\\frac{a^{2} + a - 3}{a - 1}\end{matrix}\right)$ \\ \textbf{Método de Cramer} \\ $x=\frac{\left|\begin{matrix}4 & 1 & 1\\1 & - a & 1\\a + 2 & 1 & 1\end{matrix}\right|}{1 - a^{2}}=\frac{a^{2} - a - 2}{1 - a^{2}}=\frac{a^{2} - a - 2}{1 - a^{2}}$ \\ $y=\frac{\left|\begin{matrix}a & 4 & 1\\1 & 1 & 1\\1 & a + 2 & 1\end{matrix}\right|}{1 - a^{2}}=\frac{1 - a^{2}}{1 - a^{2}}=1$ \\ $z=\frac{\left|\begin{matrix}a & 1 & 4\\1 & - a & 1\\1 & 1 & a + 2\end{matrix}\right|}{1 - a^{2}}=\frac{- a^{3} - 2 a^{2} + 2 a + 3}{1 - a^{2}}=\frac{- a^{3} - 2 a^{2} + 2 a + 3}{1 - a^{2}}$\\* Si $a=-1\to |A| = 0 \to \nexists A^{-1}$\\Como $rg(A)=rg(A^*)=2$ --> S.C. (El sistema tiene solución) \\ Como además $rg(A)=2$ no coincide con el número de incógnitas--> S.C.I --> Solo se puede resolver por Gauss \\ \textbf{Resolución por Gauss} \\ $A^*=\left(\begin{matrix}-1 & 1 & 1 & 4\\1 & 1 & 1 & 1\\1 & 1 & 1 & 1\end{matrix}\right)\sim\left(\begin{matrix}-1 & 1 & 1 & 4\\0 & 2 & 2 & 5\\0 & 0 & 0 & 0\end{matrix}\right)\to x=- \frac{3}{2}, y=\frac{5}{2} - \lambda, z=\lambda$\\* Si $a=1\to |A| = 0 \to \nexists A^{-1}$\\ \\ Como $rg(A)=2 \land rg(A^*)=3$ --> S.I. . Observa que escalonando la matriz ampliada obtenemos $A^*=\left(\begin{matrix}1 & 1 & 1 & 4\\1 & -1 & 1 & 1\\1 & 1 & 1 & 3\end{matrix}\right)\sim\left(\begin{matrix}1 & 1 & 1 & 4\\0 & -2 & 0 & -3\\0 & 0 & 0 & -1\end{matrix}\right)$
% \end{solution}


% % Murcia Julio 20
% \question Considere las matrices: \\
% $A=\left(\begin{matrix}2 & 3\\-1 & -2\end{matrix}\right)
% $ y $B=\left(\begin{matrix}-1 & -3\\1 & 2\end{matrix}\right)$ 
% \begin{parts}
%     \part[2] Compruebe que las matrices A y B tienen inversa y calcúlelas
%     \part[1] Resuelva la ecuación matricial $AXB=A^t-3B$, donde $A^t$ denota la traspuesta de $A$
% \end{parts}
% \begin{solution}
%     $\left|A\right|=-1 \land \left|A\right|=-1$ \\
%     $A^{-1}=\left(\begin{matrix}2 & 3\\-1 & -2\end{matrix}\right)$ \\
%     $B^{-1}=\left(\begin{matrix}2 & 3\\-1 & -1\end{matrix}\right)$
%     $X=A^{-1}\left(A^t+3B\right)B^{-1}=\left(\begin{matrix}28 & 38\\-18 & -23\end{matrix}\right)$
% \end{solution}

% Murcia Julio 20
\question Dado el sistema de ecuaciones en función del parámetro $a$:$$\left\{ \begin{matrix}x + y - z = 4 \\ x + a^{2} y - z = 3 - a \\  x - y + a z = 1 \\ \end{matrix}\right.$$
\begin{parts}
\part[2] Determine para qué valores de a el sistema tiene solución única. Si es posible, calcule dicha solución para $a = 0$
\part[1] Determine para qué valor de $a$ el sistema tiene infinitas soluciones y resuélvalo en ese caso
\part[1] Determine para qué valor de $a$ el sistema no tiene solución
\end{parts}
\begin{solution}
$\left\{ \begin{matrix}x + y - z = 4 \\ a^{2} y + x - z = 3 - a \\ a z + x - y = 1 \\ \end{matrix}\right.$\\$|A|=\left|\begin{matrix}1 & 1 & -1\\1 & a^{2} & -1\\1 & -1 & a\end{matrix}\right|=\left(a - 1\right) \left(a + 1\right)^{2}$\\* Si $a \neq -1, 1\to |A| \neq 0 \to \exists A^{-1}$.\\Como $rg(A)=rg(A^*)=3$ --> S.C. (El sistema tiene solución) \\ Como además $rg(A)=3$ coincide con el número de incógnitas --> S.C.D -- > Se puede resolver por Gauss, Matriz inversa o Cramer \\ \textbf{Resolución por Gauss} \\ $A^*=\left(\begin{matrix}1 & 1 & -1 & 4\\1 & a^{2} & -1 & 3 - a\\1 & -1 & a & 1\end{matrix}\right)\sim\left(\begin{matrix}1 & 1 & -1 & 4\\0 & -2 & a + 1 & -3\\0 & 0 & \frac{\left(a + 1\right) \left(a^{2} - 1\right)}{2} & - \frac{3 a^{2}}{2} - a + \frac{1}{2}\end{matrix}\right)\to x=\frac{4 a + 2}{a + 1}, y=- \frac{1}{a - 1}, z=\frac{1 - 3 a}{a^{2} - 1}$ \\ \textbf{Método de la matríz inversa} \\ $A^{-1}=\left(\begin{matrix}\frac{a^{2} + a + 1}{a^{2} + 2 a + 1} & - \frac{1}{a^{2} + 2 a + 1} & \frac{1}{a + 1}\\- \frac{1}{a^{2} - 1} & \frac{1}{a^{2} - 1} & 0\\\frac{- a^{2} - 1}{a^{3} + a^{2} - a - 1} & - \frac{2}{- a^{3} - a^{2} + a + 1} & \frac{1}{a + 1}\end{matrix}\right) \to  X=A^{-1}\cdot b =\left(\begin{matrix}\frac{a^{2} + a + 1}{a^{2} + 2 a + 1} & - \frac{1}{a^{2} + 2 a + 1} & \frac{1}{a + 1}\\- \frac{1}{a^{2} - 1} & \frac{1}{a^{2} - 1} & 0\\\frac{- a^{2} - 1}{a^{3} + a^{2} - a - 1} & - \frac{2}{- a^{3} - a^{2} + a + 1} & \frac{1}{a + 1}\end{matrix}\right)\cdot \left(\begin{matrix}4\\3 - a\\1\end{matrix}\right) =\left(\begin{matrix}\frac{2 \cdot \left(2 a + 1\right)}{a + 1}\\- \frac{1}{a - 1}\\\frac{1 - 3 a}{a^{2} - 1}\end{matrix}\right)$ \\ \textbf{Método de Cramer} \\ $x=\frac{\left|\begin{matrix}4 & 1 & -1\\3 - a & a^{2} & -1\\1 & -1 & a\end{matrix}\right|}{a^{3} + a^{2} - a - 1}=\frac{4 a^{3} + 2 a^{2} - 4 a - 2}{a^{3} + a^{2} - a - 1}=\frac{4 a^{3} + 2 a^{2} - 4 a - 2}{a^{3} + a^{2} - a - 1}$ \\ $y=\frac{\left|\begin{matrix}1 & 4 & -1\\1 & 3 - a & -1\\1 & 1 & a\end{matrix}\right|}{a^{3} + a^{2} - a - 1}=\frac{- a^{2} - 2 a - 1}{a^{3} + a^{2} - a - 1}=\frac{- a^{2} - 2 a - 1}{a^{3} + a^{2} - a - 1}$ \\ $z=\frac{\left|\begin{matrix}1 & 1 & 4\\1 & a^{2} & 3 - a\\1 & -1 & 1\end{matrix}\right|}{a^{3} + a^{2} - a - 1}=\frac{- 3 a^{2} - 2 a + 1}{a^{3} + a^{2} - a - 1}=\frac{- 3 a^{2} - 2 a + 1}{a^{3} + a^{2} - a - 1}$\\* Si $a=-1\to |A| = 0 \to \nexists A^{-1}$\\ \\ Como $rg(A)=2 \land rg(A^*)=3$ --> S.I. . Observa que escalonando la matriz ampliada obtenemos $A^*=\left(\begin{matrix}1 & 1 & -1 & 4\\1 & -1 & -1 & 4\\1 & -1 & -1 & 1\end{matrix}\right)\sim\left(\begin{matrix}1 & 1 & -1 & 4\\0 & -2 & 0 & 0\\0 & 0 & 0 & -3\end{matrix}\right)$\\* Si $a=1\to |A| = 0 \to \nexists A^{-1}$\\ \\ Como $rg(A)=2 \land rg(A^*)=3$ --> S.I. . Observa que escalonando la matriz ampliada obtenemos $A^*=\left(\begin{matrix}1 & 1 & -1 & 4\\1 & 1 & -1 & 2\\1 & -1 & 1 & 1\end{matrix}\right)\sim\left(\begin{matrix}1 & 1 & -1 & 4\\0 & -2 & 2 & -3\\0 & 0 & 0 & -2\end{matrix}\right)$
\\
Si $a=0$ :
\\ \textbf{Resolución por Gauss} \\ $A^*=\left(\begin{matrix}1 & 1 & -1 & 4\\1 & 0 & -1 & 3\\1 & -1 & 0 & 1\end{matrix}\right)\sim\left(\begin{matrix}1 & 1 & -1 & 4\\0 & -1 & 0 & -1\\0 & 0 & 1 & -1\end{matrix}\right)\to x=2, y=1, z=-1$ \\ \textbf{Método de la matríz inversa} \\ $A^{-1}=\left(\begin{matrix}1 & -1 & 1\\1 & -1 & 0\\1 & -2 & 1\end{matrix}\right) \to  X=A^{-1}\cdot b =\left(\begin{matrix}1 & -1 & 1\\1 & -1 & 0\\1 & -2 & 1\end{matrix}\right)\cdot \left(\begin{matrix}4\\3\\1\end{matrix}\right) =\left(\begin{matrix}2\\1\\-1\end{matrix}\right)$ \\ \textbf{Método de Cramer} \\ $x=\frac{\left|\begin{matrix}4 & 1 & -1\\3 & 0 & -1\\1 & -1 & 0\end{matrix}\right|}{-1}=\frac{-2}{-1}=2$ \\ $y=\frac{\left|\begin{matrix}1 & 4 & -1\\1 & 3 & -1\\1 & 1 & 0\end{matrix}\right|}{-1}=\frac{-1}{-1}=1$ \\ $z=\frac{\left|\begin{matrix}1 & 1 & 4\\1 & 0 & 3\\1 & -1 & 1\end{matrix}\right|}{-1}=\frac{1}{-1}=-1$
\end{solution}

% Murcia Junio 21
\question Considere las matrices: \\
$A=\left(\begin{matrix}1 & 1 & 1\\0 & 1 & 0\\1 & 2 & 2\end{matrix}\right)
$, $B=\left(\begin{matrix}1 & 0\\0 & -1\\2 & 1\end{matrix}\right)$ y $C=\left(\begin{matrix}-2 & 0 & -1\\1 & -1 & 1\end{matrix}\right)$
\begin{parts}
    \part[2] Compruebe que la matriz A es regular (o inversible) y calcule su inversa
    \part[1] Resuelva la ecuación matricial $AX-B=C^t$, donde $C^t$ denota la traspuesta de $C$
\end{parts}
\begin{solution}
    $\left|A\right|=1$ \\
    $A^{-1}=\left(\begin{matrix}2 & 0 & -1\\0 & 1 & 0\\-1 & -1 & 1\end{matrix}\right)$ \\
    $X=A^{-1}\left(C^t+B\right)=\left(\begin{matrix}-3 & 0\\0 & -2\\2 & 3\end{matrix}\right)$
\end{solution}

\question[3] Si $\left| \begin{matrix}
   a & b & c  \\
   d & e & f  \\
   g & h & i  \\
\end{matrix} \right|=7$
 , calcula el valor del siguiente determinante sin desarrollarlo, indicando las propiedades que aplicas:
 $$\left| \begin{matrix}
   3a & 3b & 3c  \\
   a+d & b+e & c+f  \\
   -g+a & -h+b & -i+c  \\
\end{matrix} \right|$$

% \question[2] Demuestra que $\left| \begin{matrix}
%    2 & 3 & 5  \\
%    4 & 9 & 13  \\
%    8 & 27 & 35  \\
% \end{matrix} \right|\text{ }=\text{ 0}$, sin desarrollar, indica las propiedades de los determinantes que aplicas.

% \question	Discutir, en función del parámetro a, el siguiente sistema de ecuaciones y resolverlo para   a= 1 y   a= 5.
% 			$$\,\left. \begin{matrix}
%   & x+y-z=1 \\ 
%  & 3x+ay+az=5 \\ 
%  & 4x+ay=5 \\ 
% \end{matrix} \right\}$$

\hline


\question Los puntos $A= (0,-1,1)$, $B= (1,1,1)$ son dos vértices de un triángulo. El tercer vértice, $C$, está contenido en la recta $r$ que pasa por el punto $B$ y es perpendicular al plano $\pi \equiv 2x - y + z = 1$. 
\begin{parts}
    \part Calcule la ecuación de la recta $r$
    \part Calcule las coordenadas del vértice C sabiendo que el área del triángulo es $3\sqrt{30}$
\end{parts}

\question Los puntos $A = (2, 0, 0)$, $B = (-1, 12, 4)$ son dos vértices de un triángulo. El tercer vértice se encuentra en la recta $r\equiv \left\{\begin{matrix}
    4x+3z= 33 \\
    y=0
\end{matrix}\right.$
\begin{parts}
    \part Calcule las coordenadas del tercer vértice $C$, sabiendo que la recta $r$ es perpendicular a la recta que pasa por los puntos $A$ y $C$
    \part Determine el ángulo que forman los vectores $\overrightarrow{AB}$ y $\overrightarrow{AC}$
    \part Calcule el área del triángulo $ABC$
    
\end{parts}


\question Problema 19.1.3 (2,5 puntos) Dados los puntos $A(1, 2, -3)$; $B(1, 5, 0)$; $C(5, 6, -1)$ y $D(4, -1, 3)$,
se pide:
\begin{parts}
    \part Calcular el plano $\pi$ que contiene a los puntos $A$,$B$ y $C$
    \part Calcular el volumen del tetraedro determinado por los cuatro puntos
    \part Calcular el área del triángulo determinado por $A$,$B$ y $C$ 
\end{parts}
\begin{solution}
    $\left( x - 2 y + 2 z = -9, \  21, \  9\right)$
\end{solution}

\question Se consideran las matrices $A=\left( \begin{matrix}
   1 & 0  \\
   2 & k  \\
   0 & 1  \\
\end{matrix} \right)$. y $B=\left( \begin{matrix}
   k & 0 & -1  \\
   1 & 1 & 2  \\
\end{matrix} \right)$.
Discutir en función de los valores que pueda tomar k, si la matriz AB tiene inversa.	(5 puntos)
Discutir, en función de los valores de k, si la matriz BA tiene inversa.	(5 puntos)

\question Considera la matriz $A=\left( \begin{matrix}
   a & b & c  \\
   2a & -b & 3c  \\
   3a & 0 & 4c  \\
\end{matrix} \right)$ donde a, b y c son no nulos.
Determina el número de columnas de A que son linealmente independientes.	(5 puntos)
	Calcula el rango de A y razona si la matriz tiene inversa.	(5 puntos)


\question Discutir y resolver el siguiente sistema de acuerdo con los valores del parámetro m.
	$\left\{ \begin{matrix}
  & 5x+4y+2z=0 \\ 
 & 2x+3y+z=0 \\ 
 & 4x-y+{{m}^{2}}z=m-1 \\ 
\end{matrix} \right.$	(10 puntos)


\addpoints



\end{questions}

\end{document}
\grid
