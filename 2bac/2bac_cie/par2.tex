\documentclass[addpoints,spanish, 12pt,a4paper]{exam}
%\documentclass[answers, spanish, 12pt,a4paper]{exam}
\printanswers
\pointpoints{punto}{puntos}
\hpword{Puntos:}
\vpword{Puntos:}
\htword{Total}
\vtword{Total}
\hsword{Resultado:}
\hqword{Ejercicio:}
\vqword{Ejercicio:}

\usepackage[utf8]{inputenc}
\usepackage[spanish]{babel}
\usepackage{eurosym}
%\usepackage[spanish,es-lcroman, es-tabla, es-noshorthands]{babel}


\usepackage[margin=1in]{geometry}
\usepackage{amsmath,amssymb}
\usepackage{multicol}
\usepackage{yhmath}

\pointsinrightmargin % Para poner las puntuaciones a la derecha. Se puede cambiar. Si se comenta, sale a la izquierda.
\extrawidth{-2.4cm} %Un poquito más de margen por si ponemos textos largos.
\marginpointname{ \emph{\points}}

\usepackage{graphicx}

\graphicspath{{../../img/}} 

\newcommand{\class}{2º Bachillerato CIT}
\newcommand{\examdate}{\today}
\newcommand{\examnum}{Parcial 2ªEv.}
\newcommand{\tipo}{A}


\newcommand{\timelimit}{45 minutos}

\renewcommand{\solutiontitle}{\noindent\textbf{Solución:}\enspace}


\pagestyle{head}
\firstpageheader{\includegraphics[width=0.2\columnwidth]{header_left}}{\textbf{Departamento de Matemáticas\linebreak \class}\linebreak \examnum}{\includegraphics[width=0.1\columnwidth]{header_right}}
\runningheader{\class}{\examnum}{Página \thepage\ of \numpages}
\runningheadrule


\usepackage{pgf,tikz,pgfplots}
\pgfplotsset{compat=1.15}
\usepackage{mathrsfs}
\usetikzlibrary{arrows}


\begin{document}

\noindent
\begin{tabular*}{\textwidth}{l @{\extracolsep{\fill}} r @{\extracolsep{6pt}} }
\textbf{Nombre:} \makebox[3.5in]{\hrulefill} & \textbf{Fecha:}\makebox[1in]{\hrulefill} \\
 & \\
\textbf{Tiempo: \timelimit} & Tipo: \tipo 
\end{tabular*}
\rule[2ex]{\textwidth}{2pt}
Esta prueba tiene \numquestions\ ejercicios. La puntuación máxima es de \numpoints. 
La nota final de la prueba será la parte proporcional de la puntuación obtenida sobre la puntuación máxima. 

\begin{center}


\addpoints
 %\gradetable[h][questions]
	\pointtable[h][questions]
\end{center}

\noindent
\rule[2ex]{\textwidth}{2pt}

\begin{questions}

%\question 
%
%\begin{parts}
%\part[2] 
%\begin{solution}
%\end{solution}
%
%
%\end{parts}
%\addpoints

\question[1] Determine los valores de $a$ y $b$ para sea
continua la función:
$$f(x)=\left\{\begin{array}{ll}
\dfrac{1}{e^x} & si \ x\leq 0 \\
a\cos(x) + b & si \  0 < x \leq \pi \\
\sen(x)-ax & si \ \pi<x
\end{array}\right.$$
\begin{solution}
Fuera de $x=0$ y $x=\pi$ la función es continua por serlo sus trozos en sus dominios. \\

En $x=0$: \\
$\lim_{x \to 0^-}f=\frac{1}{e^0}=1$\\
$\lim_{x \to 0^+}f=a\cdot \cos(0) +b =a+b \to a+b=1$\\

En $x=\pi$: \\
$\lim_{x \to \pi^-}f=a\cdot \cos(\pi) +b =b-a$\\
$\lim_{x \to \pi^+}f=\sen(\pi) - a\cdot \pi =- a\cdot \pi \to b-a=-a\cdot \pi$\\

$(2-\pi)\cdot a= 1 \to a=\frac{1}{2- \pi} \to b=\frac{1-\pi}{2-\pi}$

\end{solution}

\question Calcula las siguientes integrales:
\begin{parts}
\part[1]
$$\int x^{2} \ln{\left(x \right)}\, dx$$
\begin{solution}
$\left|\begin{matrix}
u = \ln x & du=\frac{1}{x} \, dx \\ dv= x^2 \, dx & v=\frac{x^3}{3}
\end{matrix}\right| \to \frac{x^{3} \ln{\left(x \right)}}{3} - \frac{x^{3}}{9}+K$
\end{solution}

\part[2]
$$\int x^{2} \ln^2{\left(x \right)}\, dx$$
\begin{solution}
$\left|\begin{matrix}
u = \ln^2 x & du=\frac{2}{x}\cdot \ln x \, dx \\ dv= x^2 \, dx & v=\frac{x^3}{3}
\end{matrix}\right| \to \frac{x^{3} \ln{\left(x \right)}^{2}}{3} - \frac{2 x^{3} \ln{\left(x \right)}}{9} + \frac{2 x^{3}}{27}+K$
\end{solution}

\part[1]
$$\int\limits_{2}^{3} \frac{1}{2 x^{2} - 4 x + 2}\, dx$$
\begin{solution}
$\frac{1}{4}  \land F(x)= -\frac{1}{2 x - 2}$
\end{solution}
\end{parts}


\addpoints

\end{questions}

\end{document}
\grid
