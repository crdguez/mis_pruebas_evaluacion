\documentclass[addpoints,spanish, 12pt,a4paper]{exam}
%\documentclass[answers, spanish, 12pt,a4paper]{exam}
\printanswers
\pointpoints{punto}{puntos}
\hpword{Puntos:}
\vpword{Puntos:}
\htword{Total}
\vtword{Total}
\hsword{Resultado:}
\hqword{Ejercicio:}
\vqword{Ejercicio:}

\usepackage[utf8]{inputenc}
\usepackage[spanish]{babel}
\usepackage{eurosym}
%\usepackage[spanish,es-lcroman, es-tabla, es-noshorthands]{babel}


\usepackage[margin=1in]{geometry}
\usepackage{amsmath,amssymb}
\usepackage{multicol}
\usepackage{yhmath}

\pointsinrightmargin % Para poner las puntuaciones a la derecha. Se puede cambiar. Si se comenta, sale a la izquierda.
\extrawidth{-2.4cm} %Un poquito más de margen por si ponemos textos largos.
\marginpointname{ \emph{\points}}

\usepackage{graphicx}

\graphicspath{{../../img/}} 

\newcommand{\class}{2º Bachillerato CIT}
\newcommand{\examdate}{\today}
\newcommand{\examnum}{Parcial 2ªEv.}
\newcommand{\tipo}{A}


\newcommand{\timelimit}{45 minutos}

\renewcommand{\solutiontitle}{\noindent\textbf{Solución:}\enspace}


\pagestyle{head}
\firstpageheader{\includegraphics[width=0.2\columnwidth]{header_left}}{\textbf{Departamento de Matemáticas\linebreak \class}\linebreak \examnum}{\includegraphics[width=0.1\columnwidth]{header_right}}
\runningheader{\class}{\examnum}{Página \thepage\ of \numpages}
\runningheadrule


\usepackage{pgf,tikz,pgfplots}
\pgfplotsset{compat=1.15}
\usepackage{mathrsfs}
\usetikzlibrary{arrows}


\begin{document}

\noindent
\begin{tabular*}{\textwidth}{l @{\extracolsep{\fill}} r @{\extracolsep{6pt}} }
\textbf{Nombre:} \makebox[3.5in]{\hrulefill} & \textbf{Fecha:}\makebox[1in]{\hrulefill} \\
 & \\
\textbf{Tiempo: \timelimit} & Tipo: \tipo 
\end{tabular*}
\rule[2ex]{\textwidth}{2pt}
Esta prueba tiene \numquestions\ ejercicios. La puntuación máxima es de \numpoints. 
La nota final de la prueba será la parte proporcional de la puntuación obtenida sobre la puntuación máxima. 

\begin{center}


\addpoints
 %\gradetable[h][questions]
	\pointtable[h][questions]
\end{center}

\noindent
\rule[2ex]{\textwidth}{2pt}

\begin{questions}

%\question 
%
%\begin{parts}
%\part[2] 
%\begin{solution}
%\end{solution}
%
%
%\end{parts}
%\addpoints

\question[1] Determine los valores de $a$ y $b$ para sea
continua la función:
$$f(x)=\left\{\begin{array}{ll}
\dfrac{1}{e^x} & si \ x\leq 0 \\
a\cos(x) + b & si \  0 < x \leq \pi \\
\sen(x)-ax & si \ \pi<x
\end{array}\right.$$
\begin{solution}
Fuera de $x=0$ y $x=\pi$ la función es continua por serlo sus trozos en sus dominios. \\

En $x=0$: \\
$\lim_{x \to 0^-}f=\frac{1}{e^0}=1$\\
$\lim_{x \to 0^+}f=a\cdot \cos(0) +b =a+b \to a+b=1$\\

En $x=\pi$: \\
$\lim_{x \to \pi^-}f=a\cdot \cos(\pi) +b =b-a$\\
$\lim_{x \to \pi^+}f=\sen(\pi) - a\cdot \pi =- a\cdot \pi \to b-a=-a\cdot \pi$\\

$(2-\pi)\cdot a= 1 \to a=\frac{1}{2- \pi} \to b=\frac{1-\pi}{2-\pi}$

\end{solution}

\question Calcula las siguientes integrales:
\begin{parts}
\part[1]
$$\int x \ln{\left(x \right)}\, dx$$
\begin{solution}
$\left|\begin{matrix}
u = \ln x & du=\frac{1}{x} \, dx \\ dv= x \, dx & v=\frac{x^3}{3}
\end{matrix}\right| \to \frac{x^{3} \ln{\left(x \right)}}{3} - \frac{x^{3}}{9}+K$
\end{solution}

\part[2]
$$\int x^{2} \ln^2{\left(x \right)}\, dx$$
\begin{solution}
$\left|\begin{matrix}
u = \ln^2 x & du=\frac{2}{x}\cdot \ln x \, dx \\ dv= x^2 \, dx & v=\frac{x^3}{3}
\end{matrix}\right| \to \frac{x^{3} \ln{\left(x \right)}^{2}}{3} - \frac{2 x^{3} \ln{\left(x \right)}}{9} + \frac{2 x^{3}}{27}+K$
\end{solution}

\part[2]
$$\int\limits_{2}^{3} \frac{1}{2 x^{2} - 4 x + 2}\, dx$$
\begin{solution}
$\frac{1}{4}  \land F(x)= -\frac{1}{2 x - 2}$
\end{solution}
\end{parts}


\question[3] Calcule las dimensiones de tres campos cuadrados que no tienen ningún lado común y que
satisfacen que el perímetro de uno de ellos es triple que el de otro y, además, se necesitan 1248 metros de
valla para vallar completamente los tres campos, de manera que la suma de las áreas es la mínima posible.
\begin{solution}
    Lados= $x$, $3x$ e $y$ \\
    $4x+4(3x)+4y=1248 \to 4(x+3x+y)=1248 \to 4x+y=312 \to y=312-4x$ \\
    maximizar $f(x)=x^2+(3x)^2+(312-4x)^2=26 x^{2} - 2496 x + 97344= x^{2} - 96 x + 3744$ \\
    $f'(x)=0 \to 2x-96=0 \to x=48 \land f''(48)=2>0 \to$ en $x=48$ hay un mínimo.
    Solución: 48, 144, 120 m  
    
\end{solution}

\question[2] Descomponer el número 12 en dos sumandos positivos de forma que el producto
del primero por el cuadrado del segundo sea máximo.
\begin{solution}
    $x$ e $y \to x+y =12 \to y=12-x$ \\
    Maximizar $f(x)=x\cdot y^2 \to f(x)=x\left(12-x\right)^2$ \\
    $f(x)=x^{3} - 24 x^{2} + 144 x \to f'(x)=3 x^{2} - 48 x + 144$ \\
    $f'(x)=0 \to x=4 y x=12 \to f''(x)=6x-48 \to f''(4)=-24<0 \land f''(12)=24>0 \to$ Solución. $x=4, y=8$
\end{solution}

\question[2] Calcule el valor de $a \in \mathbb{R}$, con $a\neq 0$, para que: $$\lim_{x \to 0}\dfrac{e^{a\cdot x^2}-\cos{x}}{\ln\left({x^2+1}\right)}=\frac{5}{6}$$
\begin{solution}
    L'H $\lim = \frac{2a+1}{2} \to a = \frac{1}{3}$
\end{solution}

\question[2] Calcule el valor de $k$ para que: $$\lim_{x \to 0}\dfrac{e^x-e^{-x}+kx}{x-\sen{x}}=2$$
\begin{solution}
    L'H $\lim =\frac{2+k}{0}  \to k = -2$
\end{solution}
\addpoints

% problema 23.2.2 p523
\question[2] Determine el área de la región limitada inferiormente por la parábola $y=9x-x^2$ y superiormente por las rectas tangentes a la parábola en los puntos de corte de ella con el eje $OX$

% problema 3.3.2 p69
\question Sea la función $f(x)=x\cdot e^{x^2}$
\begin{parts}
    \part[1]Hallar la ecuación de la recta tangente a la gráfica de $f (x)$ en el punto de abscisa $x = 1$.
    \part[1] Calcula el área del recinto plano acotado limitado por la gráfica de $f(x)$ para $x \geq 0$, el eje $OX$ y la recta $x = 2$.
\end{parts}
\addpoints



\end{questions}

\end{document}
\grid
