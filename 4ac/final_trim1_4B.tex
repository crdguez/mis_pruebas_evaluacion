\documentclass[addpoints,spanish, 12pt,a4paper]{exam}
%\documentclass[answers, spanish, 12pt,a4paper]{exam}
% \printanswers
\pointpoints{punto}{puntos}
\hpword{Puntos:}
\vpword{Puntos:}
\htword{Total}
\vtword{Total}
\hsword{Resultado:}
\hqword{Ejercicio:}
\vqword{Ejercicio:}

\usepackage[utf8]{inputenc}
\usepackage[spanish]{babel}
\usepackage{eurosym}
%\usepackage[spanish,es-lcroman, es-tabla, es-noshorthands]{babel}


\usepackage[margin=1in]{geometry}
\usepackage{amsmath,amssymb}
\usepackage{multicol}
\usepackage{yhmath}

\pointsinrightmargin % Para poner las puntuaciones a la derecha. Se puede cambiar. Si se comenta, sale a la izquierda.
\extrawidth{-2.4cm} %Un poquito más de margen por si ponemos textos largos.
\marginpointname{ \emph{\points}}

\usepackage{graphicx}

\graphicspath{{../img/}} 

\newcommand{\class}{4º Académicas}
\newcommand{\examdate}{\today}
\newcommand{\examnum}{Examen final de trimestre 1}
\newcommand{\tipo}{B}


\newcommand{\timelimit}{50 minutos}

\renewcommand{\solutiontitle}{\noindent\textbf{Solución:}\enspace}


\pagestyle{head}
\firstpageheader{\includegraphics[width=0.2\columnwidth]{header_left}}{\textbf{Departamento de Matemáticas\linebreak \class}\linebreak \examnum}{\includegraphics[width=0.1\columnwidth]{header_right}}
\runningheader{\class}{\examnum}{Página \thepage\ of \numpages}
\runningheadrule


\begin{document}

\noindent
\begin{tabular*}{\textwidth}{l @{\extracolsep{\fill}} r @{\extracolsep{6pt}} }
\textbf{Nombre:} \makebox[3.5in]{\hrulefill} & \textbf{Fecha:}\makebox[1in]{\hrulefill} \\
 & \\
\textbf{Tiempo: \timelimit} & Tipo: \tipo 
\end{tabular*}
\rule[2ex]{\textwidth}{2pt}
Esta prueba tiene \numquestions\ ejercicios. La puntuación máxima es de \numpoints. 
La nota final de la prueba será la parte proporcional de la puntuación obtenida sobre la puntuación máxima. Para la recuperación de pendientes se tendrán en cuenta los apartados:\textbf{1,2a,5a}

\begin{center}


\addpoints
 %\gradetable[h][questions]
	\pointtable[h][questions]
\end{center}

\noindent
\rule[2ex]{\textwidth}{2pt}

\begin{questions}


% \question[1] 
% \begin{solution} \end{solution}
% \addpoints

% \question[1] Resuelve las siguientes ecuaciones sin usar la fórmula de la ecuación general de segundo grado
% \begin{parts}
% \part $100-4x^2=0$ \begin{solution}$x=5$\end{solution}
% \part $100x-4x^2=0$ \begin{solution}$x=0 \land x= 25$\end{solution}
% \end{parts}
\question[1] Resuelve las siguientes ecuaciones sin usar la fórmula de la ecuación general de segundo grado
\begin{parts}
\part $200-8x^2=0$ \begin{solution}$x=5$\end{solution}
\part $200x-8x^2=0$ \begin{solution}$x=0 \land x= 25$\end{solution}
\end{parts}

\question Calcula: 
\begin{parts}
% \part[1] Racionaliza y simplifica:  $\dfrac{3\sqrt{5}-4}{\sqrt{5}-2}$ 
% \begin{solution} $7+2\sqrt{5}$ \end{solution}
% \part[1] Racionaliza y simplifica:  $\dfrac{2\sqrt{8}-3\sqrt{2}}{2\sqrt{8}+3\sqrt{2}}$ 
% \begin{solution} $\frac{1}{7}$ \end{solution}
% \part[1] Racionaliza y simplifica:  $\dfrac{\sqrt{3}}{2\sqrt{3}-\sqrt{2}}$ 
% \begin{solution} $=\dfrac{\sqrt{3}\cdot\left(2\sqrt{3}+\sqrt{2}\right)}{\left(2\sqrt{3}-\sqrt{2}\right)\left(2\sqrt{3}+\sqrt{2}\right)}=\dfrac{6\sqrt{6}}{12-2}=\dfrac{6\sqrt{6}}{10}$ \end{solution}
\part[1] Racionaliza y simplifica:  $\dfrac{\sqrt{5}}{2\sqrt{5}+\sqrt{2}}$
\begin{solution} $\frac{10 - \sqrt{10}}{18}$ \end{solution}
% \part[1] Racionaliza y simplifica:  $\dfrac{\sqrt{5}}{3\sqrt{5}+\sqrt{3}}$
% \begin{solution} $\frac{15 - \sqrt{15}}{42}$ \end{solution}
\part[1] Aplica la definición de logaritmo para calcular: $\log_4 \sqrt{0,25}$
\begin{solution} $\to 4^x=\sqrt{\frac{1}{4}}\to4^x=4^{-1/2}\to\log_4 \sqrt{0,25}=-\frac{1}{2}$ \end{solution}
% \part[1] Aplica la definición de logaritmo para calcular: $\log_{4}{\sqrt{0,125}}$
% \begin{solution} $- \frac{3}{4}$ \end{solution}
% \part[1] Aplica la definición de logaritmo para calcular: $\log_5 \sqrt[3]{25}$
% \begin{solution} $\to 5^x=\sqrt[3]{5^2}\to5^x=5^{2/3}\to\log_5 \sqrt[3]{25}=\frac{2}{3}$ \end{solution}
% \part[1] Aplica la definición de logaritmo para calcular: $\log_4 \sqrt[3]{16}$
% \begin{solution} $\frac{2}{3}$ \end{solution}
% \part[2] Sabiendo que $\log x=1$ y $\log y=-2$, calcula: $\log (\dfrac{100\cdot x^2}{\sqrt{x\cdot y}})$
% \begin{solution}
% $\log (\frac{100\cdot x^2}{\sqrt{x\cdot y}})=\frac{3 \log{\left (x \right )}}{2} - \frac{\log{\left (y \right )}}{2} + 2=2 - \frac{-2}{2} + \frac{3 \cdot 1}{2}=\frac{9}{2}$      
% \end{solution}
% \part[1] Sabiendo que $\log x=2$ y $\log y=-1$, calcula: $\log (\dfrac{\sqrt{x\cdot y}}{100\cdot x^2})$
% \begin{solution}
% $\log (\frac{\sqrt{x\cdot y}}{100\cdot x^2})=\frac{3 \log{\left (x \right )}}{2} - \frac{\log{\left (y \right )}}{2} + 2=2 - \frac{-1}{2} + \frac{3 \cdot 2}{2}=\frac{11}{2}$      
% \end{solution}
\end{parts}

\addpoints

% \question[1] Halla el valor de $k$ para que la división $\left( 5x^3-2kx+k \right): \left(x - 2\right)$  tenga resto 1
% \begin{solution} $- 3 k + 40=1 \to k = 13 $ \end{solution}

% \question[1] Halla el valor de $k$ para que la división $\left( 5x^3-2kx+k \right): \left(x - 3\right)$  tenga resto 5
% \begin{solution} $- 5 k + 135=5 \to k = 26 $ \end{solution}

% \addpoints


\question[1] Dado $P(x)=-2x^3 + x^2 - 3x - 6$ y utilizando el teorema del resto, resuelve:
\begin{parts}
\part Valor numérico para $x=-1$ \begin{solution} $0$ \end{solution}
\part ¿Es divisible $P(x)$ por $x+1$? Justifica tu respuesta 
\begin{solution} Sí. Por el teorema del resto \end{solution}
\end{parts}

\addpoints


% \question[1] Halla el valor de \emph{k} para que la siguiente división sea exacta: $$(3x^2+kx-2):(x+2)$$
% \begin{solution} $\to 10-2k=0 \to k=5 $ \end{solution}

% \question[1] Halla el valor de \emph{m} para que el polinomio $mx^3-3x^2+5x+9m$ sea divisible por $x+2$
% \begin{solution}
%     $-8m-12-10+9m=0 \to m=22$
% \end{solution}

% \question[1] Halla el valor de \emph{k} para que la siguiente división sea exacta: $$(2x^4-6x^3+kx^2-11):(x+1)$$
% \begin{solution} $\to -3+k=0 \to k=3 $ \end{solution}

% \question[1] Halla el valor de \emph{k} para que $3x^2+kx-2$ sea divisible por $x+2$
% \begin{solution} $\to 10-2k=0 \to k=5 $ \end{solution}

\addpoints

\question[2] Simplifica la fracción algebraica: $$\dfrac{2x^3-5x^2+3x}{2x^2+x-6} $$
\begin{solution}$=\dfrac{2x\left(x-1\right)\left(x-\dfrac{3}{2}\right)}{2\left(x+2\right)\left(x-\dfrac{3}{2}\right)}=\dfrac{x(x-1)}{x+2}$  \end{solution}

% \question[2] Simplifica la fracción algebraica: $$\frac{2 x^{3} + 2 x^{2} - 4 x}{3 x^{4} + 3 x^{3} - 6 x^{2}}$$
% \begin{solution} $\frac{2 x^{3} + 2 x^{2} - 4 x}{3 x^{4} + 3 x^{3} - 6 x^{2}}=\frac{2 x \left(x - 1\right) \left(x + 2\right)}{3 x^{2} \left(x - 1\right) \left(x + 2\right)}=\frac{2}{3 x}$  \end{solution}

% \question[1] Simplifica la fracción algebraica: $$\frac{3 x^{4} - 3 x^{3} - 6 x^{2}}{2 x^{3} - 2 x^{2} - 4 x}$$
% \begin{solution} $\frac{3 x^{4} - 3 x^{3} - 6 x^{2}}{2 x^{3} - 2 x^{2} - 4 x}=\frac{3 x^{2} \left(x - 2\right) \left(x + 1\right)}{2 x \left(x - 2\right) \left(x + 1\right)}=\frac{3 x}{2}$  \end{solution}

% \addpoints

% \question[1] Simplifica la fracción algebraica: $$\dfrac{2x^4-6x^3+6x^2-2x}{6x^3-12x^2+6x} $$
% \begin{solution}$=\dfrac{2x\left(x-1\right)^3}{6x\left(x-1\right)^2}=\dfrac{x-1}{3}$ \end{solution}

% \addpoints

\question Resuelve las siguientes ecuaciones: 
\begin{parts}
\part[2] $$x^{4} - 13 x^{2} = 36$$
\begin{solution}$x = +- 3 , \land x=+-2$\end{solution}
% \part[1] Factoriza el polinomio de la izquierda de la ecuación anterior
% \begin{solution}$\left(x - 3\right) \left(x - 2\right) \left(x + 2\right) \left(x + 3\right)$\end{solution}
%\part[2] $$x^{4} - 5 x^{2} - 36=0$$
%\begin{solution}$x = 3 \land x=-3$\end{solution}
%\part[1] Factoriza el polinomio de la izquierda de la ecuación anterior
%\begin{solution}
%$\left(x - 3\right) \left(x + 3\right) \left(x^{2} + 4\right)$
%\end{solution}


\part[2] $$\dfrac{2x}{x+1}-\dfrac{1}{x}=\dfrac{5}{6}$$
\begin{solution}$\to \dfrac{12x^2}{6x\left(x+1\right)}-\dfrac{6\left(x+1\right)}{5x\left(x+1\right)\left(x+1\right)}=\dfrac{5}{6x\left(x+1\right)}\to 12x^2-6x-6=5x^2+5x\to 7x^2-11x-6=0\to x=2 \ x=-\frac{3}{7}$ \end{solution}
% \part[2] $$\dfrac{6x+1}{x^2-4}-\dfrac{x}{x-2}=\dfrac{x+1}{x+2}$$
% \begin{solution}$\to \dfrac{6x+1}{\left(x+2\right)\left(x-2\right)}-\dfrac{x\left(x+2\right)}{\left(x+2\right)\left(x-2\right)}=\dfrac{\left(x+1\right)\left(x-2\right)}{\left(x+2\right)\left(x-2\right)} \to 6x+1-x^2-2x=x^2-2x+x-2\to 0=2x^2-5x-3 \to x=\frac{5 \pm \sqrt{25+24}}{4}=\frac{5 \pm 7}{4}=\left\lbrace \begin{gathered}
% 	  x=-\frac{1}{2}  \hfill\\
% 	  x=3  
% 	\end{gathered} \right.$ \end{solution}

% \part[2] $$\frac{6x+1}{x^2-4}+\frac{x}{2-x}=\frac{x+1}{x+2}$$
% \begin{solution}$\frac{6x+1}{x^2-4}+\frac{x}{2-x}=\frac{x+1}{x+2} \to x=- \frac{1}{2}, x=3$ \end{solution}

% \part[2] $$\frac{x+1}{3x-6}-\frac{x+1}{2x+4}=\frac{10-x^2}{6x^2-24}$$
% \begin{solution}$\frac{x+1}{3x-6}-\frac{x+1}{2x+4}=\frac{10-x^2}{6x^2-24} \to x=0$ \end{solution}


% \part[2] $$2x^4-6x^3+6x^2=2x$$
% \begin{solution}$P(x)2x^4-6x^3+6x^2-2x=2x\left(x-1\right)^3$. Soluciones: $x=0$ y $x=1$ triple\end{solution}
% \part[2] $$6x^3-12x^2+6x=0$$
% \begin{solution}$P(x)=6x^3-12x^2+6x=6x\left(x-1\right)^2$. Soluciones: $x=0$ y $x=1$ doble \end{solution}

% \part[2] $$x^5-10x^4+31x^3-30x^2$$
% \begin{solution}$x^5-10x^4+31x^3-30x^2 \to x=0, x=2, x=3, x=5$ \end{solution}
% \part[2] $$-2x^5+10x^4-12x^3-8x^2+16x=0$$
% \begin{solution}$$-2x^5+10x^4-12x^3-8x^2+16x=0 \to x=-1, x=0, x=2$$ \end{solution}



% \part[2] $$\sqrt{x+1}+5=x$$
% \begin{solution}$\to x+1=\left(x-5\right)^2\to x+1=x^2+25-10x\to 0=x^2-11x+24$ Soluciones: $x=8$ válida y $x=3$ no válida \end{solution}
% \part[2] $$\sqrt{3x-2}+\sqrt{x-1}=3$$
% \begin{solution}$\to\sqrt{3x-2}=3-\sqrt{x-1}\to 3x-2=9+x-1-6\sqrt{x-1}\to6\sqrt{x-1}=9+x-1-3x+32\to6\sqrt{x-1}=10-2x\to3\sqrt{x-1}=5-x\to x-1=25+x^2-10x\to x^2-19x+34=0$. Soluciones: $x=2$ (Sí) y $x=17$ No  \end{solution}

% \part[2] $$2x+1=2\sqrt{1+x}+x$$
% \begin{solution}$2x+1=2\sqrt{1+x}+x \to x=-1, x=3$ \end{solution}

% \part[2] $$2x-3=\sqrt{3x-3}-2+x$$
% \begin{solution}$2x-3=\sqrt{3x-3}-2+x \to x=1, x=4$ \end{solution}

% \part[2] $$2+\sqrt{2x+3}=2x-1$$
% \begin{solution}$2+\sqrt{2x+3}=2x-1 \to x=3$ \end{solution}

% \part[2] $$2+\sqrt{2x+3}=2x-1$$
% \begin{solution}$2+\sqrt{2x+3}=2x-1 \to x=3$ \end{solution}


% \part[2] $$\sqrt{3x-2}+\sqrt{x-1}=3$$
% \begin{solution}$\sqrt{3x-2}+\sqrt{x-1}=3 \to x=2$ \end{solution}

\part[2] $$\sqrt{x+3}+\sqrt{x-2}=5$$
\begin{solution}$\sqrt{x+3}+\sqrt{x-2}=5 \to x=6$ \end{solution}

% \part[1] $$5^{3x-3}=625$$
% \begin{solution}$5^{3x-3}=5^4 \to x=\frac{7}{3}$ \end{solution}

\part[2] $$2^{x^2-4x+1}=\frac{1}{4}$$
\begin{solution}$2^{x^2-4x+1}=\frac{1}{4} \to x=1, x=3$ \end{solution}

% \part[2] $$3^{x-1}+3^{x}+3^{x+1}=117$$
% \begin{solution}$3^{x-1}+3^{x}+3^{x+1}=117 \to x=3$ \end{solution}	

% \part[2] $$\log {\left(x-1\right)}+ \log{2}=\log{\left(x^2+3\right)} - \log x$$ \\

% \begin{solution} $\to 2\left(x-1\right)=\frac{x^2+3}{x}\to 2x^2 -2x=x^2+3\to x^2-2x-3=0 \to x=\frac{2\pm\sqrt{4+12}}{2}=\left\lbrace \begin{gathered}
% 	  x=3  \to \textup{es solución} \hfill\\
% 	  x=-1  \to \textup{no es solución, no existen los logaritmos de negativos}
% 	\end{gathered} \right. $  \end{solution}

% \part[2] $$\left(x^2-5x+5\right)\log 5 + \log{20}=\log 4$$
% \begin{solution} $\to 5^{\left(x^2-5x+5\right)}\cdot20=4 \to 5^{\left(x^2-5x+5\right)}=\frac{1}{5} \to 5^{\left(x^2-5x+5\right)}=5^{-1}\to x^2-5x+5 = -1 \to x^2-5x+6=0 \to x=\frac{5 \pm \sqrt{25-24}}{2}=\left\lbrace \begin{gathered}
% 	  x=3  \to \textup{es solución} \hfill\\
% 	  x=2  \to \textup{es solución}
% 	\end{gathered} \right. $  \end{solution}

% \part[2] $$2\log x - \log (3x-5) = \log(5x)-1 $$
% \begin{solution}$2\log x - \log (3x-5) = \log(5x)-1  \to x=5$ \end{solution}	

% \part[2] $$\log(x-1)+\log 2 = \log (x^2+3) -\log x$$
% \begin{solution}$\log(x-1)+\log 2 = \log (x^2+3) -\log x \to x=3$ \end{solution}
	
\part[2] $$\log x + \log (x+3)=2 \log (x+1)$$
\begin{solution}
    $x(x+3)=(x+1)^2 \to x^2+3x=x^2+2x+1 \to x=1$
\end{solution}

\end{parts}

\addpoints

\end{questions}

\end{document}
\grid
