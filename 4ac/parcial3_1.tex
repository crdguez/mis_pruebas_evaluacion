\documentclass[addpoints,spanish, 12pt,a4paper]{exam}
%\documentclass[answers, spanish, 12pt,a4paper]{exam}
% \printanswers
\pointpoints{punto}{puntos}
\hpword{Puntos:}
\vpword{Puntos:}
\htword{Total}
\vtword{Total}
\hsword{Resultado:}
\hqword{Ejercicio:}
\vqword{Ejercicio:}

\usepackage[utf8]{inputenc}
\usepackage[spanish]{babel}
\usepackage{eurosym}
%\usepackage[spanish,es-lcroman, es-tabla, es-noshorthands]{babel}


\usepackage[margin=1in]{geometry}
\usepackage{amsmath,amssymb}
\usepackage{multicol}
\usepackage{yhmath}

\pointsinrightmargin % Para poner las puntuaciones a la derecha. Se puede cambiar. Si se comenta, sale a la izquierda.
\extrawidth{-2.4cm} %Un poquito más de margen por si ponemos textos largos.
\marginpointname{ \emph{\points}}

\usepackage{graphicx}

\graphicspath{{../img/}} 

\newcommand{\class}{4º Académicas}
\newcommand{\examdate}{\today}
\newcommand{\examnum}{Examen parcial 3ª evaluación}
\newcommand{\tipo}{A}


\newcommand{\timelimit}{50 minutos}

\renewcommand{\solutiontitle}{\noindent\textbf{Solución:}\enspace}


\pagestyle{head}
\firstpageheader{\includegraphics[width=0.2\columnwidth]{header_left}}{\textbf{Departamento de Matemáticas\linebreak \class}\linebreak \examnum}{\includegraphics[width=0.1\columnwidth]{header_right}}
\runningheader{\class}{\examnum}{Página \thepage\ of \numpages}
\runningheadrule


\begin{document}

\noindent
\begin{tabular*}{\textwidth}{l @{\extracolsep{\fill}} r @{\extracolsep{6pt}} }
\textbf{Nombre:} \makebox[3.5in]{\hrulefill} & \textbf{Fecha:}\makebox[1in]{\hrulefill} \\
 & \\
\textbf{Tiempo: \timelimit} & Tipo: \tipo 
\end{tabular*}
\rule[2ex]{\textwidth}{2pt}
Esta prueba tiene \numquestions\ ejercicios. La puntuación máxima es de \numpoints. 
La nota final de la prueba será la parte proporcional de la puntuación obtenida sobre la puntuación máxima. 

\begin{center}


\addpoints
 %\gradetable[h][questions]
	\pointtable[h][questions]
\end{center}

\noindent
\rule[2ex]{\textwidth}{2pt}

\begin{questions}


% \question[1] 
% \begin{solution} \end{solution}
% \addpoints

%\question 
%\begin{parts}
%\part[1]
%\begin{solution} \end{solution}
%\end{parts}
%\addpoints


%\question 
%\begin{parts}
%\part[1]
%\begin{solution} \end{solution}
%\end{parts}
%\addpointsquestion 

% ESTE ES DE SEMEJANZA

\question[1] Para medir la altura de una antena, Rosa se sitúa a 25 m de la misma. Desde esta posición, su visual al extremo superior de la antena pasa por la copa de un árbol de 2,5 m de alto. Calcula la altura de la antena sabiendo que Rosa mide 1,68 m y que se encuentra a 3 m de distancia del árbol.
\begin{solution}
    $6,83 + 1,68 = 8,51 m$
\end{solution}

% ESTE ES DE SEMEJANZA

\question[2] Calcula la altura de Juan sabiendo que proyecta una sombra de 2 metros en el momento en
que Pedro, que mide 1,80 m, proyecta una sombra de 2,25 metros.
\begin{solution}
    $1.60 m mide Juan$
\end{solution}


\question[2] Calcula, usando las identidades fundamentales de la trigonometría, las razones trigonométricas de un ángulo agudo $\alpha$ sabiendo que:\begin{parts} 
% \part[1] $\sen{\alpha}=\dfrac{\sqrt{2}}{2}$\begin{solution}  \\ $\sen{\alpha}=\dfrac{\sqrt{2}}{2}, \cos{\alpha}=\dfrac{\sqrt{2}}{2}, \tan{\alpha}=1$. \\ El ángulo agudo con esas razones es: $45^{\circ}$.\end{solution} 
\part[1] $\cos{\alpha}=\dfrac{\sqrt{2}}{2}$\begin{solution} $\sen{\alpha}=\dfrac{\sqrt{2}}{2}, \cos{\alpha}=\dfrac{\sqrt{2}}{2}, \tan{\alpha}=1$. \\ El ángulo agudo con esas razones es: $45^{\circ}$. \end{solution}
\part[2] $\tan{\alpha}=\sqrt{3}$\begin{solution}  \\ $\sen{\alpha}=\dfrac{\sqrt{3}}{2}, \cos{\alpha}=\dfrac{1}{2}, \tan{\alpha}=\sqrt{3}$. \\ El ángulo agudo con esas razones es: $60^{\circ}$.\end{solution} 
\end{parts} 

% \question Indica en qué cuadrante se encuentra y calcula, usando las identidades fundamentales de la trigonometría, las razones trigonométricas del ángulo $\alpha$ si:\begin{parts} 
% \part[2] $\tg{\alpha}=\sqrt{3} \ y \cos{\alpha}<0$ \begin{solution}  \\ $\sin{\alpha}=- \dfrac{\sqrt{3}}{2}, \cos{\alpha}=- \dfrac{1}{2}, \tan{\alpha}=\sqrt{3}$. \\ El ángulo que cumple las condiciones del ejercicio es: $240^{\circ}$\end{solution} 
% % \part[2] $\tg{\alpha}=\dfrac{\sqrt{3}}{3} \ y \cos{\alpha}<0$ \begin{solution}  \\ $\sin{\alpha}=- \dfrac{1}{2}, \cos{\alpha}=- \dfrac{\sqrt{3}}{2}, \tan{\alpha}=\dfrac{\sqrt{3}}{3}$. \\ El ángulo que cumple las condiciones del ejercicio es: $210^{\circ}$\end{solution} 

% \end{parts} 

\question Indica en qué cuadrante se encuentra y calcula las razones trigonométricas del ángulo $\alpha$ si:\begin{parts} 
\part[1] $\tg{\alpha}=\sqrt{3} \ y \cos{\alpha}<0$ \begin{solution}  \\ $\sin{\alpha}=- \dfrac{\sqrt{3}}{2}, \cos{\alpha}=- \dfrac{1}{2}, \tan{\alpha}=\sqrt{3}$. \\ El ángulo que cumple las condiciones del ejercicio es: $240^{\circ}$\end{solution} 
% \part[1] $\tg{\alpha}=\dfrac{\sqrt{3}}{3} \ y \cos{\alpha}<0$ \begin{solution}  \\ $\sin{\alpha}=- \dfrac{1}{2}, \cos{\alpha}=- \dfrac{\sqrt{3}}{2}, \tan{\alpha}=\dfrac{\sqrt{3}}{3}$. \\ El ángulo que cumple las condiciones del ejercicio es: $210^{\circ}$\end{solution} 

\end{parts} 



\question Calcula los lados y los ángulos del triángulo rectángulo:
\begin{parts} 
% \part[1] Sabiendo que la hipotenusa mide 17 y un cateto 8 cm.\begin{solution} LLos lados del triángulo miden: $8$, $15$, $17$ cm. Y los ángulos: $28.07$, $61.93$, $90$ º\end{solution} 
% \part[1] Sabiendo que un cateto mide 15 cm. y su ángulo opuesto 30º\begin{solution} Los lados del triángulo miden: $15$, $25.98$, $30$ cm. Y los ángulos: $30$, $60$, $90$ º\end{solution} 
% \part[1] Sabiendo que un cateto mide 30 cm. y su ángulo opuesto 45º\begin{solution} Los lados del triángulo miden: $30$, $30$, $42.43$ cm. Y los ángulos: $45$, $45$, $90$ º\end{solution} 
% \part[1] Sabiendo que la hipotenusa mide 20 cm. y un ángulo 60º\begin{solution} Los lados del triángulo miden: $17.32$, $10$, $20$ cm. Y los ángulos: $60$, $30$, $90$ º\end{solution} 
\part[1] Sabiendo que un cateto mide 20 cm. y el ángulo opuesto al otro cateto 30º\begin{solution} Los lados del triángulo miden: $20$, $11.55$, $23.09$ cm. Y los ángulos: $60$, $30$, $90$ º\end{solution} 
\end{parts} 



% \question Resuelve: \begin{parts} 
% \part[2] La diagonal menor de un rombo mide 20 cm y el ángulo menor es de 60$^{\circ}$. ¿Cuánto mide la diagonal?¿Y el lado del rombo?\begin{solution} Miden $34.64$ y $20.0$ respectivamente\end{solution} 
% % \part[2] La diagonal menor de un rombo mide 40 cm y el ángulo menor es de 60$^{\circ}$. ¿Cuánto mide la diagonal?¿Y el lado del rombo?\begin{solution} Miden $69.28$ y $40.0$ respectivamente\end{solution} 
% \end{parts} 
    
    
\question Resuelve\begin{parts} 
\part[2] Un carpintero quiere construir una escalera de tijera cuyos brazos, una vez abiertos, formen un ángulo de 60º. Si la altura de la escalera, estando abierta es de 3m, ¿qué longitud deberá tener cada brazo?\begin{solution} $2 \sqrt{3} \to 3.46\ m $\end{solution} 
\end{parts}
    
    
\question Resuelve: \begin{parts} 

\part[1] Calcula los ángulos de un rombo cuyas diagonales midan 12 y 8 cm, respectivamente
\part[2] Desde la torre de control de un aeropuerto se establece comunicación con un avión que va a aterrizar. En ese momento el avión se encuentra a una altura de 1200 m y el ángulo de observación desde la torre es de 30º. ¿A qué distancia está el avión del pie de la torre si ésta mide 40 m de altura?


    % \part[1] Un carpintero quiere construir una escalera de tijera cuyos brazos, una vez abiertos, formen un ángulo de 60º. Si la altura de la escalera, estando abierta es de 2m, ¿qué longitud deberá tener cada brazo?\begin{solution} $\dfrac{4 \sqrt{3}}{3} \to 2.31\ m $\end{solution}
% \part[2] Desde el punto donde estoy, la visual al punto más alto de una torre de 100 m que tengo 
%         enfrente forma un ángulo de $30^{\circ}$ con la horizontal. ¿Cuántos m me tengo que acercar para que el ángulo
%         sea de $60^{\circ}$?¿A cuántos metros estaba inicialmente?.\begin{solution} $ \left\{\begin{matrix}\tan{\left(30 \right)} = \dfrac{100}{x} \\ \tan{\left(60 \right)} = \dfrac{100}{y} \\ \end{matrix}\right. \to \left\{ x : 100 \sqrt{3}, \  y : \dfrac{100 \sqrt{3}}{3}\right\}\to 100 \sqrt{3} \land \dfrac{100 \sqrt{3}}{3} \to 115.47 \land 173.21\ m $\end{solution} 
        
%         \part[2] Para  hallar  el  ancho  de  un  río  procedemos  así:  Nos  situamos  en  un  punto  A,  en  una  orilla del  río,  y medimos  el  ángulo  (60º)  bajo el  cual  se  ve  un  árbol  que  está  frente  a  nosotros,  en  la  otra  orilla.  Nos  alejamos 20 m de la orilla en dirección perpendicular a ella y volvemos a medir el ángulo bajo el cual se ve el árbol, 30º. ¿Cuánto mide el ancho del rio? 
         
% \begin{solution}$ \left. \begin{gathered}
% 	  \tg 60 = \frac{y}{x} \hfill \\
% 	  \tg 30 = \frac{y}{x+20} \hfill \\ 
% 	\end{gathered}  \right\} \to \begin{Bmatrix}x=\frac{20\tan30}{\tan60 - \tan30} =10 m, & y= 10 \sqrt{3}\end{Bmatrix}$
% \end{solution}	  
        
        
          \part[2] Una antena de radio está sujeta al suelo con dos cables. Los ángulos que forman los cables con la antena son 30º y 45º. Los puntos de sujeción de los cables están alineados  con el pie de la antena  y distan entre sí 98 m.
Calcula la altura de la antena y la longitud de los cables.
\begin{solution}
$\left. \begin{gathered}
	  \tg 60 = \frac{y}{x} \\
	  \tg 45 = \frac{y}{98-x} \hfill
	 \end{gathered}  \right\rbrace \to \\
	 x=\frac{98\tg45}{\tg45+tg60}\approx35.870489570875 m\\
	 y=\frac{98\tg45\tg60}{\tg45+tg60}\approx62.129510429125 m\\
	 x_1=\frac{y}{\sen60}\approx71.74097914175 m \\
	 x_2=\frac{y}{\sen45}\approx87.8643962724692m$\\
Si has pensado que los ángulos eran sobre el suelo 
	$\left. \begin{gathered}
	  \tg 30 = \frac{y}{x} \\
	  \tg 45 = \frac{y}{98-x} \hfill
	 \end{gathered}  \right\rbrace \to \\
	 x=\frac{98\tg45}{\tg45+tg30}\approx62.129510429125 m\\
	 y=\frac{98\tg45\tg30}{\tg45+tg30}\approx35.870489570875 m\\
	 x_1=\frac{y}{\sen30}\approx71.74097914175 m \\
	 x_2=\frac{y}{\sen45}\approx50.7285328400941m$\end{solution}

        \end{parts} 
        
   

\question Resuelve las siguientes ecuaciones para ángulos que  estén entre 0º y 360º:\begin{parts} 
% \part[1] $\cos{x}=\dfrac{1}{2}$\begin{solution} $x=60^{\circ}, x=300^{\circ}$\end{solution} 
\part[1] $\cos{x}=-\dfrac{1}{2}$\begin{solution} $x=120^{\circ}, x=240^{\circ}$\end{solution} 
% \part[1] $4(\cos{x})^2-3=0$\begin{solution} $x=30^{\circ}, x=150^{\circ}, x=210^{\circ}, x=330^{\circ}$\end{solution} 
% \part[1] $4(\cos{x})^2-2=0$\begin{solution} $x=45^{\circ}, x=135^{\circ}, x=225^{\circ}, x=315^{\circ}$\end{solution} 
\part[1] $4(\cos{x})^2-3=0$\begin{solution} \end{solution} 
\end{parts}


% \question Usando las identidades fundamentales, demuestra:\begin{parts} 
% % \part[1] $\cos{x}=\dfrac{1}{2}$\begin{solution} $x=60^{\circ}, x=300^{\circ}$\end{solution} 
% % \part[1] $\cos{x}=-\dfrac{1}{2}$\begin{solution} $x=120^{\circ}, x=240^{\circ}$\end{solution} 
% % % \part[1] $4(\cos{x})^2-3=0$\begin{solution} $x=30^{\circ}, x=150^{\circ}, x=210^{\circ}, x=330^{\circ}$\end{solution} 
% % \part[1] $4(\cos{x})^2-2=0$\begin{solution} $x=45^{\circ}, x=135^{\circ}, x=225^{\circ}, x=315^{\circ}$\end{solution} 
% % \part[1] $\dfrac{1-\sin^2{\alpha}}{\cos{\alpha}}=\cos{\alpha}$
% \part[1] $\dfrac{1- \sen{\alpha}}{cos{\alpha}}=\dfrac{cos{\alpha}}{1+ \sen{\alpha}}$
% \end{parts}
\addpoints

\end{questions}

\end{document}
%\grid
