\documentclass{exam}
\usepackage{amsmath, amsthm, amssymb} 

\begin{document}

\begin{center}
\bfseries Prueba inicial
\end{center}
\textbf{Nombre:} \\
\textbf{Fecha:} \\
\textbf{Curso:} \\
\hline

\begin{questions}
\question Escribe en forma de intervalo y representa en la recta real los siguientes conjuntos:
\begin{parts}
    \part $A=\left\{ x \in \mathbf{R}| -1 < x \leq 5 \right\}$
        \vspace{50}

    \part $B= \left\{ x \in \mathbf{R}| |x-3|<5\right\}$
        \vspace{50}

\end{parts}

\question Dados los conjuntos $A=(1,8]$ y $B=(-\infty,5]$, calcula razonadamente:
\begin{parts}
    \part  $A \bigcap B=$
        \vspace{30}

    \part  $A \bigcup B=$
        \vspace{30}

\end{parts}

\question Simplifica, de manera detallada:
\begin{parts}
    \part $\dfrac{\sqrt{9}}{\sqrt[3]{3}}=$ 
    \vspace{40}
    \part $\dfrac{\sqrt[5]{16}}{\sqrt{2}}=$
        \vspace{40}

    \part $\dfrac{\sqrt[4]{a^3b^5c}}{\sqrt{ab^3c^3}}=$
        \vspace{40}

    \part $\left(\sqrt{\sqrt{\sqrt{2}}}\right)^8=$
        \vspace{40}

\end{parts}

\question Calcula razonadamente:\\
    $\sqrt{18}+\sqrt{50}-\sqrt{2}-\sqrt{8}=$

\end{questions}

\end{document}
