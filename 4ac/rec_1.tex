\documentclass[addpoints,spanish, 12pt,a4paper]{exam}
%\documentclass[answers, spanish, 12pt,a4paper]{exam}
\printanswers
\pointsinrightmargin
\marginpointname{ \emph{\points}}

\pointpoints{punto}{puntos}
\hpword{Puntos:}
\vpword{Puntos:}
\htword{Total}
\vtword{Total}
\hsword{Resultado:}
\hqword{Ejercicio:}
\vqword{Ejercicio:}

\usepackage[utf8]{inputenc}
\usepackage[spanish]{babel}
\usepackage{eurosym}
%\usepackage[spanish,es-lcroman, es-tabla, es-noshorthands]{babel}


\usepackage[margin=1in]{geometry}
\usepackage{amsmath,amssymb}
\usepackage{multicol}
\usepackage{yhmath}

\usepackage{verbatim}
%\usepackage{pstricks}


\usepackage{graphicx}
\graphicspath{{../img/}} 

\newcommand{\class}{4º Académicas}
\newcommand{\examdate}{\today}
\newcommand{\examnum}{Recuperación 1ª Evaluación}
\newcommand{\tipo}{A}


\newcommand{\timelimit}{50 minutos}

\renewcommand{\solutiontitle}{\noindent\textbf{Solución:}\enspace}


\pagestyle{head}
\firstpageheader{\includegraphics[width=0.2\columnwidth]{header_left}}{\textbf{Departamento de Matemáticas\linebreak \class}\linebreak \examnum}{\includegraphics[width=0.1\columnwidth]{header_right}}
\runningheader{\class}{\examnum}{Página \thepage\ of \numpages}
\runningheadrule


\begin{document}

\noindent
\begin{tabular*}{\textwidth}{l @{\extracolsep{\fill}} r @{\extracolsep{6pt}} }
\textbf{Nombre:} \makebox[3.5in]{\hrulefill} & \textbf{Fecha:}\makebox[1in]{\hrulefill} \\
 & \\
\textbf{Tiempo: \timelimit} & Tipo: \tipo 
\end{tabular*}
\rule[2ex]{\textwidth}{2pt}
Esta prueba tiene \numquestions\ ejercicios. La puntuación máxima es de \numpoints. 
La nota final de la prueba será la parte proporcional de la puntuación obtenida sobre la puntuación máxima. 

\begin{center}


\addpoints
 %\gradetable[h][questions]
	\pointtable[h][questions]
\end{center}

\noindent
\rule[2ex]{\textwidth}{2pt}

\begin{questions}

\begin{comment}
\question[1] 
\begin{solution} \end{solution}

\addpoints
\end{comment}

% \question[2] Indica a cuáles de los conjuntos
% $\mathbb{N}$, $\mathbb{Z}$, $\mathbb{Q}$, $\mathbb{R}$ pertenecen cada uno de los siguientes números:
% \begin{center}
% \begin{tabular}{|c |c |c |c |c|}\hline
% &$\mathbb{N}$& $\mathbb{Z}$& $\mathbb{Q}$&$\mathbb{R}$\\ 
% \hline
% $\frac{8}{16}$&&&&\\
% \hline
% $\sqrt[3]{-27}$&&&&\\
% \hline
% $3.0\wideparen{1}$&&&&\\
% \hline
% $-\frac{12}{4}$&&&&\\
% \hline
% $-\sqrt{25}$&&&&\\
% \hline
% $\sqrt{8}$&&&&\\
% \hline
% $4$&&&&\\
% \hline
% $\pi$&&&&\\
% \hline
% $\sqrt{-4}$&&&&\\
% \hline
% $\frac{39}{13}$&&&&\\
% \hline
% \end{tabular}

% \end{center}

% \addpoints


\question[1] Representa en la recta real y en forma de intervalo el siguiente conjunto numérico:
\addpoints % to omit double points count
$$\left\{ x \in \mathbb{R} \left| -5 \leqslant x < -1 \right. \right\}$$

\begin{solution}
$ $ 
\end{solution}

\question Calcula, justificadamente (sin usar la calculadora): 
\begin{parts}
%
%radsimp(2/3*sqrt(45)-sqrt(20)/2+4*sqrt(125)-sqrt(5))
% \part[1] $ \frac{2}{3}\sqrt{45}-\frac{\sqrt{20}}{2}+4\sqrt{125}-\sqrt{5}$ 
% \begin{solution} $20 \sqrt{5}$\end{solution}
\part[1] $ 2\sqrt{45}-\sqrt{20}+4\sqrt{125}-\sqrt{5}$ 
\begin{solution} $23 \sqrt{5}$\end{solution}
% \part[1] $\dfrac{2\sqrt[3]{2\sqrt{2}}}{\sqrt[4]{8}}$
% \begin{solution} $\sqrt[4]{2^3}$ \end{solution}
\part[1] $\dfrac{2+\sqrt{2}}{2-\sqrt{2}}$
\begin{solution} $2 \sqrt{2} + 3 $ \end{solution}
\part[1] $\sqrt[8]{4}\cdot\sqrt[6]{16}\cdot\sqrt[12]{8^5}$
\begin{solution} $ 4 \sqrt[6]{2}$ \end{solution}
\end{parts}

\addpoints
\question Calcula aplicando la definición de logaritmo:
\begin{parts}
\part[1] $\log_3 9$
\begin{solution} $2$ \end{solution}
%simplify(log(3,9))
\part[1] $\log_9 3$
\begin{solution} $ \frac{1}{2}$ \end{solution}
% \part[1] $\log_4 \sqrt{0,25}$
% \begin{solution} $ - \frac{1}{2}$ \end{solution}
\end{parts}

% \question[2] Calcula y simplifica: $$\left(\dfrac{3}{x}-\dfrac{2}{x+1}\right):\dfrac{x^2+x}{x-1}$$
% \begin{solution}$ =\frac{x + 3}{x \left(x + 1\right)}:\frac{x^2+x}{x-1}=\frac{\left(x - 1\right) \left(x + 3\right)}{x^{2} \left(x + 1\right)^{2}}=\dfrac{x^{2} + 2 x - 3}{x^{4} + 2 x^{3} + x^{2}}$  \end{solution}

% \addpoints



\question Resuelve las siguientes ecuaciones: 
\begin{parts}
\part[2] $$\dfrac{8}{x+6}+\dfrac{12-x}{x-6}=1$$
\begin{solution}$ \to\left(- 2 x^{2} + 14 x + 60\right)=0\to x=-3 \lor x=10$ \end{solution}

\part[2] $$2x^4-6x^3+6x^2-2x=0$$
\begin{solution}$P(x)2x^4-6x^3+6x^2-2x=2x\left(x-1\right)^3$. Soluciones: $x=0$ y $x=1$ triple\end{solution}

\part[2] $$\sqrt{3x-2}+\sqrt{x-1}=3$$
\begin{solution}$\to\sqrt{3x-2}=3-\sqrt{x-1}\to 3x-2=9+x-1-6\sqrt{x-1}\to6\sqrt{x-1}=9+x-1-3x+32\to6\sqrt{x-1}=10-2x\to3\sqrt{x-1}=5-x\to x-1=25+x^2-10x\to x^2-19x+34=0$. Soluciones: $x=2$ (Sí) y $x=17$ No  \end{solution}

\part[2] $$2\log x- \log{\left(3x-5\right)}=\log{5x} - 1$$
\begin{solution}$\to 2\log x- \log{\left(3x-5\right)}=\log{5x} - \log {10} \to \log {\frac{x^2}{3x-5}}=\log {\frac{5x}{10}}\to$ \\
$\to \frac{x^2}{3x-5}=\frac{x}{2}\to2x^2=3x^2-5x\to 0=x^2-5 \to 0=x\left(x-5\right)\to x=0 \ y \ x=5 \to$ \\ $\to$ de las dos soluciones, la única válida es $x=5$ ya que $\log 0$ no existe \end{solution}

\end{parts}

\addpoints

\end{questions}

\end{document}
\grid
